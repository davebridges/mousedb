% Generated by Sphinx.
\def\sphinxdocclass{report}
\documentclass[letterpaper,10pt,english]{sphinxmanual}
\usepackage[utf8]{inputenc}
\DeclareUnicodeCharacter{00A0}{\nobreakspace}
\usepackage[T1]{fontenc}
\usepackage{babel}
\usepackage{times}
\usepackage[Bjarne]{fncychap}
\usepackage{longtable}
\usepackage{sphinx}


\title{MouseDB Documentation}
\date{August 22, 2010}
\release{0.1}
\author{Dave Bridges, Ph.D.}
\newcommand{\sphinxlogo}{}
\renewcommand{\releasename}{Release}
\makeindex

\makeatletter
\def\PYG@reset{\let\PYG@it=\relax \let\PYG@bf=\relax%
    \let\PYG@ul=\relax \let\PYG@tc=\relax%
    \let\PYG@bc=\relax \let\PYG@ff=\relax}
\def\PYG@tok#1{\csname PYG@tok@#1\endcsname}
\def\PYG@toks#1+{\ifx\relax#1\empty\else%
    \PYG@tok{#1}\expandafter\PYG@toks\fi}
\def\PYG@do#1{\PYG@bc{\PYG@tc{\PYG@ul{%
    \PYG@it{\PYG@bf{\PYG@ff{#1}}}}}}}
\def\PYG#1#2{\PYG@reset\PYG@toks#1+\relax+\PYG@do{#2}}

\def\PYG@tok@gd{\def\PYG@tc##1{\textcolor[rgb]{0.63,0.00,0.00}{##1}}}
\def\PYG@tok@gu{\let\PYG@bf=\textbf\def\PYG@tc##1{\textcolor[rgb]{0.50,0.00,0.50}{##1}}}
\def\PYG@tok@gt{\def\PYG@tc##1{\textcolor[rgb]{0.00,0.25,0.82}{##1}}}
\def\PYG@tok@gs{\let\PYG@bf=\textbf}
\def\PYG@tok@gr{\def\PYG@tc##1{\textcolor[rgb]{1.00,0.00,0.00}{##1}}}
\def\PYG@tok@cm{\let\PYG@it=\textit\def\PYG@tc##1{\textcolor[rgb]{0.25,0.50,0.56}{##1}}}
\def\PYG@tok@vg{\def\PYG@tc##1{\textcolor[rgb]{0.73,0.38,0.84}{##1}}}
\def\PYG@tok@m{\def\PYG@tc##1{\textcolor[rgb]{0.13,0.50,0.31}{##1}}}
\def\PYG@tok@mh{\def\PYG@tc##1{\textcolor[rgb]{0.13,0.50,0.31}{##1}}}
\def\PYG@tok@cs{\def\PYG@tc##1{\textcolor[rgb]{0.25,0.50,0.56}{##1}}\def\PYG@bc##1{\colorbox[rgb]{1.00,0.94,0.94}{##1}}}
\def\PYG@tok@ge{\let\PYG@it=\textit}
\def\PYG@tok@vc{\def\PYG@tc##1{\textcolor[rgb]{0.73,0.38,0.84}{##1}}}
\def\PYG@tok@il{\def\PYG@tc##1{\textcolor[rgb]{0.13,0.50,0.31}{##1}}}
\def\PYG@tok@go{\def\PYG@tc##1{\textcolor[rgb]{0.19,0.19,0.19}{##1}}}
\def\PYG@tok@cp{\def\PYG@tc##1{\textcolor[rgb]{0.00,0.44,0.13}{##1}}}
\def\PYG@tok@gi{\def\PYG@tc##1{\textcolor[rgb]{0.00,0.63,0.00}{##1}}}
\def\PYG@tok@gh{\let\PYG@bf=\textbf\def\PYG@tc##1{\textcolor[rgb]{0.00,0.00,0.50}{##1}}}
\def\PYG@tok@ni{\let\PYG@bf=\textbf\def\PYG@tc##1{\textcolor[rgb]{0.84,0.33,0.22}{##1}}}
\def\PYG@tok@nl{\let\PYG@bf=\textbf\def\PYG@tc##1{\textcolor[rgb]{0.00,0.13,0.44}{##1}}}
\def\PYG@tok@nn{\let\PYG@bf=\textbf\def\PYG@tc##1{\textcolor[rgb]{0.05,0.52,0.71}{##1}}}
\def\PYG@tok@no{\def\PYG@tc##1{\textcolor[rgb]{0.38,0.68,0.84}{##1}}}
\def\PYG@tok@na{\def\PYG@tc##1{\textcolor[rgb]{0.25,0.44,0.63}{##1}}}
\def\PYG@tok@nb{\def\PYG@tc##1{\textcolor[rgb]{0.00,0.44,0.13}{##1}}}
\def\PYG@tok@nc{\let\PYG@bf=\textbf\def\PYG@tc##1{\textcolor[rgb]{0.05,0.52,0.71}{##1}}}
\def\PYG@tok@nd{\let\PYG@bf=\textbf\def\PYG@tc##1{\textcolor[rgb]{0.33,0.33,0.33}{##1}}}
\def\PYG@tok@ne{\def\PYG@tc##1{\textcolor[rgb]{0.00,0.44,0.13}{##1}}}
\def\PYG@tok@nf{\def\PYG@tc##1{\textcolor[rgb]{0.02,0.16,0.49}{##1}}}
\def\PYG@tok@si{\let\PYG@it=\textit\def\PYG@tc##1{\textcolor[rgb]{0.44,0.63,0.82}{##1}}}
\def\PYG@tok@s2{\def\PYG@tc##1{\textcolor[rgb]{0.25,0.44,0.63}{##1}}}
\def\PYG@tok@vi{\def\PYG@tc##1{\textcolor[rgb]{0.73,0.38,0.84}{##1}}}
\def\PYG@tok@nt{\let\PYG@bf=\textbf\def\PYG@tc##1{\textcolor[rgb]{0.02,0.16,0.45}{##1}}}
\def\PYG@tok@nv{\def\PYG@tc##1{\textcolor[rgb]{0.73,0.38,0.84}{##1}}}
\def\PYG@tok@s1{\def\PYG@tc##1{\textcolor[rgb]{0.25,0.44,0.63}{##1}}}
\def\PYG@tok@gp{\let\PYG@bf=\textbf\def\PYG@tc##1{\textcolor[rgb]{0.78,0.36,0.04}{##1}}}
\def\PYG@tok@sh{\def\PYG@tc##1{\textcolor[rgb]{0.25,0.44,0.63}{##1}}}
\def\PYG@tok@ow{\let\PYG@bf=\textbf\def\PYG@tc##1{\textcolor[rgb]{0.00,0.44,0.13}{##1}}}
\def\PYG@tok@sx{\def\PYG@tc##1{\textcolor[rgb]{0.78,0.36,0.04}{##1}}}
\def\PYG@tok@bp{\def\PYG@tc##1{\textcolor[rgb]{0.00,0.44,0.13}{##1}}}
\def\PYG@tok@c1{\let\PYG@it=\textit\def\PYG@tc##1{\textcolor[rgb]{0.25,0.50,0.56}{##1}}}
\def\PYG@tok@kc{\let\PYG@bf=\textbf\def\PYG@tc##1{\textcolor[rgb]{0.00,0.44,0.13}{##1}}}
\def\PYG@tok@c{\let\PYG@it=\textit\def\PYG@tc##1{\textcolor[rgb]{0.25,0.50,0.56}{##1}}}
\def\PYG@tok@mf{\def\PYG@tc##1{\textcolor[rgb]{0.13,0.50,0.31}{##1}}}
\def\PYG@tok@err{\def\PYG@bc##1{\fcolorbox[rgb]{1.00,0.00,0.00}{1,1,1}{##1}}}
\def\PYG@tok@kd{\let\PYG@bf=\textbf\def\PYG@tc##1{\textcolor[rgb]{0.00,0.44,0.13}{##1}}}
\def\PYG@tok@ss{\def\PYG@tc##1{\textcolor[rgb]{0.32,0.47,0.09}{##1}}}
\def\PYG@tok@sr{\def\PYG@tc##1{\textcolor[rgb]{0.14,0.33,0.53}{##1}}}
\def\PYG@tok@mo{\def\PYG@tc##1{\textcolor[rgb]{0.13,0.50,0.31}{##1}}}
\def\PYG@tok@mi{\def\PYG@tc##1{\textcolor[rgb]{0.13,0.50,0.31}{##1}}}
\def\PYG@tok@kn{\let\PYG@bf=\textbf\def\PYG@tc##1{\textcolor[rgb]{0.00,0.44,0.13}{##1}}}
\def\PYG@tok@o{\def\PYG@tc##1{\textcolor[rgb]{0.40,0.40,0.40}{##1}}}
\def\PYG@tok@kr{\let\PYG@bf=\textbf\def\PYG@tc##1{\textcolor[rgb]{0.00,0.44,0.13}{##1}}}
\def\PYG@tok@s{\def\PYG@tc##1{\textcolor[rgb]{0.25,0.44,0.63}{##1}}}
\def\PYG@tok@kp{\def\PYG@tc##1{\textcolor[rgb]{0.00,0.44,0.13}{##1}}}
\def\PYG@tok@w{\def\PYG@tc##1{\textcolor[rgb]{0.73,0.73,0.73}{##1}}}
\def\PYG@tok@kt{\def\PYG@tc##1{\textcolor[rgb]{0.56,0.13,0.00}{##1}}}
\def\PYG@tok@sc{\def\PYG@tc##1{\textcolor[rgb]{0.25,0.44,0.63}{##1}}}
\def\PYG@tok@sb{\def\PYG@tc##1{\textcolor[rgb]{0.25,0.44,0.63}{##1}}}
\def\PYG@tok@k{\let\PYG@bf=\textbf\def\PYG@tc##1{\textcolor[rgb]{0.00,0.44,0.13}{##1}}}
\def\PYG@tok@se{\let\PYG@bf=\textbf\def\PYG@tc##1{\textcolor[rgb]{0.25,0.44,0.63}{##1}}}
\def\PYG@tok@sd{\let\PYG@it=\textit\def\PYG@tc##1{\textcolor[rgb]{0.25,0.44,0.63}{##1}}}

\def\PYGZbs{\char`\\}
\def\PYGZus{\char`\_}
\def\PYGZob{\char`\{}
\def\PYGZcb{\char`\}}
\def\PYGZca{\char`\^}
% for compatibility with earlier versions
\def\PYGZat{@}
\def\PYGZlb{[}
\def\PYGZrb{]}
\makeatother

\begin{document}

\maketitle
\tableofcontents
\phantomsection\label{index::doc}


Contents:


\chapter{MouseDB Concepts}
\label{concepts:welcome-to-mousedb-s-documentation}\label{concepts::doc}\label{concepts:mousedb-concepts}
Data storage for MouseDB is separated into packages which contain information about animals, and information collected about animals.  There is also a separate module for timed matings of animals.  This document will describe the basics of how data is stored in each of these modules.


\section{Animal Module}
\label{concepts:animal-module}
Animals are tracked as individual entities, and given associations to breeding cages to follow ancestry, and strains.


\subsection{Animal}
\label{concepts:animal}
Most parameters about an animal are set within the animal object.  Here is where the animals strain, breeding, parentage and many other parameters are included.  Animals have foreignkey relationships with both Strain and Breeding, so an animal may only belong to one of each of those.  As an example, a mouse cannot come from more than one Breeding set, and cannot belong to more than one strain.


\subsubsection{Backcrosses and Generations}
\label{concepts:backcrosses-and-generations}
For this software, optional tracking of backcrosses and generations is available and is stored as an attribute of an animal.  When an inbred cross is made against a pure background, the backcross increases by 1.  When a heterozygote cross is made, the generation increases by one.  As an example, for every time a mouse in a C57/BL6 background is crossed against a wildtype C57/B6 mouse, the backcross (but not the generation) increases by one.  For every time a mutant strain is crosses against itself (either vs a heterozygote or homozygote of that strain), the generation will increase by one.  Backcrosses should typically be performed against a separate colony of purebred mouse, rather than against wild-type alleles of the mutant strain.


\subsection{Breeding Cages}
\label{concepts:breeding-cages}
A breeding cage is defined as a set of one or more male and one or more female mice.  Because of this, it is not always clear who the precise parentage of an animal is.  If the parentage is known, then the Mother and Father fields can be set for a particular animal.  In the case of Active, if an End field is specified, then the Active field is set to False.  In the case of Cage, if a Cage is provided, and animals are specified under Male or Females for a Breeding object, then the Cage field for those animals is set to that of the breeding cage.  The same is true for both Rack and Rack Position.


\subsection{Strains}
\label{concepts:strains}
A strain is a set of mice with a similar genetics.  Importantly strains are separated from Backgrounds.  For example, one might have mice with the genotype ob/ob but these mice may be in either a C57-Black6 or a mixed background.  This difference is set at the individual animal level.
The result of this is that a query for a particular strain may then need to be filtered to a specific background.


\section{Data Module}
\label{concepts:data-module}
Data (or measurements) can be stored for any type of measurement.  Conceptually, several pieces of data belong to an experiment (for example several mice are measured at some time) and several experiments belong to a study.  Measurements can be stored independent of experiments and experiments can be performed outside of the context of a study.  It is however, perfered that measurements are stored within an experiment and experiments are stored within studies as this will greatly facilitate the organization of the data.


\subsection{Studies}
\label{concepts:studies}
In general studies are a collection of experiments.  These can be grouped together on the basis of animals and/or treatment groups.  A study must have at least one treatment group, which defines the animals and their conditions.


\subsection{Experiments}
\label{concepts:experiments}
An experiment is a collection of measurements for a given set of animals.  In general, an experiment is defined as a number of measurements take in a given day.


\subsection{Measurements}
\label{concepts:measurements}
A measurement is an animal, an assay and a measurement value.  It can be associated with an experiment, or can stand alone as an individual value.  Measurements can be viewed in the context of a study, an experiment, a treatment group or an animal by going to the appropriate page.


\section{Timed Matings Module}
\label{concepts:timed-matings-module}
Timed matings are a specific type of breeding set.  Generally, for these experiments a mating cage is set up and pregnancy is defined by a plug event.  Based on this information, the age of an embryo can be estimated.  When a breeding cage is defined, one option is to set this cage as a timed mating cage (ie Timed\_Mating=True).  If this is the case, then a plug event can be registered and recorded for this mating set.  If the mother gives birth then this cage is implicitly set as a normal breeding cage.


\section{Groups Module}
\label{concepts:groups-module}
This app defines generic Group and License information for a particular installation of MouseDB.  Because every page on this site identifies both the Group and data restrictions, at a minimum, group information must be provided upon installation (see installation instructions).


\chapter{MouseDB Installation}
\label{installation:mousedb-installation}\label{installation::doc}

\section{Configuration}
\label{installation:configuration}
MouseDB requires both a database and a webserver to be set up.  Ideally, the database should be hosted separately from the webserver and MouseDB installation, but this is not necessary, as both can be used from the same server.  If you are using a remote server for the database, it is best to set up a user for this database that can only be accessed from the webserver.  If you want to set up several installations (ie for different users or different laboratories), you need separate databases and MouseDB installations for each.  You will also need to set up the webserver with different addresses for each installation.


\section{Software Dependencies}
\label{installation:software-dependencies}\begin{enumerate}
\item {} 
\textbf{MouseDB source code}.  Download from one of the following:

\end{enumerate}
\begin{enumerate}
\item {} 
\href{http://github.com/davebridges/mousedb/downloads}{http://github.com/davebridges/mousedb/downloads} for the current release

\item {} 
\href{http://github.com/davebridges/mousedb}{http://github.com/davebridges/mousedb} for the source code via Git

\end{enumerate}

Downloading and/or unzipping will create a directory named mousedb.  You can update to the newest revision at any time either using git or downloading and re-installing the newer version.  Changing or updating software versions will not alter any saved data, but you will have to update the localsettings.py file (described below).
\begin{enumerate}
\setcounter{enumi}{1}
\item {} 
\textbf{Python}.  Requires Version 2.6, is not yet compatible with Python 3.0.  Download from \href{http://www.python.org/download/}{http://www.python.org/download/}.

\item {} 
\textbf{Django}.  Download from \href{http://www.djangoproject.com/download/}{http://www.djangoproject.com/download/}.  MouseDB currently requires at least Django 1.2.

\item {} 
\textbf{Database software}.  Typically MySQL is used, but PostgreSQL, Oracle or SQLite can also be used.  You also need to install the python driver for this database (unless you are using SQLite, which is internal to Python 2.5+).  See \href{http://docs.djangoproject.com/en/dev/topics/install/database-installation}{http://docs.djangoproject.com/en/dev/topics/install/database-installation} - Django Database Installation Documentation for more information.

\end{enumerate}


\section{Database Setup}
\label{installation:database-setup}\begin{enumerate}
\item {} 
Create a new database.  You need to record the user, password, host and database name.  If you are using SQLite this step is not required.

\item {} 
Go to localsettings\_empty.py and edit the settings:

\end{enumerate}
\begin{itemize}
\item {} 
ENGINE: Choose one of `django.db.backends.postgresql\_psycopg2','django.db.backends.postgresql', `django.db.backends.mysql', `django.db.backends.sqlite3', `django.db.backends.oracle' depending on the database software used.

\item {} 
NAME: database name

\item {} 
USER: database user

\item {} 
PASSWORD: database password

\item {} 
HOST: database host

\end{itemize}
\begin{enumerate}
\setcounter{enumi}{2}
\item {} 
Save this file as localsettings.py in the main MouseDB directory.

\end{enumerate}


\section{Web Server Setup}
\label{installation:web-server-setup}
You need to set up a server to serve both the django installation and saved files.  For the saved files.  I recommend using apache for both.  The preferred setup is to use Apache2 with mod\_wsgi.  See \href{http://code.google.com/p/modwsgi/wiki/InstallationInstructions}{http://code.google.com/p/modwsgi/wiki/InstallationInstructions} for instructions on using mod\_wsgi.  The following is a httpd.conf example where the code is placed in /usr/src/mousedb:

Alias /robots.txt ``usrsrcmousedbmediarobots.txt''
Alias /favicon.ico ``usrsrcmediafavicon.ico''

Alias /mousedb-media ``usrsrcmousedbmedia''
\begin{description}
\item[{\textless{}Directory ``usrsrcmousedb''\textgreater{}}] \leavevmode
Order allow,deny
Allow from all

\end{description}

\textless{}/Directory\textgreater{}

WSGIScriptAlias /mousedb ``usrsrcmousedbapachedjango.wsgi''

Then open up the mousedb/apache/django.wsgi file and replace the location of the mousedb directory on the sys.path.append line.

If you want to restrict access to these files, change the Allow from all directive to specific domains or ip addresses (for example Allow from 192.168.0.0/99 would allow from 192.168.0.0 to 192.168.0.99)


\section{Final Configuration and User Setup}
\label{installation:final-configuration-and-user-setup}
Go to a command prompt, navigate to inside the mousedb directory and enter the following to get to a python prompt:

\begin{Verbatim}[commandchars=@\[\]]
python manage.py shell
\end{Verbatim}

Go to servername/mousedb/admin/groups/group/1 and name your research group and select a license if desired

Go to servername/mousedb/admin/auth/users/ and create users, selecting usernames, full names, password (or have the user set the password) and then choose group permissions.


\chapter{Animal Data Entry}
\label{usage:animal-data-entry}\label{usage::doc}

\section{Newborn Mice or Newly Weaned Mice}
\label{usage:newborn-mice-or-newly-weaned-mice}\begin{enumerate}
\item {} 
Go to Breeding Cages Tab

\item {} 
Click on Add/Wean Pups Button

\item {} 
Each row is a new animal.  If you accidentaly enter an extra animal, check off the delete box then submit.

\item {} 
Leave extra lines blank if you have less than 10 mice to enter

\item {} 
If you need to enter more than 10 mice, enter the first ten and submit them.  Go back and enter up to 10 more animals (10 more blank spaces will appear)

\end{enumerate}


\section{Newborn Mice}
\label{usage:newborn-mice}\begin{enumerate}
\item {} 
Enter Breeding Cage under Cage

\item {} 
Enter Strain

\item {} 
Enter Background (normally Mixed or C57BL/6-BA unless from the LY breeding cages in which case it is C57BL/6-LY5.2)

\item {} 
Enter Birthdate in format YYYY-MM-DD

\item {} 
Enter Generation and Backcross

\end{enumerate}


\section{Weaning Mice}
\label{usage:weaning-mice}\begin{enumerate}
\item {} 
If not previously entered, enter data as if newborn mice

\item {} 
Enter gender

\item {} 
Enter Wean Date in format YYYY-MM-DD

\item {} 
Enter new Cage number for Cage

\end{enumerate}


\section{Cage Changes (Not Weaning)}
\label{usage:cage-changes-not-weaning}\begin{enumerate}
\item {} 
Find mouse either from animal list or strain list

\item {} 
Click the edit mouse button

\item {} 
Change the Cage, Rack and Rack Position as Necessary

\end{enumerate}


\section{Genotyping or Ear Tagging}
\label{usage:genotyping-or-ear-tagging}\begin{enumerate}
\item {} 
Find mouse either from animal list or strain list, or through breeding cage

\item {} 
Click the edit mouse button or the Eartag/Genotype/Cage Change/Death Button

\item {} 
Enter the Ear Tag and/or select the Genotype from the Pull Down List

\end{enumerate}


\section{Marking Mice as Dead}
\label{usage:marking-mice-as-dead}

\subsection{Dead Mice (Single Mouse)}
\label{usage:dead-mice-single-mouse}\begin{enumerate}
\item {} 
Find mouse from animal list or strain list

\item {} 
Click the edit mouse button

\item {} 
Enter the death date in format YYYY-MM-DD

\item {} 
Choose Cause of Death from Pull Down List

\end{enumerate}


\subsection{Dead Mice (Several Mice)}
\label{usage:dead-mice-several-mice}\begin{enumerate}
\item {} 
Find mice from breeding cages

\item {} 
Click the Eartag/Genotype/Cage Change/Death Button

\item {} 
Enter the death date in format YYYY-MM-DD

\item {} 
Choose the Cause of Death from Pull Down List

\end{enumerate}


\chapter{Studies and Experimental Setup}
\label{usage:studies-and-experimental-setup}
Set up a new study at /mousedb/admin/data/study/ selecting animals

You must put a description and select animals in one or more treatment groups

If you have more than 2 treatment groups save the first two, then two more empty slots will appear. For animals, click on the magnifying glass then find the animal in that treatment group and click on the MouseID. The number displayed now in that field will not be the MouseID, but don't worry its just a different number to describe the mouse. To add more animals, click on the magnifying glass again and select the next animal. There should be now two numbers, separated by commas in this field. Repeat to fill all your treatment groups. You must enter a diet and environment for each treatment. The other fields are optional, and should only be used if appropriate. Ensure for pharmaceutical, you include a saline treatment group.


\chapter{Measurment Entry}
\label{usage:measurment-entry}

\section{Studies}
\label{usage:studies}
If this measurement is part of a study (ie a group of experiments) then click on the plus sign beside the study field and enter in the details about the study and treatment groups.  Unfortunately until i can figure out how to filter the treatment group animals in the admin interface, at each of the subsequent steps you will see all the animals in the database (soon hopefully it will only be the ones as part of the study group).


\section{Experiment Details}
\label{usage:experiment-details}\begin{itemize}
\item {} 
Pick experiment date, feeding state and resarchers

\item {} 
Pick animals used in this experiment (the search box will filter results)

\item {} 
Fasting state, time, injections, concentration, experimentID and notes are all optional

\end{itemize}


\section{Measurements}
\label{usage:measurements}\begin{itemize}
\item {} 
There is room to enter 14 measurements.  If you need more rows, enter the first 14 and select ``Save and Continue Editing'' and 14 more blank spots will appear.

\item {} 
Each row is a measurement, so if you have glucose and weight for some animal that is two rows entered.

\item {} 
For animals, click on the magnifying glass then find the animal in that treatment group and click on the MouseID. The number displayed now in that field will not be the MouseID, but don't worry its just a different number to describe the mouse.

\item {} 
For values, the standard units (defined by each assay) are mg for weights, mg/dL for glucose and pg/mL for insulin).  You must enter integers here (no decimal places).  If you have several measurements (ie several glucose readings during a GTT, enter them all in one measurement row, separated by commas and \emph{NO spaces}).

\end{itemize}


\chapter{Automated Documentation}
\label{api:automated-documentation}\label{api::doc}

\section{Data Package}
\label{api:data-package}\label{api:module-data}\index{data (module)}
The data module describes the conditions and collection of data regarding experimental animals.

Data (or measurements) can be stored for any type of measurement.  Conceptually, several pieces of data belong to an experiment (for example several mice are measured at some time) and several experiments belong to a study.  Measurements can be stored independent of experiments and experiments can be performed outside of the context of a study.  It is however, perfered that measurements are stored within an experiment and experiments are stored within studies as this will greatly facilitate the organization of the data.


\subsection{Studies}
\label{api:studies}
In general studies are a collection of experiments.  These can be grouped together on the basis of animals and/or treatment groups.  A study must have at least one treatment group, which defines the animals and their conditions.


\subsection{Experiments}
\label{api:experiments}
An experiment is a collection of measurements for a given set of animals.  In general, an experiment is defined as a number of measurements take in a given day.


\subsection{Measurements}
\label{api:measurements}
A measurement is an animal, an assay and a measurement value.  It can be associated with an experiment, or can stand alone as an individual value.  Measurements can be viewed in the context of a study, an experiment, a treatment group or an animal by going to the appropriate page.


\subsection{Models}
\label{api:models}\label{api:module-data.models}\index{data.models (module)}\index{Assay (class in data.models)}

\begin{fulllineitems}
\phantomsection\label{api:data.models.Assay}\pysiglinewithargsret{\strong{class }\code{data.models.}\bfcode{Assay}}{\emph{*args}, \emph{**kwargs}}{}
Bases: \code{django.db.models.base.Model}

Assay(id, assay, assay\_slug, notes, measurement\_units)
\index{Assay.DoesNotExist}

\begin{fulllineitems}
\phantomsection\label{api:data.models.Assay.DoesNotExist}\pysigline{\strong{exception }\bfcode{DoesNotExist}}{}
Bases: \code{django.core.exceptions.ObjectDoesNotExist}

\end{fulllineitems}

\index{Assay.MultipleObjectsReturned}

\begin{fulllineitems}
\phantomsection\label{api:data.models.Assay.MultipleObjectsReturned}\pysigline{\strong{exception }\code{Assay.}\bfcode{MultipleObjectsReturned}}{}
Bases: \code{django.core.exceptions.MultipleObjectsReturned}

\end{fulllineitems}

\index{measurement\_set (data.models.Assay attribute)}

\begin{fulllineitems}
\phantomsection\label{api:data.models.Assay.measurement_set}\pysigline{\code{Assay.}\bfcode{measurement\_set}}{}
\end{fulllineitems}


\end{fulllineitems}

\index{Diet (class in data.models)}

\begin{fulllineitems}
\phantomsection\label{api:data.models.Diet}\pysiglinewithargsret{\strong{class }\code{data.models.}\bfcode{Diet}}{\emph{*args}, \emph{**kwargs}}{}
Bases: \code{django.db.models.base.Model}

Diet(id, vendor\_id, description, product\_id, fat\_content, protein\_content, carb\_content, irradiated, notes)
\index{Diet.DoesNotExist}

\begin{fulllineitems}
\phantomsection\label{api:data.models.Diet.DoesNotExist}\pysigline{\strong{exception }\bfcode{DoesNotExist}}{}
Bases: \code{django.core.exceptions.ObjectDoesNotExist}

\end{fulllineitems}

\index{Diet.MultipleObjectsReturned}

\begin{fulllineitems}
\phantomsection\label{api:data.models.Diet.MultipleObjectsReturned}\pysigline{\strong{exception }\code{Diet.}\bfcode{MultipleObjectsReturned}}{}
Bases: \code{django.core.exceptions.MultipleObjectsReturned}

\end{fulllineitems}

\index{treatment\_set (data.models.Diet attribute)}

\begin{fulllineitems}
\phantomsection\label{api:data.models.Diet.treatment_set}\pysigline{\code{Diet.}\bfcode{treatment\_set}}{}
\end{fulllineitems}

\index{vendor (data.models.Diet attribute)}

\begin{fulllineitems}
\phantomsection\label{api:data.models.Diet.vendor}\pysigline{\code{Diet.}\bfcode{vendor}}{}
\end{fulllineitems}


\end{fulllineitems}

\index{Environment (class in data.models)}

\begin{fulllineitems}
\phantomsection\label{api:data.models.Environment}\pysiglinewithargsret{\strong{class }\code{data.models.}\bfcode{Environment}}{\emph{*args}, \emph{**kwargs}}{}
Bases: \code{django.db.models.base.Model}

Environment(id, building, room, temperature, humidity, notes)
\index{Environment.DoesNotExist}

\begin{fulllineitems}
\phantomsection\label{api:data.models.Environment.DoesNotExist}\pysigline{\strong{exception }\bfcode{DoesNotExist}}{}
Bases: \code{django.core.exceptions.ObjectDoesNotExist}

\end{fulllineitems}

\index{Environment.MultipleObjectsReturned}

\begin{fulllineitems}
\phantomsection\label{api:data.models.Environment.MultipleObjectsReturned}\pysigline{\strong{exception }\code{Environment.}\bfcode{MultipleObjectsReturned}}{}
Bases: \code{django.core.exceptions.MultipleObjectsReturned}

\end{fulllineitems}

\index{contact (data.models.Environment attribute)}

\begin{fulllineitems}
\phantomsection\label{api:data.models.Environment.contact}\pysigline{\code{Environment.}\bfcode{contact}}{}
\end{fulllineitems}

\index{treatment\_set (data.models.Environment attribute)}

\begin{fulllineitems}
\phantomsection\label{api:data.models.Environment.treatment_set}\pysigline{\code{Environment.}\bfcode{treatment\_set}}{}
\end{fulllineitems}


\end{fulllineitems}

\index{Experiment (class in data.models)}

\begin{fulllineitems}
\phantomsection\label{api:data.models.Experiment}\pysiglinewithargsret{\strong{class }\code{data.models.}\bfcode{Experiment}}{\emph{*args}, \emph{**kwargs}}{}
Bases: \code{django.db.models.base.Model}

Experiment(id, date, notes, experimentID, feeding\_state, fasting\_time, injection, concentration, study\_id)
\index{Experiment.DoesNotExist}

\begin{fulllineitems}
\phantomsection\label{api:data.models.Experiment.DoesNotExist}\pysigline{\strong{exception }\bfcode{DoesNotExist}}{}
Bases: \code{django.core.exceptions.ObjectDoesNotExist}

\end{fulllineitems}

\index{Experiment.MultipleObjectsReturned}

\begin{fulllineitems}
\phantomsection\label{api:data.models.Experiment.MultipleObjectsReturned}\pysigline{\strong{exception }\code{Experiment.}\bfcode{MultipleObjectsReturned}}{}
Bases: \code{django.core.exceptions.MultipleObjectsReturned}

\end{fulllineitems}

\index{animals (data.models.Experiment attribute)}

\begin{fulllineitems}
\phantomsection\label{api:data.models.Experiment.animals}\pysigline{\code{Experiment.}\bfcode{animals}}{}
\end{fulllineitems}

\index{get\_absolute\_url() (data.models.Experiment method)}

\begin{fulllineitems}
\phantomsection\label{api:data.models.Experiment.get_absolute_url}\pysiglinewithargsret{\code{Experiment.}\bfcode{get\_absolute\_url}}{\emph{*moreargs}, \emph{**morekwargs}}{}
\end{fulllineitems}

\index{get\_feeding\_state\_display() (data.models.Experiment method)}

\begin{fulllineitems}
\phantomsection\label{api:data.models.Experiment.get_feeding_state_display}\pysiglinewithargsret{\code{Experiment.}\bfcode{get\_feeding\_state\_display}}{\emph{*moreargs}, \emph{**morekwargs}}{}
\end{fulllineitems}

\index{get\_injection\_display() (data.models.Experiment method)}

\begin{fulllineitems}
\phantomsection\label{api:data.models.Experiment.get_injection_display}\pysiglinewithargsret{\code{Experiment.}\bfcode{get\_injection\_display}}{\emph{*moreargs}, \emph{**morekwargs}}{}
\end{fulllineitems}

\index{get\_next\_by\_date() (data.models.Experiment method)}

\begin{fulllineitems}
\phantomsection\label{api:data.models.Experiment.get_next_by_date}\pysiglinewithargsret{\code{Experiment.}\bfcode{get\_next\_by\_date}}{\emph{*moreargs}, \emph{**morekwargs}}{}
\end{fulllineitems}

\index{get\_previous\_by\_date() (data.models.Experiment method)}

\begin{fulllineitems}
\phantomsection\label{api:data.models.Experiment.get_previous_by_date}\pysiglinewithargsret{\code{Experiment.}\bfcode{get\_previous\_by\_date}}{\emph{*moreargs}, \emph{**morekwargs}}{}
\end{fulllineitems}

\index{measurement\_set (data.models.Experiment attribute)}

\begin{fulllineitems}
\phantomsection\label{api:data.models.Experiment.measurement_set}\pysigline{\code{Experiment.}\bfcode{measurement\_set}}{}
\end{fulllineitems}

\index{researchers (data.models.Experiment attribute)}

\begin{fulllineitems}
\phantomsection\label{api:data.models.Experiment.researchers}\pysigline{\code{Experiment.}\bfcode{researchers}}{}
\end{fulllineitems}

\index{study (data.models.Experiment attribute)}

\begin{fulllineitems}
\phantomsection\label{api:data.models.Experiment.study}\pysigline{\code{Experiment.}\bfcode{study}}{}
\end{fulllineitems}


\end{fulllineitems}

\index{Implantation (class in data.models)}

\begin{fulllineitems}
\phantomsection\label{api:data.models.Implantation}\pysiglinewithargsret{\strong{class }\code{data.models.}\bfcode{Implantation}}{\emph{*args}, \emph{**kwargs}}{}
Bases: \code{django.db.models.base.Model}

Implantation(id, implant, vendor\_id, product\_id, notes)
\index{Implantation.DoesNotExist}

\begin{fulllineitems}
\phantomsection\label{api:data.models.Implantation.DoesNotExist}\pysigline{\strong{exception }\bfcode{DoesNotExist}}{}
Bases: \code{django.core.exceptions.ObjectDoesNotExist}

\end{fulllineitems}

\index{Implantation.MultipleObjectsReturned}

\begin{fulllineitems}
\phantomsection\label{api:data.models.Implantation.MultipleObjectsReturned}\pysigline{\strong{exception }\code{Implantation.}\bfcode{MultipleObjectsReturned}}{}
Bases: \code{django.core.exceptions.MultipleObjectsReturned}

\end{fulllineitems}

\index{surgeon (data.models.Implantation attribute)}

\begin{fulllineitems}
\phantomsection\label{api:data.models.Implantation.surgeon}\pysigline{\code{Implantation.}\bfcode{surgeon}}{}
\end{fulllineitems}

\index{treatment\_set (data.models.Implantation attribute)}

\begin{fulllineitems}
\phantomsection\label{api:data.models.Implantation.treatment_set}\pysigline{\code{Implantation.}\bfcode{treatment\_set}}{}
\end{fulllineitems}

\index{vendor (data.models.Implantation attribute)}

\begin{fulllineitems}
\phantomsection\label{api:data.models.Implantation.vendor}\pysigline{\code{Implantation.}\bfcode{vendor}}{}
\end{fulllineitems}


\end{fulllineitems}

\index{Measurement (class in data.models)}

\begin{fulllineitems}
\phantomsection\label{api:data.models.Measurement}\pysiglinewithargsret{\strong{class }\code{data.models.}\bfcode{Measurement}}{\emph{*args}, \emph{**kwargs}}{}
Bases: \code{django.db.models.base.Model}

Measurement(id, animal\_id, experiment\_id, assay\_id, values)
\index{Measurement.DoesNotExist}

\begin{fulllineitems}
\phantomsection\label{api:data.models.Measurement.DoesNotExist}\pysigline{\strong{exception }\bfcode{DoesNotExist}}{}
Bases: \code{django.core.exceptions.ObjectDoesNotExist}

\end{fulllineitems}

\index{Measurement.MultipleObjectsReturned}

\begin{fulllineitems}
\phantomsection\label{api:data.models.Measurement.MultipleObjectsReturned}\pysigline{\strong{exception }\code{Measurement.}\bfcode{MultipleObjectsReturned}}{}
Bases: \code{django.core.exceptions.MultipleObjectsReturned}

\end{fulllineitems}

\index{age() (data.models.Measurement method)}

\begin{fulllineitems}
\phantomsection\label{api:data.models.Measurement.age}\pysiglinewithargsret{\code{Measurement.}\bfcode{age}}{}{}
\end{fulllineitems}

\index{animal (data.models.Measurement attribute)}

\begin{fulllineitems}
\phantomsection\label{api:data.models.Measurement.animal}\pysigline{\code{Measurement.}\bfcode{animal}}{}
\end{fulllineitems}

\index{assay (data.models.Measurement attribute)}

\begin{fulllineitems}
\phantomsection\label{api:data.models.Measurement.assay}\pysigline{\code{Measurement.}\bfcode{assay}}{}
\end{fulllineitems}

\index{experiment (data.models.Measurement attribute)}

\begin{fulllineitems}
\phantomsection\label{api:data.models.Measurement.experiment}\pysigline{\code{Measurement.}\bfcode{experiment}}{}
\end{fulllineitems}


\end{fulllineitems}

\index{Pharmaceutical (class in data.models)}

\begin{fulllineitems}
\phantomsection\label{api:data.models.Pharmaceutical}\pysiglinewithargsret{\strong{class }\code{data.models.}\bfcode{Pharmaceutical}}{\emph{*args}, \emph{**kwargs}}{}
Bases: \code{django.db.models.base.Model}

Pharmaceutical(id, drug, dose, recurrance, mode, vendor\_id, notes)
\index{Pharmaceutical.DoesNotExist}

\begin{fulllineitems}
\phantomsection\label{api:data.models.Pharmaceutical.DoesNotExist}\pysigline{\strong{exception }\bfcode{DoesNotExist}}{}
Bases: \code{django.core.exceptions.ObjectDoesNotExist}

\end{fulllineitems}

\index{Pharmaceutical.MultipleObjectsReturned}

\begin{fulllineitems}
\phantomsection\label{api:data.models.Pharmaceutical.MultipleObjectsReturned}\pysigline{\strong{exception }\code{Pharmaceutical.}\bfcode{MultipleObjectsReturned}}{}
Bases: \code{django.core.exceptions.MultipleObjectsReturned}

\end{fulllineitems}

\index{get\_mode\_display() (data.models.Pharmaceutical method)}

\begin{fulllineitems}
\phantomsection\label{api:data.models.Pharmaceutical.get_mode_display}\pysiglinewithargsret{\code{Pharmaceutical.}\bfcode{get\_mode\_display}}{\emph{*moreargs}, \emph{**morekwargs}}{}
\end{fulllineitems}

\index{treatment\_set (data.models.Pharmaceutical attribute)}

\begin{fulllineitems}
\phantomsection\label{api:data.models.Pharmaceutical.treatment_set}\pysigline{\code{Pharmaceutical.}\bfcode{treatment\_set}}{}
\end{fulllineitems}

\index{vendor (data.models.Pharmaceutical attribute)}

\begin{fulllineitems}
\phantomsection\label{api:data.models.Pharmaceutical.vendor}\pysigline{\code{Pharmaceutical.}\bfcode{vendor}}{}
\end{fulllineitems}


\end{fulllineitems}

\index{Researcher (class in data.models)}

\begin{fulllineitems}
\phantomsection\label{api:data.models.Researcher}\pysiglinewithargsret{\strong{class }\code{data.models.}\bfcode{Researcher}}{\emph{*args}, \emph{**kwargs}}{}
Bases: \code{django.db.models.base.Model}

Researcher(id, first\_name, last\_name, name\_slug, email, active)
\index{Researcher.DoesNotExist}

\begin{fulllineitems}
\phantomsection\label{api:data.models.Researcher.DoesNotExist}\pysigline{\strong{exception }\bfcode{DoesNotExist}}{}
Bases: \code{django.core.exceptions.ObjectDoesNotExist}

\end{fulllineitems}

\index{Researcher.MultipleObjectsReturned}

\begin{fulllineitems}
\phantomsection\label{api:data.models.Researcher.MultipleObjectsReturned}\pysigline{\strong{exception }\code{Researcher.}\bfcode{MultipleObjectsReturned}}{}
Bases: \code{django.core.exceptions.MultipleObjectsReturned}

\end{fulllineitems}

\index{environment\_set (data.models.Researcher attribute)}

\begin{fulllineitems}
\phantomsection\label{api:data.models.Researcher.environment_set}\pysigline{\code{Researcher.}\bfcode{environment\_set}}{}
\end{fulllineitems}

\index{experiment\_set (data.models.Researcher attribute)}

\begin{fulllineitems}
\phantomsection\label{api:data.models.Researcher.experiment_set}\pysigline{\code{Researcher.}\bfcode{experiment\_set}}{}
\end{fulllineitems}

\index{implantation\_set (data.models.Researcher attribute)}

\begin{fulllineitems}
\phantomsection\label{api:data.models.Researcher.implantation_set}\pysigline{\code{Researcher.}\bfcode{implantation\_set}}{}
\end{fulllineitems}

\index{transplantation\_set (data.models.Researcher attribute)}

\begin{fulllineitems}
\phantomsection\label{api:data.models.Researcher.transplantation_set}\pysigline{\code{Researcher.}\bfcode{transplantation\_set}}{}
\end{fulllineitems}

\index{treatment\_set (data.models.Researcher attribute)}

\begin{fulllineitems}
\phantomsection\label{api:data.models.Researcher.treatment_set}\pysigline{\code{Researcher.}\bfcode{treatment\_set}}{}
\end{fulllineitems}


\end{fulllineitems}

\index{Study (class in data.models)}

\begin{fulllineitems}
\phantomsection\label{api:data.models.Study}\pysiglinewithargsret{\strong{class }\code{data.models.}\bfcode{Study}}{\emph{*args}, \emph{**kwargs}}{}
Bases: \code{django.db.models.base.Model}

Study(id, description, start\_date, stop\_date, notes)
\index{Study.DoesNotExist}

\begin{fulllineitems}
\phantomsection\label{api:data.models.Study.DoesNotExist}\pysigline{\strong{exception }\bfcode{DoesNotExist}}{}
Bases: \code{django.core.exceptions.ObjectDoesNotExist}

\end{fulllineitems}

\index{Study.MultipleObjectsReturned}

\begin{fulllineitems}
\phantomsection\label{api:data.models.Study.MultipleObjectsReturned}\pysigline{\strong{exception }\code{Study.}\bfcode{MultipleObjectsReturned}}{}
Bases: \code{django.core.exceptions.MultipleObjectsReturned}

\end{fulllineitems}

\index{experiment\_set (data.models.Study attribute)}

\begin{fulllineitems}
\phantomsection\label{api:data.models.Study.experiment_set}\pysigline{\code{Study.}\bfcode{experiment\_set}}{}
\end{fulllineitems}

\index{get\_absolute\_url() (data.models.Study method)}

\begin{fulllineitems}
\phantomsection\label{api:data.models.Study.get_absolute_url}\pysiglinewithargsret{\code{Study.}\bfcode{get\_absolute\_url}}{\emph{*moreargs}, \emph{**morekwargs}}{}
\end{fulllineitems}

\index{strain (data.models.Study attribute)}

\begin{fulllineitems}
\phantomsection\label{api:data.models.Study.strain}\pysigline{\code{Study.}\bfcode{strain}}{}
\end{fulllineitems}

\index{treatment\_set (data.models.Study attribute)}

\begin{fulllineitems}
\phantomsection\label{api:data.models.Study.treatment_set}\pysigline{\code{Study.}\bfcode{treatment\_set}}{}
\end{fulllineitems}


\end{fulllineitems}

\index{Transplantation (class in data.models)}

\begin{fulllineitems}
\phantomsection\label{api:data.models.Transplantation}\pysiglinewithargsret{\strong{class }\code{data.models.}\bfcode{Transplantation}}{\emph{*args}, \emph{**kwargs}}{}
Bases: \code{django.db.models.base.Model}

Transplantation(id, tissue, transplant\_date, notes)
\index{Transplantation.DoesNotExist}

\begin{fulllineitems}
\phantomsection\label{api:data.models.Transplantation.DoesNotExist}\pysigline{\strong{exception }\bfcode{DoesNotExist}}{}
Bases: \code{django.core.exceptions.ObjectDoesNotExist}

\end{fulllineitems}

\index{Transplantation.MultipleObjectsReturned}

\begin{fulllineitems}
\phantomsection\label{api:data.models.Transplantation.MultipleObjectsReturned}\pysigline{\strong{exception }\code{Transplantation.}\bfcode{MultipleObjectsReturned}}{}
Bases: \code{django.core.exceptions.MultipleObjectsReturned}

\end{fulllineitems}

\index{donor (data.models.Transplantation attribute)}

\begin{fulllineitems}
\phantomsection\label{api:data.models.Transplantation.donor}\pysigline{\code{Transplantation.}\bfcode{donor}}{}
\end{fulllineitems}

\index{get\_next\_by\_transplant\_date() (data.models.Transplantation method)}

\begin{fulllineitems}
\phantomsection\label{api:data.models.Transplantation.get_next_by_transplant_date}\pysiglinewithargsret{\code{Transplantation.}\bfcode{get\_next\_by\_transplant\_date}}{\emph{*moreargs}, \emph{**morekwargs}}{}
\end{fulllineitems}

\index{get\_previous\_by\_transplant\_date() (data.models.Transplantation method)}

\begin{fulllineitems}
\phantomsection\label{api:data.models.Transplantation.get_previous_by_transplant_date}\pysiglinewithargsret{\code{Transplantation.}\bfcode{get\_previous\_by\_transplant\_date}}{\emph{*moreargs}, \emph{**morekwargs}}{}
\end{fulllineitems}

\index{surgeon (data.models.Transplantation attribute)}

\begin{fulllineitems}
\phantomsection\label{api:data.models.Transplantation.surgeon}\pysigline{\code{Transplantation.}\bfcode{surgeon}}{}
\end{fulllineitems}

\index{treatment\_set (data.models.Transplantation attribute)}

\begin{fulllineitems}
\phantomsection\label{api:data.models.Transplantation.treatment_set}\pysigline{\code{Transplantation.}\bfcode{treatment\_set}}{}
\end{fulllineitems}


\end{fulllineitems}

\index{Treatment (class in data.models)}

\begin{fulllineitems}
\phantomsection\label{api:data.models.Treatment}\pysiglinewithargsret{\strong{class }\code{data.models.}\bfcode{Treatment}}{\emph{*args}, \emph{**kwargs}}{}
Bases: \code{django.db.models.base.Model}

Treatment(id, treatment, study\_id, diet\_id, environment\_id, transplantation\_id, notes)
\index{Treatment.DoesNotExist}

\begin{fulllineitems}
\phantomsection\label{api:data.models.Treatment.DoesNotExist}\pysigline{\strong{exception }\bfcode{DoesNotExist}}{}
Bases: \code{django.core.exceptions.ObjectDoesNotExist}

\end{fulllineitems}

\index{Treatment.MultipleObjectsReturned}

\begin{fulllineitems}
\phantomsection\label{api:data.models.Treatment.MultipleObjectsReturned}\pysigline{\strong{exception }\code{Treatment.}\bfcode{MultipleObjectsReturned}}{}
Bases: \code{django.core.exceptions.MultipleObjectsReturned}

\end{fulllineitems}

\index{animals (data.models.Treatment attribute)}

\begin{fulllineitems}
\phantomsection\label{api:data.models.Treatment.animals}\pysigline{\code{Treatment.}\bfcode{animals}}{}
\end{fulllineitems}

\index{diet (data.models.Treatment attribute)}

\begin{fulllineitems}
\phantomsection\label{api:data.models.Treatment.diet}\pysigline{\code{Treatment.}\bfcode{diet}}{}
\end{fulllineitems}

\index{environment (data.models.Treatment attribute)}

\begin{fulllineitems}
\phantomsection\label{api:data.models.Treatment.environment}\pysigline{\code{Treatment.}\bfcode{environment}}{}
\end{fulllineitems}

\index{get\_absolute\_url() (data.models.Treatment method)}

\begin{fulllineitems}
\phantomsection\label{api:data.models.Treatment.get_absolute_url}\pysiglinewithargsret{\code{Treatment.}\bfcode{get\_absolute\_url}}{\emph{*moreargs}, \emph{**morekwargs}}{}
\end{fulllineitems}

\index{implantation (data.models.Treatment attribute)}

\begin{fulllineitems}
\phantomsection\label{api:data.models.Treatment.implantation}\pysigline{\code{Treatment.}\bfcode{implantation}}{}
\end{fulllineitems}

\index{pharmaceutical (data.models.Treatment attribute)}

\begin{fulllineitems}
\phantomsection\label{api:data.models.Treatment.pharmaceutical}\pysigline{\code{Treatment.}\bfcode{pharmaceutical}}{}
\end{fulllineitems}

\index{researchers (data.models.Treatment attribute)}

\begin{fulllineitems}
\phantomsection\label{api:data.models.Treatment.researchers}\pysigline{\code{Treatment.}\bfcode{researchers}}{}
\end{fulllineitems}

\index{study (data.models.Treatment attribute)}

\begin{fulllineitems}
\phantomsection\label{api:data.models.Treatment.study}\pysigline{\code{Treatment.}\bfcode{study}}{}
\end{fulllineitems}

\index{transplantation (data.models.Treatment attribute)}

\begin{fulllineitems}
\phantomsection\label{api:data.models.Treatment.transplantation}\pysigline{\code{Treatment.}\bfcode{transplantation}}{}
\end{fulllineitems}


\end{fulllineitems}

\index{Vendor (class in data.models)}

\begin{fulllineitems}
\phantomsection\label{api:data.models.Vendor}\pysiglinewithargsret{\strong{class }\code{data.models.}\bfcode{Vendor}}{\emph{*args}, \emph{**kwargs}}{}
Bases: \code{django.db.models.base.Model}

Vendor(id, vendor, website, email, ordering, notes)
\index{Vendor.DoesNotExist}

\begin{fulllineitems}
\phantomsection\label{api:data.models.Vendor.DoesNotExist}\pysigline{\strong{exception }\bfcode{DoesNotExist}}{}
Bases: \code{django.core.exceptions.ObjectDoesNotExist}

\end{fulllineitems}

\index{Vendor.MultipleObjectsReturned}

\begin{fulllineitems}
\phantomsection\label{api:data.models.Vendor.MultipleObjectsReturned}\pysigline{\strong{exception }\code{Vendor.}\bfcode{MultipleObjectsReturned}}{}
Bases: \code{django.core.exceptions.MultipleObjectsReturned}

\end{fulllineitems}

\index{diet\_set (data.models.Vendor attribute)}

\begin{fulllineitems}
\phantomsection\label{api:data.models.Vendor.diet_set}\pysigline{\code{Vendor.}\bfcode{diet\_set}}{}
\end{fulllineitems}

\index{get\_ordering\_display() (data.models.Vendor method)}

\begin{fulllineitems}
\phantomsection\label{api:data.models.Vendor.get_ordering_display}\pysiglinewithargsret{\code{Vendor.}\bfcode{get\_ordering\_display}}{\emph{*moreargs}, \emph{**morekwargs}}{}
\end{fulllineitems}

\index{implantation\_set (data.models.Vendor attribute)}

\begin{fulllineitems}
\phantomsection\label{api:data.models.Vendor.implantation_set}\pysigline{\code{Vendor.}\bfcode{implantation\_set}}{}
\end{fulllineitems}

\index{pharmaceutical\_set (data.models.Vendor attribute)}

\begin{fulllineitems}
\phantomsection\label{api:data.models.Vendor.pharmaceutical_set}\pysigline{\code{Vendor.}\bfcode{pharmaceutical\_set}}{}
\end{fulllineitems}


\end{fulllineitems}



\subsection{Forms}
\label{api:forms}\label{api:module-data.forms}\index{data.forms (module)}\index{ExperimentForm (class in data.forms)}

\begin{fulllineitems}
\phantomsection\label{api:data.forms.ExperimentForm}\pysiglinewithargsret{\strong{class }\code{data.forms.}\bfcode{ExperimentForm}}{\emph{data=None}, \emph{files=None}, \emph{auto\_id='id\_\%s'}, \emph{prefix=None}, \emph{initial=None}, \emph{error\_class=\textless{}class `django.forms.util.ErrorList'\textgreater{}}, \emph{label\_suffix=':'}, \emph{empty\_permitted=False}, \emph{instance=None}}{}
Bases: \code{django.forms.models.ModelForm}

This is the configuration for the experiment form.

This form is used to set up and modify an experiment.  It uses a datepicker widget for the date, and autocomplete forms for the animals.
\index{ExperimentForm.Media (class in data.forms)}

\begin{fulllineitems}
\phantomsection\label{api:data.forms.ExperimentForm.Media}\pysigline{\strong{class }\bfcode{Media}}{}
\end{fulllineitems}

\index{ExperimentForm.Meta (class in data.forms)}

\begin{fulllineitems}
\phantomsection\label{api:data.forms.ExperimentForm.Meta}\pysigline{\strong{class }\code{ExperimentForm.}\bfcode{Meta}}{}~\index{model (data.forms.ExperimentForm.Meta attribute)}

\begin{fulllineitems}
\phantomsection\label{api:data.forms.ExperimentForm.Meta.model}\pysigline{\bfcode{model}}{}
alias of \code{Experiment}

\end{fulllineitems}


\end{fulllineitems}

\index{media (data.forms.ExperimentForm attribute)}

\begin{fulllineitems}
\phantomsection\label{api:data.forms.ExperimentForm.media}\pysigline{\code{ExperimentForm.}\bfcode{media}}{}
\end{fulllineitems}


\end{fulllineitems}

\index{MeasurementForm (class in data.forms)}

\begin{fulllineitems}
\phantomsection\label{api:data.forms.MeasurementForm}\pysiglinewithargsret{\strong{class }\code{data.forms.}\bfcode{MeasurementForm}}{\emph{data=None}, \emph{files=None}, \emph{auto\_id='id\_\%s'}, \emph{prefix=None}, \emph{initial=None}, \emph{error\_class=\textless{}class `django.forms.util.ErrorList'\textgreater{}}, \emph{label\_suffix=':'}, \emph{empty\_permitted=False}, \emph{instance=None}}{}
Bases: \code{django.forms.models.ModelForm}

Form definition for adding and editing measurements.

This form excludes experiment, which must be passed as a filtering parameter from the view.  
This form is used for formsets to add or modify measurements from within an experiment.
\index{MeasurementForm.Meta (class in data.forms)}

\begin{fulllineitems}
\phantomsection\label{api:data.forms.MeasurementForm.Meta}\pysigline{\strong{class }\bfcode{Meta}}{}~\index{model (data.forms.MeasurementForm.Meta attribute)}

\begin{fulllineitems}
\phantomsection\label{api:data.forms.MeasurementForm.Meta.model}\pysigline{\bfcode{model}}{}
alias of \code{Measurement}

\end{fulllineitems}


\end{fulllineitems}

\index{media (data.forms.MeasurementForm attribute)}

\begin{fulllineitems}
\phantomsection\label{api:data.forms.MeasurementForm.media}\pysigline{\code{MeasurementForm.}\bfcode{media}}{}
\end{fulllineitems}


\end{fulllineitems}

\index{StudyExperimentForm (class in data.forms)}

\begin{fulllineitems}
\phantomsection\label{api:data.forms.StudyExperimentForm}\pysiglinewithargsret{\strong{class }\code{data.forms.}\bfcode{StudyExperimentForm}}{\emph{data=None}, \emph{files=None}, \emph{auto\_id='id\_\%s'}, \emph{prefix=None}, \emph{initial=None}, \emph{error\_class=\textless{}class `django.forms.util.ErrorList'\textgreater{}}, \emph{label\_suffix=':'}, \emph{empty\_permitted=False}, \emph{instance=None}}{}
Bases: \code{django.forms.models.ModelForm}

This is the configuration for a study form (From an experiment).

This form provides an autocomplete field for the animals, and hides the study field which will be automatically set upon save.
\index{StudyExperimentForm.Media (class in data.forms)}

\begin{fulllineitems}
\phantomsection\label{api:data.forms.StudyExperimentForm.Media}\pysigline{\strong{class }\bfcode{Media}}{}
\end{fulllineitems}

\index{StudyExperimentForm.Meta (class in data.forms)}

\begin{fulllineitems}
\phantomsection\label{api:data.forms.StudyExperimentForm.Meta}\pysigline{\strong{class }\code{StudyExperimentForm.}\bfcode{Meta}}{}~\index{model (data.forms.StudyExperimentForm.Meta attribute)}

\begin{fulllineitems}
\phantomsection\label{api:data.forms.StudyExperimentForm.Meta.model}\pysigline{\bfcode{model}}{}
alias of \code{Experiment}

\end{fulllineitems}


\end{fulllineitems}

\index{media (data.forms.StudyExperimentForm attribute)}

\begin{fulllineitems}
\phantomsection\label{api:data.forms.StudyExperimentForm.media}\pysigline{\code{StudyExperimentForm.}\bfcode{media}}{}
\end{fulllineitems}


\end{fulllineitems}

\index{StudyForm (class in data.forms)}

\begin{fulllineitems}
\phantomsection\label{api:data.forms.StudyForm}\pysiglinewithargsret{\strong{class }\code{data.forms.}\bfcode{StudyForm}}{\emph{data=None}, \emph{files=None}, \emph{auto\_id='id\_\%s'}, \emph{prefix=None}, \emph{initial=None}, \emph{error\_class=\textless{}class `django.forms.util.ErrorList'\textgreater{}}, \emph{label\_suffix=':'}, \emph{empty\_permitted=False}, \emph{instance=None}}{}
Bases: \code{django.forms.models.ModelForm}

This is the configuration for the study form.

This form is used to create and modify studies.  It uses an autocomplete widget for the animals.
\index{StudyForm.Media (class in data.forms)}

\begin{fulllineitems}
\phantomsection\label{api:data.forms.StudyForm.Media}\pysigline{\strong{class }\bfcode{Media}}{}
\end{fulllineitems}

\index{StudyForm.Meta (class in data.forms)}

\begin{fulllineitems}
\phantomsection\label{api:data.forms.StudyForm.Meta}\pysigline{\strong{class }\code{StudyForm.}\bfcode{Meta}}{}~\index{model (data.forms.StudyForm.Meta attribute)}

\begin{fulllineitems}
\phantomsection\label{api:data.forms.StudyForm.Meta.model}\pysigline{\bfcode{model}}{}
alias of \code{Study}

\end{fulllineitems}


\end{fulllineitems}

\index{media (data.forms.StudyForm attribute)}

\begin{fulllineitems}
\phantomsection\label{api:data.forms.StudyForm.media}\pysigline{\code{StudyForm.}\bfcode{media}}{}
\end{fulllineitems}


\end{fulllineitems}

\index{TreatmentForm (class in data.forms)}

\begin{fulllineitems}
\phantomsection\label{api:data.forms.TreatmentForm}\pysiglinewithargsret{\strong{class }\code{data.forms.}\bfcode{TreatmentForm}}{\emph{data=None}, \emph{files=None}, \emph{auto\_id='id\_\%s'}, \emph{prefix=None}, \emph{initial=None}, \emph{error\_class=\textless{}class `django.forms.util.ErrorList'\textgreater{}}, \emph{label\_suffix=':'}, \emph{empty\_permitted=False}, \emph{instance=None}}{}
Bases: \code{django.forms.models.ModelForm}

Form class for study treatment groups.

In the case of studies, animals are defined in the treatment group rather than in the study group.  A treatment consists of a study, a set of animals and the conditions which define that treatment.  This includes related fields for environment, diet, implants and transplants.
\index{TreatmentForm.Media (class in data.forms)}

\begin{fulllineitems}
\phantomsection\label{api:data.forms.TreatmentForm.Media}\pysigline{\strong{class }\bfcode{Media}}{}
\end{fulllineitems}

\index{TreatmentForm.Meta (class in data.forms)}

\begin{fulllineitems}
\phantomsection\label{api:data.forms.TreatmentForm.Meta}\pysigline{\strong{class }\code{TreatmentForm.}\bfcode{Meta}}{}~\index{model (data.forms.TreatmentForm.Meta attribute)}

\begin{fulllineitems}
\phantomsection\label{api:data.forms.TreatmentForm.Meta.model}\pysigline{\bfcode{model}}{}
alias of \code{Treatment}

\end{fulllineitems}


\end{fulllineitems}

\index{media (data.forms.TreatmentForm attribute)}

\begin{fulllineitems}
\phantomsection\label{api:data.forms.TreatmentForm.media}\pysigline{\code{TreatmentForm.}\bfcode{media}}{}
\end{fulllineitems}


\end{fulllineitems}



\subsection{Views and URLs}
\label{api:views-and-urls}\label{api:module-data.views}\index{data.views (module)}\index{add\_measurement() (in module data.views)}

\begin{fulllineitems}
\phantomsection\label{api:data.views.add_measurement}\pysiglinewithargsret{\code{data.views.}\bfcode{add\_measurement}}{\emph{request}, \emph{*args}, \emph{**kwargs}}{}
\end{fulllineitems}

\index{experiment\_detail() (in module data.views)}

\begin{fulllineitems}
\phantomsection\label{api:data.views.experiment_detail}\pysiglinewithargsret{\code{data.views.}\bfcode{experiment\_detail}}{\emph{request}, \emph{*args}, \emph{**kwargs}}{}
\end{fulllineitems}

\index{experiment\_detail\_all() (in module data.views)}

\begin{fulllineitems}
\phantomsection\label{api:data.views.experiment_detail_all}\pysiglinewithargsret{\code{data.views.}\bfcode{experiment\_detail\_all}}{\emph{request}, \emph{*args}, \emph{**kwargs}}{}
\end{fulllineitems}

\index{experiment\_details\_csv() (in module data.views)}

\begin{fulllineitems}
\phantomsection\label{api:data.views.experiment_details_csv}\pysiglinewithargsret{\code{data.views.}\bfcode{experiment\_details\_csv}}{\emph{request}, \emph{*args}, \emph{**kwargs}}{}
This view generates a csv output file of an experiment.

The view writes to a csv table the animal, genotype, age (in days), assay and values.

\end{fulllineitems}

\index{experiment\_list() (in module data.views)}

\begin{fulllineitems}
\phantomsection\label{api:data.views.experiment_list}\pysiglinewithargsret{\code{data.views.}\bfcode{experiment\_list}}{\emph{request}, \emph{*args}, \emph{**kwargs}}{}
\end{fulllineitems}

\index{study\_experiment() (in module data.views)}

\begin{fulllineitems}
\phantomsection\label{api:data.views.study_experiment}\pysiglinewithargsret{\code{data.views.}\bfcode{study\_experiment}}{\emph{request}, \emph{*args}, \emph{**kwargs}}{}
\end{fulllineitems}

\phantomsection\label{api:module-data.urls}\index{data.urls (module)}

\subsection{Administrative Site Configuration}
\label{api:administrative-site-configuration}\label{api:module-data.admin}\index{data.admin (module)}

\subsection{Test Files}
\label{api:test-files}\label{api:module-data.tests}\index{data.tests (module)}
This file contains tests for the data application.

These tests will verify generation and function of a all objects in the data application.
\index{ExperimentModelTests (class in data.tests)}

\begin{fulllineitems}
\phantomsection\label{api:data.tests.ExperimentModelTests}\pysiglinewithargsret{\strong{class }\code{data.tests.}\bfcode{ExperimentModelTests}}{\emph{methodName='runTest'}}{}
Bases: \code{django.test.testcases.TestCase}

Test the Experiment model contained in the `data' application.
\index{setUp() (data.tests.ExperimentModelTests method)}

\begin{fulllineitems}
\phantomsection\label{api:data.tests.ExperimentModelTests.setUp}\pysiglinewithargsret{\bfcode{setUp}}{}{}
Instantiate the test client.

\end{fulllineitems}

\index{tearDown() (data.tests.ExperimentModelTests method)}

\begin{fulllineitems}
\phantomsection\label{api:data.tests.ExperimentModelTests.tearDown}\pysiglinewithargsret{\bfcode{tearDown}}{}{}
Depopulate created model instances from test database.

\end{fulllineitems}

\index{test\_create\_experiment\_minimal() (data.tests.ExperimentModelTests method)}

\begin{fulllineitems}
\phantomsection\label{api:data.tests.ExperimentModelTests.test_create_experiment_minimal}\pysiglinewithargsret{\bfcode{test\_create\_experiment\_minimal}}{}{}
This is a test for creating a new experiment object, with only the minimum fields being entered

\end{fulllineitems}

\index{test\_experiment\_get\_absolute\_url() (data.tests.ExperimentModelTests method)}

\begin{fulllineitems}
\phantomsection\label{api:data.tests.ExperimentModelTests.test_experiment_get_absolute_url}\pysiglinewithargsret{\bfcode{test\_experiment\_get\_absolute\_url}}{}{}
This test verifies the absolute url of an experiment object.

\end{fulllineitems}

\index{test\_experiment\_unicode() (data.tests.ExperimentModelTests method)}

\begin{fulllineitems}
\phantomsection\label{api:data.tests.ExperimentModelTests.test_experiment_unicode}\pysiglinewithargsret{\bfcode{test\_experiment\_unicode}}{}{}
This test verifies that the unicode representation of an experiment is date-feeding state

\end{fulllineitems}


\end{fulllineitems}



\section{Animals Package}
\label{api:animals-package}\label{api:module-animal}\index{animal (module)}
The animal app contains and controls the display of data about animals.

Animals are tracked as individual entities, and given associations to breeding cages to follow ancestry, and strains.


\subsection{Animal}
\label{api:animal}
Most parameters about an animal are set within the animal object.  Here is where the animals strain, breeding, parentage and many other parameters are included.  Animals have foreignkey relationships with both Strain and Breeding, so an animal may only belong to one of each of those.  As an example, a mouse cannot come from more than one Breeding set, and cannot belong to more than one strain.


\subsubsection{Backcrosses and Generations}
\label{api:backcrosses-and-generations}
For this software, optional tracking of backcrosses and generations is available and is stored as an attribute of an animal.  When an inbred cross is made against a pure background, the backcross increases by 1.  When a heterozygote cross is made, the generation increases by one.  As an example, for every time a mouse in a C57/BL6 background is crossed against a wildtype C57/B6 mouse, the backcross (but not the generation) increases by one.  For every time a mutant strain is crosses against itself (either vs a heterozygote or homozygote of that strain), the generation will increase by one.  Backcrosses should typically be performed against a separate colony of purebred mouse, rather than against wild-type alleles of the mutant strain.


\subsection{Breeding Cages}
\label{api:breeding-cages}
A breeding cage is defined as a set of one or more male and one or more female mice.  Because of this, it is not always clear who the precise parentage of an animal is.  If the parentage is known, then the Mother and Father fields can be set for a particular animal.


\subsection{Strains}
\label{api:strains}
A strain is a set of mice with a similar genetics.  Importantly strains are separated from Backgrounds.  For example, one might have mice with the genotype ob/ob but these mice may be in either a C57-Black6 or a mixed background.  This difference is set at the individual animal level.  
The result of this is that a query for a particular strain may then need to be filtered to a specific background.


\subsection{Models}
\label{api:id1}\phantomsection\label{api:module-animal.models}\index{animal.models (module)}
This module describes the Strain, Animal, Breeding and Cage data models.

This module stores all data regarding a particular laboratory animal.  Information about experimental data and timed matings are stored in the data and timed\_matings packages.  This module describes the database structure for each data model.
\index{Strain (class in animal.models)}

\begin{fulllineitems}
\phantomsection\label{api:animal.models.Strain}\pysiglinewithargsret{\strong{class }\code{animal.models.}\bfcode{Strain}}{\emph{*args}, \emph{**kwargs}}{}
Bases: \code{django.db.models.base.Model}

A data model describing a mouse strain.

This is separate from the background of a mouse.  For example a ob/ob mouse on a mixed or a black-6 background still have the same strain.  The background is defined in the animal and breeding cages.  Strain and Strain\_slug are required.

\end{fulllineitems}

\index{Animal (class in animal.models)}

\begin{fulllineitems}
\phantomsection\label{api:animal.models.Animal}\pysiglinewithargsret{\strong{class }\code{animal.models.}\bfcode{Animal}}{\emph{*args}, \emph{**kwargs}}{}
Bases: \code{django.db.models.base.Model}

A data model describing an animal.

This data model describes a wide variety of parameters of an experimental animal.  This model is linked to the Strain and Cage models via 1:1 relationships.  If the parentage of a mouse is known, this can be identified (the breeding set may not be clear on this matter). Mice are automatically marked as not alive when a Death date is provided and the object is saved.  Strain, Background and Genotype are required field.  By default, querysets are ordered first by strain then by MouseID.

\end{fulllineitems}

\index{Breeding (class in animal.models)}

\begin{fulllineitems}
\phantomsection\label{api:animal.models.Breeding}\pysiglinewithargsret{\strong{class }\code{animal.models.}\bfcode{Breeding}}{\emph{*args}, \emph{**kwargs}}{}
Bases: \code{django.db.models.base.Model}

This data model stores information about a particular breeding set

A breeding set may contain one ore more males and females and must be defined via the progeny strain.  For example, in the case of generating a new strain, the strain indicates the new strain not the parental strains.  A breeding cage is defined as one male with one or more females.  If the breeding set is part of a timed mating experiment, then Timed\_Mating must be selected.  Breeding cages are automatically inactivated upon saving when a End date is provided.  The only required field is Strain.  By default, querysets are ordered by Strain, then Start.

\end{fulllineitems}



\subsection{Forms}
\label{api:id2}\phantomsection\label{api:module-animal.forms}\index{animal.forms (module)}
Forms for use in manipulating objects in the animal app.
\index{AnimalForm (class in animal.forms)}

\begin{fulllineitems}
\phantomsection\label{api:animal.forms.AnimalForm}\pysiglinewithargsret{\strong{class }\code{animal.forms.}\bfcode{AnimalForm}}{\emph{data=None}, \emph{files=None}, \emph{auto\_id='id\_\%s'}, \emph{prefix=None}, \emph{initial=None}, \emph{error\_class=\textless{}class `django.forms.util.ErrorList'\textgreater{}}, \emph{label\_suffix=':'}, \emph{empty\_permitted=False}, \emph{instance=None}}{}
Bases: \code{django.forms.models.ModelForm}

This modelform provides fields for modifying animal data.

This form also automatically loads javascript and css for the datepicker jquery-ui widget.  It also includes auto
\index{AnimalForm.Media (class in animal.forms)}

\begin{fulllineitems}
\phantomsection\label{api:animal.forms.AnimalForm.Media}\pysigline{\strong{class }\bfcode{Media}}{}
\end{fulllineitems}

\index{AnimalForm.Meta (class in animal.forms)}

\begin{fulllineitems}
\phantomsection\label{api:animal.forms.AnimalForm.Meta}\pysigline{\strong{class }\code{AnimalForm.}\bfcode{Meta}}{}~\index{model (animal.forms.AnimalForm.Meta attribute)}

\begin{fulllineitems}
\phantomsection\label{api:animal.forms.AnimalForm.Meta.model}\pysigline{\bfcode{model}}{}
alias of \code{Animal}

\end{fulllineitems}


\end{fulllineitems}

\index{media (animal.forms.AnimalForm attribute)}

\begin{fulllineitems}
\phantomsection\label{api:animal.forms.AnimalForm.media}\pysigline{\code{AnimalForm.}\bfcode{media}}{}
\end{fulllineitems}


\end{fulllineitems}

\index{BreedingForm (class in animal.forms)}

\begin{fulllineitems}
\phantomsection\label{api:animal.forms.BreedingForm}\pysiglinewithargsret{\strong{class }\code{animal.forms.}\bfcode{BreedingForm}}{\emph{data=None}, \emph{files=None}, \emph{auto\_id='id\_\%s'}, \emph{prefix=None}, \emph{initial=None}, \emph{error\_class=\textless{}class `django.forms.util.ErrorList'\textgreater{}}, \emph{label\_suffix=':'}, \emph{empty\_permitted=False}, \emph{instance=None}}{}
Bases: \code{django.forms.models.ModelForm}

This form provides most fields for creating and entring breeding cage data.

This form is used from the url /mousedb/breeding/new and is a generic create view.  This view includes a datepicker widget for Stat and End dates and autocomplete fields for the Females and Male fields
\index{BreedingForm.Media (class in animal.forms)}

\begin{fulllineitems}
\phantomsection\label{api:animal.forms.BreedingForm.Media}\pysigline{\strong{class }\bfcode{Media}}{}
\end{fulllineitems}

\index{BreedingForm.Meta (class in animal.forms)}

\begin{fulllineitems}
\phantomsection\label{api:animal.forms.BreedingForm.Meta}\pysigline{\strong{class }\code{BreedingForm.}\bfcode{Meta}}{}~\index{model (animal.forms.BreedingForm.Meta attribute)}

\begin{fulllineitems}
\phantomsection\label{api:animal.forms.BreedingForm.Meta.model}\pysigline{\bfcode{model}}{}
alias of \code{Breeding}

\end{fulllineitems}


\end{fulllineitems}

\index{media (animal.forms.BreedingForm attribute)}

\begin{fulllineitems}
\phantomsection\label{api:animal.forms.BreedingForm.media}\pysigline{\code{BreedingForm.}\bfcode{media}}{}
\end{fulllineitems}


\end{fulllineitems}

\index{MultipleAnimalForm (class in animal.forms)}

\begin{fulllineitems}
\phantomsection\label{api:animal.forms.MultipleAnimalForm}\pysiglinewithargsret{\strong{class }\code{animal.forms.}\bfcode{MultipleAnimalForm}}{\emph{data=None}, \emph{files=None}, \emph{auto\_id='id\_\%s'}, \emph{prefix=None}, \emph{initial=None}, \emph{error\_class=\textless{}class `django.forms.util.ErrorList'\textgreater{}}, \emph{label\_suffix=':'}, \emph{empty\_permitted=False}, \emph{instance=None}}{}
Bases: \code{django.forms.models.ModelForm}

This modelform provides fields for entering multiple identical copies of a set of mice.

This form only includes the required fields Background and Strain.
\index{MultipleAnimalForm.Meta (class in animal.forms)}

\begin{fulllineitems}
\phantomsection\label{api:animal.forms.MultipleAnimalForm.Meta}\pysigline{\strong{class }\bfcode{Meta}}{}~\index{model (animal.forms.MultipleAnimalForm.Meta attribute)}

\begin{fulllineitems}
\phantomsection\label{api:animal.forms.MultipleAnimalForm.Meta.model}\pysigline{\bfcode{model}}{}
alias of \code{Animal}

\end{fulllineitems}


\end{fulllineitems}

\index{media (animal.forms.MultipleAnimalForm attribute)}

\begin{fulllineitems}
\phantomsection\label{api:animal.forms.MultipleAnimalForm.media}\pysigline{\code{MultipleAnimalForm.}\bfcode{media}}{}
\end{fulllineitems}


\end{fulllineitems}

\index{MultipleBreedingAnimalForm (class in animal.forms)}

\begin{fulllineitems}
\phantomsection\label{api:animal.forms.MultipleBreedingAnimalForm}\pysiglinewithargsret{\strong{class }\code{animal.forms.}\bfcode{MultipleBreedingAnimalForm}}{\emph{data=None}, \emph{files=None}, \emph{auto\_id='id\_\%s'}, \emph{prefix=None}, \emph{initial=None}, \emph{error\_class=\textless{}class `django.forms.util.ErrorList'\textgreater{}}, \emph{label\_suffix=':'}, \emph{empty\_permitted=False}, \emph{instance=None}}{}
Bases: \code{django.forms.models.ModelForm}

This modelform provides fields for entering multiple pups within a breeding set.

The only fields presented are Born, Weaned, Gender and Count.  Several other fields will be automatically entered based on the Breeding Set entries.
\index{MultipleBreedingAnimalForm.Meta (class in animal.forms)}

\begin{fulllineitems}
\phantomsection\label{api:animal.forms.MultipleBreedingAnimalForm.Meta}\pysigline{\strong{class }\bfcode{Meta}}{}~\index{model (animal.forms.MultipleBreedingAnimalForm.Meta attribute)}

\begin{fulllineitems}
\phantomsection\label{api:animal.forms.MultipleBreedingAnimalForm.Meta.model}\pysigline{\bfcode{model}}{}
alias of \code{Animal}

\end{fulllineitems}


\end{fulllineitems}

\index{media (animal.forms.MultipleBreedingAnimalForm attribute)}

\begin{fulllineitems}
\phantomsection\label{api:animal.forms.MultipleBreedingAnimalForm.media}\pysigline{\code{MultipleBreedingAnimalForm.}\bfcode{media}}{}
\end{fulllineitems}


\end{fulllineitems}



\subsection{Views and URLs}
\label{api:id3}\phantomsection\label{api:module-animal.views}\index{animal.views (module)}
These views define template redirects for the animal app.

This module contains only non-generic views.  Several generic views are also used and are defined in animal/urls/.
\index{animal\_detail() (in module animal.views)}

\begin{fulllineitems}
\phantomsection\label{api:animal.views.animal_detail}\pysiglinewithargsret{\code{animal.views.}\bfcode{animal\_detail}}{\emph{request}, \emph{*args}, \emph{**kwargs}}{}
This view displays specific details about an animal.

It takes a request in the form animal/(id)/, mice/(id) or mouse/(id)/ and renders the detail page for that mouse.  The request is defined for id not MouseID (or barcode) because this allows for details to be displayed for mice without barcode identification.
Therefore care must be taken that animal/4932 is id=4932 and not barcode=4932.  The animal name is defined at the top of the page.
This page is restricted to logged-in users.

\end{fulllineitems}

\index{breeding\_change() (in module animal.views)}

\begin{fulllineitems}
\phantomsection\label{api:animal.views.breeding_change}\pysiglinewithargsret{\code{animal.views.}\bfcode{breeding\_change}}{\emph{request}, \emph{*args}, \emph{**kwargs}}{}
This view is used to generate a form by which to change pups which belong to a particular breeding set.

This view typically is used to modify existing pups.  This might include marking animals as sacrificed, entering genotype or marking information or entering movement of mice to another cage.  It is used to show and modify several animals at once.
It takes a request in the form /breeding/(breeding\_id)/change/ and returns a form specific to the breeding set defined in breeding\_id.  breeding\_id is the background identification number of the breeding set and does not refer to the barcode of any breeding cage.
This view returns a formset in which one row represents one animal.  To add extra animals to a breeding set use /breeding/(breeding\_id)/pups/.
This view is restricted to those with the permission animal.change\_animal.

\end{fulllineitems}

\index{breeding\_detail() (in module animal.views)}

\begin{fulllineitems}
\phantomsection\label{api:animal.views.breeding_detail}\pysiglinewithargsret{\code{animal.views.}\bfcode{breeding\_detail}}{\emph{request}, \emph{*args}, \emph{**kwargs}}{}
This view displays specific details about a breeding set.

It takes a request in the form /breeding/(breeding\_id)/all and renders the detail page for that breeding set.
The breeding\_id refers to the background id of the breeding set, and not the breeding cage barcode.
This view also passes along a dictionary of all pups belonging to that breeding set.
This page is restricted to logged-in users.

\end{fulllineitems}

\index{breeding\_pups() (in module animal.views)}

\begin{fulllineitems}
\phantomsection\label{api:animal.views.breeding_pups}\pysiglinewithargsret{\code{animal.views.}\bfcode{breeding\_pups}}{\emph{request}, \emph{*args}, \emph{**kwargs}}{}
This view is used to generate a form by which to add pups which belong to a particular breeding set.

This view is intended to be used to add initial information about pups, including eartag, genotype, gender and birth or weaning information.
It takes a request in the form /breeding/(breeding\_id)/pups/ and returns a form specific to the breeding set defined in breeding\_id.  breeding\_id is the background identification number of the breeding set and does not refer to the barcode of any breeding cage.
This view is restricted to those with the permission animal.add\_animal.

\end{fulllineitems}

\index{multiple\_breeding\_pups() (in module animal.views)}

\begin{fulllineitems}
\phantomsection\label{api:animal.views.multiple_breeding_pups}\pysiglinewithargsret{\code{animal.views.}\bfcode{multiple\_breeding\_pups}}{\emph{request}, \emph{breeding\_id}}{}
This view is used to enter multiple animals at the same time from a breeding cage.
\begin{description}
\item[{It will generate a form containing animal information and a number of mice.  It is intended to create several identical animals with the same attributes.}] \leavevmode
This view requres an input of a breeding\_id to generate the correct form.

\end{description}

\end{fulllineitems}

\index{multiple\_pups() (in module animal.views)}

\begin{fulllineitems}
\phantomsection\label{api:animal.views.multiple_pups}\pysiglinewithargsret{\code{animal.views.}\bfcode{multiple\_pups}}{\emph{request}}{}
This view is used to enter multiple animals at the same time.

It will generate a form containing animal information and a number of mice.  It is intended to create several identical animals with the same attributes.

\end{fulllineitems}

\index{strain\_detail() (in module animal.views)}

\begin{fulllineitems}
\phantomsection\label{api:animal.views.strain_detail}\pysiglinewithargsret{\code{animal.views.}\bfcode{strain\_detail}}{\emph{request}, \emph{*args}, \emph{**kwargs}}{}
This view displays specific details about a strain.

It takes a request in the form strain/(strain\_slug)/ and renders the detail page for that strain.
This view also passes along a dictionary of alive animals belonging to that strain.
This page is restricted to logged-in users.

\end{fulllineitems}

\index{strain\_detail\_all() (in module animal.views)}

\begin{fulllineitems}
\phantomsection\label{api:animal.views.strain_detail_all}\pysiglinewithargsret{\code{animal.views.}\bfcode{strain\_detail\_all}}{\emph{request}, \emph{*args}, \emph{**kwargs}}{}
This view displays specific details about a strain.

It takes a request in the form /strain/(strain\_slug)/all and renders the detail page for that strain.
This view also passes along a dictionary of all animals belonging to that strain.
This page is restricted to logged-in users.

\end{fulllineitems}

\index{strain\_list() (in module animal.views)}

\begin{fulllineitems}
\phantomsection\label{api:animal.views.strain_list}\pysiglinewithargsret{\code{animal.views.}\bfcode{strain\_list}}{\emph{request}, \emph{*args}, \emph{**kwargs}}{}
This view presents a list of strains currently present in the database and annotates this list with a count of alive and total animals.

This view redirects from a /strain/ request.
This view is restricted to logged-in users.

\end{fulllineitems}

\phantomsection\label{api:module-animal.urls}\index{animal.urls (module)}

\subsection{Administrative Site Configuration}
\label{api:id4}\phantomsection\label{api:module-animal.admin}\index{animal.admin (module)}
Admin site settings for the animal app.
\index{AnimalAdmin (class in animal.admin)}

\begin{fulllineitems}
\phantomsection\label{api:animal.admin.AnimalAdmin}\pysiglinewithargsret{\strong{class }\code{animal.admin.}\bfcode{AnimalAdmin}}{\emph{model}, \emph{admin\_site}}{}
Bases: \code{django.contrib.admin.options.ModelAdmin}

Provides parameters for animal objects within the admin interface.
\index{mark\_sacrificed() (animal.admin.AnimalAdmin method)}

\begin{fulllineitems}
\phantomsection\label{api:animal.admin.AnimalAdmin.mark_sacrificed}\pysiglinewithargsret{\bfcode{mark\_sacrificed}}{\emph{request}, \emph{queryset}}{}
An admin action for marking several animals as sacrificed.

This action sets the selected animals as Alive=False, Death=today and Cause\_of\_Death as sacrificed.  To use other paramters, mice muse be individually marked as sacrificed.
This admin action also shows as the output the number of mice sacrificed.

\end{fulllineitems}

\index{media (animal.admin.AnimalAdmin attribute)}

\begin{fulllineitems}
\phantomsection\label{api:animal.admin.AnimalAdmin.media}\pysigline{\bfcode{media}}{}
\end{fulllineitems}


\end{fulllineitems}

\index{AnimalInline (class in animal.admin)}

\begin{fulllineitems}
\phantomsection\label{api:animal.admin.AnimalInline}\pysiglinewithargsret{\strong{class }\code{animal.admin.}\bfcode{AnimalInline}}{\emph{parent\_model}, \emph{admin\_site}}{}
Bases: \code{django.contrib.admin.options.TabularInline}

Provides an inline tabular formset for animal objects.

Currently used with the breeding admin page.
\index{media (animal.admin.AnimalInline attribute)}

\begin{fulllineitems}
\phantomsection\label{api:animal.admin.AnimalInline.media}\pysigline{\bfcode{media}}{}
\end{fulllineitems}

\index{model (animal.admin.AnimalInline attribute)}

\begin{fulllineitems}
\phantomsection\label{api:animal.admin.AnimalInline.model}\pysigline{\bfcode{model}}{}
alias of \code{Animal}

\end{fulllineitems}


\end{fulllineitems}

\index{BreedingAdmin (class in animal.admin)}

\begin{fulllineitems}
\phantomsection\label{api:animal.admin.BreedingAdmin}\pysiglinewithargsret{\strong{class }\code{animal.admin.}\bfcode{BreedingAdmin}}{\emph{model}, \emph{admin\_site}}{}
Bases: \code{django.contrib.admin.options.ModelAdmin}

Settings in the admin interface for dealing with Breeding objects.

This interface also includes an form for adding objects associated with this breeding cage.
\index{mark\_deactivated() (animal.admin.BreedingAdmin method)}

\begin{fulllineitems}
\phantomsection\label{api:animal.admin.BreedingAdmin.mark_deactivated}\pysiglinewithargsret{\bfcode{mark\_deactivated}}{\emph{request}, \emph{queryset}}{}
An admin action for marking several cages as inactive.

This action sets the selected cages as Active=False and Death=today.
This admin action also shows as the output the number of mice sacrificed.

\end{fulllineitems}

\index{media (animal.admin.BreedingAdmin attribute)}

\begin{fulllineitems}
\phantomsection\label{api:animal.admin.BreedingAdmin.media}\pysigline{\bfcode{media}}{}
\end{fulllineitems}


\end{fulllineitems}

\index{StrainAdmin (class in animal.admin)}

\begin{fulllineitems}
\phantomsection\label{api:animal.admin.StrainAdmin}\pysiglinewithargsret{\strong{class }\code{animal.admin.}\bfcode{StrainAdmin}}{\emph{model}, \emph{admin\_site}}{}
Bases: \code{django.contrib.admin.options.ModelAdmin}

Settings in the admin interface for dealing with Strain objects.
\index{media (animal.admin.StrainAdmin attribute)}

\begin{fulllineitems}
\phantomsection\label{api:animal.admin.StrainAdmin.media}\pysigline{\bfcode{media}}{}
\end{fulllineitems}


\end{fulllineitems}



\subsection{Test Files}
\label{api:id5}\phantomsection\label{api:module-animal.tests}\index{animal.tests (module)}
This file contains tests for the animal application.

These tests will verify generation and function of a new breeding, strain and animal object.
\index{BreedingModelTests (class in animal.tests)}

\begin{fulllineitems}
\phantomsection\label{api:animal.tests.BreedingModelTests}\pysiglinewithargsret{\strong{class }\code{animal.tests.}\bfcode{BreedingModelTests}}{\emph{methodName='runTest'}}{}
Bases: \code{django.test.testcases.TestCase}

Test the models contained in the `groups' app.
\index{setUp() (animal.tests.BreedingModelTests method)}

\begin{fulllineitems}
\phantomsection\label{api:animal.tests.BreedingModelTests.setUp}\pysiglinewithargsret{\bfcode{setUp}}{}{}
Instantiate the test client.

\end{fulllineitems}

\index{tearDown() (animal.tests.BreedingModelTests method)}

\begin{fulllineitems}
\phantomsection\label{api:animal.tests.BreedingModelTests.tearDown}\pysiglinewithargsret{\bfcode{tearDown}}{}{}
Depopulate created model instances from test database.

\end{fulllineitems}

\index{test\_autoset\_active\_state() (animal.tests.BreedingModelTests method)}

\begin{fulllineitems}
\phantomsection\label{api:animal.tests.BreedingModelTests.test_autoset_active_state}\pysiglinewithargsret{\bfcode{test\_autoset\_active\_state}}{}{}
This is a test for creating a new breeding object, with only the minimum being entered.  That object is then tested for the active state being automatically set when a End date is specified.

\end{fulllineitems}

\index{test\_create\_animal\_minimal() (animal.tests.BreedingModelTests method)}

\begin{fulllineitems}
\phantomsection\label{api:animal.tests.BreedingModelTests.test_create_animal_minimal}\pysiglinewithargsret{\bfcode{test\_create\_animal\_minimal}}{}{}
This is a test for creating a new animal object, with only the minimum fields being entered

\end{fulllineitems}

\index{test\_create\_breeding\_minimal() (animal.tests.BreedingModelTests method)}

\begin{fulllineitems}
\phantomsection\label{api:animal.tests.BreedingModelTests.test_create_breeding_minimal}\pysiglinewithargsret{\bfcode{test\_create\_breeding\_minimal}}{}{}
This is a test for creating a new breeding object, with only the minimum being entered.

\end{fulllineitems}

\index{test\_unicode() (animal.tests.BreedingModelTests method)}

\begin{fulllineitems}
\phantomsection\label{api:animal.tests.BreedingModelTests.test_unicode}\pysiglinewithargsret{\bfcode{test\_unicode}}{}{}
This is a test for creating a new animal object, with only the minimum fields being entered.  It then tests that the correct unicode representation is being generated.

\end{fulllineitems}


\end{fulllineitems}

\index{BreedingViewTests (class in animal.tests)}

\begin{fulllineitems}
\phantomsection\label{api:animal.tests.BreedingViewTests}\pysiglinewithargsret{\strong{class }\code{animal.tests.}\bfcode{BreedingViewTests}}{\emph{methodName='runTest'}}{}
Bases: \code{django.test.testcases.TestCase}
\index{setUp() (animal.tests.BreedingViewTests method)}

\begin{fulllineitems}
\phantomsection\label{api:animal.tests.BreedingViewTests.setUp}\pysiglinewithargsret{\bfcode{setUp}}{}{}
\end{fulllineitems}

\index{tearDown() (animal.tests.BreedingViewTests method)}

\begin{fulllineitems}
\phantomsection\label{api:animal.tests.BreedingViewTests.tearDown}\pysiglinewithargsret{\bfcode{tearDown}}{}{}
\end{fulllineitems}

\index{test\_breeding\_list() (animal.tests.BreedingViewTests method)}

\begin{fulllineitems}
\phantomsection\label{api:animal.tests.BreedingViewTests.test_breeding_list}\pysiglinewithargsret{\bfcode{test\_breeding\_list}}{}{}
\end{fulllineitems}

\index{test\_breeding\_list\_all() (animal.tests.BreedingViewTests method)}

\begin{fulllineitems}
\phantomsection\label{api:animal.tests.BreedingViewTests.test_breeding_list_all}\pysiglinewithargsret{\bfcode{test\_breeding\_list\_all}}{}{}
\end{fulllineitems}

\index{test\_breeding\_new() (animal.tests.BreedingViewTests method)}

\begin{fulllineitems}
\phantomsection\label{api:animal.tests.BreedingViewTests.test_breeding_new}\pysiglinewithargsret{\bfcode{test\_breeding\_new}}{}{}
\end{fulllineitems}


\end{fulllineitems}



\section{Timed Mating Package}
\label{api:module-timed_mating}\label{api:timed-mating-package}\index{timed\_mating (module)}
This package defines the timed\_mating app.

Timed matings are a specific type of breeding set.  
Generally, for these experiments a mating cage is set up and pregnancy is defined by a plug event.  
Based on this information, the age of an embryo can be estimated.  
When a breeding cage is defined, one option is to set this cage as a timed mating cage (ie Timed\_Mating=True).  
If this is the case, then a plug event can be registered and recorded for this mating set.  
If the mother gives birth then this cage is implicitly set as a normal breeding cage.


\subsection{Models}
\label{api:id6}\phantomsection\label{api:module-timed_mating.models}\index{timed\_mating.models (module)}
This defines the data model for the timed\_mating app.

Currently the only data model is for PlugEvents.
\index{PlugEvents (class in timed\_mating.models)}

\begin{fulllineitems}
\phantomsection\label{api:timed_mating.models.PlugEvents}\pysiglinewithargsret{\strong{class }\code{timed\_mating.models.}\bfcode{PlugEvents}}{\emph{*args}, \emph{**kwargs}}{}
Bases: \code{django.db.models.base.Model}

This defines the model for PlugEvents.

A PlugEvent requires a date.  All other fields are optional.
Upon observation of a plug event, the PlugDate, Breeding Cage, Femalem, Male, Researcher and Notes can be set.
Upon sacrifice of the mother, then genotyped alive and dead embryos can be entered, along with the SacrificeDate, Researcher and Notes.

\end{fulllineitems}



\subsection{Forms}
\label{api:id7}\phantomsection\label{api:module-timed_mating.forms}\index{timed\_mating.forms (module)}
This package describes forms used by the Timed Mating app.
\index{BreedingPlugForm (class in timed\_mating.forms)}

\begin{fulllineitems}
\phantomsection\label{api:timed_mating.forms.BreedingPlugForm}\pysiglinewithargsret{\strong{class }\code{timed\_mating.forms.}\bfcode{BreedingPlugForm}}{\emph{data=None}, \emph{files=None}, \emph{auto\_id='id\_\%s'}, \emph{prefix=None}, \emph{initial=None}, \emph{error\_class=\textless{}class `django.forms.util.ErrorList'\textgreater{}}, \emph{label\_suffix=':'}, \emph{empty\_permitted=False}, \emph{instance=None}}{}
Bases: \code{django.forms.models.ModelForm}

This form is used to enter Plug Events from a specific breeding cage.
\index{BreedingPlugForm.Meta (class in timed\_mating.forms)}

\begin{fulllineitems}
\phantomsection\label{api:timed_mating.forms.BreedingPlugForm.Meta}\pysigline{\strong{class }\bfcode{Meta}}{}~\index{model (timed\_mating.forms.BreedingPlugForm.Meta attribute)}

\begin{fulllineitems}
\phantomsection\label{api:timed_mating.forms.BreedingPlugForm.Meta.model}\pysigline{\bfcode{model}}{}
alias of \code{PlugEvents}

\end{fulllineitems}


\end{fulllineitems}

\index{media (timed\_mating.forms.BreedingPlugForm attribute)}

\begin{fulllineitems}
\phantomsection\label{api:timed_mating.forms.BreedingPlugForm.media}\pysigline{\code{BreedingPlugForm.}\bfcode{media}}{}
\end{fulllineitems}


\end{fulllineitems}



\subsection{Views and URLs}
\label{api:id8}\phantomsection\label{api:module-timed_mating.views}\index{timed\_mating.views (module)}
This package defines custom views for the timed\_mating application.

Currently all views are generic CRUD views except for the view in which a plug event is defined from a breeding cage.
\index{breeding\_plugevent() (in module timed\_mating.views)}

\begin{fulllineitems}
\phantomsection\label{api:timed_mating.views.breeding_plugevent}\pysiglinewithargsret{\code{timed\_mating.views.}\bfcode{breeding\_plugevent}}{\emph{request}, \emph{*args}, \emph{**kwargs}}{}
This view defines a form for adding new plug events from a breeding cage.

This form requires a breeding\_id from a breeding set and restricts the PlugFemale and PlugMale to animals that are defined in that breeding cage.

\end{fulllineitems}

\phantomsection\label{api:module-timed_mating.urls}\index{timed\_mating.urls (module)}\index{change\_plugevents() (in module timed\_mating.urls)}

\begin{fulllineitems}
\phantomsection\label{api:timed_mating.urls.change_plugevents}\pysiglinewithargsret{\code{timed\_mating.urls.}\bfcode{change\_plugevents}}{\emph{request}, \emph{*args}, \emph{**kwargs}}{}
\end{fulllineitems}

\index{create\_plugevents() (in module timed\_mating.urls)}

\begin{fulllineitems}
\phantomsection\label{api:timed_mating.urls.create_plugevents}\pysiglinewithargsret{\code{timed\_mating.urls.}\bfcode{create\_plugevents}}{\emph{request}, \emph{*args}, \emph{**kwargs}}{}
\end{fulllineitems}

\index{delete\_plugevents() (in module timed\_mating.urls)}

\begin{fulllineitems}
\phantomsection\label{api:timed_mating.urls.delete_plugevents}\pysiglinewithargsret{\code{timed\_mating.urls.}\bfcode{delete\_plugevents}}{\emph{request}, \emph{*args}, \emph{**kwargs}}{}
\end{fulllineitems}

\index{limited\_object\_detail() (in module timed\_mating.urls)}

\begin{fulllineitems}
\phantomsection\label{api:timed_mating.urls.limited_object_detail}\pysiglinewithargsret{\code{timed\_mating.urls.}\bfcode{limited\_object\_detail}}{\emph{request}, \emph{*args}, \emph{**kwargs}}{}
\end{fulllineitems}

\index{limited\_object\_list() (in module timed\_mating.urls)}

\begin{fulllineitems}
\phantomsection\label{api:timed_mating.urls.limited_object_list}\pysiglinewithargsret{\code{timed\_mating.urls.}\bfcode{limited\_object\_list}}{\emph{request}, \emph{*args}, \emph{**kwargs}}{}
\end{fulllineitems}



\subsection{Administrative Site Configuration}
\label{api:id9}\phantomsection\label{api:module-timed_mating.admin}\index{timed\_mating.admin (module)}
Settings to control the admin interface for the timed\_mating app.

This file defines a PlugEventsAdmin object to enter parameters about individual plug events/
\index{PlugEventsAdmin (class in timed\_mating.admin)}

\begin{fulllineitems}
\phantomsection\label{api:timed_mating.admin.PlugEventsAdmin}\pysiglinewithargsret{\strong{class }\code{timed\_mating.admin.}\bfcode{PlugEventsAdmin}}{\emph{model}, \emph{admin\_site}}{}
Bases: \code{django.contrib.admin.options.ModelAdmin}

This class defines the admin interface for the PlugEvents model.
\index{media (timed\_mating.admin.PlugEventsAdmin attribute)}

\begin{fulllineitems}
\phantomsection\label{api:timed_mating.admin.PlugEventsAdmin.media}\pysigline{\bfcode{media}}{}
\end{fulllineitems}


\end{fulllineitems}



\subsection{Test Files}
\label{api:id10}\phantomsection\label{api:module-timed_mating.tests}\index{timed\_mating.tests (module)}
This file contains tests for the timed\_mating application.

These tests will verify generation of a new PlugEvent object.
\index{GroupsModelTests (class in timed\_mating.tests)}

\begin{fulllineitems}
\phantomsection\label{api:timed_mating.tests.GroupsModelTests}\pysiglinewithargsret{\strong{class }\code{timed\_mating.tests.}\bfcode{GroupsModelTests}}{\emph{methodName='runTest'}}{}
Bases: \code{django.test.testcases.TestCase}

Test the models contained in the `groups' app.
\index{setUp() (timed\_mating.tests.GroupsModelTests method)}

\begin{fulllineitems}
\phantomsection\label{api:timed_mating.tests.GroupsModelTests.setUp}\pysiglinewithargsret{\bfcode{setUp}}{}{}
Instantiate the test client.

\end{fulllineitems}

\index{tearDown() (timed\_mating.tests.GroupsModelTests method)}

\begin{fulllineitems}
\phantomsection\label{api:timed_mating.tests.GroupsModelTests.tearDown}\pysiglinewithargsret{\bfcode{tearDown}}{}{}
Depopulate created model instances from test database.

\end{fulllineitems}

\index{test\_create\_plugevent\_minimal() (timed\_mating.tests.GroupsModelTests method)}

\begin{fulllineitems}
\phantomsection\label{api:timed_mating.tests.GroupsModelTests.test_create_plugevent_minimal}\pysiglinewithargsret{\bfcode{test\_create\_plugevent\_minimal}}{}{}
This is a test for creating a new PlugEvent object, with only the minimum being entered.

\end{fulllineitems}

\index{test\_create\_plugevent\_most\_fields() (timed\_mating.tests.GroupsModelTests method)}

\begin{fulllineitems}
\phantomsection\label{api:timed_mating.tests.GroupsModelTests.test_create_plugevent_most_fields}\pysiglinewithargsret{\bfcode{test\_create\_plugevent\_most\_fields}}{}{}
This is a test for creating a new PlugEvent object.

This test uses a Breeding, PlugDate, PlugMale and PlugFemale field.

\end{fulllineitems}

\index{test\_set\_plugevet\_inactive() (timed\_mating.tests.GroupsModelTests method)}

\begin{fulllineitems}
\phantomsection\label{api:timed_mating.tests.GroupsModelTests.test_set_plugevet_inactive}\pysiglinewithargsret{\bfcode{test\_set\_plugevet\_inactive}}{}{}
This is a test for the automatic inactivation of a cage when the SacrificeDate is entered.

\end{fulllineitems}


\end{fulllineitems}



\section{Groups Package}
\label{api:module-groups}\label{api:groups-package}\index{groups (module)}
This package defines the Group application.
This app defines generic Group and License information for a particular installation of MouseDB.  
Because every page on this site identifies both the Group and data restrictions, at a minimum, group information must be provided upon installation (see installation instructions).


\subsection{Models}
\label{api:id11}\phantomsection\label{api:module-groups.models}\index{groups.models (module)}\index{Group (class in groups.models)}

\begin{fulllineitems}
\phantomsection\label{api:groups.models.Group}\pysiglinewithargsret{\strong{class }\code{groups.models.}\bfcode{Group}}{\emph{*args}, \emph{**kwargs}}{}
Bases: \code{django.db.models.base.Model}

This defines the data structure for the Group model.

The only required field is group.
All other fields (group\_slug, group\_url, license, contact\_title, contact\_first, contact\_last and contact\_email) are optional.

\end{fulllineitems}



\subsection{Views and URLs}
\label{api:id12}\phantomsection\label{api:module-groups.views}\index{groups.views (module)}

\subsection{Administrative Site Configuration}
\label{api:id13}\phantomsection\label{api:module-groups.admin}\index{groups.admin (module)}\index{GroupAdmin (class in groups.admin)}

\begin{fulllineitems}
\phantomsection\label{api:groups.admin.GroupAdmin}\pysiglinewithargsret{\strong{class }\code{groups.admin.}\bfcode{GroupAdmin}}{\emph{model}, \emph{admin\_site}}{}
Bases: \code{django.contrib.admin.options.ModelAdmin}

Defines the admin interface for Groups.

Currently set as default.
\index{media (groups.admin.GroupAdmin attribute)}

\begin{fulllineitems}
\phantomsection\label{api:groups.admin.GroupAdmin.media}\pysigline{\bfcode{media}}{}
\end{fulllineitems}


\end{fulllineitems}

\index{LicenseAdmin (class in groups.admin)}

\begin{fulllineitems}
\phantomsection\label{api:groups.admin.LicenseAdmin}\pysiglinewithargsret{\strong{class }\code{groups.admin.}\bfcode{LicenseAdmin}}{\emph{model}, \emph{admin\_site}}{}
Bases: \code{django.contrib.admin.options.ModelAdmin}

Defines the admin interface for Licences.

Currently set as default.
\index{media (groups.admin.LicenseAdmin attribute)}

\begin{fulllineitems}
\phantomsection\label{api:groups.admin.LicenseAdmin.media}\pysigline{\bfcode{media}}{}
\end{fulllineitems}


\end{fulllineitems}



\subsection{Test Files}
\label{api:id14}\phantomsection\label{api:module-groups.tests}\index{groups.tests (module)}
This file contains tests for the groups application.

These tests will verify generation of a new group and license object.
\index{GroupsModelTests (class in groups.tests)}

\begin{fulllineitems}
\phantomsection\label{api:groups.tests.GroupsModelTests}\pysiglinewithargsret{\strong{class }\code{groups.tests.}\bfcode{GroupsModelTests}}{\emph{methodName='runTest'}}{}
Bases: \code{django.test.testcases.TestCase}

Test the models contained in the `groups' app.
\index{setUp() (groups.tests.GroupsModelTests method)}

\begin{fulllineitems}
\phantomsection\label{api:groups.tests.GroupsModelTests.setUp}\pysiglinewithargsret{\bfcode{setUp}}{}{}
Instantiate the test client.

\end{fulllineitems}

\index{tearDown() (groups.tests.GroupsModelTests method)}

\begin{fulllineitems}
\phantomsection\label{api:groups.tests.GroupsModelTests.tearDown}\pysiglinewithargsret{\bfcode{tearDown}}{}{}
Depopulate created model instances from test database.

\end{fulllineitems}

\index{test\_create\_group\_all\_fields() (groups.tests.GroupsModelTests method)}

\begin{fulllineitems}
\phantomsection\label{api:groups.tests.GroupsModelTests.test_create_group_all_fields}\pysiglinewithargsret{\bfcode{test\_create\_group\_all\_fields}}{}{}
This is a test for creating a new group object, with all fields being entered, except license.

\end{fulllineitems}

\index{test\_create\_group\_minimal() (groups.tests.GroupsModelTests method)}

\begin{fulllineitems}
\phantomsection\label{api:groups.tests.GroupsModelTests.test_create_group_minimal}\pysiglinewithargsret{\bfcode{test\_create\_group\_minimal}}{}{}
This is a test for creating a new group object, with only the minimum being entered.

\end{fulllineitems}

\index{test\_create\_license\_all\_fields() (groups.tests.GroupsModelTests method)}

\begin{fulllineitems}
\phantomsection\label{api:groups.tests.GroupsModelTests.test_create_license_all_fields}\pysiglinewithargsret{\bfcode{test\_create\_license\_all\_fields}}{}{}
This is a test for creating a new license object, with all fields being entered.

\end{fulllineitems}

\index{test\_create\_license\_minimal() (groups.tests.GroupsModelTests method)}

\begin{fulllineitems}
\phantomsection\label{api:groups.tests.GroupsModelTests.test_create_license_minimal}\pysiglinewithargsret{\bfcode{test\_create\_license\_minimal}}{}{}
This is a test for creating a new license object, with only the minimum being entered.

\end{fulllineitems}


\end{fulllineitems}



\chapter{Indices and tables}
\label{index:indices-and-tables}\begin{itemize}
\item {} 
\emph{genindex}

\item {} 
\emph{modindex}

\item {} 
\emph{search}

\end{itemize}


\renewcommand{\indexname}{Python Module Index}
\begin{theindex}
\def\bigletter#1{{\Large\sffamily#1}\nopagebreak\vspace{1mm}}
\bigletter{a}
\item {\texttt{animal}}, \pageref{api:module-animal}
\item {\texttt{animal.admin}}, \pageref{api:module-animal.admin}
\item {\texttt{animal.forms}}, \pageref{api:module-animal.forms}
\item {\texttt{animal.models}}, \pageref{api:module-animal.models}
\item {\texttt{animal.tests}}, \pageref{api:module-animal.tests}
\item {\texttt{animal.urls}}, \pageref{api:module-animal.urls}
\item {\texttt{animal.views}}, \pageref{api:module-animal.views}
\indexspace
\bigletter{d}
\item {\texttt{data}}, \pageref{api:module-data}
\item {\texttt{data.admin}}, \pageref{api:module-data.admin}
\item {\texttt{data.forms}}, \pageref{api:module-data.forms}
\item {\texttt{data.models}}, \pageref{api:module-data.models}
\item {\texttt{data.tests}}, \pageref{api:module-data.tests}
\item {\texttt{data.urls}}, \pageref{api:module-data.urls}
\item {\texttt{data.views}}, \pageref{api:module-data.views}
\indexspace
\bigletter{g}
\item {\texttt{groups}}, \pageref{api:module-groups}
\item {\texttt{groups.admin}}, \pageref{api:module-groups.admin}
\item {\texttt{groups.models}}, \pageref{api:module-groups.models}
\item {\texttt{groups.tests}}, \pageref{api:module-groups.tests}
\item {\texttt{groups.views}}, \pageref{api:module-groups.views}
\indexspace
\bigletter{t}
\item {\texttt{timed\_mating}}, \pageref{api:module-timed_mating}
\item {\texttt{timed\_mating.admin}}, \pageref{api:module-timed_mating.admin}
\item {\texttt{timed\_mating.forms}}, \pageref{api:module-timed_mating.forms}
\item {\texttt{timed\_mating.models}}, \pageref{api:module-timed_mating.models}
\item {\texttt{timed\_mating.tests}}, \pageref{api:module-timed_mating.tests}
\item {\texttt{timed\_mating.urls}}, \pageref{api:module-timed_mating.urls}
\item {\texttt{timed\_mating.views}}, \pageref{api:module-timed_mating.views}
\end{theindex}

\renewcommand{\indexname}{Index}
\printindex
\end{document}
