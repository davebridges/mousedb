% Generated by Sphinx.
\def\sphinxdocclass{report}
\documentclass[letterpaper,10pt,english]{sphinxmanual}
\usepackage[utf8]{inputenc}
\DeclareUnicodeCharacter{00A0}{\nobreakspace}
\usepackage[T1]{fontenc}
\usepackage{babel}
\usepackage{times}
\usepackage[Bjarne]{fncychap}
\usepackage{longtable}
\usepackage{sphinx}


\title{MouseDB Documentation}
\date{December 27, 2009}
\release{0.1}
\author{Dave Bridges, Ph.D.}
\newcommand{\sphinxlogo}{}
\renewcommand{\releasename}{Release}
\makeindex
\makemodindex

\makeatletter
\def\PYG@reset{\let\PYG@it=\relax \let\PYG@bf=\relax%
    \let\PYG@ul=\relax \let\PYG@tc=\relax%
    \let\PYG@bc=\relax \let\PYG@ff=\relax}
\def\PYG@tok#1{\csname PYG@tok@#1\endcsname}
\def\PYG@toks#1+{\ifx\relax#1\empty\else%
    \PYG@tok{#1}\expandafter\PYG@toks\fi}
\def\PYG@do#1{\PYG@bc{\PYG@tc{\PYG@ul{%
    \PYG@it{\PYG@bf{\PYG@ff{#1}}}}}}}
\def\PYG#1#2{\PYG@reset\PYG@toks#1+\relax+\PYG@do{#2}}

\def\PYG@tok@gd{\def\PYG@tc##1{\textcolor[rgb]{0.63,0.00,0.00}{##1}}}
\def\PYG@tok@gu{\let\PYG@bf=\textbf\def\PYG@tc##1{\textcolor[rgb]{0.50,0.00,0.50}{##1}}}
\def\PYG@tok@gt{\def\PYG@tc##1{\textcolor[rgb]{0.00,0.25,0.82}{##1}}}
\def\PYG@tok@gs{\let\PYG@bf=\textbf}
\def\PYG@tok@gr{\def\PYG@tc##1{\textcolor[rgb]{1.00,0.00,0.00}{##1}}}
\def\PYG@tok@cm{\let\PYG@it=\textit\def\PYG@tc##1{\textcolor[rgb]{0.25,0.50,0.56}{##1}}}
\def\PYG@tok@vg{\def\PYG@tc##1{\textcolor[rgb]{0.73,0.38,0.84}{##1}}}
\def\PYG@tok@m{\def\PYG@tc##1{\textcolor[rgb]{0.13,0.50,0.31}{##1}}}
\def\PYG@tok@mh{\def\PYG@tc##1{\textcolor[rgb]{0.13,0.50,0.31}{##1}}}
\def\PYG@tok@cs{\def\PYG@tc##1{\textcolor[rgb]{0.25,0.50,0.56}{##1}}\def\PYG@bc##1{\colorbox[rgb]{1.00,0.94,0.94}{##1}}}
\def\PYG@tok@ge{\let\PYG@it=\textit}
\def\PYG@tok@vc{\def\PYG@tc##1{\textcolor[rgb]{0.73,0.38,0.84}{##1}}}
\def\PYG@tok@il{\def\PYG@tc##1{\textcolor[rgb]{0.13,0.50,0.31}{##1}}}
\def\PYG@tok@go{\def\PYG@tc##1{\textcolor[rgb]{0.19,0.19,0.19}{##1}}}
\def\PYG@tok@cp{\def\PYG@tc##1{\textcolor[rgb]{0.00,0.44,0.13}{##1}}}
\def\PYG@tok@gi{\def\PYG@tc##1{\textcolor[rgb]{0.00,0.63,0.00}{##1}}}
\def\PYG@tok@gh{\let\PYG@bf=\textbf\def\PYG@tc##1{\textcolor[rgb]{0.00,0.00,0.50}{##1}}}
\def\PYG@tok@ni{\let\PYG@bf=\textbf\def\PYG@tc##1{\textcolor[rgb]{0.84,0.33,0.22}{##1}}}
\def\PYG@tok@nl{\let\PYG@bf=\textbf\def\PYG@tc##1{\textcolor[rgb]{0.00,0.13,0.44}{##1}}}
\def\PYG@tok@nn{\let\PYG@bf=\textbf\def\PYG@tc##1{\textcolor[rgb]{0.05,0.52,0.71}{##1}}}
\def\PYG@tok@no{\def\PYG@tc##1{\textcolor[rgb]{0.38,0.68,0.84}{##1}}}
\def\PYG@tok@na{\def\PYG@tc##1{\textcolor[rgb]{0.25,0.44,0.63}{##1}}}
\def\PYG@tok@nb{\def\PYG@tc##1{\textcolor[rgb]{0.00,0.44,0.13}{##1}}}
\def\PYG@tok@nc{\let\PYG@bf=\textbf\def\PYG@tc##1{\textcolor[rgb]{0.05,0.52,0.71}{##1}}}
\def\PYG@tok@nd{\let\PYG@bf=\textbf\def\PYG@tc##1{\textcolor[rgb]{0.33,0.33,0.33}{##1}}}
\def\PYG@tok@ne{\def\PYG@tc##1{\textcolor[rgb]{0.00,0.44,0.13}{##1}}}
\def\PYG@tok@nf{\def\PYG@tc##1{\textcolor[rgb]{0.02,0.16,0.49}{##1}}}
\def\PYG@tok@si{\let\PYG@it=\textit\def\PYG@tc##1{\textcolor[rgb]{0.44,0.63,0.82}{##1}}}
\def\PYG@tok@s2{\def\PYG@tc##1{\textcolor[rgb]{0.25,0.44,0.63}{##1}}}
\def\PYG@tok@vi{\def\PYG@tc##1{\textcolor[rgb]{0.73,0.38,0.84}{##1}}}
\def\PYG@tok@nt{\let\PYG@bf=\textbf\def\PYG@tc##1{\textcolor[rgb]{0.02,0.16,0.45}{##1}}}
\def\PYG@tok@nv{\def\PYG@tc##1{\textcolor[rgb]{0.73,0.38,0.84}{##1}}}
\def\PYG@tok@s1{\def\PYG@tc##1{\textcolor[rgb]{0.25,0.44,0.63}{##1}}}
\def\PYG@tok@gp{\let\PYG@bf=\textbf\def\PYG@tc##1{\textcolor[rgb]{0.78,0.36,0.04}{##1}}}
\def\PYG@tok@sh{\def\PYG@tc##1{\textcolor[rgb]{0.25,0.44,0.63}{##1}}}
\def\PYG@tok@ow{\let\PYG@bf=\textbf\def\PYG@tc##1{\textcolor[rgb]{0.00,0.44,0.13}{##1}}}
\def\PYG@tok@sx{\def\PYG@tc##1{\textcolor[rgb]{0.78,0.36,0.04}{##1}}}
\def\PYG@tok@bp{\def\PYG@tc##1{\textcolor[rgb]{0.00,0.44,0.13}{##1}}}
\def\PYG@tok@c1{\let\PYG@it=\textit\def\PYG@tc##1{\textcolor[rgb]{0.25,0.50,0.56}{##1}}}
\def\PYG@tok@kc{\let\PYG@bf=\textbf\def\PYG@tc##1{\textcolor[rgb]{0.00,0.44,0.13}{##1}}}
\def\PYG@tok@c{\let\PYG@it=\textit\def\PYG@tc##1{\textcolor[rgb]{0.25,0.50,0.56}{##1}}}
\def\PYG@tok@mf{\def\PYG@tc##1{\textcolor[rgb]{0.13,0.50,0.31}{##1}}}
\def\PYG@tok@err{\def\PYG@bc##1{\fcolorbox[rgb]{1.00,0.00,0.00}{1,1,1}{##1}}}
\def\PYG@tok@kd{\let\PYG@bf=\textbf\def\PYG@tc##1{\textcolor[rgb]{0.00,0.44,0.13}{##1}}}
\def\PYG@tok@ss{\def\PYG@tc##1{\textcolor[rgb]{0.32,0.47,0.09}{##1}}}
\def\PYG@tok@sr{\def\PYG@tc##1{\textcolor[rgb]{0.14,0.33,0.53}{##1}}}
\def\PYG@tok@mo{\def\PYG@tc##1{\textcolor[rgb]{0.13,0.50,0.31}{##1}}}
\def\PYG@tok@mi{\def\PYG@tc##1{\textcolor[rgb]{0.13,0.50,0.31}{##1}}}
\def\PYG@tok@kn{\let\PYG@bf=\textbf\def\PYG@tc##1{\textcolor[rgb]{0.00,0.44,0.13}{##1}}}
\def\PYG@tok@o{\def\PYG@tc##1{\textcolor[rgb]{0.40,0.40,0.40}{##1}}}
\def\PYG@tok@kr{\let\PYG@bf=\textbf\def\PYG@tc##1{\textcolor[rgb]{0.00,0.44,0.13}{##1}}}
\def\PYG@tok@s{\def\PYG@tc##1{\textcolor[rgb]{0.25,0.44,0.63}{##1}}}
\def\PYG@tok@kp{\def\PYG@tc##1{\textcolor[rgb]{0.00,0.44,0.13}{##1}}}
\def\PYG@tok@w{\def\PYG@tc##1{\textcolor[rgb]{0.73,0.73,0.73}{##1}}}
\def\PYG@tok@kt{\def\PYG@tc##1{\textcolor[rgb]{0.56,0.13,0.00}{##1}}}
\def\PYG@tok@sc{\def\PYG@tc##1{\textcolor[rgb]{0.25,0.44,0.63}{##1}}}
\def\PYG@tok@sb{\def\PYG@tc##1{\textcolor[rgb]{0.25,0.44,0.63}{##1}}}
\def\PYG@tok@k{\let\PYG@bf=\textbf\def\PYG@tc##1{\textcolor[rgb]{0.00,0.44,0.13}{##1}}}
\def\PYG@tok@se{\let\PYG@bf=\textbf\def\PYG@tc##1{\textcolor[rgb]{0.25,0.44,0.63}{##1}}}
\def\PYG@tok@sd{\let\PYG@it=\textit\def\PYG@tc##1{\textcolor[rgb]{0.25,0.44,0.63}{##1}}}

\def\PYGZat{@}
\def\PYGZlb{[}
\def\PYGZrb{]}
\makeatother

\begin{document}

\maketitle
\tableofcontents
\hypertarget{--doc-index}{}


Contents:

\resetcurrentobjects
\hypertarget{--doc-concepts}{}

\chapter{MouseDB Concepts}

Data storage for MouseDB is separated into packages which contain information about animals, and information collected about animals.  There is also a separate module for timed matings of animals.  This document will describe the basics of how data is stored in each of these modules.


\section{Animal Module}

Animals are tracked as individual entities, and given associations to breeding cages to follow ancestry, and strains.


\subsection{Animal}


\subsection{Breeding Cages}

A breeding cage is defined as a set of one or more male and one or more female mice.  Because of this, it is not always clear who the precise parentage of an animal is.  If the parentage is known, then the Mother and Father fields can be set for a particular animal.


\subsection{Strains}


\section{Data Module}

Data (or measurements) can be stored for any type of measurement.  Conceptually, several pieces of data belong to an experiment (for example several mice are measured at some time) and several experiments belong to a study.  Measurements can be stored independent of experiments and experiments can be performed outside of the context of a study.  It is however, perfered that measurements are stored within an experiment and experiments are stored within studies as this will greatly facilitate the organization of the data.


\subsection{Studies}

In general studies are a collection of experiments.  These can be grouped together on the basis of animals and/or treatment groups.  A study must have at least one treatment group, which defines the animals and their conditions.


\subsection{Experiments}

An experiment is a collection of measurements for a given set of animals.  In general, an experiment is defined as a number of measurements take in a given day.


\subsection{Measurements}

A measurement is an animal, an assay and a measurement value.  It can be associated with an experiment, or can stand alone as an individual value.  Measurements can be viewed in the context of a study, an experiment, a treatment group or an animal by going to the appropriate page.

\resetcurrentobjects
\hypertarget{--doc-installation}{}

\chapter{MouseDB Installation}


\section{Configuration}

MouseDB requires both a database and a webserver to be set up.  Ideally, the database should be hosted separately from the webserver and MouseDB installation, but this is not necessary, as both can be used from the same server.  If you are using a remote server for the database, it is best to set up a user for this database that can only be accessed from the webserver.  If you want to set up several installations (ie for different users or different laboratories), you need separate databases and MouseDB installations for each.  You will also need to set up the webserver with different addresses for each installation.


\section{Software Dependencies}
\begin{enumerate}
\item {} 
\textbf{MouseDB source code}.  Download from one of the following:

\end{enumerate}
\begin{enumerate}
\item {} 
\href{http://github.com/davebridges/mousedb/downloads}{http://github.com/davebridges/mousedb/downloads} for the current release

\item {} 
\href{http://github.com/davebridges/mousedb}{http://github.com/davebridges/mousedb} for the source code via Git

\end{enumerate}

Downloading and/or unzipping will create a directory named mousedb.  You can update to the newest revision at any time either using git or downloading and re-installing the newer version.  Changing or updating software versions will not alter any saved data, but you will have to update the localsettings.py file (described below).
\begin{enumerate}
\item {} 
\textbf{Python}.  Requires Version 2.6, is not yet compatible with Python 3.0.  Download from \href{http://www.python.org/download/}{http://www.python.org/download/}.

\item {} 
\textbf{Django}.  Download from \href{http://www.djangoproject.com/download/}{http://www.djangoproject.com/download/}

\item {} 
\textbf{Database software}.  Typically MySQL is used, but PostgreSQL, Oracle or SQLite can also be used.  You also need to install the python driver for this database (unless you are using SQLite, which is internal to Python 2.5+).  See \href{http://docs.djangoproject.com/en/dev/topics/install/database-installation}{http://docs.djangoproject.com/en/dev/topics/install/database-installation} - Django Database Installation Documentation for more information.

\end{enumerate}


\section{Database Setup}
\begin{enumerate}
\item {} 
Create a new database.  You need to record the user, password, host and database name.  If you are using SQLite this step is not required.

\item {} 
Go to localsettings\_empty.py and edit the settings:

\end{enumerate}
\begin{itemize}
\item {} 
DATABASE\_ENGINE: `mysql', `postgresql\_psycopg2' or `sqlite3 depending on the database software used.

\item {} 
DATABASE\_NAME: database name

\item {} 
DATABASE\_USER: database user

\item {} 
DATABASE\_PASSWORD: database password

\item {} 
DATABASE\_HOST: database host

\end{itemize}
\begin{enumerate}
\item {} 
Save this file as localsettings.py in the main MouseDB directory.

\end{enumerate}


\section{Web Server Setup}

You need to set up a server to serve both the django installation and saved files.  For the saved files.  I recommend using apache for both.  The preferred setup is to use Apache2 with mod\_python.  The following is a httpd.conf example where the code is placed in /usr/src/mousedb:

\begin{Verbatim}[commandchars=@\[\]]
Alias /static /usr/src/mousedb/media
Alias /media /usr/src/mousedb/media
@textless[]Directory /usr/mousedb/media@textgreater[]
   Order allow,deny
   Allow from all
@textless[]/Directory@textgreater[]
@textless[]Location "/mousedb/"@textgreater[]
   SetHandler python-program
   PythonHandler django.core.handlers.modpython
   SetEnv DJANGO@_SETTINGS@_MODULE mousedb.settings
   SetEnv PYTHON@_EGG@_CACHE /var/www/eggs
   PythonOption django.root /mousedb
   PythonDebug On
   PythonPath "@PYGZlb[]'/usr/src'@PYGZrb[] + sys.path"
   PythonInterpreter mousedb
@textless[]/Location@textgreater[]
\end{Verbatim}

If you want to restrict access to these files, change the Allow from all directive to specific domains or ip addresses (for example Allow from 192.168.0.0/99 would allow from 192.168.0.0 to 192.168.0.99)


\section{Final Configuration and User Setup}
\begin{enumerate}
\item {} 
Go to mousedb/admin/auth/users/ and create users, selecting usernames, full names, password (or have the user set the password) and then choose group permissions.

\end{enumerate}

\resetcurrentobjects
\hypertarget{--doc-usage}{}

\chapter{Animal Data Entry}


\section{Newborn Mice or Newly Weaned Mice}
\begin{enumerate}
\item {} 
Go to Breeding Cages Tab

\item {} 
Click on Add/Wean Pups Button

\item {} 
Each row is a new animal.  If you accidentaly enter an extra animal, check off the delete box then submit.

\item {} 
Leave extra lines blank if you have less than 10 mice to enter

\item {} 
If you need to enter more than 10 mice, enter the first ten and submit them.  Go back and enter up to 10 more animals (10 more blank spaces will appear)

\end{enumerate}


\section{Newborn Mice}
\begin{enumerate}
\item {} 
Enter Breeding Cage under Cage

\item {} 
Enter Strain

\item {} 
Enter Background (normally Mixed or C57BL/6-BA unless from the LY breeding cages in which case it is C57BL/6-LY5.2)

\item {} 
Enter Birthdate in format YYYY-MM-DD

\item {} 
Enter Generation and Backcross

\end{enumerate}


\section{Weaning Mice}
\begin{enumerate}
\item {} 
If not previously entered, enter data as if newborn mice

\item {} 
Enter gender

\item {} 
Enter Wean Date in format YYYY-MM-DD

\item {} 
Enter new Cage number for Cage

\end{enumerate}


\section{Cage Changes (Not Weaning)}
\begin{enumerate}
\item {} 
Find mouse either from animal list or strain list

\item {} 
Click the edit mouse button

\item {} 
Change the Cage, Rack and Rack Position as Necessary

\end{enumerate}


\section{Genotyping or Ear Tagging}
\begin{enumerate}
\item {} 
Find mouse either from animal list or strain list, or through breeding cage

\item {} 
Click the edit mouse button or the Eartag/Genotype/Cage Change/Death Button

\item {} 
Enter the Ear Tag and/or select the Genotype from the Pull Down List

\end{enumerate}


\section{Marking Mice as Dead}


\subsection{Dead Mice (Single Mouse)}
\begin{enumerate}
\item {} 
Find mouse from animal list or strain list

\item {} 
Click the edit mouse button

\item {} 
Enter the death date in format YYYY-MM-DD

\item {} 
Choose Cause of Death from Pull Down List

\end{enumerate}


\subsection{Dead Mice (Several Mice)}
\begin{enumerate}
\item {} 
Find mice from breeding cages

\item {} 
Click the Eartag/Genotype/Cage Change/Death Button

\item {} 
Enter the death date in format YYYY-MM-DD

\item {} 
Choose the Cause of Death from Pull Down List

\end{enumerate}


\chapter{Studies and Experimental Setup}

Set up a new study at /mousedb/admin/data/study/ selecting animals

You must put a description and select animals in one or more treatment groups

If you have more than 2 treatment groups save the first two, then two more empty slots will appear. For animals, click on the magnifying glass then find the animal in that treatment group and click on the MouseID. The number displayed now in that field will not be the MouseID, but don't worry its just a different number to describe the mouse. To add more animals, click on the magnifying glass again and select the next animal. There should be now two numbers, separated by commas in this field. Repeat to fill all your treatment groups. You must enter a diet and environment for each treatment. The other fields are optional, and should only be used if appropriate. Ensure for pharmaceutical, you include a saline treatment group.


\chapter{Measurment Entry}


\section{Studies}

If this measurement is part of a study (ie a group of experiments) then click on the plus sign beside the study field and enter in the details about the study and treatment groups.  Unfortunately until i can figure out how to filter the treatment group animals in the admin interface, at each of the subsequent steps you will see all the animals in the database (soon hopefully it will only be the ones as part of the study group).


\section{Experiment Details}
\begin{itemize}
\item {} 
Pick experiment date, feeding state and resarchers

\item {} 
Pick animals used in this experiment (the search box will filter results)

\item {} 
Fasting state, time, injections, concentration, experimentID and notes are all optional

\end{itemize}


\section{Measurements}
\begin{itemize}
\item {} 
There is room to enter 14 measurements.  If you need more rows, enter the first 14 and select ``Save and Continue Editing'' and 14 more blank spots will appear.

\item {} 
Each row is a measurement, so if you have glucose and weight for some animal that is two rows entered.

\item {} 
For animals, click on the magnifying glass then find the animal in that treatment group and click on the MouseID. The number displayed now in that field will not be the MouseID, but don't worry its just a different number to describe the mouse.

\item {} 
For values, the standard units (defined by each assay) are mg for weights, mg/dL for glucose and pg/mL for insulin).  You must enter integers here (no decimal places).  If you have several measurements (ie several glucose readings during a GTT, enter them all in one measurement row, separated by commas and \emph{NO spaces}).

\end{itemize}

\resetcurrentobjects
\hypertarget{--doc-api}{}

\chapter{Automated Documentation}


\section{Data Package}
\index{data (module)}
\hypertarget{module-data}{}
\declaremodule[data]{}{data}
\modulesynopsis{}

\subsection{Models}
\index{data.models (module)}
\hypertarget{module-data.models}{}
\declaremodule[data.models]{}{data.models}
\modulesynopsis{}\index{Assay (class in data.models)}

\hypertarget{data.models.Assay}{}\begin{classdesc}{Assay}{*args, **kwargs}
Assay(id, assay, assay\_slug, notes, measurement\_units)
\index{Assay.DoesNotExist}

\hypertarget{data.models.Assay.DoesNotExist}{}\begin{excdesc}{DoesNotExist}~\index{args (data.models.Assay.DoesNotExist attribute)}

\hypertarget{data.models.Assay.DoesNotExist.args}{}\begin{memberdesc}{args}\end{memberdesc}
\index{message (data.models.Assay.DoesNotExist attribute)}

\hypertarget{data.models.Assay.DoesNotExist.message}{}\begin{memberdesc}{message}\end{memberdesc}
\end{excdesc}
\index{Assay.MultipleObjectsReturned}

\hypertarget{data.models.Assay.MultipleObjectsReturned}{}\begin{excdesc}{MultipleObjectsReturned}~\index{args (data.models.Assay.MultipleObjectsReturned attribute)}

\hypertarget{data.models.Assay.MultipleObjectsReturned.args}{}\begin{memberdesc}{args}\end{memberdesc}
\index{message (data.models.Assay.MultipleObjectsReturned attribute)}

\hypertarget{data.models.Assay.MultipleObjectsReturned.message}{}\begin{memberdesc}{message}\end{memberdesc}
\end{excdesc}
\index{delete() (data.models.Assay method)}

\hypertarget{data.models.Assay.delete}{}\begin{methoddesc}[Assay]{delete}{}\end{methoddesc}
\index{measurement\_set (data.models.Assay attribute)}

\hypertarget{data.models.Assay.measurement_set}{}\begin{memberdesc}[Assay]{measurement\_set}\end{memberdesc}
\index{pk (data.models.Assay attribute)}

\hypertarget{data.models.Assay.pk}{}\begin{memberdesc}[Assay]{pk}\end{memberdesc}
\index{prepare\_database\_save() (data.models.Assay method)}

\hypertarget{data.models.Assay.prepare_database_save}{}\begin{methoddesc}[Assay]{prepare\_database\_save}{unused}\end{methoddesc}
\index{save() (data.models.Assay method)}

\hypertarget{data.models.Assay.save}{}\begin{methoddesc}[Assay]{save}{force\_insert=False, force\_update=False}
Saves the current instance. Override this in a subclass if you want to
control the saving process.

The `force\_insert' and `force\_update' parameters can be used to insist
that the ``save'' must be an SQL insert or update (or equivalent for
non-SQL backends), respectively. Normally, they should not be set.
\end{methoddesc}
\index{save\_base() (data.models.Assay method)}

\hypertarget{data.models.Assay.save_base}{}\begin{methoddesc}[Assay]{save\_base}{raw=False, cls=None, origin=None, force\_insert=False, force\_update=False}
Does the heavy-lifting involved in saving. Subclasses shouldn't need to
override this method. It's separate from save() in order to hide the
need for overrides of save() to pass around internal-only parameters
(`raw', `cls', and `origin').
\end{methoddesc}
\index{serializable\_value() (data.models.Assay method)}

\hypertarget{data.models.Assay.serializable_value}{}\begin{methoddesc}[Assay]{serializable\_value}{field\_name}
Returns the value of the field name for this instance. If the field is
a foreign key, returns the id value, instead of the object. If there's
no Field object with this name on the model, the model attribute's
value is returned directly.

Used to serialize a field's value (in the serializer, or form output,
for example). Normally, you would just access the attribute directly
and not use this method.
\end{methoddesc}
\end{classdesc}
\index{Diet (class in data.models)}

\hypertarget{data.models.Diet}{}\begin{classdesc}{Diet}{*args, **kwargs}
Diet(id, vendor\_id, description, product\_id, fat\_content, protein\_content, carb\_content, irradiated, notes)
\index{Diet.DoesNotExist}

\hypertarget{data.models.Diet.DoesNotExist}{}\begin{excdesc}{DoesNotExist}~\index{args (data.models.Diet.DoesNotExist attribute)}

\hypertarget{data.models.Diet.DoesNotExist.args}{}\begin{memberdesc}{args}\end{memberdesc}
\index{message (data.models.Diet.DoesNotExist attribute)}

\hypertarget{data.models.Diet.DoesNotExist.message}{}\begin{memberdesc}{message}\end{memberdesc}
\end{excdesc}
\index{Diet.MultipleObjectsReturned}

\hypertarget{data.models.Diet.MultipleObjectsReturned}{}\begin{excdesc}{MultipleObjectsReturned}~\index{args (data.models.Diet.MultipleObjectsReturned attribute)}

\hypertarget{data.models.Diet.MultipleObjectsReturned.args}{}\begin{memberdesc}{args}\end{memberdesc}
\index{message (data.models.Diet.MultipleObjectsReturned attribute)}

\hypertarget{data.models.Diet.MultipleObjectsReturned.message}{}\begin{memberdesc}{message}\end{memberdesc}
\end{excdesc}
\index{delete() (data.models.Diet method)}

\hypertarget{data.models.Diet.delete}{}\begin{methoddesc}[Diet]{delete}{}\end{methoddesc}
\index{pk (data.models.Diet attribute)}

\hypertarget{data.models.Diet.pk}{}\begin{memberdesc}[Diet]{pk}\end{memberdesc}
\index{prepare\_database\_save() (data.models.Diet method)}

\hypertarget{data.models.Diet.prepare_database_save}{}\begin{methoddesc}[Diet]{prepare\_database\_save}{unused}\end{methoddesc}
\index{save() (data.models.Diet method)}

\hypertarget{data.models.Diet.save}{}\begin{methoddesc}[Diet]{save}{force\_insert=False, force\_update=False}
Saves the current instance. Override this in a subclass if you want to
control the saving process.

The `force\_insert' and `force\_update' parameters can be used to insist
that the ``save'' must be an SQL insert or update (or equivalent for
non-SQL backends), respectively. Normally, they should not be set.
\end{methoddesc}
\index{save\_base() (data.models.Diet method)}

\hypertarget{data.models.Diet.save_base}{}\begin{methoddesc}[Diet]{save\_base}{raw=False, cls=None, origin=None, force\_insert=False, force\_update=False}
Does the heavy-lifting involved in saving. Subclasses shouldn't need to
override this method. It's separate from save() in order to hide the
need for overrides of save() to pass around internal-only parameters
(`raw', `cls', and `origin').
\end{methoddesc}
\index{serializable\_value() (data.models.Diet method)}

\hypertarget{data.models.Diet.serializable_value}{}\begin{methoddesc}[Diet]{serializable\_value}{field\_name}
Returns the value of the field name for this instance. If the field is
a foreign key, returns the id value, instead of the object. If there's
no Field object with this name on the model, the model attribute's
value is returned directly.

Used to serialize a field's value (in the serializer, or form output,
for example). Normally, you would just access the attribute directly
and not use this method.
\end{methoddesc}
\index{treatment\_set (data.models.Diet attribute)}

\hypertarget{data.models.Diet.treatment_set}{}\begin{memberdesc}[Diet]{treatment\_set}\end{memberdesc}
\index{vendor (data.models.Diet attribute)}

\hypertarget{data.models.Diet.vendor}{}\begin{memberdesc}[Diet]{vendor}\end{memberdesc}
\end{classdesc}
\index{Environment (class in data.models)}

\hypertarget{data.models.Environment}{}\begin{classdesc}{Environment}{*args, **kwargs}
Environment(id, building, room, temperature, humidity, notes)
\index{Environment.DoesNotExist}

\hypertarget{data.models.Environment.DoesNotExist}{}\begin{excdesc}{DoesNotExist}~\index{args (data.models.Environment.DoesNotExist attribute)}

\hypertarget{data.models.Environment.DoesNotExist.args}{}\begin{memberdesc}{args}\end{memberdesc}
\index{message (data.models.Environment.DoesNotExist attribute)}

\hypertarget{data.models.Environment.DoesNotExist.message}{}\begin{memberdesc}{message}\end{memberdesc}
\end{excdesc}
\index{Environment.MultipleObjectsReturned}

\hypertarget{data.models.Environment.MultipleObjectsReturned}{}\begin{excdesc}{MultipleObjectsReturned}~\index{args (data.models.Environment.MultipleObjectsReturned attribute)}

\hypertarget{data.models.Environment.MultipleObjectsReturned.args}{}\begin{memberdesc}{args}\end{memberdesc}
\index{message (data.models.Environment.MultipleObjectsReturned attribute)}

\hypertarget{data.models.Environment.MultipleObjectsReturned.message}{}\begin{memberdesc}{message}\end{memberdesc}
\end{excdesc}
\index{contact (data.models.Environment attribute)}

\hypertarget{data.models.Environment.contact}{}\begin{memberdesc}[Environment]{contact}\end{memberdesc}
\index{delete() (data.models.Environment method)}

\hypertarget{data.models.Environment.delete}{}\begin{methoddesc}[Environment]{delete}{}\end{methoddesc}
\index{pk (data.models.Environment attribute)}

\hypertarget{data.models.Environment.pk}{}\begin{memberdesc}[Environment]{pk}\end{memberdesc}
\index{prepare\_database\_save() (data.models.Environment method)}

\hypertarget{data.models.Environment.prepare_database_save}{}\begin{methoddesc}[Environment]{prepare\_database\_save}{unused}\end{methoddesc}
\index{save() (data.models.Environment method)}

\hypertarget{data.models.Environment.save}{}\begin{methoddesc}[Environment]{save}{force\_insert=False, force\_update=False}
Saves the current instance. Override this in a subclass if you want to
control the saving process.

The `force\_insert' and `force\_update' parameters can be used to insist
that the ``save'' must be an SQL insert or update (or equivalent for
non-SQL backends), respectively. Normally, they should not be set.
\end{methoddesc}
\index{save\_base() (data.models.Environment method)}

\hypertarget{data.models.Environment.save_base}{}\begin{methoddesc}[Environment]{save\_base}{raw=False, cls=None, origin=None, force\_insert=False, force\_update=False}
Does the heavy-lifting involved in saving. Subclasses shouldn't need to
override this method. It's separate from save() in order to hide the
need for overrides of save() to pass around internal-only parameters
(`raw', `cls', and `origin').
\end{methoddesc}
\index{serializable\_value() (data.models.Environment method)}

\hypertarget{data.models.Environment.serializable_value}{}\begin{methoddesc}[Environment]{serializable\_value}{field\_name}
Returns the value of the field name for this instance. If the field is
a foreign key, returns the id value, instead of the object. If there's
no Field object with this name on the model, the model attribute's
value is returned directly.

Used to serialize a field's value (in the serializer, or form output,
for example). Normally, you would just access the attribute directly
and not use this method.
\end{methoddesc}
\index{treatment\_set (data.models.Environment attribute)}

\hypertarget{data.models.Environment.treatment_set}{}\begin{memberdesc}[Environment]{treatment\_set}\end{memberdesc}
\end{classdesc}
\index{Experiment (class in data.models)}

\hypertarget{data.models.Experiment}{}\begin{classdesc}{Experiment}{*args, **kwargs}
Experiment(id, date, notes, experimentID, feeding\_state, fasting\_time, injection, concentration, study\_id)
\index{Experiment.DoesNotExist}

\hypertarget{data.models.Experiment.DoesNotExist}{}\begin{excdesc}{DoesNotExist}~\index{args (data.models.Experiment.DoesNotExist attribute)}

\hypertarget{data.models.Experiment.DoesNotExist.args}{}\begin{memberdesc}{args}\end{memberdesc}
\index{message (data.models.Experiment.DoesNotExist attribute)}

\hypertarget{data.models.Experiment.DoesNotExist.message}{}\begin{memberdesc}{message}\end{memberdesc}
\end{excdesc}
\index{Experiment.MultipleObjectsReturned}

\hypertarget{data.models.Experiment.MultipleObjectsReturned}{}\begin{excdesc}{MultipleObjectsReturned}~\index{args (data.models.Experiment.MultipleObjectsReturned attribute)}

\hypertarget{data.models.Experiment.MultipleObjectsReturned.args}{}\begin{memberdesc}{args}\end{memberdesc}
\index{message (data.models.Experiment.MultipleObjectsReturned attribute)}

\hypertarget{data.models.Experiment.MultipleObjectsReturned.message}{}\begin{memberdesc}{message}\end{memberdesc}
\end{excdesc}
\index{animals (data.models.Experiment attribute)}

\hypertarget{data.models.Experiment.animals}{}\begin{memberdesc}[Experiment]{animals}\end{memberdesc}
\index{delete() (data.models.Experiment method)}

\hypertarget{data.models.Experiment.delete}{}\begin{methoddesc}[Experiment]{delete}{}\end{methoddesc}
\index{get\_feeding\_state\_display() (data.models.Experiment method)}

\hypertarget{data.models.Experiment.get_feeding_state_display}{}\begin{methoddesc}[Experiment]{get\_feeding\_state\_display}{*moreargs, **morekwargs}\end{methoddesc}
\index{get\_injection\_display() (data.models.Experiment method)}

\hypertarget{data.models.Experiment.get_injection_display}{}\begin{methoddesc}[Experiment]{get\_injection\_display}{*moreargs, **morekwargs}\end{methoddesc}
\index{get\_next\_by\_date() (data.models.Experiment method)}

\hypertarget{data.models.Experiment.get_next_by_date}{}\begin{methoddesc}[Experiment]{get\_next\_by\_date}{*moreargs, **morekwargs}\end{methoddesc}
\index{get\_previous\_by\_date() (data.models.Experiment method)}

\hypertarget{data.models.Experiment.get_previous_by_date}{}\begin{methoddesc}[Experiment]{get\_previous\_by\_date}{*moreargs, **morekwargs}\end{methoddesc}
\index{measurement\_set (data.models.Experiment attribute)}

\hypertarget{data.models.Experiment.measurement_set}{}\begin{memberdesc}[Experiment]{measurement\_set}\end{memberdesc}
\index{pk (data.models.Experiment attribute)}

\hypertarget{data.models.Experiment.pk}{}\begin{memberdesc}[Experiment]{pk}\end{memberdesc}
\index{prepare\_database\_save() (data.models.Experiment method)}

\hypertarget{data.models.Experiment.prepare_database_save}{}\begin{methoddesc}[Experiment]{prepare\_database\_save}{unused}\end{methoddesc}
\index{researchers (data.models.Experiment attribute)}

\hypertarget{data.models.Experiment.researchers}{}\begin{memberdesc}[Experiment]{researchers}\end{memberdesc}
\index{save() (data.models.Experiment method)}

\hypertarget{data.models.Experiment.save}{}\begin{methoddesc}[Experiment]{save}{force\_insert=False, force\_update=False}
Saves the current instance. Override this in a subclass if you want to
control the saving process.

The `force\_insert' and `force\_update' parameters can be used to insist
that the ``save'' must be an SQL insert or update (or equivalent for
non-SQL backends), respectively. Normally, they should not be set.
\end{methoddesc}
\index{save\_base() (data.models.Experiment method)}

\hypertarget{data.models.Experiment.save_base}{}\begin{methoddesc}[Experiment]{save\_base}{raw=False, cls=None, origin=None, force\_insert=False, force\_update=False}
Does the heavy-lifting involved in saving. Subclasses shouldn't need to
override this method. It's separate from save() in order to hide the
need for overrides of save() to pass around internal-only parameters
(`raw', `cls', and `origin').
\end{methoddesc}
\index{serializable\_value() (data.models.Experiment method)}

\hypertarget{data.models.Experiment.serializable_value}{}\begin{methoddesc}[Experiment]{serializable\_value}{field\_name}
Returns the value of the field name for this instance. If the field is
a foreign key, returns the id value, instead of the object. If there's
no Field object with this name on the model, the model attribute's
value is returned directly.

Used to serialize a field's value (in the serializer, or form output,
for example). Normally, you would just access the attribute directly
and not use this method.
\end{methoddesc}
\index{study (data.models.Experiment attribute)}

\hypertarget{data.models.Experiment.study}{}\begin{memberdesc}[Experiment]{study}\end{memberdesc}
\end{classdesc}
\index{Implantation (class in data.models)}

\hypertarget{data.models.Implantation}{}\begin{classdesc}{Implantation}{*args, **kwargs}
Implantation(id, implant, vendor\_id, product\_id, notes)
\index{Implantation.DoesNotExist}

\hypertarget{data.models.Implantation.DoesNotExist}{}\begin{excdesc}{DoesNotExist}~\index{args (data.models.Implantation.DoesNotExist attribute)}

\hypertarget{data.models.Implantation.DoesNotExist.args}{}\begin{memberdesc}{args}\end{memberdesc}
\index{message (data.models.Implantation.DoesNotExist attribute)}

\hypertarget{data.models.Implantation.DoesNotExist.message}{}\begin{memberdesc}{message}\end{memberdesc}
\end{excdesc}
\index{Implantation.MultipleObjectsReturned}

\hypertarget{data.models.Implantation.MultipleObjectsReturned}{}\begin{excdesc}{MultipleObjectsReturned}~\index{args (data.models.Implantation.MultipleObjectsReturned attribute)}

\hypertarget{data.models.Implantation.MultipleObjectsReturned.args}{}\begin{memberdesc}{args}\end{memberdesc}
\index{message (data.models.Implantation.MultipleObjectsReturned attribute)}

\hypertarget{data.models.Implantation.MultipleObjectsReturned.message}{}\begin{memberdesc}{message}\end{memberdesc}
\end{excdesc}
\index{delete() (data.models.Implantation method)}

\hypertarget{data.models.Implantation.delete}{}\begin{methoddesc}[Implantation]{delete}{}\end{methoddesc}
\index{pk (data.models.Implantation attribute)}

\hypertarget{data.models.Implantation.pk}{}\begin{memberdesc}[Implantation]{pk}\end{memberdesc}
\index{prepare\_database\_save() (data.models.Implantation method)}

\hypertarget{data.models.Implantation.prepare_database_save}{}\begin{methoddesc}[Implantation]{prepare\_database\_save}{unused}\end{methoddesc}
\index{save() (data.models.Implantation method)}

\hypertarget{data.models.Implantation.save}{}\begin{methoddesc}[Implantation]{save}{force\_insert=False, force\_update=False}
Saves the current instance. Override this in a subclass if you want to
control the saving process.

The `force\_insert' and `force\_update' parameters can be used to insist
that the ``save'' must be an SQL insert or update (or equivalent for
non-SQL backends), respectively. Normally, they should not be set.
\end{methoddesc}
\index{save\_base() (data.models.Implantation method)}

\hypertarget{data.models.Implantation.save_base}{}\begin{methoddesc}[Implantation]{save\_base}{raw=False, cls=None, origin=None, force\_insert=False, force\_update=False}
Does the heavy-lifting involved in saving. Subclasses shouldn't need to
override this method. It's separate from save() in order to hide the
need for overrides of save() to pass around internal-only parameters
(`raw', `cls', and `origin').
\end{methoddesc}
\index{serializable\_value() (data.models.Implantation method)}

\hypertarget{data.models.Implantation.serializable_value}{}\begin{methoddesc}[Implantation]{serializable\_value}{field\_name}
Returns the value of the field name for this instance. If the field is
a foreign key, returns the id value, instead of the object. If there's
no Field object with this name on the model, the model attribute's
value is returned directly.

Used to serialize a field's value (in the serializer, or form output,
for example). Normally, you would just access the attribute directly
and not use this method.
\end{methoddesc}
\index{surgeon (data.models.Implantation attribute)}

\hypertarget{data.models.Implantation.surgeon}{}\begin{memberdesc}[Implantation]{surgeon}\end{memberdesc}
\index{treatment\_set (data.models.Implantation attribute)}

\hypertarget{data.models.Implantation.treatment_set}{}\begin{memberdesc}[Implantation]{treatment\_set}\end{memberdesc}
\index{vendor (data.models.Implantation attribute)}

\hypertarget{data.models.Implantation.vendor}{}\begin{memberdesc}[Implantation]{vendor}\end{memberdesc}
\end{classdesc}
\index{Measurement (class in data.models)}

\hypertarget{data.models.Measurement}{}\begin{classdesc}{Measurement}{*args, **kwargs}
Measurement(id, animal\_id, experiment\_id, assay\_id, values)
\index{Measurement.DoesNotExist}

\hypertarget{data.models.Measurement.DoesNotExist}{}\begin{excdesc}{DoesNotExist}~\index{args (data.models.Measurement.DoesNotExist attribute)}

\hypertarget{data.models.Measurement.DoesNotExist.args}{}\begin{memberdesc}{args}\end{memberdesc}
\index{message (data.models.Measurement.DoesNotExist attribute)}

\hypertarget{data.models.Measurement.DoesNotExist.message}{}\begin{memberdesc}{message}\end{memberdesc}
\end{excdesc}
\index{Measurement.MultipleObjectsReturned}

\hypertarget{data.models.Measurement.MultipleObjectsReturned}{}\begin{excdesc}{MultipleObjectsReturned}~\index{args (data.models.Measurement.MultipleObjectsReturned attribute)}

\hypertarget{data.models.Measurement.MultipleObjectsReturned.args}{}\begin{memberdesc}{args}\end{memberdesc}
\index{message (data.models.Measurement.MultipleObjectsReturned attribute)}

\hypertarget{data.models.Measurement.MultipleObjectsReturned.message}{}\begin{memberdesc}{message}\end{memberdesc}
\end{excdesc}
\index{animal (data.models.Measurement attribute)}

\hypertarget{data.models.Measurement.animal}{}\begin{memberdesc}[Measurement]{animal}\end{memberdesc}
\index{assay (data.models.Measurement attribute)}

\hypertarget{data.models.Measurement.assay}{}\begin{memberdesc}[Measurement]{assay}\end{memberdesc}
\index{delete() (data.models.Measurement method)}

\hypertarget{data.models.Measurement.delete}{}\begin{methoddesc}[Measurement]{delete}{}\end{methoddesc}
\index{experiment (data.models.Measurement attribute)}

\hypertarget{data.models.Measurement.experiment}{}\begin{memberdesc}[Measurement]{experiment}\end{memberdesc}
\index{pk (data.models.Measurement attribute)}

\hypertarget{data.models.Measurement.pk}{}\begin{memberdesc}[Measurement]{pk}\end{memberdesc}
\index{prepare\_database\_save() (data.models.Measurement method)}

\hypertarget{data.models.Measurement.prepare_database_save}{}\begin{methoddesc}[Measurement]{prepare\_database\_save}{unused}\end{methoddesc}
\index{save() (data.models.Measurement method)}

\hypertarget{data.models.Measurement.save}{}\begin{methoddesc}[Measurement]{save}{force\_insert=False, force\_update=False}
Saves the current instance. Override this in a subclass if you want to
control the saving process.

The `force\_insert' and `force\_update' parameters can be used to insist
that the ``save'' must be an SQL insert or update (or equivalent for
non-SQL backends), respectively. Normally, they should not be set.
\end{methoddesc}
\index{save\_base() (data.models.Measurement method)}

\hypertarget{data.models.Measurement.save_base}{}\begin{methoddesc}[Measurement]{save\_base}{raw=False, cls=None, origin=None, force\_insert=False, force\_update=False}
Does the heavy-lifting involved in saving. Subclasses shouldn't need to
override this method. It's separate from save() in order to hide the
need for overrides of save() to pass around internal-only parameters
(`raw', `cls', and `origin').
\end{methoddesc}
\index{serializable\_value() (data.models.Measurement method)}

\hypertarget{data.models.Measurement.serializable_value}{}\begin{methoddesc}[Measurement]{serializable\_value}{field\_name}
Returns the value of the field name for this instance. If the field is
a foreign key, returns the id value, instead of the object. If there's
no Field object with this name on the model, the model attribute's
value is returned directly.

Used to serialize a field's value (in the serializer, or form output,
for example). Normally, you would just access the attribute directly
and not use this method.
\end{methoddesc}
\end{classdesc}
\index{Pharmaceutical (class in data.models)}

\hypertarget{data.models.Pharmaceutical}{}\begin{classdesc}{Pharmaceutical}{*args, **kwargs}
Pharmaceutical(id, drug, dose, recurrance, mode, vendor\_id, notes)
\index{Pharmaceutical.DoesNotExist}

\hypertarget{data.models.Pharmaceutical.DoesNotExist}{}\begin{excdesc}{DoesNotExist}~\index{args (data.models.Pharmaceutical.DoesNotExist attribute)}

\hypertarget{data.models.Pharmaceutical.DoesNotExist.args}{}\begin{memberdesc}{args}\end{memberdesc}
\index{message (data.models.Pharmaceutical.DoesNotExist attribute)}

\hypertarget{data.models.Pharmaceutical.DoesNotExist.message}{}\begin{memberdesc}{message}\end{memberdesc}
\end{excdesc}
\index{Pharmaceutical.MultipleObjectsReturned}

\hypertarget{data.models.Pharmaceutical.MultipleObjectsReturned}{}\begin{excdesc}{MultipleObjectsReturned}~\index{args (data.models.Pharmaceutical.MultipleObjectsReturned attribute)}

\hypertarget{data.models.Pharmaceutical.MultipleObjectsReturned.args}{}\begin{memberdesc}{args}\end{memberdesc}
\index{message (data.models.Pharmaceutical.MultipleObjectsReturned attribute)}

\hypertarget{data.models.Pharmaceutical.MultipleObjectsReturned.message}{}\begin{memberdesc}{message}\end{memberdesc}
\end{excdesc}
\index{delete() (data.models.Pharmaceutical method)}

\hypertarget{data.models.Pharmaceutical.delete}{}\begin{methoddesc}[Pharmaceutical]{delete}{}\end{methoddesc}
\index{get\_mode\_display() (data.models.Pharmaceutical method)}

\hypertarget{data.models.Pharmaceutical.get_mode_display}{}\begin{methoddesc}[Pharmaceutical]{get\_mode\_display}{*moreargs, **morekwargs}\end{methoddesc}
\index{pk (data.models.Pharmaceutical attribute)}

\hypertarget{data.models.Pharmaceutical.pk}{}\begin{memberdesc}[Pharmaceutical]{pk}\end{memberdesc}
\index{prepare\_database\_save() (data.models.Pharmaceutical method)}

\hypertarget{data.models.Pharmaceutical.prepare_database_save}{}\begin{methoddesc}[Pharmaceutical]{prepare\_database\_save}{unused}\end{methoddesc}
\index{save() (data.models.Pharmaceutical method)}

\hypertarget{data.models.Pharmaceutical.save}{}\begin{methoddesc}[Pharmaceutical]{save}{force\_insert=False, force\_update=False}
Saves the current instance. Override this in a subclass if you want to
control the saving process.

The `force\_insert' and `force\_update' parameters can be used to insist
that the ``save'' must be an SQL insert or update (or equivalent for
non-SQL backends), respectively. Normally, they should not be set.
\end{methoddesc}
\index{save\_base() (data.models.Pharmaceutical method)}

\hypertarget{data.models.Pharmaceutical.save_base}{}\begin{methoddesc}[Pharmaceutical]{save\_base}{raw=False, cls=None, origin=None, force\_insert=False, force\_update=False}
Does the heavy-lifting involved in saving. Subclasses shouldn't need to
override this method. It's separate from save() in order to hide the
need for overrides of save() to pass around internal-only parameters
(`raw', `cls', and `origin').
\end{methoddesc}
\index{serializable\_value() (data.models.Pharmaceutical method)}

\hypertarget{data.models.Pharmaceutical.serializable_value}{}\begin{methoddesc}[Pharmaceutical]{serializable\_value}{field\_name}
Returns the value of the field name for this instance. If the field is
a foreign key, returns the id value, instead of the object. If there's
no Field object with this name on the model, the model attribute's
value is returned directly.

Used to serialize a field's value (in the serializer, or form output,
for example). Normally, you would just access the attribute directly
and not use this method.
\end{methoddesc}
\index{treatment\_set (data.models.Pharmaceutical attribute)}

\hypertarget{data.models.Pharmaceutical.treatment_set}{}\begin{memberdesc}[Pharmaceutical]{treatment\_set}\end{memberdesc}
\index{vendor (data.models.Pharmaceutical attribute)}

\hypertarget{data.models.Pharmaceutical.vendor}{}\begin{memberdesc}[Pharmaceutical]{vendor}\end{memberdesc}
\end{classdesc}
\index{Researcher (class in data.models)}

\hypertarget{data.models.Researcher}{}\begin{classdesc}{Researcher}{*args, **kwargs}
Researcher(id, first\_name, last\_name, name\_slug, email, active)
\index{Researcher.DoesNotExist}

\hypertarget{data.models.Researcher.DoesNotExist}{}\begin{excdesc}{DoesNotExist}~\index{args (data.models.Researcher.DoesNotExist attribute)}

\hypertarget{data.models.Researcher.DoesNotExist.args}{}\begin{memberdesc}{args}\end{memberdesc}
\index{message (data.models.Researcher.DoesNotExist attribute)}

\hypertarget{data.models.Researcher.DoesNotExist.message}{}\begin{memberdesc}{message}\end{memberdesc}
\end{excdesc}
\index{Researcher.MultipleObjectsReturned}

\hypertarget{data.models.Researcher.MultipleObjectsReturned}{}\begin{excdesc}{MultipleObjectsReturned}~\index{args (data.models.Researcher.MultipleObjectsReturned attribute)}

\hypertarget{data.models.Researcher.MultipleObjectsReturned.args}{}\begin{memberdesc}{args}\end{memberdesc}
\index{message (data.models.Researcher.MultipleObjectsReturned attribute)}

\hypertarget{data.models.Researcher.MultipleObjectsReturned.message}{}\begin{memberdesc}{message}\end{memberdesc}
\end{excdesc}
\index{delete() (data.models.Researcher method)}

\hypertarget{data.models.Researcher.delete}{}\begin{methoddesc}[Researcher]{delete}{}\end{methoddesc}
\index{environment\_set (data.models.Researcher attribute)}

\hypertarget{data.models.Researcher.environment_set}{}\begin{memberdesc}[Researcher]{environment\_set}\end{memberdesc}
\index{experiment\_set (data.models.Researcher attribute)}

\hypertarget{data.models.Researcher.experiment_set}{}\begin{memberdesc}[Researcher]{experiment\_set}\end{memberdesc}
\index{get\_absolute\_url() (data.models.Researcher method)}

\hypertarget{data.models.Researcher.get_absolute_url}{}\begin{methoddesc}[Researcher]{get\_absolute\_url}{*moreargs, **morekwargs}\end{methoddesc}
\index{implantation\_set (data.models.Researcher attribute)}

\hypertarget{data.models.Researcher.implantation_set}{}\begin{memberdesc}[Researcher]{implantation\_set}\end{memberdesc}
\index{pk (data.models.Researcher attribute)}

\hypertarget{data.models.Researcher.pk}{}\begin{memberdesc}[Researcher]{pk}\end{memberdesc}
\index{prepare\_database\_save() (data.models.Researcher method)}

\hypertarget{data.models.Researcher.prepare_database_save}{}\begin{methoddesc}[Researcher]{prepare\_database\_save}{unused}\end{methoddesc}
\index{save() (data.models.Researcher method)}

\hypertarget{data.models.Researcher.save}{}\begin{methoddesc}[Researcher]{save}{force\_insert=False, force\_update=False}
Saves the current instance. Override this in a subclass if you want to
control the saving process.

The `force\_insert' and `force\_update' parameters can be used to insist
that the ``save'' must be an SQL insert or update (or equivalent for
non-SQL backends), respectively. Normally, they should not be set.
\end{methoddesc}
\index{save\_base() (data.models.Researcher method)}

\hypertarget{data.models.Researcher.save_base}{}\begin{methoddesc}[Researcher]{save\_base}{raw=False, cls=None, origin=None, force\_insert=False, force\_update=False}
Does the heavy-lifting involved in saving. Subclasses shouldn't need to
override this method. It's separate from save() in order to hide the
need for overrides of save() to pass around internal-only parameters
(`raw', `cls', and `origin').
\end{methoddesc}
\index{serializable\_value() (data.models.Researcher method)}

\hypertarget{data.models.Researcher.serializable_value}{}\begin{methoddesc}[Researcher]{serializable\_value}{field\_name}
Returns the value of the field name for this instance. If the field is
a foreign key, returns the id value, instead of the object. If there's
no Field object with this name on the model, the model attribute's
value is returned directly.

Used to serialize a field's value (in the serializer, or form output,
for example). Normally, you would just access the attribute directly
and not use this method.
\end{methoddesc}
\index{transplantation\_set (data.models.Researcher attribute)}

\hypertarget{data.models.Researcher.transplantation_set}{}\begin{memberdesc}[Researcher]{transplantation\_set}\end{memberdesc}
\index{treatment\_set (data.models.Researcher attribute)}

\hypertarget{data.models.Researcher.treatment_set}{}\begin{memberdesc}[Researcher]{treatment\_set}\end{memberdesc}
\end{classdesc}
\index{Study (class in data.models)}

\hypertarget{data.models.Study}{}\begin{classdesc}{Study}{*args, **kwargs}
Study(id, description, start\_date, stop\_date, notes)
\index{Study.DoesNotExist}

\hypertarget{data.models.Study.DoesNotExist}{}\begin{excdesc}{DoesNotExist}~\index{args (data.models.Study.DoesNotExist attribute)}

\hypertarget{data.models.Study.DoesNotExist.args}{}\begin{memberdesc}{args}\end{memberdesc}
\index{message (data.models.Study.DoesNotExist attribute)}

\hypertarget{data.models.Study.DoesNotExist.message}{}\begin{memberdesc}{message}\end{memberdesc}
\end{excdesc}
\index{Study.MultipleObjectsReturned}

\hypertarget{data.models.Study.MultipleObjectsReturned}{}\begin{excdesc}{MultipleObjectsReturned}~\index{args (data.models.Study.MultipleObjectsReturned attribute)}

\hypertarget{data.models.Study.MultipleObjectsReturned.args}{}\begin{memberdesc}{args}\end{memberdesc}
\index{message (data.models.Study.MultipleObjectsReturned attribute)}

\hypertarget{data.models.Study.MultipleObjectsReturned.message}{}\begin{memberdesc}{message}\end{memberdesc}
\end{excdesc}
\index{delete() (data.models.Study method)}

\hypertarget{data.models.Study.delete}{}\begin{methoddesc}[Study]{delete}{}\end{methoddesc}
\index{experiment\_set (data.models.Study attribute)}

\hypertarget{data.models.Study.experiment_set}{}\begin{memberdesc}[Study]{experiment\_set}\end{memberdesc}
\index{pk (data.models.Study attribute)}

\hypertarget{data.models.Study.pk}{}\begin{memberdesc}[Study]{pk}\end{memberdesc}
\index{prepare\_database\_save() (data.models.Study method)}

\hypertarget{data.models.Study.prepare_database_save}{}\begin{methoddesc}[Study]{prepare\_database\_save}{unused}\end{methoddesc}
\index{save() (data.models.Study method)}

\hypertarget{data.models.Study.save}{}\begin{methoddesc}[Study]{save}{force\_insert=False, force\_update=False}
Saves the current instance. Override this in a subclass if you want to
control the saving process.

The `force\_insert' and `force\_update' parameters can be used to insist
that the ``save'' must be an SQL insert or update (or equivalent for
non-SQL backends), respectively. Normally, they should not be set.
\end{methoddesc}
\index{save\_base() (data.models.Study method)}

\hypertarget{data.models.Study.save_base}{}\begin{methoddesc}[Study]{save\_base}{raw=False, cls=None, origin=None, force\_insert=False, force\_update=False}
Does the heavy-lifting involved in saving. Subclasses shouldn't need to
override this method. It's separate from save() in order to hide the
need for overrides of save() to pass around internal-only parameters
(`raw', `cls', and `origin').
\end{methoddesc}
\index{serializable\_value() (data.models.Study method)}

\hypertarget{data.models.Study.serializable_value}{}\begin{methoddesc}[Study]{serializable\_value}{field\_name}
Returns the value of the field name for this instance. If the field is
a foreign key, returns the id value, instead of the object. If there's
no Field object with this name on the model, the model attribute's
value is returned directly.

Used to serialize a field's value (in the serializer, or form output,
for example). Normally, you would just access the attribute directly
and not use this method.
\end{methoddesc}
\index{strain (data.models.Study attribute)}

\hypertarget{data.models.Study.strain}{}\begin{memberdesc}[Study]{strain}\end{memberdesc}
\index{treatment\_set (data.models.Study attribute)}

\hypertarget{data.models.Study.treatment_set}{}\begin{memberdesc}[Study]{treatment\_set}\end{memberdesc}
\end{classdesc}
\index{Transplantation (class in data.models)}

\hypertarget{data.models.Transplantation}{}\begin{classdesc}{Transplantation}{*args, **kwargs}
Transplantation(id, tissue, transplant\_date, notes)
\index{Transplantation.DoesNotExist}

\hypertarget{data.models.Transplantation.DoesNotExist}{}\begin{excdesc}{DoesNotExist}~\index{args (data.models.Transplantation.DoesNotExist attribute)}

\hypertarget{data.models.Transplantation.DoesNotExist.args}{}\begin{memberdesc}{args}\end{memberdesc}
\index{message (data.models.Transplantation.DoesNotExist attribute)}

\hypertarget{data.models.Transplantation.DoesNotExist.message}{}\begin{memberdesc}{message}\end{memberdesc}
\end{excdesc}
\index{Transplantation.MultipleObjectsReturned}

\hypertarget{data.models.Transplantation.MultipleObjectsReturned}{}\begin{excdesc}{MultipleObjectsReturned}~\index{args (data.models.Transplantation.MultipleObjectsReturned attribute)}

\hypertarget{data.models.Transplantation.MultipleObjectsReturned.args}{}\begin{memberdesc}{args}\end{memberdesc}
\index{message (data.models.Transplantation.MultipleObjectsReturned attribute)}

\hypertarget{data.models.Transplantation.MultipleObjectsReturned.message}{}\begin{memberdesc}{message}\end{memberdesc}
\end{excdesc}
\index{delete() (data.models.Transplantation method)}

\hypertarget{data.models.Transplantation.delete}{}\begin{methoddesc}[Transplantation]{delete}{}\end{methoddesc}
\index{donor (data.models.Transplantation attribute)}

\hypertarget{data.models.Transplantation.donor}{}\begin{memberdesc}[Transplantation]{donor}\end{memberdesc}
\index{get\_next\_by\_transplant\_date() (data.models.Transplantation method)}

\hypertarget{data.models.Transplantation.get_next_by_transplant_date}{}\begin{methoddesc}[Transplantation]{get\_next\_by\_transplant\_date}{*moreargs, **morekwargs}\end{methoddesc}
\index{get\_previous\_by\_transplant\_date() (data.models.Transplantation method)}

\hypertarget{data.models.Transplantation.get_previous_by_transplant_date}{}\begin{methoddesc}[Transplantation]{get\_previous\_by\_transplant\_date}{*moreargs, **morekwargs}\end{methoddesc}
\index{pk (data.models.Transplantation attribute)}

\hypertarget{data.models.Transplantation.pk}{}\begin{memberdesc}[Transplantation]{pk}\end{memberdesc}
\index{prepare\_database\_save() (data.models.Transplantation method)}

\hypertarget{data.models.Transplantation.prepare_database_save}{}\begin{methoddesc}[Transplantation]{prepare\_database\_save}{unused}\end{methoddesc}
\index{save() (data.models.Transplantation method)}

\hypertarget{data.models.Transplantation.save}{}\begin{methoddesc}[Transplantation]{save}{force\_insert=False, force\_update=False}
Saves the current instance. Override this in a subclass if you want to
control the saving process.

The `force\_insert' and `force\_update' parameters can be used to insist
that the ``save'' must be an SQL insert or update (or equivalent for
non-SQL backends), respectively. Normally, they should not be set.
\end{methoddesc}
\index{save\_base() (data.models.Transplantation method)}

\hypertarget{data.models.Transplantation.save_base}{}\begin{methoddesc}[Transplantation]{save\_base}{raw=False, cls=None, origin=None, force\_insert=False, force\_update=False}
Does the heavy-lifting involved in saving. Subclasses shouldn't need to
override this method. It's separate from save() in order to hide the
need for overrides of save() to pass around internal-only parameters
(`raw', `cls', and `origin').
\end{methoddesc}
\index{serializable\_value() (data.models.Transplantation method)}

\hypertarget{data.models.Transplantation.serializable_value}{}\begin{methoddesc}[Transplantation]{serializable\_value}{field\_name}
Returns the value of the field name for this instance. If the field is
a foreign key, returns the id value, instead of the object. If there's
no Field object with this name on the model, the model attribute's
value is returned directly.

Used to serialize a field's value (in the serializer, or form output,
for example). Normally, you would just access the attribute directly
and not use this method.
\end{methoddesc}
\index{surgeon (data.models.Transplantation attribute)}

\hypertarget{data.models.Transplantation.surgeon}{}\begin{memberdesc}[Transplantation]{surgeon}\end{memberdesc}
\index{treatment\_set (data.models.Transplantation attribute)}

\hypertarget{data.models.Transplantation.treatment_set}{}\begin{memberdesc}[Transplantation]{treatment\_set}\end{memberdesc}
\end{classdesc}
\index{Treatment (class in data.models)}

\hypertarget{data.models.Treatment}{}\begin{classdesc}{Treatment}{*args, **kwargs}
Treatment(id, treatment, study\_id, diet\_id, environment\_id, transplantation\_id, notes)
\index{Treatment.DoesNotExist}

\hypertarget{data.models.Treatment.DoesNotExist}{}\begin{excdesc}{DoesNotExist}~\index{args (data.models.Treatment.DoesNotExist attribute)}

\hypertarget{data.models.Treatment.DoesNotExist.args}{}\begin{memberdesc}{args}\end{memberdesc}
\index{message (data.models.Treatment.DoesNotExist attribute)}

\hypertarget{data.models.Treatment.DoesNotExist.message}{}\begin{memberdesc}{message}\end{memberdesc}
\end{excdesc}
\index{Treatment.MultipleObjectsReturned}

\hypertarget{data.models.Treatment.MultipleObjectsReturned}{}\begin{excdesc}{MultipleObjectsReturned}~\index{args (data.models.Treatment.MultipleObjectsReturned attribute)}

\hypertarget{data.models.Treatment.MultipleObjectsReturned.args}{}\begin{memberdesc}{args}\end{memberdesc}
\index{message (data.models.Treatment.MultipleObjectsReturned attribute)}

\hypertarget{data.models.Treatment.MultipleObjectsReturned.message}{}\begin{memberdesc}{message}\end{memberdesc}
\end{excdesc}
\index{animals (data.models.Treatment attribute)}

\hypertarget{data.models.Treatment.animals}{}\begin{memberdesc}[Treatment]{animals}\end{memberdesc}
\index{delete() (data.models.Treatment method)}

\hypertarget{data.models.Treatment.delete}{}\begin{methoddesc}[Treatment]{delete}{}\end{methoddesc}
\index{diet (data.models.Treatment attribute)}

\hypertarget{data.models.Treatment.diet}{}\begin{memberdesc}[Treatment]{diet}\end{memberdesc}
\index{environment (data.models.Treatment attribute)}

\hypertarget{data.models.Treatment.environment}{}\begin{memberdesc}[Treatment]{environment}\end{memberdesc}
\index{implantation (data.models.Treatment attribute)}

\hypertarget{data.models.Treatment.implantation}{}\begin{memberdesc}[Treatment]{implantation}\end{memberdesc}
\index{pharmaceutical (data.models.Treatment attribute)}

\hypertarget{data.models.Treatment.pharmaceutical}{}\begin{memberdesc}[Treatment]{pharmaceutical}\end{memberdesc}
\index{pk (data.models.Treatment attribute)}

\hypertarget{data.models.Treatment.pk}{}\begin{memberdesc}[Treatment]{pk}\end{memberdesc}
\index{prepare\_database\_save() (data.models.Treatment method)}

\hypertarget{data.models.Treatment.prepare_database_save}{}\begin{methoddesc}[Treatment]{prepare\_database\_save}{unused}\end{methoddesc}
\index{researchers (data.models.Treatment attribute)}

\hypertarget{data.models.Treatment.researchers}{}\begin{memberdesc}[Treatment]{researchers}\end{memberdesc}
\index{save() (data.models.Treatment method)}

\hypertarget{data.models.Treatment.save}{}\begin{methoddesc}[Treatment]{save}{force\_insert=False, force\_update=False}
Saves the current instance. Override this in a subclass if you want to
control the saving process.

The `force\_insert' and `force\_update' parameters can be used to insist
that the ``save'' must be an SQL insert or update (or equivalent for
non-SQL backends), respectively. Normally, they should not be set.
\end{methoddesc}
\index{save\_base() (data.models.Treatment method)}

\hypertarget{data.models.Treatment.save_base}{}\begin{methoddesc}[Treatment]{save\_base}{raw=False, cls=None, origin=None, force\_insert=False, force\_update=False}
Does the heavy-lifting involved in saving. Subclasses shouldn't need to
override this method. It's separate from save() in order to hide the
need for overrides of save() to pass around internal-only parameters
(`raw', `cls', and `origin').
\end{methoddesc}
\index{serializable\_value() (data.models.Treatment method)}

\hypertarget{data.models.Treatment.serializable_value}{}\begin{methoddesc}[Treatment]{serializable\_value}{field\_name}
Returns the value of the field name for this instance. If the field is
a foreign key, returns the id value, instead of the object. If there's
no Field object with this name on the model, the model attribute's
value is returned directly.

Used to serialize a field's value (in the serializer, or form output,
for example). Normally, you would just access the attribute directly
and not use this method.
\end{methoddesc}
\index{study (data.models.Treatment attribute)}

\hypertarget{data.models.Treatment.study}{}\begin{memberdesc}[Treatment]{study}\end{memberdesc}
\index{transplantation (data.models.Treatment attribute)}

\hypertarget{data.models.Treatment.transplantation}{}\begin{memberdesc}[Treatment]{transplantation}\end{memberdesc}
\end{classdesc}
\index{Vendor (class in data.models)}

\hypertarget{data.models.Vendor}{}\begin{classdesc}{Vendor}{*args, **kwargs}
Vendor(id, vendor, website, email, ordering, notes)
\index{Vendor.DoesNotExist}

\hypertarget{data.models.Vendor.DoesNotExist}{}\begin{excdesc}{DoesNotExist}~\index{args (data.models.Vendor.DoesNotExist attribute)}

\hypertarget{data.models.Vendor.DoesNotExist.args}{}\begin{memberdesc}{args}\end{memberdesc}
\index{message (data.models.Vendor.DoesNotExist attribute)}

\hypertarget{data.models.Vendor.DoesNotExist.message}{}\begin{memberdesc}{message}\end{memberdesc}
\end{excdesc}
\index{Vendor.MultipleObjectsReturned}

\hypertarget{data.models.Vendor.MultipleObjectsReturned}{}\begin{excdesc}{MultipleObjectsReturned}~\index{args (data.models.Vendor.MultipleObjectsReturned attribute)}

\hypertarget{data.models.Vendor.MultipleObjectsReturned.args}{}\begin{memberdesc}{args}\end{memberdesc}
\index{message (data.models.Vendor.MultipleObjectsReturned attribute)}

\hypertarget{data.models.Vendor.MultipleObjectsReturned.message}{}\begin{memberdesc}{message}\end{memberdesc}
\end{excdesc}
\index{delete() (data.models.Vendor method)}

\hypertarget{data.models.Vendor.delete}{}\begin{methoddesc}[Vendor]{delete}{}\end{methoddesc}
\index{diet\_set (data.models.Vendor attribute)}

\hypertarget{data.models.Vendor.diet_set}{}\begin{memberdesc}[Vendor]{diet\_set}\end{memberdesc}
\index{get\_ordering\_display() (data.models.Vendor method)}

\hypertarget{data.models.Vendor.get_ordering_display}{}\begin{methoddesc}[Vendor]{get\_ordering\_display}{*moreargs, **morekwargs}\end{methoddesc}
\index{implantation\_set (data.models.Vendor attribute)}

\hypertarget{data.models.Vendor.implantation_set}{}\begin{memberdesc}[Vendor]{implantation\_set}\end{memberdesc}
\index{pharmaceutical\_set (data.models.Vendor attribute)}

\hypertarget{data.models.Vendor.pharmaceutical_set}{}\begin{memberdesc}[Vendor]{pharmaceutical\_set}\end{memberdesc}
\index{pk (data.models.Vendor attribute)}

\hypertarget{data.models.Vendor.pk}{}\begin{memberdesc}[Vendor]{pk}\end{memberdesc}
\index{prepare\_database\_save() (data.models.Vendor method)}

\hypertarget{data.models.Vendor.prepare_database_save}{}\begin{methoddesc}[Vendor]{prepare\_database\_save}{unused}\end{methoddesc}
\index{save() (data.models.Vendor method)}

\hypertarget{data.models.Vendor.save}{}\begin{methoddesc}[Vendor]{save}{force\_insert=False, force\_update=False}
Saves the current instance. Override this in a subclass if you want to
control the saving process.

The `force\_insert' and `force\_update' parameters can be used to insist
that the ``save'' must be an SQL insert or update (or equivalent for
non-SQL backends), respectively. Normally, they should not be set.
\end{methoddesc}
\index{save\_base() (data.models.Vendor method)}

\hypertarget{data.models.Vendor.save_base}{}\begin{methoddesc}[Vendor]{save\_base}{raw=False, cls=None, origin=None, force\_insert=False, force\_update=False}
Does the heavy-lifting involved in saving. Subclasses shouldn't need to
override this method. It's separate from save() in order to hide the
need for overrides of save() to pass around internal-only parameters
(`raw', `cls', and `origin').
\end{methoddesc}
\index{serializable\_value() (data.models.Vendor method)}

\hypertarget{data.models.Vendor.serializable_value}{}\begin{methoddesc}[Vendor]{serializable\_value}{field\_name}
Returns the value of the field name for this instance. If the field is
a foreign key, returns the id value, instead of the object. If there's
no Field object with this name on the model, the model attribute's
value is returned directly.

Used to serialize a field's value (in the serializer, or form output,
for example). Normally, you would just access the attribute directly
and not use this method.
\end{methoddesc}
\end{classdesc}


\subsection{Forms}
\index{data.forms (module)}
\hypertarget{module-data.forms}{}
\declaremodule[data.forms]{}{data.forms}
\modulesynopsis{}\index{ExperimentForm (class in data.forms)}

\hypertarget{data.forms.ExperimentForm}{}\begin{classdesc}{ExperimentForm}{data=None, files=None, auto\_id='id\_\%s', prefix=None, initial=None, error\_class=\textless{}class 'django.forms.util.ErrorList'\textgreater{}, label\_suffix=':', empty\_permitted=False, instance=None}~\index{ExperimentForm.Meta (class in data.forms)}

\hypertarget{data.forms.ExperimentForm.Meta}{}\begin{classdesc}{Meta}{}~\index{model (data.forms.ExperimentForm.Meta attribute)}

\hypertarget{data.forms.ExperimentForm.Meta.model}{}\begin{memberdesc}{model}
alias of \code{Experiment}
\end{memberdesc}
\end{classdesc}
\index{add\_initial\_prefix() (data.forms.ExperimentForm method)}

\hypertarget{data.forms.ExperimentForm.add_initial_prefix}{}\begin{methoddesc}[ExperimentForm]{add\_initial\_prefix}{field\_name}
Add a `initial' prefix for checking dynamic initial values
\end{methoddesc}
\index{add\_prefix() (data.forms.ExperimentForm method)}

\hypertarget{data.forms.ExperimentForm.add_prefix}{}\begin{methoddesc}[ExperimentForm]{add\_prefix}{field\_name}
Returns the field name with a prefix appended, if this Form has a
prefix set.

Subclasses may wish to override.
\end{methoddesc}
\index{as\_p() (data.forms.ExperimentForm method)}

\hypertarget{data.forms.ExperimentForm.as_p}{}\begin{methoddesc}[ExperimentForm]{as\_p}{}
Returns this form rendered as HTML \textless{}p\textgreater{}s.
\end{methoddesc}
\index{as\_table() (data.forms.ExperimentForm method)}

\hypertarget{data.forms.ExperimentForm.as_table}{}\begin{methoddesc}[ExperimentForm]{as\_table}{}
Returns this form rendered as HTML \textless{}tr\textgreater{}s -- excluding the \textless{}table\textgreater{}\textless{}/table\textgreater{}.
\end{methoddesc}
\index{as\_ul() (data.forms.ExperimentForm method)}

\hypertarget{data.forms.ExperimentForm.as_ul}{}\begin{methoddesc}[ExperimentForm]{as\_ul}{}
Returns this form rendered as HTML \textless{}li\textgreater{}s -- excluding the \textless{}ul\textgreater{}\textless{}/ul\textgreater{}.
\end{methoddesc}
\index{changed\_data (data.forms.ExperimentForm attribute)}

\hypertarget{data.forms.ExperimentForm.changed_data}{}\begin{memberdesc}[ExperimentForm]{changed\_data}\end{memberdesc}
\index{clean() (data.forms.ExperimentForm method)}

\hypertarget{data.forms.ExperimentForm.clean}{}\begin{methoddesc}[ExperimentForm]{clean}{}\end{methoddesc}
\index{date\_error\_message() (data.forms.ExperimentForm method)}

\hypertarget{data.forms.ExperimentForm.date_error_message}{}\begin{methoddesc}[ExperimentForm]{date\_error\_message}{lookup\_type, field, unique\_for}\end{methoddesc}
\index{errors (data.forms.ExperimentForm attribute)}

\hypertarget{data.forms.ExperimentForm.errors}{}\begin{memberdesc}[ExperimentForm]{errors}
Returns an ErrorDict for the data provided for the form
\end{memberdesc}
\index{full\_clean() (data.forms.ExperimentForm method)}

\hypertarget{data.forms.ExperimentForm.full_clean}{}\begin{methoddesc}[ExperimentForm]{full\_clean}{}
Cleans all of self.data and populates self.\_errors and
self.cleaned\_data.
\end{methoddesc}
\index{has\_changed() (data.forms.ExperimentForm method)}

\hypertarget{data.forms.ExperimentForm.has_changed}{}\begin{methoddesc}[ExperimentForm]{has\_changed}{}
Returns True if data differs from initial.
\end{methoddesc}
\index{hidden\_fields() (data.forms.ExperimentForm method)}

\hypertarget{data.forms.ExperimentForm.hidden_fields}{}\begin{methoddesc}[ExperimentForm]{hidden\_fields}{}
Returns a list of all the BoundField objects that are hidden fields.
Useful for manual form layout in templates.
\end{methoddesc}
\index{is\_multipart() (data.forms.ExperimentForm method)}

\hypertarget{data.forms.ExperimentForm.is_multipart}{}\begin{methoddesc}[ExperimentForm]{is\_multipart}{}
Returns True if the form needs to be multipart-encrypted, i.e. it has
FileInput. Otherwise, False.
\end{methoddesc}
\index{is\_valid() (data.forms.ExperimentForm method)}

\hypertarget{data.forms.ExperimentForm.is_valid}{}\begin{methoddesc}[ExperimentForm]{is\_valid}{}
Returns True if the form has no errors. Otherwise, False. If errors are
being ignored, returns False.
\end{methoddesc}
\index{media (data.forms.ExperimentForm attribute)}

\hypertarget{data.forms.ExperimentForm.media}{}\begin{memberdesc}[ExperimentForm]{media}\end{memberdesc}
\index{non\_field\_errors() (data.forms.ExperimentForm method)}

\hypertarget{data.forms.ExperimentForm.non_field_errors}{}\begin{methoddesc}[ExperimentForm]{non\_field\_errors}{}
Returns an ErrorList of errors that aren't associated with a particular
field -- i.e., from Form.clean(). Returns an empty ErrorList if there
are none.
\end{methoddesc}
\index{save() (data.forms.ExperimentForm method)}

\hypertarget{data.forms.ExperimentForm.save}{}\begin{methoddesc}[ExperimentForm]{save}{commit=True}
Saves this \code{form}`s cleaned\_data into model instance
\code{self.instance}.

If commit=True, then the changes to \code{instance} will be saved to the
database. Returns \code{instance}.
\end{methoddesc}
\index{unique\_error\_message() (data.forms.ExperimentForm method)}

\hypertarget{data.forms.ExperimentForm.unique_error_message}{}\begin{methoddesc}[ExperimentForm]{unique\_error\_message}{unique\_check}\end{methoddesc}
\index{validate\_unique() (data.forms.ExperimentForm method)}

\hypertarget{data.forms.ExperimentForm.validate_unique}{}\begin{methoddesc}[ExperimentForm]{validate\_unique}{}\end{methoddesc}
\index{visible\_fields() (data.forms.ExperimentForm method)}

\hypertarget{data.forms.ExperimentForm.visible_fields}{}\begin{methoddesc}[ExperimentForm]{visible\_fields}{}
Returns a list of BoundField objects that aren't hidden fields.
The opposite of the hidden\_fields() method.
\end{methoddesc}
\end{classdesc}
\index{MeasurementForm (class in data.forms)}

\hypertarget{data.forms.MeasurementForm}{}\begin{classdesc}{MeasurementForm}{data=None, files=None, auto\_id='id\_\%s', prefix=None, initial=None, error\_class=\textless{}class 'django.forms.util.ErrorList'\textgreater{}, label\_suffix=':', empty\_permitted=False, instance=None}~\index{MeasurementForm.Meta (class in data.forms)}

\hypertarget{data.forms.MeasurementForm.Meta}{}\begin{classdesc}{Meta}{}~\index{model (data.forms.MeasurementForm.Meta attribute)}

\hypertarget{data.forms.MeasurementForm.Meta.model}{}\begin{memberdesc}{model}
alias of \code{Measurement}
\end{memberdesc}
\end{classdesc}
\index{add\_initial\_prefix() (data.forms.MeasurementForm method)}

\hypertarget{data.forms.MeasurementForm.add_initial_prefix}{}\begin{methoddesc}[MeasurementForm]{add\_initial\_prefix}{field\_name}
Add a `initial' prefix for checking dynamic initial values
\end{methoddesc}
\index{add\_prefix() (data.forms.MeasurementForm method)}

\hypertarget{data.forms.MeasurementForm.add_prefix}{}\begin{methoddesc}[MeasurementForm]{add\_prefix}{field\_name}
Returns the field name with a prefix appended, if this Form has a
prefix set.

Subclasses may wish to override.
\end{methoddesc}
\index{as\_p() (data.forms.MeasurementForm method)}

\hypertarget{data.forms.MeasurementForm.as_p}{}\begin{methoddesc}[MeasurementForm]{as\_p}{}
Returns this form rendered as HTML \textless{}p\textgreater{}s.
\end{methoddesc}
\index{as\_table() (data.forms.MeasurementForm method)}

\hypertarget{data.forms.MeasurementForm.as_table}{}\begin{methoddesc}[MeasurementForm]{as\_table}{}
Returns this form rendered as HTML \textless{}tr\textgreater{}s -- excluding the \textless{}table\textgreater{}\textless{}/table\textgreater{}.
\end{methoddesc}
\index{as\_ul() (data.forms.MeasurementForm method)}

\hypertarget{data.forms.MeasurementForm.as_ul}{}\begin{methoddesc}[MeasurementForm]{as\_ul}{}
Returns this form rendered as HTML \textless{}li\textgreater{}s -- excluding the \textless{}ul\textgreater{}\textless{}/ul\textgreater{}.
\end{methoddesc}
\index{changed\_data (data.forms.MeasurementForm attribute)}

\hypertarget{data.forms.MeasurementForm.changed_data}{}\begin{memberdesc}[MeasurementForm]{changed\_data}\end{memberdesc}
\index{clean() (data.forms.MeasurementForm method)}

\hypertarget{data.forms.MeasurementForm.clean}{}\begin{methoddesc}[MeasurementForm]{clean}{}\end{methoddesc}
\index{date\_error\_message() (data.forms.MeasurementForm method)}

\hypertarget{data.forms.MeasurementForm.date_error_message}{}\begin{methoddesc}[MeasurementForm]{date\_error\_message}{lookup\_type, field, unique\_for}\end{methoddesc}
\index{errors (data.forms.MeasurementForm attribute)}

\hypertarget{data.forms.MeasurementForm.errors}{}\begin{memberdesc}[MeasurementForm]{errors}
Returns an ErrorDict for the data provided for the form
\end{memberdesc}
\index{full\_clean() (data.forms.MeasurementForm method)}

\hypertarget{data.forms.MeasurementForm.full_clean}{}\begin{methoddesc}[MeasurementForm]{full\_clean}{}
Cleans all of self.data and populates self.\_errors and
self.cleaned\_data.
\end{methoddesc}
\index{has\_changed() (data.forms.MeasurementForm method)}

\hypertarget{data.forms.MeasurementForm.has_changed}{}\begin{methoddesc}[MeasurementForm]{has\_changed}{}
Returns True if data differs from initial.
\end{methoddesc}
\index{hidden\_fields() (data.forms.MeasurementForm method)}

\hypertarget{data.forms.MeasurementForm.hidden_fields}{}\begin{methoddesc}[MeasurementForm]{hidden\_fields}{}
Returns a list of all the BoundField objects that are hidden fields.
Useful for manual form layout in templates.
\end{methoddesc}
\index{is\_multipart() (data.forms.MeasurementForm method)}

\hypertarget{data.forms.MeasurementForm.is_multipart}{}\begin{methoddesc}[MeasurementForm]{is\_multipart}{}
Returns True if the form needs to be multipart-encrypted, i.e. it has
FileInput. Otherwise, False.
\end{methoddesc}
\index{is\_valid() (data.forms.MeasurementForm method)}

\hypertarget{data.forms.MeasurementForm.is_valid}{}\begin{methoddesc}[MeasurementForm]{is\_valid}{}
Returns True if the form has no errors. Otherwise, False. If errors are
being ignored, returns False.
\end{methoddesc}
\index{media (data.forms.MeasurementForm attribute)}

\hypertarget{data.forms.MeasurementForm.media}{}\begin{memberdesc}[MeasurementForm]{media}\end{memberdesc}
\index{non\_field\_errors() (data.forms.MeasurementForm method)}

\hypertarget{data.forms.MeasurementForm.non_field_errors}{}\begin{methoddesc}[MeasurementForm]{non\_field\_errors}{}
Returns an ErrorList of errors that aren't associated with a particular
field -- i.e., from Form.clean(). Returns an empty ErrorList if there
are none.
\end{methoddesc}
\index{save() (data.forms.MeasurementForm method)}

\hypertarget{data.forms.MeasurementForm.save}{}\begin{methoddesc}[MeasurementForm]{save}{commit=True}
Saves this \code{form}`s cleaned\_data into model instance
\code{self.instance}.

If commit=True, then the changes to \code{instance} will be saved to the
database. Returns \code{instance}.
\end{methoddesc}
\index{unique\_error\_message() (data.forms.MeasurementForm method)}

\hypertarget{data.forms.MeasurementForm.unique_error_message}{}\begin{methoddesc}[MeasurementForm]{unique\_error\_message}{unique\_check}\end{methoddesc}
\index{validate\_unique() (data.forms.MeasurementForm method)}

\hypertarget{data.forms.MeasurementForm.validate_unique}{}\begin{methoddesc}[MeasurementForm]{validate\_unique}{}\end{methoddesc}
\index{visible\_fields() (data.forms.MeasurementForm method)}

\hypertarget{data.forms.MeasurementForm.visible_fields}{}\begin{methoddesc}[MeasurementForm]{visible\_fields}{}
Returns a list of BoundField objects that aren't hidden fields.
The opposite of the hidden\_fields() method.
\end{methoddesc}
\end{classdesc}
\index{StudyExperimentForm (class in data.forms)}

\hypertarget{data.forms.StudyExperimentForm}{}\begin{classdesc}{StudyExperimentForm}{data=None, files=None, auto\_id='id\_\%s', prefix=None, initial=None, error\_class=\textless{}class 'django.forms.util.ErrorList'\textgreater{}, label\_suffix=':', empty\_permitted=False, instance=None}~\index{StudyExperimentForm.Meta (class in data.forms)}

\hypertarget{data.forms.StudyExperimentForm.Meta}{}\begin{classdesc}{Meta}{}~\index{model (data.forms.StudyExperimentForm.Meta attribute)}

\hypertarget{data.forms.StudyExperimentForm.Meta.model}{}\begin{memberdesc}{model}
alias of \code{Experiment}
\end{memberdesc}
\end{classdesc}
\index{add\_initial\_prefix() (data.forms.StudyExperimentForm method)}

\hypertarget{data.forms.StudyExperimentForm.add_initial_prefix}{}\begin{methoddesc}[StudyExperimentForm]{add\_initial\_prefix}{field\_name}
Add a `initial' prefix for checking dynamic initial values
\end{methoddesc}
\index{add\_prefix() (data.forms.StudyExperimentForm method)}

\hypertarget{data.forms.StudyExperimentForm.add_prefix}{}\begin{methoddesc}[StudyExperimentForm]{add\_prefix}{field\_name}
Returns the field name with a prefix appended, if this Form has a
prefix set.

Subclasses may wish to override.
\end{methoddesc}
\index{as\_p() (data.forms.StudyExperimentForm method)}

\hypertarget{data.forms.StudyExperimentForm.as_p}{}\begin{methoddesc}[StudyExperimentForm]{as\_p}{}
Returns this form rendered as HTML \textless{}p\textgreater{}s.
\end{methoddesc}
\index{as\_table() (data.forms.StudyExperimentForm method)}

\hypertarget{data.forms.StudyExperimentForm.as_table}{}\begin{methoddesc}[StudyExperimentForm]{as\_table}{}
Returns this form rendered as HTML \textless{}tr\textgreater{}s -- excluding the \textless{}table\textgreater{}\textless{}/table\textgreater{}.
\end{methoddesc}
\index{as\_ul() (data.forms.StudyExperimentForm method)}

\hypertarget{data.forms.StudyExperimentForm.as_ul}{}\begin{methoddesc}[StudyExperimentForm]{as\_ul}{}
Returns this form rendered as HTML \textless{}li\textgreater{}s -- excluding the \textless{}ul\textgreater{}\textless{}/ul\textgreater{}.
\end{methoddesc}
\index{changed\_data (data.forms.StudyExperimentForm attribute)}

\hypertarget{data.forms.StudyExperimentForm.changed_data}{}\begin{memberdesc}[StudyExperimentForm]{changed\_data}\end{memberdesc}
\index{clean() (data.forms.StudyExperimentForm method)}

\hypertarget{data.forms.StudyExperimentForm.clean}{}\begin{methoddesc}[StudyExperimentForm]{clean}{}\end{methoddesc}
\index{date\_error\_message() (data.forms.StudyExperimentForm method)}

\hypertarget{data.forms.StudyExperimentForm.date_error_message}{}\begin{methoddesc}[StudyExperimentForm]{date\_error\_message}{lookup\_type, field, unique\_for}\end{methoddesc}
\index{errors (data.forms.StudyExperimentForm attribute)}

\hypertarget{data.forms.StudyExperimentForm.errors}{}\begin{memberdesc}[StudyExperimentForm]{errors}
Returns an ErrorDict for the data provided for the form
\end{memberdesc}
\index{full\_clean() (data.forms.StudyExperimentForm method)}

\hypertarget{data.forms.StudyExperimentForm.full_clean}{}\begin{methoddesc}[StudyExperimentForm]{full\_clean}{}
Cleans all of self.data and populates self.\_errors and
self.cleaned\_data.
\end{methoddesc}
\index{has\_changed() (data.forms.StudyExperimentForm method)}

\hypertarget{data.forms.StudyExperimentForm.has_changed}{}\begin{methoddesc}[StudyExperimentForm]{has\_changed}{}
Returns True if data differs from initial.
\end{methoddesc}
\index{hidden\_fields() (data.forms.StudyExperimentForm method)}

\hypertarget{data.forms.StudyExperimentForm.hidden_fields}{}\begin{methoddesc}[StudyExperimentForm]{hidden\_fields}{}
Returns a list of all the BoundField objects that are hidden fields.
Useful for manual form layout in templates.
\end{methoddesc}
\index{is\_multipart() (data.forms.StudyExperimentForm method)}

\hypertarget{data.forms.StudyExperimentForm.is_multipart}{}\begin{methoddesc}[StudyExperimentForm]{is\_multipart}{}
Returns True if the form needs to be multipart-encrypted, i.e. it has
FileInput. Otherwise, False.
\end{methoddesc}
\index{is\_valid() (data.forms.StudyExperimentForm method)}

\hypertarget{data.forms.StudyExperimentForm.is_valid}{}\begin{methoddesc}[StudyExperimentForm]{is\_valid}{}
Returns True if the form has no errors. Otherwise, False. If errors are
being ignored, returns False.
\end{methoddesc}
\index{media (data.forms.StudyExperimentForm attribute)}

\hypertarget{data.forms.StudyExperimentForm.media}{}\begin{memberdesc}[StudyExperimentForm]{media}\end{memberdesc}
\index{non\_field\_errors() (data.forms.StudyExperimentForm method)}

\hypertarget{data.forms.StudyExperimentForm.non_field_errors}{}\begin{methoddesc}[StudyExperimentForm]{non\_field\_errors}{}
Returns an ErrorList of errors that aren't associated with a particular
field -- i.e., from Form.clean(). Returns an empty ErrorList if there
are none.
\end{methoddesc}
\index{save() (data.forms.StudyExperimentForm method)}

\hypertarget{data.forms.StudyExperimentForm.save}{}\begin{methoddesc}[StudyExperimentForm]{save}{commit=True}
Saves this \code{form}`s cleaned\_data into model instance
\code{self.instance}.

If commit=True, then the changes to \code{instance} will be saved to the
database. Returns \code{instance}.
\end{methoddesc}
\index{unique\_error\_message() (data.forms.StudyExperimentForm method)}

\hypertarget{data.forms.StudyExperimentForm.unique_error_message}{}\begin{methoddesc}[StudyExperimentForm]{unique\_error\_message}{unique\_check}\end{methoddesc}
\index{validate\_unique() (data.forms.StudyExperimentForm method)}

\hypertarget{data.forms.StudyExperimentForm.validate_unique}{}\begin{methoddesc}[StudyExperimentForm]{validate\_unique}{}\end{methoddesc}
\index{visible\_fields() (data.forms.StudyExperimentForm method)}

\hypertarget{data.forms.StudyExperimentForm.visible_fields}{}\begin{methoddesc}[StudyExperimentForm]{visible\_fields}{}
Returns a list of BoundField objects that aren't hidden fields.
The opposite of the hidden\_fields() method.
\end{methoddesc}
\end{classdesc}
\index{StudyForm (class in data.forms)}

\hypertarget{data.forms.StudyForm}{}\begin{classdesc}{StudyForm}{data=None, files=None, auto\_id='id\_\%s', prefix=None, initial=None, error\_class=\textless{}class 'django.forms.util.ErrorList'\textgreater{}, label\_suffix=':', empty\_permitted=False, instance=None}~\index{StudyForm.Meta (class in data.forms)}

\hypertarget{data.forms.StudyForm.Meta}{}\begin{classdesc}{Meta}{}~\index{model (data.forms.StudyForm.Meta attribute)}

\hypertarget{data.forms.StudyForm.Meta.model}{}\begin{memberdesc}{model}
alias of \code{Study}
\end{memberdesc}
\end{classdesc}
\index{add\_initial\_prefix() (data.forms.StudyForm method)}

\hypertarget{data.forms.StudyForm.add_initial_prefix}{}\begin{methoddesc}[StudyForm]{add\_initial\_prefix}{field\_name}
Add a `initial' prefix for checking dynamic initial values
\end{methoddesc}
\index{add\_prefix() (data.forms.StudyForm method)}

\hypertarget{data.forms.StudyForm.add_prefix}{}\begin{methoddesc}[StudyForm]{add\_prefix}{field\_name}
Returns the field name with a prefix appended, if this Form has a
prefix set.

Subclasses may wish to override.
\end{methoddesc}
\index{as\_p() (data.forms.StudyForm method)}

\hypertarget{data.forms.StudyForm.as_p}{}\begin{methoddesc}[StudyForm]{as\_p}{}
Returns this form rendered as HTML \textless{}p\textgreater{}s.
\end{methoddesc}
\index{as\_table() (data.forms.StudyForm method)}

\hypertarget{data.forms.StudyForm.as_table}{}\begin{methoddesc}[StudyForm]{as\_table}{}
Returns this form rendered as HTML \textless{}tr\textgreater{}s -- excluding the \textless{}table\textgreater{}\textless{}/table\textgreater{}.
\end{methoddesc}
\index{as\_ul() (data.forms.StudyForm method)}

\hypertarget{data.forms.StudyForm.as_ul}{}\begin{methoddesc}[StudyForm]{as\_ul}{}
Returns this form rendered as HTML \textless{}li\textgreater{}s -- excluding the \textless{}ul\textgreater{}\textless{}/ul\textgreater{}.
\end{methoddesc}
\index{changed\_data (data.forms.StudyForm attribute)}

\hypertarget{data.forms.StudyForm.changed_data}{}\begin{memberdesc}[StudyForm]{changed\_data}\end{memberdesc}
\index{clean() (data.forms.StudyForm method)}

\hypertarget{data.forms.StudyForm.clean}{}\begin{methoddesc}[StudyForm]{clean}{}\end{methoddesc}
\index{date\_error\_message() (data.forms.StudyForm method)}

\hypertarget{data.forms.StudyForm.date_error_message}{}\begin{methoddesc}[StudyForm]{date\_error\_message}{lookup\_type, field, unique\_for}\end{methoddesc}
\index{errors (data.forms.StudyForm attribute)}

\hypertarget{data.forms.StudyForm.errors}{}\begin{memberdesc}[StudyForm]{errors}
Returns an ErrorDict for the data provided for the form
\end{memberdesc}
\index{full\_clean() (data.forms.StudyForm method)}

\hypertarget{data.forms.StudyForm.full_clean}{}\begin{methoddesc}[StudyForm]{full\_clean}{}
Cleans all of self.data and populates self.\_errors and
self.cleaned\_data.
\end{methoddesc}
\index{has\_changed() (data.forms.StudyForm method)}

\hypertarget{data.forms.StudyForm.has_changed}{}\begin{methoddesc}[StudyForm]{has\_changed}{}
Returns True if data differs from initial.
\end{methoddesc}
\index{hidden\_fields() (data.forms.StudyForm method)}

\hypertarget{data.forms.StudyForm.hidden_fields}{}\begin{methoddesc}[StudyForm]{hidden\_fields}{}
Returns a list of all the BoundField objects that are hidden fields.
Useful for manual form layout in templates.
\end{methoddesc}
\index{is\_multipart() (data.forms.StudyForm method)}

\hypertarget{data.forms.StudyForm.is_multipart}{}\begin{methoddesc}[StudyForm]{is\_multipart}{}
Returns True if the form needs to be multipart-encrypted, i.e. it has
FileInput. Otherwise, False.
\end{methoddesc}
\index{is\_valid() (data.forms.StudyForm method)}

\hypertarget{data.forms.StudyForm.is_valid}{}\begin{methoddesc}[StudyForm]{is\_valid}{}
Returns True if the form has no errors. Otherwise, False. If errors are
being ignored, returns False.
\end{methoddesc}
\index{media (data.forms.StudyForm attribute)}

\hypertarget{data.forms.StudyForm.media}{}\begin{memberdesc}[StudyForm]{media}\end{memberdesc}
\index{non\_field\_errors() (data.forms.StudyForm method)}

\hypertarget{data.forms.StudyForm.non_field_errors}{}\begin{methoddesc}[StudyForm]{non\_field\_errors}{}
Returns an ErrorList of errors that aren't associated with a particular
field -- i.e., from Form.clean(). Returns an empty ErrorList if there
are none.
\end{methoddesc}
\index{save() (data.forms.StudyForm method)}

\hypertarget{data.forms.StudyForm.save}{}\begin{methoddesc}[StudyForm]{save}{commit=True}
Saves this \code{form}`s cleaned\_data into model instance
\code{self.instance}.

If commit=True, then the changes to \code{instance} will be saved to the
database. Returns \code{instance}.
\end{methoddesc}
\index{unique\_error\_message() (data.forms.StudyForm method)}

\hypertarget{data.forms.StudyForm.unique_error_message}{}\begin{methoddesc}[StudyForm]{unique\_error\_message}{unique\_check}\end{methoddesc}
\index{validate\_unique() (data.forms.StudyForm method)}

\hypertarget{data.forms.StudyForm.validate_unique}{}\begin{methoddesc}[StudyForm]{validate\_unique}{}\end{methoddesc}
\index{visible\_fields() (data.forms.StudyForm method)}

\hypertarget{data.forms.StudyForm.visible_fields}{}\begin{methoddesc}[StudyForm]{visible\_fields}{}
Returns a list of BoundField objects that aren't hidden fields.
The opposite of the hidden\_fields() method.
\end{methoddesc}
\end{classdesc}


\subsection{Views and URLs}
\index{data.views (module)}
\hypertarget{module-data.views}{}
\declaremodule[data.views]{}{data.views}
\modulesynopsis{}\index{add\_measurement (in module data.views)}

\hypertarget{data.views.add_measurement}{}\begin{memberdesc}[data.views]{add\_measurement}\end{memberdesc}
\index{experiment\_detail (in module data.views)}

\hypertarget{data.views.experiment_detail}{}\begin{memberdesc}[data.views]{experiment\_detail}\end{memberdesc}
\index{experiment\_detail\_all (in module data.views)}

\hypertarget{data.views.experiment_detail_all}{}\begin{memberdesc}[data.views]{experiment\_detail\_all}\end{memberdesc}
\index{experiment\_list (in module data.views)}

\hypertarget{data.views.experiment_list}{}\begin{memberdesc}[data.views]{experiment\_list}\end{memberdesc}
\index{study\_experiment (in module data.views)}

\hypertarget{data.views.study_experiment}{}\begin{memberdesc}[data.views]{study\_experiment}\end{memberdesc}
\index{data.urls (module)}
\hypertarget{module-data.urls}{}
\declaremodule[data.urls]{}{data.urls}
\modulesynopsis{}

\subsection{Administrative Site Configuration}
\index{data.admin (module)}
\hypertarget{module-data.admin}{}
\declaremodule[data.admin]{}{data.admin}
\modulesynopsis{}\index{AssayAdmin (class in data.admin)}

\hypertarget{data.admin.AssayAdmin}{}\begin{classdesc}{AssayAdmin}{model, admin\_site}~\index{action\_checkbox() (data.admin.AssayAdmin method)}

\hypertarget{data.admin.AssayAdmin.action_checkbox}{}\begin{methoddesc}{action\_checkbox}{obj}
A list\_display column containing a checkbox widget.
\end{methoddesc}
\index{action\_form (data.admin.AssayAdmin attribute)}

\hypertarget{data.admin.AssayAdmin.action_form}{}\begin{memberdesc}{action\_form}
alias of \code{ActionForm}
\end{memberdesc}
\index{add\_view() (data.admin.AssayAdmin method)}

\hypertarget{data.admin.AssayAdmin.add_view}{}\begin{methoddesc}{add\_view}{*args, **kw}
The `add' admin view for this model.
\end{methoddesc}
\index{change\_view() (data.admin.AssayAdmin method)}

\hypertarget{data.admin.AssayAdmin.change_view}{}\begin{methoddesc}{change\_view}{*args, **kw}
The `change' admin view for this model.
\end{methoddesc}
\index{changelist\_view() (data.admin.AssayAdmin method)}

\hypertarget{data.admin.AssayAdmin.changelist_view}{}\begin{methoddesc}{changelist\_view}{request, extra\_context=None}
The `change list' admin view for this model.
\end{methoddesc}
\index{construct\_change\_message() (data.admin.AssayAdmin method)}

\hypertarget{data.admin.AssayAdmin.construct_change_message}{}\begin{methoddesc}{construct\_change\_message}{request, form, formsets}
Construct a change message from a changed object.
\end{methoddesc}
\index{declared\_fieldsets (data.admin.AssayAdmin attribute)}

\hypertarget{data.admin.AssayAdmin.declared_fieldsets}{}\begin{memberdesc}{declared\_fieldsets}\end{memberdesc}
\index{delete\_view() (data.admin.AssayAdmin method)}

\hypertarget{data.admin.AssayAdmin.delete_view}{}\begin{methoddesc}{delete\_view}{request, object\_id, extra\_context=None}
The `delete' admin view for this model.
\end{methoddesc}
\index{form (data.admin.AssayAdmin attribute)}

\hypertarget{data.admin.AssayAdmin.form}{}\begin{memberdesc}{form}
alias of \code{ModelForm}
\end{memberdesc}
\index{formfield\_for\_choice\_field() (data.admin.AssayAdmin method)}

\hypertarget{data.admin.AssayAdmin.formfield_for_choice_field}{}\begin{methoddesc}{formfield\_for\_choice\_field}{db\_field, request=None, **kwargs}
Get a form Field for a database Field that has declared choices.
\end{methoddesc}
\index{formfield\_for\_dbfield() (data.admin.AssayAdmin method)}

\hypertarget{data.admin.AssayAdmin.formfield_for_dbfield}{}\begin{methoddesc}{formfield\_for\_dbfield}{db\_field, **kwargs}
Hook for specifying the form Field instance for a given database Field
instance.

If kwargs are given, they're passed to the form Field's constructor.
\end{methoddesc}
\index{formfield\_for\_foreignkey() (data.admin.AssayAdmin method)}

\hypertarget{data.admin.AssayAdmin.formfield_for_foreignkey}{}\begin{methoddesc}{formfield\_for\_foreignkey}{db\_field, request=None, **kwargs}
Get a form Field for a ForeignKey.
\end{methoddesc}
\index{formfield\_for\_manytomany() (data.admin.AssayAdmin method)}

\hypertarget{data.admin.AssayAdmin.formfield_for_manytomany}{}\begin{methoddesc}{formfield\_for\_manytomany}{db\_field, request=None, **kwargs}
Get a form Field for a ManyToManyField.
\end{methoddesc}
\index{get\_action() (data.admin.AssayAdmin method)}

\hypertarget{data.admin.AssayAdmin.get_action}{}\begin{methoddesc}{get\_action}{action}
Return a given action from a parameter, which can either be a callable,
or the name of a method on the ModelAdmin.  Return is a tuple of
(callable, name, description).
\end{methoddesc}
\index{get\_action\_choices() (data.admin.AssayAdmin method)}

\hypertarget{data.admin.AssayAdmin.get_action_choices}{}\begin{methoddesc}{get\_action\_choices}{request, default\_choices=, {[}('', '---------'){]}}
Return a list of choices for use in a form object.  Each choice is a
tuple (name, description).
\end{methoddesc}
\index{get\_actions() (data.admin.AssayAdmin method)}

\hypertarget{data.admin.AssayAdmin.get_actions}{}\begin{methoddesc}{get\_actions}{request}
Return a dictionary mapping the names of all actions for this
ModelAdmin to a tuple of (callable, name, description) for each action.
\end{methoddesc}
\index{get\_changelist\_form() (data.admin.AssayAdmin method)}

\hypertarget{data.admin.AssayAdmin.get_changelist_form}{}\begin{methoddesc}{get\_changelist\_form}{request, **kwargs}
Returns a Form class for use in the Formset on the changelist page.
\end{methoddesc}
\index{get\_changelist\_formset() (data.admin.AssayAdmin method)}

\hypertarget{data.admin.AssayAdmin.get_changelist_formset}{}\begin{methoddesc}{get\_changelist\_formset}{request, **kwargs}
Returns a FormSet class for use on the changelist page if list\_editable
is used.
\end{methoddesc}
\index{get\_fieldsets() (data.admin.AssayAdmin method)}

\hypertarget{data.admin.AssayAdmin.get_fieldsets}{}\begin{methoddesc}{get\_fieldsets}{request, obj=None}
Hook for specifying fieldsets for the add form.
\end{methoddesc}
\index{get\_form() (data.admin.AssayAdmin method)}

\hypertarget{data.admin.AssayAdmin.get_form}{}\begin{methoddesc}{get\_form}{request, obj=None, **kwargs}
Returns a Form class for use in the admin add view. This is used by
add\_view and change\_view.
\end{methoddesc}
\index{get\_formsets() (data.admin.AssayAdmin method)}

\hypertarget{data.admin.AssayAdmin.get_formsets}{}\begin{methoddesc}{get\_formsets}{request, obj=None}\end{methoddesc}
\index{get\_model\_perms() (data.admin.AssayAdmin method)}

\hypertarget{data.admin.AssayAdmin.get_model_perms}{}\begin{methoddesc}{get\_model\_perms}{request}
Returns a dict of all perms for this model. This dict has the keys
\code{add}, \code{change}, and \code{delete} mapping to the True/False for each
of those actions.
\end{methoddesc}
\index{get\_urls() (data.admin.AssayAdmin method)}

\hypertarget{data.admin.AssayAdmin.get_urls}{}\begin{methoddesc}{get\_urls}{}\end{methoddesc}
\index{has\_add\_permission() (data.admin.AssayAdmin method)}

\hypertarget{data.admin.AssayAdmin.has_add_permission}{}\begin{methoddesc}{has\_add\_permission}{request}
Returns True if the given request has permission to add an object.
\end{methoddesc}
\index{has\_change\_permission() (data.admin.AssayAdmin method)}

\hypertarget{data.admin.AssayAdmin.has_change_permission}{}\begin{methoddesc}{has\_change\_permission}{request, obj=None}
Returns True if the given request has permission to change the given
Django model instance.

If \emph{obj} is None, this should return True if the given request has
permission to change \emph{any} object of the given type.
\end{methoddesc}
\index{has\_delete\_permission() (data.admin.AssayAdmin method)}

\hypertarget{data.admin.AssayAdmin.has_delete_permission}{}\begin{methoddesc}{has\_delete\_permission}{request, obj=None}
Returns True if the given request has permission to change the given
Django model instance.

If \emph{obj} is None, this should return True if the given request has
permission to delete \emph{any} object of the given type.
\end{methoddesc}
\index{history\_view() (data.admin.AssayAdmin method)}

\hypertarget{data.admin.AssayAdmin.history_view}{}\begin{methoddesc}{history\_view}{request, object\_id, extra\_context=None}
The `history' admin view for this model.
\end{methoddesc}
\index{log\_addition() (data.admin.AssayAdmin method)}

\hypertarget{data.admin.AssayAdmin.log_addition}{}\begin{methoddesc}{log\_addition}{request, object}
Log that an object has been successfully added.

The default implementation creates an admin LogEntry object.
\end{methoddesc}
\index{log\_change() (data.admin.AssayAdmin method)}

\hypertarget{data.admin.AssayAdmin.log_change}{}\begin{methoddesc}{log\_change}{request, object, message}
Log that an object has been successfully changed.

The default implementation creates an admin LogEntry object.
\end{methoddesc}
\index{log\_deletion() (data.admin.AssayAdmin method)}

\hypertarget{data.admin.AssayAdmin.log_deletion}{}\begin{methoddesc}{log\_deletion}{request, object, object\_repr}
Log that an object has been successfully deleted. Note that since the
object is deleted, it might no longer be safe to call \emph{any} methods
on the object, hence this method getting object\_repr.

The default implementation creates an admin LogEntry object.
\end{methoddesc}
\index{media (data.admin.AssayAdmin attribute)}

\hypertarget{data.admin.AssayAdmin.media}{}\begin{memberdesc}{media}\end{memberdesc}
\index{message\_user() (data.admin.AssayAdmin method)}

\hypertarget{data.admin.AssayAdmin.message_user}{}\begin{methoddesc}{message\_user}{request, message}
Send a message to the user. The default implementation
posts a message using the auth Message object.
\end{methoddesc}
\index{queryset() (data.admin.AssayAdmin method)}

\hypertarget{data.admin.AssayAdmin.queryset}{}\begin{methoddesc}{queryset}{request}
Returns a QuerySet of all model instances that can be edited by the
admin site. This is used by changelist\_view.
\end{methoddesc}
\index{render\_change\_form() (data.admin.AssayAdmin method)}

\hypertarget{data.admin.AssayAdmin.render_change_form}{}\begin{methoddesc}{render\_change\_form}{request, context, add=False, change=False, form\_url='', obj=None}\end{methoddesc}
\index{response\_action() (data.admin.AssayAdmin method)}

\hypertarget{data.admin.AssayAdmin.response_action}{}\begin{methoddesc}{response\_action}{request, queryset}
Handle an admin action. This is called if a request is POSTed to the
changelist; it returns an HttpResponse if the action was handled, and
None otherwise.
\end{methoddesc}
\index{response\_add() (data.admin.AssayAdmin method)}

\hypertarget{data.admin.AssayAdmin.response_add}{}\begin{methoddesc}{response\_add}{request, obj, post\_url\_continue='../\%s/'}
Determines the HttpResponse for the add\_view stage.
\end{methoddesc}
\index{response\_change() (data.admin.AssayAdmin method)}

\hypertarget{data.admin.AssayAdmin.response_change}{}\begin{methoddesc}{response\_change}{request, obj}
Determines the HttpResponse for the change\_view stage.
\end{methoddesc}
\index{save\_form() (data.admin.AssayAdmin method)}

\hypertarget{data.admin.AssayAdmin.save_form}{}\begin{methoddesc}{save\_form}{request, form, change}
Given a ModelForm return an unsaved instance. \code{change} is True if
the object is being changed, and False if it's being added.
\end{methoddesc}
\index{save\_formset() (data.admin.AssayAdmin method)}

\hypertarget{data.admin.AssayAdmin.save_formset}{}\begin{methoddesc}{save\_formset}{request, form, formset, change}
Given an inline formset save it to the database.
\end{methoddesc}
\index{save\_model() (data.admin.AssayAdmin method)}

\hypertarget{data.admin.AssayAdmin.save_model}{}\begin{methoddesc}{save\_model}{request, obj, form, change}
Given a model instance save it to the database.
\end{methoddesc}
\index{urls (data.admin.AssayAdmin attribute)}

\hypertarget{data.admin.AssayAdmin.urls}{}\begin{memberdesc}{urls}\end{memberdesc}
\end{classdesc}
\index{DietAdmin (class in data.admin)}

\hypertarget{data.admin.DietAdmin}{}\begin{classdesc}{DietAdmin}{model, admin\_site}~\index{action\_checkbox() (data.admin.DietAdmin method)}

\hypertarget{data.admin.DietAdmin.action_checkbox}{}\begin{methoddesc}{action\_checkbox}{obj}
A list\_display column containing a checkbox widget.
\end{methoddesc}
\index{action\_form (data.admin.DietAdmin attribute)}

\hypertarget{data.admin.DietAdmin.action_form}{}\begin{memberdesc}{action\_form}
alias of \code{ActionForm}
\end{memberdesc}
\index{add\_view() (data.admin.DietAdmin method)}

\hypertarget{data.admin.DietAdmin.add_view}{}\begin{methoddesc}{add\_view}{*args, **kw}
The `add' admin view for this model.
\end{methoddesc}
\index{change\_view() (data.admin.DietAdmin method)}

\hypertarget{data.admin.DietAdmin.change_view}{}\begin{methoddesc}{change\_view}{*args, **kw}
The `change' admin view for this model.
\end{methoddesc}
\index{changelist\_view() (data.admin.DietAdmin method)}

\hypertarget{data.admin.DietAdmin.changelist_view}{}\begin{methoddesc}{changelist\_view}{request, extra\_context=None}
The `change list' admin view for this model.
\end{methoddesc}
\index{construct\_change\_message() (data.admin.DietAdmin method)}

\hypertarget{data.admin.DietAdmin.construct_change_message}{}\begin{methoddesc}{construct\_change\_message}{request, form, formsets}
Construct a change message from a changed object.
\end{methoddesc}
\index{declared\_fieldsets (data.admin.DietAdmin attribute)}

\hypertarget{data.admin.DietAdmin.declared_fieldsets}{}\begin{memberdesc}{declared\_fieldsets}\end{memberdesc}
\index{delete\_view() (data.admin.DietAdmin method)}

\hypertarget{data.admin.DietAdmin.delete_view}{}\begin{methoddesc}{delete\_view}{request, object\_id, extra\_context=None}
The `delete' admin view for this model.
\end{methoddesc}
\index{form (data.admin.DietAdmin attribute)}

\hypertarget{data.admin.DietAdmin.form}{}\begin{memberdesc}{form}
alias of \code{ModelForm}
\end{memberdesc}
\index{formfield\_for\_choice\_field() (data.admin.DietAdmin method)}

\hypertarget{data.admin.DietAdmin.formfield_for_choice_field}{}\begin{methoddesc}{formfield\_for\_choice\_field}{db\_field, request=None, **kwargs}
Get a form Field for a database Field that has declared choices.
\end{methoddesc}
\index{formfield\_for\_dbfield() (data.admin.DietAdmin method)}

\hypertarget{data.admin.DietAdmin.formfield_for_dbfield}{}\begin{methoddesc}{formfield\_for\_dbfield}{db\_field, **kwargs}
Hook for specifying the form Field instance for a given database Field
instance.

If kwargs are given, they're passed to the form Field's constructor.
\end{methoddesc}
\index{formfield\_for\_foreignkey() (data.admin.DietAdmin method)}

\hypertarget{data.admin.DietAdmin.formfield_for_foreignkey}{}\begin{methoddesc}{formfield\_for\_foreignkey}{db\_field, request=None, **kwargs}
Get a form Field for a ForeignKey.
\end{methoddesc}
\index{formfield\_for\_manytomany() (data.admin.DietAdmin method)}

\hypertarget{data.admin.DietAdmin.formfield_for_manytomany}{}\begin{methoddesc}{formfield\_for\_manytomany}{db\_field, request=None, **kwargs}
Get a form Field for a ManyToManyField.
\end{methoddesc}
\index{get\_action() (data.admin.DietAdmin method)}

\hypertarget{data.admin.DietAdmin.get_action}{}\begin{methoddesc}{get\_action}{action}
Return a given action from a parameter, which can either be a callable,
or the name of a method on the ModelAdmin.  Return is a tuple of
(callable, name, description).
\end{methoddesc}
\index{get\_action\_choices() (data.admin.DietAdmin method)}

\hypertarget{data.admin.DietAdmin.get_action_choices}{}\begin{methoddesc}{get\_action\_choices}{request, default\_choices=, {[}('', '---------'){]}}
Return a list of choices for use in a form object.  Each choice is a
tuple (name, description).
\end{methoddesc}
\index{get\_actions() (data.admin.DietAdmin method)}

\hypertarget{data.admin.DietAdmin.get_actions}{}\begin{methoddesc}{get\_actions}{request}
Return a dictionary mapping the names of all actions for this
ModelAdmin to a tuple of (callable, name, description) for each action.
\end{methoddesc}
\index{get\_changelist\_form() (data.admin.DietAdmin method)}

\hypertarget{data.admin.DietAdmin.get_changelist_form}{}\begin{methoddesc}{get\_changelist\_form}{request, **kwargs}
Returns a Form class for use in the Formset on the changelist page.
\end{methoddesc}
\index{get\_changelist\_formset() (data.admin.DietAdmin method)}

\hypertarget{data.admin.DietAdmin.get_changelist_formset}{}\begin{methoddesc}{get\_changelist\_formset}{request, **kwargs}
Returns a FormSet class for use on the changelist page if list\_editable
is used.
\end{methoddesc}
\index{get\_fieldsets() (data.admin.DietAdmin method)}

\hypertarget{data.admin.DietAdmin.get_fieldsets}{}\begin{methoddesc}{get\_fieldsets}{request, obj=None}
Hook for specifying fieldsets for the add form.
\end{methoddesc}
\index{get\_form() (data.admin.DietAdmin method)}

\hypertarget{data.admin.DietAdmin.get_form}{}\begin{methoddesc}{get\_form}{request, obj=None, **kwargs}
Returns a Form class for use in the admin add view. This is used by
add\_view and change\_view.
\end{methoddesc}
\index{get\_formsets() (data.admin.DietAdmin method)}

\hypertarget{data.admin.DietAdmin.get_formsets}{}\begin{methoddesc}{get\_formsets}{request, obj=None}\end{methoddesc}
\index{get\_model\_perms() (data.admin.DietAdmin method)}

\hypertarget{data.admin.DietAdmin.get_model_perms}{}\begin{methoddesc}{get\_model\_perms}{request}
Returns a dict of all perms for this model. This dict has the keys
\code{add}, \code{change}, and \code{delete} mapping to the True/False for each
of those actions.
\end{methoddesc}
\index{get\_urls() (data.admin.DietAdmin method)}

\hypertarget{data.admin.DietAdmin.get_urls}{}\begin{methoddesc}{get\_urls}{}\end{methoddesc}
\index{has\_add\_permission() (data.admin.DietAdmin method)}

\hypertarget{data.admin.DietAdmin.has_add_permission}{}\begin{methoddesc}{has\_add\_permission}{request}
Returns True if the given request has permission to add an object.
\end{methoddesc}
\index{has\_change\_permission() (data.admin.DietAdmin method)}

\hypertarget{data.admin.DietAdmin.has_change_permission}{}\begin{methoddesc}{has\_change\_permission}{request, obj=None}
Returns True if the given request has permission to change the given
Django model instance.

If \emph{obj} is None, this should return True if the given request has
permission to change \emph{any} object of the given type.
\end{methoddesc}
\index{has\_delete\_permission() (data.admin.DietAdmin method)}

\hypertarget{data.admin.DietAdmin.has_delete_permission}{}\begin{methoddesc}{has\_delete\_permission}{request, obj=None}
Returns True if the given request has permission to change the given
Django model instance.

If \emph{obj} is None, this should return True if the given request has
permission to delete \emph{any} object of the given type.
\end{methoddesc}
\index{history\_view() (data.admin.DietAdmin method)}

\hypertarget{data.admin.DietAdmin.history_view}{}\begin{methoddesc}{history\_view}{request, object\_id, extra\_context=None}
The `history' admin view for this model.
\end{methoddesc}
\index{log\_addition() (data.admin.DietAdmin method)}

\hypertarget{data.admin.DietAdmin.log_addition}{}\begin{methoddesc}{log\_addition}{request, object}
Log that an object has been successfully added.

The default implementation creates an admin LogEntry object.
\end{methoddesc}
\index{log\_change() (data.admin.DietAdmin method)}

\hypertarget{data.admin.DietAdmin.log_change}{}\begin{methoddesc}{log\_change}{request, object, message}
Log that an object has been successfully changed.

The default implementation creates an admin LogEntry object.
\end{methoddesc}
\index{log\_deletion() (data.admin.DietAdmin method)}

\hypertarget{data.admin.DietAdmin.log_deletion}{}\begin{methoddesc}{log\_deletion}{request, object, object\_repr}
Log that an object has been successfully deleted. Note that since the
object is deleted, it might no longer be safe to call \emph{any} methods
on the object, hence this method getting object\_repr.

The default implementation creates an admin LogEntry object.
\end{methoddesc}
\index{media (data.admin.DietAdmin attribute)}

\hypertarget{data.admin.DietAdmin.media}{}\begin{memberdesc}{media}\end{memberdesc}
\index{message\_user() (data.admin.DietAdmin method)}

\hypertarget{data.admin.DietAdmin.message_user}{}\begin{methoddesc}{message\_user}{request, message}
Send a message to the user. The default implementation
posts a message using the auth Message object.
\end{methoddesc}
\index{queryset() (data.admin.DietAdmin method)}

\hypertarget{data.admin.DietAdmin.queryset}{}\begin{methoddesc}{queryset}{request}
Returns a QuerySet of all model instances that can be edited by the
admin site. This is used by changelist\_view.
\end{methoddesc}
\index{render\_change\_form() (data.admin.DietAdmin method)}

\hypertarget{data.admin.DietAdmin.render_change_form}{}\begin{methoddesc}{render\_change\_form}{request, context, add=False, change=False, form\_url='', obj=None}\end{methoddesc}
\index{response\_action() (data.admin.DietAdmin method)}

\hypertarget{data.admin.DietAdmin.response_action}{}\begin{methoddesc}{response\_action}{request, queryset}
Handle an admin action. This is called if a request is POSTed to the
changelist; it returns an HttpResponse if the action was handled, and
None otherwise.
\end{methoddesc}
\index{response\_add() (data.admin.DietAdmin method)}

\hypertarget{data.admin.DietAdmin.response_add}{}\begin{methoddesc}{response\_add}{request, obj, post\_url\_continue='../\%s/'}
Determines the HttpResponse for the add\_view stage.
\end{methoddesc}
\index{response\_change() (data.admin.DietAdmin method)}

\hypertarget{data.admin.DietAdmin.response_change}{}\begin{methoddesc}{response\_change}{request, obj}
Determines the HttpResponse for the change\_view stage.
\end{methoddesc}
\index{save\_form() (data.admin.DietAdmin method)}

\hypertarget{data.admin.DietAdmin.save_form}{}\begin{methoddesc}{save\_form}{request, form, change}
Given a ModelForm return an unsaved instance. \code{change} is True if
the object is being changed, and False if it's being added.
\end{methoddesc}
\index{save\_formset() (data.admin.DietAdmin method)}

\hypertarget{data.admin.DietAdmin.save_formset}{}\begin{methoddesc}{save\_formset}{request, form, formset, change}
Given an inline formset save it to the database.
\end{methoddesc}
\index{save\_model() (data.admin.DietAdmin method)}

\hypertarget{data.admin.DietAdmin.save_model}{}\begin{methoddesc}{save\_model}{request, obj, form, change}
Given a model instance save it to the database.
\end{methoddesc}
\index{urls (data.admin.DietAdmin attribute)}

\hypertarget{data.admin.DietAdmin.urls}{}\begin{memberdesc}{urls}\end{memberdesc}
\end{classdesc}
\index{EnvironmentAdmin (class in data.admin)}

\hypertarget{data.admin.EnvironmentAdmin}{}\begin{classdesc}{EnvironmentAdmin}{model, admin\_site}~\index{action\_checkbox() (data.admin.EnvironmentAdmin method)}

\hypertarget{data.admin.EnvironmentAdmin.action_checkbox}{}\begin{methoddesc}{action\_checkbox}{obj}
A list\_display column containing a checkbox widget.
\end{methoddesc}
\index{action\_form (data.admin.EnvironmentAdmin attribute)}

\hypertarget{data.admin.EnvironmentAdmin.action_form}{}\begin{memberdesc}{action\_form}
alias of \code{ActionForm}
\end{memberdesc}
\index{add\_view() (data.admin.EnvironmentAdmin method)}

\hypertarget{data.admin.EnvironmentAdmin.add_view}{}\begin{methoddesc}{add\_view}{*args, **kw}
The `add' admin view for this model.
\end{methoddesc}
\index{change\_view() (data.admin.EnvironmentAdmin method)}

\hypertarget{data.admin.EnvironmentAdmin.change_view}{}\begin{methoddesc}{change\_view}{*args, **kw}
The `change' admin view for this model.
\end{methoddesc}
\index{changelist\_view() (data.admin.EnvironmentAdmin method)}

\hypertarget{data.admin.EnvironmentAdmin.changelist_view}{}\begin{methoddesc}{changelist\_view}{request, extra\_context=None}
The `change list' admin view for this model.
\end{methoddesc}
\index{construct\_change\_message() (data.admin.EnvironmentAdmin method)}

\hypertarget{data.admin.EnvironmentAdmin.construct_change_message}{}\begin{methoddesc}{construct\_change\_message}{request, form, formsets}
Construct a change message from a changed object.
\end{methoddesc}
\index{declared\_fieldsets (data.admin.EnvironmentAdmin attribute)}

\hypertarget{data.admin.EnvironmentAdmin.declared_fieldsets}{}\begin{memberdesc}{declared\_fieldsets}\end{memberdesc}
\index{delete\_view() (data.admin.EnvironmentAdmin method)}

\hypertarget{data.admin.EnvironmentAdmin.delete_view}{}\begin{methoddesc}{delete\_view}{request, object\_id, extra\_context=None}
The `delete' admin view for this model.
\end{methoddesc}
\index{form (data.admin.EnvironmentAdmin attribute)}

\hypertarget{data.admin.EnvironmentAdmin.form}{}\begin{memberdesc}{form}
alias of \code{ModelForm}
\end{memberdesc}
\index{formfield\_for\_choice\_field() (data.admin.EnvironmentAdmin method)}

\hypertarget{data.admin.EnvironmentAdmin.formfield_for_choice_field}{}\begin{methoddesc}{formfield\_for\_choice\_field}{db\_field, request=None, **kwargs}
Get a form Field for a database Field that has declared choices.
\end{methoddesc}
\index{formfield\_for\_dbfield() (data.admin.EnvironmentAdmin method)}

\hypertarget{data.admin.EnvironmentAdmin.formfield_for_dbfield}{}\begin{methoddesc}{formfield\_for\_dbfield}{db\_field, **kwargs}
Hook for specifying the form Field instance for a given database Field
instance.

If kwargs are given, they're passed to the form Field's constructor.
\end{methoddesc}
\index{formfield\_for\_foreignkey() (data.admin.EnvironmentAdmin method)}

\hypertarget{data.admin.EnvironmentAdmin.formfield_for_foreignkey}{}\begin{methoddesc}{formfield\_for\_foreignkey}{db\_field, request=None, **kwargs}
Get a form Field for a ForeignKey.
\end{methoddesc}
\index{formfield\_for\_manytomany() (data.admin.EnvironmentAdmin method)}

\hypertarget{data.admin.EnvironmentAdmin.formfield_for_manytomany}{}\begin{methoddesc}{formfield\_for\_manytomany}{db\_field, request=None, **kwargs}
Get a form Field for a ManyToManyField.
\end{methoddesc}
\index{get\_action() (data.admin.EnvironmentAdmin method)}

\hypertarget{data.admin.EnvironmentAdmin.get_action}{}\begin{methoddesc}{get\_action}{action}
Return a given action from a parameter, which can either be a callable,
or the name of a method on the ModelAdmin.  Return is a tuple of
(callable, name, description).
\end{methoddesc}
\index{get\_action\_choices() (data.admin.EnvironmentAdmin method)}

\hypertarget{data.admin.EnvironmentAdmin.get_action_choices}{}\begin{methoddesc}{get\_action\_choices}{request, default\_choices=, {[}('', '---------'){]}}
Return a list of choices for use in a form object.  Each choice is a
tuple (name, description).
\end{methoddesc}
\index{get\_actions() (data.admin.EnvironmentAdmin method)}

\hypertarget{data.admin.EnvironmentAdmin.get_actions}{}\begin{methoddesc}{get\_actions}{request}
Return a dictionary mapping the names of all actions for this
ModelAdmin to a tuple of (callable, name, description) for each action.
\end{methoddesc}
\index{get\_changelist\_form() (data.admin.EnvironmentAdmin method)}

\hypertarget{data.admin.EnvironmentAdmin.get_changelist_form}{}\begin{methoddesc}{get\_changelist\_form}{request, **kwargs}
Returns a Form class for use in the Formset on the changelist page.
\end{methoddesc}
\index{get\_changelist\_formset() (data.admin.EnvironmentAdmin method)}

\hypertarget{data.admin.EnvironmentAdmin.get_changelist_formset}{}\begin{methoddesc}{get\_changelist\_formset}{request, **kwargs}
Returns a FormSet class for use on the changelist page if list\_editable
is used.
\end{methoddesc}
\index{get\_fieldsets() (data.admin.EnvironmentAdmin method)}

\hypertarget{data.admin.EnvironmentAdmin.get_fieldsets}{}\begin{methoddesc}{get\_fieldsets}{request, obj=None}
Hook for specifying fieldsets for the add form.
\end{methoddesc}
\index{get\_form() (data.admin.EnvironmentAdmin method)}

\hypertarget{data.admin.EnvironmentAdmin.get_form}{}\begin{methoddesc}{get\_form}{request, obj=None, **kwargs}
Returns a Form class for use in the admin add view. This is used by
add\_view and change\_view.
\end{methoddesc}
\index{get\_formsets() (data.admin.EnvironmentAdmin method)}

\hypertarget{data.admin.EnvironmentAdmin.get_formsets}{}\begin{methoddesc}{get\_formsets}{request, obj=None}\end{methoddesc}
\index{get\_model\_perms() (data.admin.EnvironmentAdmin method)}

\hypertarget{data.admin.EnvironmentAdmin.get_model_perms}{}\begin{methoddesc}{get\_model\_perms}{request}
Returns a dict of all perms for this model. This dict has the keys
\code{add}, \code{change}, and \code{delete} mapping to the True/False for each
of those actions.
\end{methoddesc}
\index{get\_urls() (data.admin.EnvironmentAdmin method)}

\hypertarget{data.admin.EnvironmentAdmin.get_urls}{}\begin{methoddesc}{get\_urls}{}\end{methoddesc}
\index{has\_add\_permission() (data.admin.EnvironmentAdmin method)}

\hypertarget{data.admin.EnvironmentAdmin.has_add_permission}{}\begin{methoddesc}{has\_add\_permission}{request}
Returns True if the given request has permission to add an object.
\end{methoddesc}
\index{has\_change\_permission() (data.admin.EnvironmentAdmin method)}

\hypertarget{data.admin.EnvironmentAdmin.has_change_permission}{}\begin{methoddesc}{has\_change\_permission}{request, obj=None}
Returns True if the given request has permission to change the given
Django model instance.

If \emph{obj} is None, this should return True if the given request has
permission to change \emph{any} object of the given type.
\end{methoddesc}
\index{has\_delete\_permission() (data.admin.EnvironmentAdmin method)}

\hypertarget{data.admin.EnvironmentAdmin.has_delete_permission}{}\begin{methoddesc}{has\_delete\_permission}{request, obj=None}
Returns True if the given request has permission to change the given
Django model instance.

If \emph{obj} is None, this should return True if the given request has
permission to delete \emph{any} object of the given type.
\end{methoddesc}
\index{history\_view() (data.admin.EnvironmentAdmin method)}

\hypertarget{data.admin.EnvironmentAdmin.history_view}{}\begin{methoddesc}{history\_view}{request, object\_id, extra\_context=None}
The `history' admin view for this model.
\end{methoddesc}
\index{log\_addition() (data.admin.EnvironmentAdmin method)}

\hypertarget{data.admin.EnvironmentAdmin.log_addition}{}\begin{methoddesc}{log\_addition}{request, object}
Log that an object has been successfully added.

The default implementation creates an admin LogEntry object.
\end{methoddesc}
\index{log\_change() (data.admin.EnvironmentAdmin method)}

\hypertarget{data.admin.EnvironmentAdmin.log_change}{}\begin{methoddesc}{log\_change}{request, object, message}
Log that an object has been successfully changed.

The default implementation creates an admin LogEntry object.
\end{methoddesc}
\index{log\_deletion() (data.admin.EnvironmentAdmin method)}

\hypertarget{data.admin.EnvironmentAdmin.log_deletion}{}\begin{methoddesc}{log\_deletion}{request, object, object\_repr}
Log that an object has been successfully deleted. Note that since the
object is deleted, it might no longer be safe to call \emph{any} methods
on the object, hence this method getting object\_repr.

The default implementation creates an admin LogEntry object.
\end{methoddesc}
\index{media (data.admin.EnvironmentAdmin attribute)}

\hypertarget{data.admin.EnvironmentAdmin.media}{}\begin{memberdesc}{media}\end{memberdesc}
\index{message\_user() (data.admin.EnvironmentAdmin method)}

\hypertarget{data.admin.EnvironmentAdmin.message_user}{}\begin{methoddesc}{message\_user}{request, message}
Send a message to the user. The default implementation
posts a message using the auth Message object.
\end{methoddesc}
\index{queryset() (data.admin.EnvironmentAdmin method)}

\hypertarget{data.admin.EnvironmentAdmin.queryset}{}\begin{methoddesc}{queryset}{request}
Returns a QuerySet of all model instances that can be edited by the
admin site. This is used by changelist\_view.
\end{methoddesc}
\index{render\_change\_form() (data.admin.EnvironmentAdmin method)}

\hypertarget{data.admin.EnvironmentAdmin.render_change_form}{}\begin{methoddesc}{render\_change\_form}{request, context, add=False, change=False, form\_url='', obj=None}\end{methoddesc}
\index{response\_action() (data.admin.EnvironmentAdmin method)}

\hypertarget{data.admin.EnvironmentAdmin.response_action}{}\begin{methoddesc}{response\_action}{request, queryset}
Handle an admin action. This is called if a request is POSTed to the
changelist; it returns an HttpResponse if the action was handled, and
None otherwise.
\end{methoddesc}
\index{response\_add() (data.admin.EnvironmentAdmin method)}

\hypertarget{data.admin.EnvironmentAdmin.response_add}{}\begin{methoddesc}{response\_add}{request, obj, post\_url\_continue='../\%s/'}
Determines the HttpResponse for the add\_view stage.
\end{methoddesc}
\index{response\_change() (data.admin.EnvironmentAdmin method)}

\hypertarget{data.admin.EnvironmentAdmin.response_change}{}\begin{methoddesc}{response\_change}{request, obj}
Determines the HttpResponse for the change\_view stage.
\end{methoddesc}
\index{save\_form() (data.admin.EnvironmentAdmin method)}

\hypertarget{data.admin.EnvironmentAdmin.save_form}{}\begin{methoddesc}{save\_form}{request, form, change}
Given a ModelForm return an unsaved instance. \code{change} is True if
the object is being changed, and False if it's being added.
\end{methoddesc}
\index{save\_formset() (data.admin.EnvironmentAdmin method)}

\hypertarget{data.admin.EnvironmentAdmin.save_formset}{}\begin{methoddesc}{save\_formset}{request, form, formset, change}
Given an inline formset save it to the database.
\end{methoddesc}
\index{save\_model() (data.admin.EnvironmentAdmin method)}

\hypertarget{data.admin.EnvironmentAdmin.save_model}{}\begin{methoddesc}{save\_model}{request, obj, form, change}
Given a model instance save it to the database.
\end{methoddesc}
\index{urls (data.admin.EnvironmentAdmin attribute)}

\hypertarget{data.admin.EnvironmentAdmin.urls}{}\begin{memberdesc}{urls}\end{memberdesc}
\end{classdesc}
\index{ExperimentAdmin (class in data.admin)}

\hypertarget{data.admin.ExperimentAdmin}{}\begin{classdesc}{ExperimentAdmin}{model, admin\_site}~\index{action\_checkbox() (data.admin.ExperimentAdmin method)}

\hypertarget{data.admin.ExperimentAdmin.action_checkbox}{}\begin{methoddesc}{action\_checkbox}{obj}
A list\_display column containing a checkbox widget.
\end{methoddesc}
\index{action\_form (data.admin.ExperimentAdmin attribute)}

\hypertarget{data.admin.ExperimentAdmin.action_form}{}\begin{memberdesc}{action\_form}
alias of \code{ActionForm}
\end{memberdesc}
\index{add\_view() (data.admin.ExperimentAdmin method)}

\hypertarget{data.admin.ExperimentAdmin.add_view}{}\begin{methoddesc}{add\_view}{*args, **kw}
The `add' admin view for this model.
\end{methoddesc}
\index{change\_view() (data.admin.ExperimentAdmin method)}

\hypertarget{data.admin.ExperimentAdmin.change_view}{}\begin{methoddesc}{change\_view}{*args, **kw}
The `change' admin view for this model.
\end{methoddesc}
\index{changelist\_view() (data.admin.ExperimentAdmin method)}

\hypertarget{data.admin.ExperimentAdmin.changelist_view}{}\begin{methoddesc}{changelist\_view}{request, extra\_context=None}
The `change list' admin view for this model.
\end{methoddesc}
\index{construct\_change\_message() (data.admin.ExperimentAdmin method)}

\hypertarget{data.admin.ExperimentAdmin.construct_change_message}{}\begin{methoddesc}{construct\_change\_message}{request, form, formsets}
Construct a change message from a changed object.
\end{methoddesc}
\index{declared\_fieldsets (data.admin.ExperimentAdmin attribute)}

\hypertarget{data.admin.ExperimentAdmin.declared_fieldsets}{}\begin{memberdesc}{declared\_fieldsets}\end{memberdesc}
\index{delete\_view() (data.admin.ExperimentAdmin method)}

\hypertarget{data.admin.ExperimentAdmin.delete_view}{}\begin{methoddesc}{delete\_view}{request, object\_id, extra\_context=None}
The `delete' admin view for this model.
\end{methoddesc}
\index{form (data.admin.ExperimentAdmin attribute)}

\hypertarget{data.admin.ExperimentAdmin.form}{}\begin{memberdesc}{form}
alias of \code{ModelForm}
\end{memberdesc}
\index{formfield\_for\_choice\_field() (data.admin.ExperimentAdmin method)}

\hypertarget{data.admin.ExperimentAdmin.formfield_for_choice_field}{}\begin{methoddesc}{formfield\_for\_choice\_field}{db\_field, request=None, **kwargs}
Get a form Field for a database Field that has declared choices.
\end{methoddesc}
\index{formfield\_for\_dbfield() (data.admin.ExperimentAdmin method)}

\hypertarget{data.admin.ExperimentAdmin.formfield_for_dbfield}{}\begin{methoddesc}{formfield\_for\_dbfield}{db\_field, **kwargs}
Hook for specifying the form Field instance for a given database Field
instance.

If kwargs are given, they're passed to the form Field's constructor.
\end{methoddesc}
\index{formfield\_for\_foreignkey() (data.admin.ExperimentAdmin method)}

\hypertarget{data.admin.ExperimentAdmin.formfield_for_foreignkey}{}\begin{methoddesc}{formfield\_for\_foreignkey}{db\_field, request=None, **kwargs}
Get a form Field for a ForeignKey.
\end{methoddesc}
\index{formfield\_for\_manytomany() (data.admin.ExperimentAdmin method)}

\hypertarget{data.admin.ExperimentAdmin.formfield_for_manytomany}{}\begin{methoddesc}{formfield\_for\_manytomany}{db\_field, request=None, **kwargs}
Get a form Field for a ManyToManyField.
\end{methoddesc}
\index{get\_action() (data.admin.ExperimentAdmin method)}

\hypertarget{data.admin.ExperimentAdmin.get_action}{}\begin{methoddesc}{get\_action}{action}
Return a given action from a parameter, which can either be a callable,
or the name of a method on the ModelAdmin.  Return is a tuple of
(callable, name, description).
\end{methoddesc}
\index{get\_action\_choices() (data.admin.ExperimentAdmin method)}

\hypertarget{data.admin.ExperimentAdmin.get_action_choices}{}\begin{methoddesc}{get\_action\_choices}{request, default\_choices=, {[}('', '---------'){]}}
Return a list of choices for use in a form object.  Each choice is a
tuple (name, description).
\end{methoddesc}
\index{get\_actions() (data.admin.ExperimentAdmin method)}

\hypertarget{data.admin.ExperimentAdmin.get_actions}{}\begin{methoddesc}{get\_actions}{request}
Return a dictionary mapping the names of all actions for this
ModelAdmin to a tuple of (callable, name, description) for each action.
\end{methoddesc}
\index{get\_changelist\_form() (data.admin.ExperimentAdmin method)}

\hypertarget{data.admin.ExperimentAdmin.get_changelist_form}{}\begin{methoddesc}{get\_changelist\_form}{request, **kwargs}
Returns a Form class for use in the Formset on the changelist page.
\end{methoddesc}
\index{get\_changelist\_formset() (data.admin.ExperimentAdmin method)}

\hypertarget{data.admin.ExperimentAdmin.get_changelist_formset}{}\begin{methoddesc}{get\_changelist\_formset}{request, **kwargs}
Returns a FormSet class for use on the changelist page if list\_editable
is used.
\end{methoddesc}
\index{get\_fieldsets() (data.admin.ExperimentAdmin method)}

\hypertarget{data.admin.ExperimentAdmin.get_fieldsets}{}\begin{methoddesc}{get\_fieldsets}{request, obj=None}
Hook for specifying fieldsets for the add form.
\end{methoddesc}
\index{get\_form() (data.admin.ExperimentAdmin method)}

\hypertarget{data.admin.ExperimentAdmin.get_form}{}\begin{methoddesc}{get\_form}{request, obj=None, **kwargs}
Returns a Form class for use in the admin add view. This is used by
add\_view and change\_view.
\end{methoddesc}
\index{get\_formsets() (data.admin.ExperimentAdmin method)}

\hypertarget{data.admin.ExperimentAdmin.get_formsets}{}\begin{methoddesc}{get\_formsets}{request, obj=None}\end{methoddesc}
\index{get\_model\_perms() (data.admin.ExperimentAdmin method)}

\hypertarget{data.admin.ExperimentAdmin.get_model_perms}{}\begin{methoddesc}{get\_model\_perms}{request}
Returns a dict of all perms for this model. This dict has the keys
\code{add}, \code{change}, and \code{delete} mapping to the True/False for each
of those actions.
\end{methoddesc}
\index{get\_urls() (data.admin.ExperimentAdmin method)}

\hypertarget{data.admin.ExperimentAdmin.get_urls}{}\begin{methoddesc}{get\_urls}{}\end{methoddesc}
\index{has\_add\_permission() (data.admin.ExperimentAdmin method)}

\hypertarget{data.admin.ExperimentAdmin.has_add_permission}{}\begin{methoddesc}{has\_add\_permission}{request}
Returns True if the given request has permission to add an object.
\end{methoddesc}
\index{has\_change\_permission() (data.admin.ExperimentAdmin method)}

\hypertarget{data.admin.ExperimentAdmin.has_change_permission}{}\begin{methoddesc}{has\_change\_permission}{request, obj=None}
Returns True if the given request has permission to change the given
Django model instance.

If \emph{obj} is None, this should return True if the given request has
permission to change \emph{any} object of the given type.
\end{methoddesc}
\index{has\_delete\_permission() (data.admin.ExperimentAdmin method)}

\hypertarget{data.admin.ExperimentAdmin.has_delete_permission}{}\begin{methoddesc}{has\_delete\_permission}{request, obj=None}
Returns True if the given request has permission to change the given
Django model instance.

If \emph{obj} is None, this should return True if the given request has
permission to delete \emph{any} object of the given type.
\end{methoddesc}
\index{history\_view() (data.admin.ExperimentAdmin method)}

\hypertarget{data.admin.ExperimentAdmin.history_view}{}\begin{methoddesc}{history\_view}{request, object\_id, extra\_context=None}
The `history' admin view for this model.
\end{methoddesc}
\index{log\_addition() (data.admin.ExperimentAdmin method)}

\hypertarget{data.admin.ExperimentAdmin.log_addition}{}\begin{methoddesc}{log\_addition}{request, object}
Log that an object has been successfully added.

The default implementation creates an admin LogEntry object.
\end{methoddesc}
\index{log\_change() (data.admin.ExperimentAdmin method)}

\hypertarget{data.admin.ExperimentAdmin.log_change}{}\begin{methoddesc}{log\_change}{request, object, message}
Log that an object has been successfully changed.

The default implementation creates an admin LogEntry object.
\end{methoddesc}
\index{log\_deletion() (data.admin.ExperimentAdmin method)}

\hypertarget{data.admin.ExperimentAdmin.log_deletion}{}\begin{methoddesc}{log\_deletion}{request, object, object\_repr}
Log that an object has been successfully deleted. Note that since the
object is deleted, it might no longer be safe to call \emph{any} methods
on the object, hence this method getting object\_repr.

The default implementation creates an admin LogEntry object.
\end{methoddesc}
\index{media (data.admin.ExperimentAdmin attribute)}

\hypertarget{data.admin.ExperimentAdmin.media}{}\begin{memberdesc}{media}\end{memberdesc}
\index{message\_user() (data.admin.ExperimentAdmin method)}

\hypertarget{data.admin.ExperimentAdmin.message_user}{}\begin{methoddesc}{message\_user}{request, message}
Send a message to the user. The default implementation
posts a message using the auth Message object.
\end{methoddesc}
\index{queryset() (data.admin.ExperimentAdmin method)}

\hypertarget{data.admin.ExperimentAdmin.queryset}{}\begin{methoddesc}{queryset}{request}
Returns a QuerySet of all model instances that can be edited by the
admin site. This is used by changelist\_view.
\end{methoddesc}
\index{render\_change\_form() (data.admin.ExperimentAdmin method)}

\hypertarget{data.admin.ExperimentAdmin.render_change_form}{}\begin{methoddesc}{render\_change\_form}{request, context, add=False, change=False, form\_url='', obj=None}\end{methoddesc}
\index{response\_action() (data.admin.ExperimentAdmin method)}

\hypertarget{data.admin.ExperimentAdmin.response_action}{}\begin{methoddesc}{response\_action}{request, queryset}
Handle an admin action. This is called if a request is POSTed to the
changelist; it returns an HttpResponse if the action was handled, and
None otherwise.
\end{methoddesc}
\index{response\_add() (data.admin.ExperimentAdmin method)}

\hypertarget{data.admin.ExperimentAdmin.response_add}{}\begin{methoddesc}{response\_add}{request, obj, post\_url\_continue='../\%s/'}
Determines the HttpResponse for the add\_view stage.
\end{methoddesc}
\index{response\_change() (data.admin.ExperimentAdmin method)}

\hypertarget{data.admin.ExperimentAdmin.response_change}{}\begin{methoddesc}{response\_change}{request, obj}
Determines the HttpResponse for the change\_view stage.
\end{methoddesc}
\index{save\_form() (data.admin.ExperimentAdmin method)}

\hypertarget{data.admin.ExperimentAdmin.save_form}{}\begin{methoddesc}{save\_form}{request, form, change}
Given a ModelForm return an unsaved instance. \code{change} is True if
the object is being changed, and False if it's being added.
\end{methoddesc}
\index{save\_formset() (data.admin.ExperimentAdmin method)}

\hypertarget{data.admin.ExperimentAdmin.save_formset}{}\begin{methoddesc}{save\_formset}{request, form, formset, change}
Given an inline formset save it to the database.
\end{methoddesc}
\index{save\_model() (data.admin.ExperimentAdmin method)}

\hypertarget{data.admin.ExperimentAdmin.save_model}{}\begin{methoddesc}{save\_model}{request, obj, form, change}
Given a model instance save it to the database.
\end{methoddesc}
\index{urls (data.admin.ExperimentAdmin attribute)}

\hypertarget{data.admin.ExperimentAdmin.urls}{}\begin{memberdesc}{urls}\end{memberdesc}
\end{classdesc}
\index{ImplantationAdmin (class in data.admin)}

\hypertarget{data.admin.ImplantationAdmin}{}\begin{classdesc}{ImplantationAdmin}{model, admin\_site}~\index{action\_checkbox() (data.admin.ImplantationAdmin method)}

\hypertarget{data.admin.ImplantationAdmin.action_checkbox}{}\begin{methoddesc}{action\_checkbox}{obj}
A list\_display column containing a checkbox widget.
\end{methoddesc}
\index{action\_form (data.admin.ImplantationAdmin attribute)}

\hypertarget{data.admin.ImplantationAdmin.action_form}{}\begin{memberdesc}{action\_form}
alias of \code{ActionForm}
\end{memberdesc}
\index{add\_view() (data.admin.ImplantationAdmin method)}

\hypertarget{data.admin.ImplantationAdmin.add_view}{}\begin{methoddesc}{add\_view}{*args, **kw}
The `add' admin view for this model.
\end{methoddesc}
\index{change\_view() (data.admin.ImplantationAdmin method)}

\hypertarget{data.admin.ImplantationAdmin.change_view}{}\begin{methoddesc}{change\_view}{*args, **kw}
The `change' admin view for this model.
\end{methoddesc}
\index{changelist\_view() (data.admin.ImplantationAdmin method)}

\hypertarget{data.admin.ImplantationAdmin.changelist_view}{}\begin{methoddesc}{changelist\_view}{request, extra\_context=None}
The `change list' admin view for this model.
\end{methoddesc}
\index{construct\_change\_message() (data.admin.ImplantationAdmin method)}

\hypertarget{data.admin.ImplantationAdmin.construct_change_message}{}\begin{methoddesc}{construct\_change\_message}{request, form, formsets}
Construct a change message from a changed object.
\end{methoddesc}
\index{declared\_fieldsets (data.admin.ImplantationAdmin attribute)}

\hypertarget{data.admin.ImplantationAdmin.declared_fieldsets}{}\begin{memberdesc}{declared\_fieldsets}\end{memberdesc}
\index{delete\_view() (data.admin.ImplantationAdmin method)}

\hypertarget{data.admin.ImplantationAdmin.delete_view}{}\begin{methoddesc}{delete\_view}{request, object\_id, extra\_context=None}
The `delete' admin view for this model.
\end{methoddesc}
\index{form (data.admin.ImplantationAdmin attribute)}

\hypertarget{data.admin.ImplantationAdmin.form}{}\begin{memberdesc}{form}
alias of \code{ModelForm}
\end{memberdesc}
\index{formfield\_for\_choice\_field() (data.admin.ImplantationAdmin method)}

\hypertarget{data.admin.ImplantationAdmin.formfield_for_choice_field}{}\begin{methoddesc}{formfield\_for\_choice\_field}{db\_field, request=None, **kwargs}
Get a form Field for a database Field that has declared choices.
\end{methoddesc}
\index{formfield\_for\_dbfield() (data.admin.ImplantationAdmin method)}

\hypertarget{data.admin.ImplantationAdmin.formfield_for_dbfield}{}\begin{methoddesc}{formfield\_for\_dbfield}{db\_field, **kwargs}
Hook for specifying the form Field instance for a given database Field
instance.

If kwargs are given, they're passed to the form Field's constructor.
\end{methoddesc}
\index{formfield\_for\_foreignkey() (data.admin.ImplantationAdmin method)}

\hypertarget{data.admin.ImplantationAdmin.formfield_for_foreignkey}{}\begin{methoddesc}{formfield\_for\_foreignkey}{db\_field, request=None, **kwargs}
Get a form Field for a ForeignKey.
\end{methoddesc}
\index{formfield\_for\_manytomany() (data.admin.ImplantationAdmin method)}

\hypertarget{data.admin.ImplantationAdmin.formfield_for_manytomany}{}\begin{methoddesc}{formfield\_for\_manytomany}{db\_field, request=None, **kwargs}
Get a form Field for a ManyToManyField.
\end{methoddesc}
\index{get\_action() (data.admin.ImplantationAdmin method)}

\hypertarget{data.admin.ImplantationAdmin.get_action}{}\begin{methoddesc}{get\_action}{action}
Return a given action from a parameter, which can either be a callable,
or the name of a method on the ModelAdmin.  Return is a tuple of
(callable, name, description).
\end{methoddesc}
\index{get\_action\_choices() (data.admin.ImplantationAdmin method)}

\hypertarget{data.admin.ImplantationAdmin.get_action_choices}{}\begin{methoddesc}{get\_action\_choices}{request, default\_choices=, {[}('', '---------'){]}}
Return a list of choices for use in a form object.  Each choice is a
tuple (name, description).
\end{methoddesc}
\index{get\_actions() (data.admin.ImplantationAdmin method)}

\hypertarget{data.admin.ImplantationAdmin.get_actions}{}\begin{methoddesc}{get\_actions}{request}
Return a dictionary mapping the names of all actions for this
ModelAdmin to a tuple of (callable, name, description) for each action.
\end{methoddesc}
\index{get\_changelist\_form() (data.admin.ImplantationAdmin method)}

\hypertarget{data.admin.ImplantationAdmin.get_changelist_form}{}\begin{methoddesc}{get\_changelist\_form}{request, **kwargs}
Returns a Form class for use in the Formset on the changelist page.
\end{methoddesc}
\index{get\_changelist\_formset() (data.admin.ImplantationAdmin method)}

\hypertarget{data.admin.ImplantationAdmin.get_changelist_formset}{}\begin{methoddesc}{get\_changelist\_formset}{request, **kwargs}
Returns a FormSet class for use on the changelist page if list\_editable
is used.
\end{methoddesc}
\index{get\_fieldsets() (data.admin.ImplantationAdmin method)}

\hypertarget{data.admin.ImplantationAdmin.get_fieldsets}{}\begin{methoddesc}{get\_fieldsets}{request, obj=None}
Hook for specifying fieldsets for the add form.
\end{methoddesc}
\index{get\_form() (data.admin.ImplantationAdmin method)}

\hypertarget{data.admin.ImplantationAdmin.get_form}{}\begin{methoddesc}{get\_form}{request, obj=None, **kwargs}
Returns a Form class for use in the admin add view. This is used by
add\_view and change\_view.
\end{methoddesc}
\index{get\_formsets() (data.admin.ImplantationAdmin method)}

\hypertarget{data.admin.ImplantationAdmin.get_formsets}{}\begin{methoddesc}{get\_formsets}{request, obj=None}\end{methoddesc}
\index{get\_model\_perms() (data.admin.ImplantationAdmin method)}

\hypertarget{data.admin.ImplantationAdmin.get_model_perms}{}\begin{methoddesc}{get\_model\_perms}{request}
Returns a dict of all perms for this model. This dict has the keys
\code{add}, \code{change}, and \code{delete} mapping to the True/False for each
of those actions.
\end{methoddesc}
\index{get\_urls() (data.admin.ImplantationAdmin method)}

\hypertarget{data.admin.ImplantationAdmin.get_urls}{}\begin{methoddesc}{get\_urls}{}\end{methoddesc}
\index{has\_add\_permission() (data.admin.ImplantationAdmin method)}

\hypertarget{data.admin.ImplantationAdmin.has_add_permission}{}\begin{methoddesc}{has\_add\_permission}{request}
Returns True if the given request has permission to add an object.
\end{methoddesc}
\index{has\_change\_permission() (data.admin.ImplantationAdmin method)}

\hypertarget{data.admin.ImplantationAdmin.has_change_permission}{}\begin{methoddesc}{has\_change\_permission}{request, obj=None}
Returns True if the given request has permission to change the given
Django model instance.

If \emph{obj} is None, this should return True if the given request has
permission to change \emph{any} object of the given type.
\end{methoddesc}
\index{has\_delete\_permission() (data.admin.ImplantationAdmin method)}

\hypertarget{data.admin.ImplantationAdmin.has_delete_permission}{}\begin{methoddesc}{has\_delete\_permission}{request, obj=None}
Returns True if the given request has permission to change the given
Django model instance.

If \emph{obj} is None, this should return True if the given request has
permission to delete \emph{any} object of the given type.
\end{methoddesc}
\index{history\_view() (data.admin.ImplantationAdmin method)}

\hypertarget{data.admin.ImplantationAdmin.history_view}{}\begin{methoddesc}{history\_view}{request, object\_id, extra\_context=None}
The `history' admin view for this model.
\end{methoddesc}
\index{log\_addition() (data.admin.ImplantationAdmin method)}

\hypertarget{data.admin.ImplantationAdmin.log_addition}{}\begin{methoddesc}{log\_addition}{request, object}
Log that an object has been successfully added.

The default implementation creates an admin LogEntry object.
\end{methoddesc}
\index{log\_change() (data.admin.ImplantationAdmin method)}

\hypertarget{data.admin.ImplantationAdmin.log_change}{}\begin{methoddesc}{log\_change}{request, object, message}
Log that an object has been successfully changed.

The default implementation creates an admin LogEntry object.
\end{methoddesc}
\index{log\_deletion() (data.admin.ImplantationAdmin method)}

\hypertarget{data.admin.ImplantationAdmin.log_deletion}{}\begin{methoddesc}{log\_deletion}{request, object, object\_repr}
Log that an object has been successfully deleted. Note that since the
object is deleted, it might no longer be safe to call \emph{any} methods
on the object, hence this method getting object\_repr.

The default implementation creates an admin LogEntry object.
\end{methoddesc}
\index{media (data.admin.ImplantationAdmin attribute)}

\hypertarget{data.admin.ImplantationAdmin.media}{}\begin{memberdesc}{media}\end{memberdesc}
\index{message\_user() (data.admin.ImplantationAdmin method)}

\hypertarget{data.admin.ImplantationAdmin.message_user}{}\begin{methoddesc}{message\_user}{request, message}
Send a message to the user. The default implementation
posts a message using the auth Message object.
\end{methoddesc}
\index{queryset() (data.admin.ImplantationAdmin method)}

\hypertarget{data.admin.ImplantationAdmin.queryset}{}\begin{methoddesc}{queryset}{request}
Returns a QuerySet of all model instances that can be edited by the
admin site. This is used by changelist\_view.
\end{methoddesc}
\index{render\_change\_form() (data.admin.ImplantationAdmin method)}

\hypertarget{data.admin.ImplantationAdmin.render_change_form}{}\begin{methoddesc}{render\_change\_form}{request, context, add=False, change=False, form\_url='', obj=None}\end{methoddesc}
\index{response\_action() (data.admin.ImplantationAdmin method)}

\hypertarget{data.admin.ImplantationAdmin.response_action}{}\begin{methoddesc}{response\_action}{request, queryset}
Handle an admin action. This is called if a request is POSTed to the
changelist; it returns an HttpResponse if the action was handled, and
None otherwise.
\end{methoddesc}
\index{response\_add() (data.admin.ImplantationAdmin method)}

\hypertarget{data.admin.ImplantationAdmin.response_add}{}\begin{methoddesc}{response\_add}{request, obj, post\_url\_continue='../\%s/'}
Determines the HttpResponse for the add\_view stage.
\end{methoddesc}
\index{response\_change() (data.admin.ImplantationAdmin method)}

\hypertarget{data.admin.ImplantationAdmin.response_change}{}\begin{methoddesc}{response\_change}{request, obj}
Determines the HttpResponse for the change\_view stage.
\end{methoddesc}
\index{save\_form() (data.admin.ImplantationAdmin method)}

\hypertarget{data.admin.ImplantationAdmin.save_form}{}\begin{methoddesc}{save\_form}{request, form, change}
Given a ModelForm return an unsaved instance. \code{change} is True if
the object is being changed, and False if it's being added.
\end{methoddesc}
\index{save\_formset() (data.admin.ImplantationAdmin method)}

\hypertarget{data.admin.ImplantationAdmin.save_formset}{}\begin{methoddesc}{save\_formset}{request, form, formset, change}
Given an inline formset save it to the database.
\end{methoddesc}
\index{save\_model() (data.admin.ImplantationAdmin method)}

\hypertarget{data.admin.ImplantationAdmin.save_model}{}\begin{methoddesc}{save\_model}{request, obj, form, change}
Given a model instance save it to the database.
\end{methoddesc}
\index{urls (data.admin.ImplantationAdmin attribute)}

\hypertarget{data.admin.ImplantationAdmin.urls}{}\begin{memberdesc}{urls}\end{memberdesc}
\end{classdesc}
\index{MeasurementAdmin (class in data.admin)}

\hypertarget{data.admin.MeasurementAdmin}{}\begin{classdesc}{MeasurementAdmin}{model, admin\_site}~\index{action\_checkbox() (data.admin.MeasurementAdmin method)}

\hypertarget{data.admin.MeasurementAdmin.action_checkbox}{}\begin{methoddesc}{action\_checkbox}{obj}
A list\_display column containing a checkbox widget.
\end{methoddesc}
\index{action\_form (data.admin.MeasurementAdmin attribute)}

\hypertarget{data.admin.MeasurementAdmin.action_form}{}\begin{memberdesc}{action\_form}
alias of \code{ActionForm}
\end{memberdesc}
\index{add\_view() (data.admin.MeasurementAdmin method)}

\hypertarget{data.admin.MeasurementAdmin.add_view}{}\begin{methoddesc}{add\_view}{*args, **kw}
The `add' admin view for this model.
\end{methoddesc}
\index{change\_view() (data.admin.MeasurementAdmin method)}

\hypertarget{data.admin.MeasurementAdmin.change_view}{}\begin{methoddesc}{change\_view}{*args, **kw}
The `change' admin view for this model.
\end{methoddesc}
\index{changelist\_view() (data.admin.MeasurementAdmin method)}

\hypertarget{data.admin.MeasurementAdmin.changelist_view}{}\begin{methoddesc}{changelist\_view}{request, extra\_context=None}
The `change list' admin view for this model.
\end{methoddesc}
\index{construct\_change\_message() (data.admin.MeasurementAdmin method)}

\hypertarget{data.admin.MeasurementAdmin.construct_change_message}{}\begin{methoddesc}{construct\_change\_message}{request, form, formsets}
Construct a change message from a changed object.
\end{methoddesc}
\index{declared\_fieldsets (data.admin.MeasurementAdmin attribute)}

\hypertarget{data.admin.MeasurementAdmin.declared_fieldsets}{}\begin{memberdesc}{declared\_fieldsets}\end{memberdesc}
\index{delete\_view() (data.admin.MeasurementAdmin method)}

\hypertarget{data.admin.MeasurementAdmin.delete_view}{}\begin{methoddesc}{delete\_view}{request, object\_id, extra\_context=None}
The `delete' admin view for this model.
\end{methoddesc}
\index{form (data.admin.MeasurementAdmin attribute)}

\hypertarget{data.admin.MeasurementAdmin.form}{}\begin{memberdesc}{form}
alias of \code{ModelForm}
\end{memberdesc}
\index{formfield\_for\_choice\_field() (data.admin.MeasurementAdmin method)}

\hypertarget{data.admin.MeasurementAdmin.formfield_for_choice_field}{}\begin{methoddesc}{formfield\_for\_choice\_field}{db\_field, request=None, **kwargs}
Get a form Field for a database Field that has declared choices.
\end{methoddesc}
\index{formfield\_for\_dbfield() (data.admin.MeasurementAdmin method)}

\hypertarget{data.admin.MeasurementAdmin.formfield_for_dbfield}{}\begin{methoddesc}{formfield\_for\_dbfield}{db\_field, **kwargs}
Hook for specifying the form Field instance for a given database Field
instance.

If kwargs are given, they're passed to the form Field's constructor.
\end{methoddesc}
\index{formfield\_for\_foreignkey() (data.admin.MeasurementAdmin method)}

\hypertarget{data.admin.MeasurementAdmin.formfield_for_foreignkey}{}\begin{methoddesc}{formfield\_for\_foreignkey}{db\_field, request=None, **kwargs}
Get a form Field for a ForeignKey.
\end{methoddesc}
\index{formfield\_for\_manytomany() (data.admin.MeasurementAdmin method)}

\hypertarget{data.admin.MeasurementAdmin.formfield_for_manytomany}{}\begin{methoddesc}{formfield\_for\_manytomany}{db\_field, request=None, **kwargs}
Get a form Field for a ManyToManyField.
\end{methoddesc}
\index{get\_action() (data.admin.MeasurementAdmin method)}

\hypertarget{data.admin.MeasurementAdmin.get_action}{}\begin{methoddesc}{get\_action}{action}
Return a given action from a parameter, which can either be a callable,
or the name of a method on the ModelAdmin.  Return is a tuple of
(callable, name, description).
\end{methoddesc}
\index{get\_action\_choices() (data.admin.MeasurementAdmin method)}

\hypertarget{data.admin.MeasurementAdmin.get_action_choices}{}\begin{methoddesc}{get\_action\_choices}{request, default\_choices=, {[}('', '---------'){]}}
Return a list of choices for use in a form object.  Each choice is a
tuple (name, description).
\end{methoddesc}
\index{get\_actions() (data.admin.MeasurementAdmin method)}

\hypertarget{data.admin.MeasurementAdmin.get_actions}{}\begin{methoddesc}{get\_actions}{request}
Return a dictionary mapping the names of all actions for this
ModelAdmin to a tuple of (callable, name, description) for each action.
\end{methoddesc}
\index{get\_changelist\_form() (data.admin.MeasurementAdmin method)}

\hypertarget{data.admin.MeasurementAdmin.get_changelist_form}{}\begin{methoddesc}{get\_changelist\_form}{request, **kwargs}
Returns a Form class for use in the Formset on the changelist page.
\end{methoddesc}
\index{get\_changelist\_formset() (data.admin.MeasurementAdmin method)}

\hypertarget{data.admin.MeasurementAdmin.get_changelist_formset}{}\begin{methoddesc}{get\_changelist\_formset}{request, **kwargs}
Returns a FormSet class for use on the changelist page if list\_editable
is used.
\end{methoddesc}
\index{get\_fieldsets() (data.admin.MeasurementAdmin method)}

\hypertarget{data.admin.MeasurementAdmin.get_fieldsets}{}\begin{methoddesc}{get\_fieldsets}{request, obj=None}
Hook for specifying fieldsets for the add form.
\end{methoddesc}
\index{get\_form() (data.admin.MeasurementAdmin method)}

\hypertarget{data.admin.MeasurementAdmin.get_form}{}\begin{methoddesc}{get\_form}{request, obj=None, **kwargs}
Returns a Form class for use in the admin add view. This is used by
add\_view and change\_view.
\end{methoddesc}
\index{get\_formsets() (data.admin.MeasurementAdmin method)}

\hypertarget{data.admin.MeasurementAdmin.get_formsets}{}\begin{methoddesc}{get\_formsets}{request, obj=None}\end{methoddesc}
\index{get\_model\_perms() (data.admin.MeasurementAdmin method)}

\hypertarget{data.admin.MeasurementAdmin.get_model_perms}{}\begin{methoddesc}{get\_model\_perms}{request}
Returns a dict of all perms for this model. This dict has the keys
\code{add}, \code{change}, and \code{delete} mapping to the True/False for each
of those actions.
\end{methoddesc}
\index{get\_urls() (data.admin.MeasurementAdmin method)}

\hypertarget{data.admin.MeasurementAdmin.get_urls}{}\begin{methoddesc}{get\_urls}{}\end{methoddesc}
\index{has\_add\_permission() (data.admin.MeasurementAdmin method)}

\hypertarget{data.admin.MeasurementAdmin.has_add_permission}{}\begin{methoddesc}{has\_add\_permission}{request}
Returns True if the given request has permission to add an object.
\end{methoddesc}
\index{has\_change\_permission() (data.admin.MeasurementAdmin method)}

\hypertarget{data.admin.MeasurementAdmin.has_change_permission}{}\begin{methoddesc}{has\_change\_permission}{request, obj=None}
Returns True if the given request has permission to change the given
Django model instance.

If \emph{obj} is None, this should return True if the given request has
permission to change \emph{any} object of the given type.
\end{methoddesc}
\index{has\_delete\_permission() (data.admin.MeasurementAdmin method)}

\hypertarget{data.admin.MeasurementAdmin.has_delete_permission}{}\begin{methoddesc}{has\_delete\_permission}{request, obj=None}
Returns True if the given request has permission to change the given
Django model instance.

If \emph{obj} is None, this should return True if the given request has
permission to delete \emph{any} object of the given type.
\end{methoddesc}
\index{history\_view() (data.admin.MeasurementAdmin method)}

\hypertarget{data.admin.MeasurementAdmin.history_view}{}\begin{methoddesc}{history\_view}{request, object\_id, extra\_context=None}
The `history' admin view for this model.
\end{methoddesc}
\index{log\_addition() (data.admin.MeasurementAdmin method)}

\hypertarget{data.admin.MeasurementAdmin.log_addition}{}\begin{methoddesc}{log\_addition}{request, object}
Log that an object has been successfully added.

The default implementation creates an admin LogEntry object.
\end{methoddesc}
\index{log\_change() (data.admin.MeasurementAdmin method)}

\hypertarget{data.admin.MeasurementAdmin.log_change}{}\begin{methoddesc}{log\_change}{request, object, message}
Log that an object has been successfully changed.

The default implementation creates an admin LogEntry object.
\end{methoddesc}
\index{log\_deletion() (data.admin.MeasurementAdmin method)}

\hypertarget{data.admin.MeasurementAdmin.log_deletion}{}\begin{methoddesc}{log\_deletion}{request, object, object\_repr}
Log that an object has been successfully deleted. Note that since the
object is deleted, it might no longer be safe to call \emph{any} methods
on the object, hence this method getting object\_repr.

The default implementation creates an admin LogEntry object.
\end{methoddesc}
\index{media (data.admin.MeasurementAdmin attribute)}

\hypertarget{data.admin.MeasurementAdmin.media}{}\begin{memberdesc}{media}\end{memberdesc}
\index{message\_user() (data.admin.MeasurementAdmin method)}

\hypertarget{data.admin.MeasurementAdmin.message_user}{}\begin{methoddesc}{message\_user}{request, message}
Send a message to the user. The default implementation
posts a message using the auth Message object.
\end{methoddesc}
\index{queryset() (data.admin.MeasurementAdmin method)}

\hypertarget{data.admin.MeasurementAdmin.queryset}{}\begin{methoddesc}{queryset}{request}
Returns a QuerySet of all model instances that can be edited by the
admin site. This is used by changelist\_view.
\end{methoddesc}
\index{render\_change\_form() (data.admin.MeasurementAdmin method)}

\hypertarget{data.admin.MeasurementAdmin.render_change_form}{}\begin{methoddesc}{render\_change\_form}{request, context, add=False, change=False, form\_url='', obj=None}\end{methoddesc}
\index{response\_action() (data.admin.MeasurementAdmin method)}

\hypertarget{data.admin.MeasurementAdmin.response_action}{}\begin{methoddesc}{response\_action}{request, queryset}
Handle an admin action. This is called if a request is POSTed to the
changelist; it returns an HttpResponse if the action was handled, and
None otherwise.
\end{methoddesc}
\index{response\_add() (data.admin.MeasurementAdmin method)}

\hypertarget{data.admin.MeasurementAdmin.response_add}{}\begin{methoddesc}{response\_add}{request, obj, post\_url\_continue='../\%s/'}
Determines the HttpResponse for the add\_view stage.
\end{methoddesc}
\index{response\_change() (data.admin.MeasurementAdmin method)}

\hypertarget{data.admin.MeasurementAdmin.response_change}{}\begin{methoddesc}{response\_change}{request, obj}
Determines the HttpResponse for the change\_view stage.
\end{methoddesc}
\index{save\_form() (data.admin.MeasurementAdmin method)}

\hypertarget{data.admin.MeasurementAdmin.save_form}{}\begin{methoddesc}{save\_form}{request, form, change}
Given a ModelForm return an unsaved instance. \code{change} is True if
the object is being changed, and False if it's being added.
\end{methoddesc}
\index{save\_formset() (data.admin.MeasurementAdmin method)}

\hypertarget{data.admin.MeasurementAdmin.save_formset}{}\begin{methoddesc}{save\_formset}{request, form, formset, change}
Given an inline formset save it to the database.
\end{methoddesc}
\index{save\_model() (data.admin.MeasurementAdmin method)}

\hypertarget{data.admin.MeasurementAdmin.save_model}{}\begin{methoddesc}{save\_model}{request, obj, form, change}
Given a model instance save it to the database.
\end{methoddesc}
\index{urls (data.admin.MeasurementAdmin attribute)}

\hypertarget{data.admin.MeasurementAdmin.urls}{}\begin{memberdesc}{urls}\end{memberdesc}
\end{classdesc}
\index{MeasurementInline (class in data.admin)}

\hypertarget{data.admin.MeasurementInline}{}\begin{classdesc}{MeasurementInline}{parent\_model, admin\_site}~\index{declared\_fieldsets (data.admin.MeasurementInline attribute)}

\hypertarget{data.admin.MeasurementInline.declared_fieldsets}{}\begin{memberdesc}{declared\_fieldsets}\end{memberdesc}
\index{form (data.admin.MeasurementInline attribute)}

\hypertarget{data.admin.MeasurementInline.form}{}\begin{memberdesc}{form}
alias of \code{ModelForm}
\end{memberdesc}
\index{formfield\_for\_choice\_field() (data.admin.MeasurementInline method)}

\hypertarget{data.admin.MeasurementInline.formfield_for_choice_field}{}\begin{methoddesc}{formfield\_for\_choice\_field}{db\_field, request=None, **kwargs}
Get a form Field for a database Field that has declared choices.
\end{methoddesc}
\index{formfield\_for\_dbfield() (data.admin.MeasurementInline method)}

\hypertarget{data.admin.MeasurementInline.formfield_for_dbfield}{}\begin{methoddesc}{formfield\_for\_dbfield}{db\_field, **kwargs}
Hook for specifying the form Field instance for a given database Field
instance.

If kwargs are given, they're passed to the form Field's constructor.
\end{methoddesc}
\index{formfield\_for\_foreignkey() (data.admin.MeasurementInline method)}

\hypertarget{data.admin.MeasurementInline.formfield_for_foreignkey}{}\begin{methoddesc}{formfield\_for\_foreignkey}{db\_field, request=None, **kwargs}
Get a form Field for a ForeignKey.
\end{methoddesc}
\index{formfield\_for\_manytomany() (data.admin.MeasurementInline method)}

\hypertarget{data.admin.MeasurementInline.formfield_for_manytomany}{}\begin{methoddesc}{formfield\_for\_manytomany}{db\_field, request=None, **kwargs}
Get a form Field for a ManyToManyField.
\end{methoddesc}
\index{formset (data.admin.MeasurementInline attribute)}

\hypertarget{data.admin.MeasurementInline.formset}{}\begin{memberdesc}{formset}
alias of \code{BaseInlineFormSet}
\end{memberdesc}
\index{get\_fieldsets() (data.admin.MeasurementInline method)}

\hypertarget{data.admin.MeasurementInline.get_fieldsets}{}\begin{methoddesc}{get\_fieldsets}{request, obj=None}\end{methoddesc}
\index{get\_formset() (data.admin.MeasurementInline method)}

\hypertarget{data.admin.MeasurementInline.get_formset}{}\begin{methoddesc}{get\_formset}{request, obj=None, **kwargs}
Returns a BaseInlineFormSet class for use in admin add/change views.
\end{methoddesc}
\index{media (data.admin.MeasurementInline attribute)}

\hypertarget{data.admin.MeasurementInline.media}{}\begin{memberdesc}{media}\end{memberdesc}
\index{model (data.admin.MeasurementInline attribute)}

\hypertarget{data.admin.MeasurementInline.model}{}\begin{memberdesc}{model}
alias of \code{Measurement}
\end{memberdesc}
\end{classdesc}
\index{PharmaceuticalAdmin (class in data.admin)}

\hypertarget{data.admin.PharmaceuticalAdmin}{}\begin{classdesc}{PharmaceuticalAdmin}{model, admin\_site}~\index{action\_checkbox() (data.admin.PharmaceuticalAdmin method)}

\hypertarget{data.admin.PharmaceuticalAdmin.action_checkbox}{}\begin{methoddesc}{action\_checkbox}{obj}
A list\_display column containing a checkbox widget.
\end{methoddesc}
\index{action\_form (data.admin.PharmaceuticalAdmin attribute)}

\hypertarget{data.admin.PharmaceuticalAdmin.action_form}{}\begin{memberdesc}{action\_form}
alias of \code{ActionForm}
\end{memberdesc}
\index{add\_view() (data.admin.PharmaceuticalAdmin method)}

\hypertarget{data.admin.PharmaceuticalAdmin.add_view}{}\begin{methoddesc}{add\_view}{*args, **kw}
The `add' admin view for this model.
\end{methoddesc}
\index{change\_view() (data.admin.PharmaceuticalAdmin method)}

\hypertarget{data.admin.PharmaceuticalAdmin.change_view}{}\begin{methoddesc}{change\_view}{*args, **kw}
The `change' admin view for this model.
\end{methoddesc}
\index{changelist\_view() (data.admin.PharmaceuticalAdmin method)}

\hypertarget{data.admin.PharmaceuticalAdmin.changelist_view}{}\begin{methoddesc}{changelist\_view}{request, extra\_context=None}
The `change list' admin view for this model.
\end{methoddesc}
\index{construct\_change\_message() (data.admin.PharmaceuticalAdmin method)}

\hypertarget{data.admin.PharmaceuticalAdmin.construct_change_message}{}\begin{methoddesc}{construct\_change\_message}{request, form, formsets}
Construct a change message from a changed object.
\end{methoddesc}
\index{declared\_fieldsets (data.admin.PharmaceuticalAdmin attribute)}

\hypertarget{data.admin.PharmaceuticalAdmin.declared_fieldsets}{}\begin{memberdesc}{declared\_fieldsets}\end{memberdesc}
\index{delete\_view() (data.admin.PharmaceuticalAdmin method)}

\hypertarget{data.admin.PharmaceuticalAdmin.delete_view}{}\begin{methoddesc}{delete\_view}{request, object\_id, extra\_context=None}
The `delete' admin view for this model.
\end{methoddesc}
\index{form (data.admin.PharmaceuticalAdmin attribute)}

\hypertarget{data.admin.PharmaceuticalAdmin.form}{}\begin{memberdesc}{form}
alias of \code{ModelForm}
\end{memberdesc}
\index{formfield\_for\_choice\_field() (data.admin.PharmaceuticalAdmin method)}

\hypertarget{data.admin.PharmaceuticalAdmin.formfield_for_choice_field}{}\begin{methoddesc}{formfield\_for\_choice\_field}{db\_field, request=None, **kwargs}
Get a form Field for a database Field that has declared choices.
\end{methoddesc}
\index{formfield\_for\_dbfield() (data.admin.PharmaceuticalAdmin method)}

\hypertarget{data.admin.PharmaceuticalAdmin.formfield_for_dbfield}{}\begin{methoddesc}{formfield\_for\_dbfield}{db\_field, **kwargs}
Hook for specifying the form Field instance for a given database Field
instance.

If kwargs are given, they're passed to the form Field's constructor.
\end{methoddesc}
\index{formfield\_for\_foreignkey() (data.admin.PharmaceuticalAdmin method)}

\hypertarget{data.admin.PharmaceuticalAdmin.formfield_for_foreignkey}{}\begin{methoddesc}{formfield\_for\_foreignkey}{db\_field, request=None, **kwargs}
Get a form Field for a ForeignKey.
\end{methoddesc}
\index{formfield\_for\_manytomany() (data.admin.PharmaceuticalAdmin method)}

\hypertarget{data.admin.PharmaceuticalAdmin.formfield_for_manytomany}{}\begin{methoddesc}{formfield\_for\_manytomany}{db\_field, request=None, **kwargs}
Get a form Field for a ManyToManyField.
\end{methoddesc}
\index{get\_action() (data.admin.PharmaceuticalAdmin method)}

\hypertarget{data.admin.PharmaceuticalAdmin.get_action}{}\begin{methoddesc}{get\_action}{action}
Return a given action from a parameter, which can either be a callable,
or the name of a method on the ModelAdmin.  Return is a tuple of
(callable, name, description).
\end{methoddesc}
\index{get\_action\_choices() (data.admin.PharmaceuticalAdmin method)}

\hypertarget{data.admin.PharmaceuticalAdmin.get_action_choices}{}\begin{methoddesc}{get\_action\_choices}{request, default\_choices=, {[}('', '---------'){]}}
Return a list of choices for use in a form object.  Each choice is a
tuple (name, description).
\end{methoddesc}
\index{get\_actions() (data.admin.PharmaceuticalAdmin method)}

\hypertarget{data.admin.PharmaceuticalAdmin.get_actions}{}\begin{methoddesc}{get\_actions}{request}
Return a dictionary mapping the names of all actions for this
ModelAdmin to a tuple of (callable, name, description) for each action.
\end{methoddesc}
\index{get\_changelist\_form() (data.admin.PharmaceuticalAdmin method)}

\hypertarget{data.admin.PharmaceuticalAdmin.get_changelist_form}{}\begin{methoddesc}{get\_changelist\_form}{request, **kwargs}
Returns a Form class for use in the Formset on the changelist page.
\end{methoddesc}
\index{get\_changelist\_formset() (data.admin.PharmaceuticalAdmin method)}

\hypertarget{data.admin.PharmaceuticalAdmin.get_changelist_formset}{}\begin{methoddesc}{get\_changelist\_formset}{request, **kwargs}
Returns a FormSet class for use on the changelist page if list\_editable
is used.
\end{methoddesc}
\index{get\_fieldsets() (data.admin.PharmaceuticalAdmin method)}

\hypertarget{data.admin.PharmaceuticalAdmin.get_fieldsets}{}\begin{methoddesc}{get\_fieldsets}{request, obj=None}
Hook for specifying fieldsets for the add form.
\end{methoddesc}
\index{get\_form() (data.admin.PharmaceuticalAdmin method)}

\hypertarget{data.admin.PharmaceuticalAdmin.get_form}{}\begin{methoddesc}{get\_form}{request, obj=None, **kwargs}
Returns a Form class for use in the admin add view. This is used by
add\_view and change\_view.
\end{methoddesc}
\index{get\_formsets() (data.admin.PharmaceuticalAdmin method)}

\hypertarget{data.admin.PharmaceuticalAdmin.get_formsets}{}\begin{methoddesc}{get\_formsets}{request, obj=None}\end{methoddesc}
\index{get\_model\_perms() (data.admin.PharmaceuticalAdmin method)}

\hypertarget{data.admin.PharmaceuticalAdmin.get_model_perms}{}\begin{methoddesc}{get\_model\_perms}{request}
Returns a dict of all perms for this model. This dict has the keys
\code{add}, \code{change}, and \code{delete} mapping to the True/False for each
of those actions.
\end{methoddesc}
\index{get\_urls() (data.admin.PharmaceuticalAdmin method)}

\hypertarget{data.admin.PharmaceuticalAdmin.get_urls}{}\begin{methoddesc}{get\_urls}{}\end{methoddesc}
\index{has\_add\_permission() (data.admin.PharmaceuticalAdmin method)}

\hypertarget{data.admin.PharmaceuticalAdmin.has_add_permission}{}\begin{methoddesc}{has\_add\_permission}{request}
Returns True if the given request has permission to add an object.
\end{methoddesc}
\index{has\_change\_permission() (data.admin.PharmaceuticalAdmin method)}

\hypertarget{data.admin.PharmaceuticalAdmin.has_change_permission}{}\begin{methoddesc}{has\_change\_permission}{request, obj=None}
Returns True if the given request has permission to change the given
Django model instance.

If \emph{obj} is None, this should return True if the given request has
permission to change \emph{any} object of the given type.
\end{methoddesc}
\index{has\_delete\_permission() (data.admin.PharmaceuticalAdmin method)}

\hypertarget{data.admin.PharmaceuticalAdmin.has_delete_permission}{}\begin{methoddesc}{has\_delete\_permission}{request, obj=None}
Returns True if the given request has permission to change the given
Django model instance.

If \emph{obj} is None, this should return True if the given request has
permission to delete \emph{any} object of the given type.
\end{methoddesc}
\index{history\_view() (data.admin.PharmaceuticalAdmin method)}

\hypertarget{data.admin.PharmaceuticalAdmin.history_view}{}\begin{methoddesc}{history\_view}{request, object\_id, extra\_context=None}
The `history' admin view for this model.
\end{methoddesc}
\index{log\_addition() (data.admin.PharmaceuticalAdmin method)}

\hypertarget{data.admin.PharmaceuticalAdmin.log_addition}{}\begin{methoddesc}{log\_addition}{request, object}
Log that an object has been successfully added.

The default implementation creates an admin LogEntry object.
\end{methoddesc}
\index{log\_change() (data.admin.PharmaceuticalAdmin method)}

\hypertarget{data.admin.PharmaceuticalAdmin.log_change}{}\begin{methoddesc}{log\_change}{request, object, message}
Log that an object has been successfully changed.

The default implementation creates an admin LogEntry object.
\end{methoddesc}
\index{log\_deletion() (data.admin.PharmaceuticalAdmin method)}

\hypertarget{data.admin.PharmaceuticalAdmin.log_deletion}{}\begin{methoddesc}{log\_deletion}{request, object, object\_repr}
Log that an object has been successfully deleted. Note that since the
object is deleted, it might no longer be safe to call \emph{any} methods
on the object, hence this method getting object\_repr.

The default implementation creates an admin LogEntry object.
\end{methoddesc}
\index{media (data.admin.PharmaceuticalAdmin attribute)}

\hypertarget{data.admin.PharmaceuticalAdmin.media}{}\begin{memberdesc}{media}\end{memberdesc}
\index{message\_user() (data.admin.PharmaceuticalAdmin method)}

\hypertarget{data.admin.PharmaceuticalAdmin.message_user}{}\begin{methoddesc}{message\_user}{request, message}
Send a message to the user. The default implementation
posts a message using the auth Message object.
\end{methoddesc}
\index{queryset() (data.admin.PharmaceuticalAdmin method)}

\hypertarget{data.admin.PharmaceuticalAdmin.queryset}{}\begin{methoddesc}{queryset}{request}
Returns a QuerySet of all model instances that can be edited by the
admin site. This is used by changelist\_view.
\end{methoddesc}
\index{render\_change\_form() (data.admin.PharmaceuticalAdmin method)}

\hypertarget{data.admin.PharmaceuticalAdmin.render_change_form}{}\begin{methoddesc}{render\_change\_form}{request, context, add=False, change=False, form\_url='', obj=None}\end{methoddesc}
\index{response\_action() (data.admin.PharmaceuticalAdmin method)}

\hypertarget{data.admin.PharmaceuticalAdmin.response_action}{}\begin{methoddesc}{response\_action}{request, queryset}
Handle an admin action. This is called if a request is POSTed to the
changelist; it returns an HttpResponse if the action was handled, and
None otherwise.
\end{methoddesc}
\index{response\_add() (data.admin.PharmaceuticalAdmin method)}

\hypertarget{data.admin.PharmaceuticalAdmin.response_add}{}\begin{methoddesc}{response\_add}{request, obj, post\_url\_continue='../\%s/'}
Determines the HttpResponse for the add\_view stage.
\end{methoddesc}
\index{response\_change() (data.admin.PharmaceuticalAdmin method)}

\hypertarget{data.admin.PharmaceuticalAdmin.response_change}{}\begin{methoddesc}{response\_change}{request, obj}
Determines the HttpResponse for the change\_view stage.
\end{methoddesc}
\index{save\_form() (data.admin.PharmaceuticalAdmin method)}

\hypertarget{data.admin.PharmaceuticalAdmin.save_form}{}\begin{methoddesc}{save\_form}{request, form, change}
Given a ModelForm return an unsaved instance. \code{change} is True if
the object is being changed, and False if it's being added.
\end{methoddesc}
\index{save\_formset() (data.admin.PharmaceuticalAdmin method)}

\hypertarget{data.admin.PharmaceuticalAdmin.save_formset}{}\begin{methoddesc}{save\_formset}{request, form, formset, change}
Given an inline formset save it to the database.
\end{methoddesc}
\index{save\_model() (data.admin.PharmaceuticalAdmin method)}

\hypertarget{data.admin.PharmaceuticalAdmin.save_model}{}\begin{methoddesc}{save\_model}{request, obj, form, change}
Given a model instance save it to the database.
\end{methoddesc}
\index{urls (data.admin.PharmaceuticalAdmin attribute)}

\hypertarget{data.admin.PharmaceuticalAdmin.urls}{}\begin{memberdesc}{urls}\end{memberdesc}
\end{classdesc}
\index{ResearcherAdmin (class in data.admin)}

\hypertarget{data.admin.ResearcherAdmin}{}\begin{classdesc}{ResearcherAdmin}{model, admin\_site}~\index{action\_checkbox() (data.admin.ResearcherAdmin method)}

\hypertarget{data.admin.ResearcherAdmin.action_checkbox}{}\begin{methoddesc}{action\_checkbox}{obj}
A list\_display column containing a checkbox widget.
\end{methoddesc}
\index{action\_form (data.admin.ResearcherAdmin attribute)}

\hypertarget{data.admin.ResearcherAdmin.action_form}{}\begin{memberdesc}{action\_form}
alias of \code{ActionForm}
\end{memberdesc}
\index{add\_view() (data.admin.ResearcherAdmin method)}

\hypertarget{data.admin.ResearcherAdmin.add_view}{}\begin{methoddesc}{add\_view}{*args, **kw}
The `add' admin view for this model.
\end{methoddesc}
\index{change\_view() (data.admin.ResearcherAdmin method)}

\hypertarget{data.admin.ResearcherAdmin.change_view}{}\begin{methoddesc}{change\_view}{*args, **kw}
The `change' admin view for this model.
\end{methoddesc}
\index{changelist\_view() (data.admin.ResearcherAdmin method)}

\hypertarget{data.admin.ResearcherAdmin.changelist_view}{}\begin{methoddesc}{changelist\_view}{request, extra\_context=None}
The `change list' admin view for this model.
\end{methoddesc}
\index{construct\_change\_message() (data.admin.ResearcherAdmin method)}

\hypertarget{data.admin.ResearcherAdmin.construct_change_message}{}\begin{methoddesc}{construct\_change\_message}{request, form, formsets}
Construct a change message from a changed object.
\end{methoddesc}
\index{declared\_fieldsets (data.admin.ResearcherAdmin attribute)}

\hypertarget{data.admin.ResearcherAdmin.declared_fieldsets}{}\begin{memberdesc}{declared\_fieldsets}\end{memberdesc}
\index{delete\_view() (data.admin.ResearcherAdmin method)}

\hypertarget{data.admin.ResearcherAdmin.delete_view}{}\begin{methoddesc}{delete\_view}{request, object\_id, extra\_context=None}
The `delete' admin view for this model.
\end{methoddesc}
\index{form (data.admin.ResearcherAdmin attribute)}

\hypertarget{data.admin.ResearcherAdmin.form}{}\begin{memberdesc}{form}
alias of \code{ModelForm}
\end{memberdesc}
\index{formfield\_for\_choice\_field() (data.admin.ResearcherAdmin method)}

\hypertarget{data.admin.ResearcherAdmin.formfield_for_choice_field}{}\begin{methoddesc}{formfield\_for\_choice\_field}{db\_field, request=None, **kwargs}
Get a form Field for a database Field that has declared choices.
\end{methoddesc}
\index{formfield\_for\_dbfield() (data.admin.ResearcherAdmin method)}

\hypertarget{data.admin.ResearcherAdmin.formfield_for_dbfield}{}\begin{methoddesc}{formfield\_for\_dbfield}{db\_field, **kwargs}
Hook for specifying the form Field instance for a given database Field
instance.

If kwargs are given, they're passed to the form Field's constructor.
\end{methoddesc}
\index{formfield\_for\_foreignkey() (data.admin.ResearcherAdmin method)}

\hypertarget{data.admin.ResearcherAdmin.formfield_for_foreignkey}{}\begin{methoddesc}{formfield\_for\_foreignkey}{db\_field, request=None, **kwargs}
Get a form Field for a ForeignKey.
\end{methoddesc}
\index{formfield\_for\_manytomany() (data.admin.ResearcherAdmin method)}

\hypertarget{data.admin.ResearcherAdmin.formfield_for_manytomany}{}\begin{methoddesc}{formfield\_for\_manytomany}{db\_field, request=None, **kwargs}
Get a form Field for a ManyToManyField.
\end{methoddesc}
\index{get\_action() (data.admin.ResearcherAdmin method)}

\hypertarget{data.admin.ResearcherAdmin.get_action}{}\begin{methoddesc}{get\_action}{action}
Return a given action from a parameter, which can either be a callable,
or the name of a method on the ModelAdmin.  Return is a tuple of
(callable, name, description).
\end{methoddesc}
\index{get\_action\_choices() (data.admin.ResearcherAdmin method)}

\hypertarget{data.admin.ResearcherAdmin.get_action_choices}{}\begin{methoddesc}{get\_action\_choices}{request, default\_choices=, {[}('', '---------'){]}}
Return a list of choices for use in a form object.  Each choice is a
tuple (name, description).
\end{methoddesc}
\index{get\_actions() (data.admin.ResearcherAdmin method)}

\hypertarget{data.admin.ResearcherAdmin.get_actions}{}\begin{methoddesc}{get\_actions}{request}
Return a dictionary mapping the names of all actions for this
ModelAdmin to a tuple of (callable, name, description) for each action.
\end{methoddesc}
\index{get\_changelist\_form() (data.admin.ResearcherAdmin method)}

\hypertarget{data.admin.ResearcherAdmin.get_changelist_form}{}\begin{methoddesc}{get\_changelist\_form}{request, **kwargs}
Returns a Form class for use in the Formset on the changelist page.
\end{methoddesc}
\index{get\_changelist\_formset() (data.admin.ResearcherAdmin method)}

\hypertarget{data.admin.ResearcherAdmin.get_changelist_formset}{}\begin{methoddesc}{get\_changelist\_formset}{request, **kwargs}
Returns a FormSet class for use on the changelist page if list\_editable
is used.
\end{methoddesc}
\index{get\_fieldsets() (data.admin.ResearcherAdmin method)}

\hypertarget{data.admin.ResearcherAdmin.get_fieldsets}{}\begin{methoddesc}{get\_fieldsets}{request, obj=None}
Hook for specifying fieldsets for the add form.
\end{methoddesc}
\index{get\_form() (data.admin.ResearcherAdmin method)}

\hypertarget{data.admin.ResearcherAdmin.get_form}{}\begin{methoddesc}{get\_form}{request, obj=None, **kwargs}
Returns a Form class for use in the admin add view. This is used by
add\_view and change\_view.
\end{methoddesc}
\index{get\_formsets() (data.admin.ResearcherAdmin method)}

\hypertarget{data.admin.ResearcherAdmin.get_formsets}{}\begin{methoddesc}{get\_formsets}{request, obj=None}\end{methoddesc}
\index{get\_model\_perms() (data.admin.ResearcherAdmin method)}

\hypertarget{data.admin.ResearcherAdmin.get_model_perms}{}\begin{methoddesc}{get\_model\_perms}{request}
Returns a dict of all perms for this model. This dict has the keys
\code{add}, \code{change}, and \code{delete} mapping to the True/False for each
of those actions.
\end{methoddesc}
\index{get\_urls() (data.admin.ResearcherAdmin method)}

\hypertarget{data.admin.ResearcherAdmin.get_urls}{}\begin{methoddesc}{get\_urls}{}\end{methoddesc}
\index{has\_add\_permission() (data.admin.ResearcherAdmin method)}

\hypertarget{data.admin.ResearcherAdmin.has_add_permission}{}\begin{methoddesc}{has\_add\_permission}{request}
Returns True if the given request has permission to add an object.
\end{methoddesc}
\index{has\_change\_permission() (data.admin.ResearcherAdmin method)}

\hypertarget{data.admin.ResearcherAdmin.has_change_permission}{}\begin{methoddesc}{has\_change\_permission}{request, obj=None}
Returns True if the given request has permission to change the given
Django model instance.

If \emph{obj} is None, this should return True if the given request has
permission to change \emph{any} object of the given type.
\end{methoddesc}
\index{has\_delete\_permission() (data.admin.ResearcherAdmin method)}

\hypertarget{data.admin.ResearcherAdmin.has_delete_permission}{}\begin{methoddesc}{has\_delete\_permission}{request, obj=None}
Returns True if the given request has permission to change the given
Django model instance.

If \emph{obj} is None, this should return True if the given request has
permission to delete \emph{any} object of the given type.
\end{methoddesc}
\index{history\_view() (data.admin.ResearcherAdmin method)}

\hypertarget{data.admin.ResearcherAdmin.history_view}{}\begin{methoddesc}{history\_view}{request, object\_id, extra\_context=None}
The `history' admin view for this model.
\end{methoddesc}
\index{log\_addition() (data.admin.ResearcherAdmin method)}

\hypertarget{data.admin.ResearcherAdmin.log_addition}{}\begin{methoddesc}{log\_addition}{request, object}
Log that an object has been successfully added.

The default implementation creates an admin LogEntry object.
\end{methoddesc}
\index{log\_change() (data.admin.ResearcherAdmin method)}

\hypertarget{data.admin.ResearcherAdmin.log_change}{}\begin{methoddesc}{log\_change}{request, object, message}
Log that an object has been successfully changed.

The default implementation creates an admin LogEntry object.
\end{methoddesc}
\index{log\_deletion() (data.admin.ResearcherAdmin method)}

\hypertarget{data.admin.ResearcherAdmin.log_deletion}{}\begin{methoddesc}{log\_deletion}{request, object, object\_repr}
Log that an object has been successfully deleted. Note that since the
object is deleted, it might no longer be safe to call \emph{any} methods
on the object, hence this method getting object\_repr.

The default implementation creates an admin LogEntry object.
\end{methoddesc}
\index{media (data.admin.ResearcherAdmin attribute)}

\hypertarget{data.admin.ResearcherAdmin.media}{}\begin{memberdesc}{media}\end{memberdesc}
\index{message\_user() (data.admin.ResearcherAdmin method)}

\hypertarget{data.admin.ResearcherAdmin.message_user}{}\begin{methoddesc}{message\_user}{request, message}
Send a message to the user. The default implementation
posts a message using the auth Message object.
\end{methoddesc}
\index{queryset() (data.admin.ResearcherAdmin method)}

\hypertarget{data.admin.ResearcherAdmin.queryset}{}\begin{methoddesc}{queryset}{request}
Returns a QuerySet of all model instances that can be edited by the
admin site. This is used by changelist\_view.
\end{methoddesc}
\index{render\_change\_form() (data.admin.ResearcherAdmin method)}

\hypertarget{data.admin.ResearcherAdmin.render_change_form}{}\begin{methoddesc}{render\_change\_form}{request, context, add=False, change=False, form\_url='', obj=None}\end{methoddesc}
\index{response\_action() (data.admin.ResearcherAdmin method)}

\hypertarget{data.admin.ResearcherAdmin.response_action}{}\begin{methoddesc}{response\_action}{request, queryset}
Handle an admin action. This is called if a request is POSTed to the
changelist; it returns an HttpResponse if the action was handled, and
None otherwise.
\end{methoddesc}
\index{response\_add() (data.admin.ResearcherAdmin method)}

\hypertarget{data.admin.ResearcherAdmin.response_add}{}\begin{methoddesc}{response\_add}{request, obj, post\_url\_continue='../\%s/'}
Determines the HttpResponse for the add\_view stage.
\end{methoddesc}
\index{response\_change() (data.admin.ResearcherAdmin method)}

\hypertarget{data.admin.ResearcherAdmin.response_change}{}\begin{methoddesc}{response\_change}{request, obj}
Determines the HttpResponse for the change\_view stage.
\end{methoddesc}
\index{save\_form() (data.admin.ResearcherAdmin method)}

\hypertarget{data.admin.ResearcherAdmin.save_form}{}\begin{methoddesc}{save\_form}{request, form, change}
Given a ModelForm return an unsaved instance. \code{change} is True if
the object is being changed, and False if it's being added.
\end{methoddesc}
\index{save\_formset() (data.admin.ResearcherAdmin method)}

\hypertarget{data.admin.ResearcherAdmin.save_formset}{}\begin{methoddesc}{save\_formset}{request, form, formset, change}
Given an inline formset save it to the database.
\end{methoddesc}
\index{save\_model() (data.admin.ResearcherAdmin method)}

\hypertarget{data.admin.ResearcherAdmin.save_model}{}\begin{methoddesc}{save\_model}{request, obj, form, change}
Given a model instance save it to the database.
\end{methoddesc}
\index{urls (data.admin.ResearcherAdmin attribute)}

\hypertarget{data.admin.ResearcherAdmin.urls}{}\begin{memberdesc}{urls}\end{memberdesc}
\end{classdesc}
\index{StudyAdmin (class in data.admin)}

\hypertarget{data.admin.StudyAdmin}{}\begin{classdesc}{StudyAdmin}{model, admin\_site}~\index{action\_checkbox() (data.admin.StudyAdmin method)}

\hypertarget{data.admin.StudyAdmin.action_checkbox}{}\begin{methoddesc}{action\_checkbox}{obj}
A list\_display column containing a checkbox widget.
\end{methoddesc}
\index{action\_form (data.admin.StudyAdmin attribute)}

\hypertarget{data.admin.StudyAdmin.action_form}{}\begin{memberdesc}{action\_form}
alias of \code{ActionForm}
\end{memberdesc}
\index{add\_view() (data.admin.StudyAdmin method)}

\hypertarget{data.admin.StudyAdmin.add_view}{}\begin{methoddesc}{add\_view}{*args, **kw}
The `add' admin view for this model.
\end{methoddesc}
\index{change\_view() (data.admin.StudyAdmin method)}

\hypertarget{data.admin.StudyAdmin.change_view}{}\begin{methoddesc}{change\_view}{*args, **kw}
The `change' admin view for this model.
\end{methoddesc}
\index{changelist\_view() (data.admin.StudyAdmin method)}

\hypertarget{data.admin.StudyAdmin.changelist_view}{}\begin{methoddesc}{changelist\_view}{request, extra\_context=None}
The `change list' admin view for this model.
\end{methoddesc}
\index{construct\_change\_message() (data.admin.StudyAdmin method)}

\hypertarget{data.admin.StudyAdmin.construct_change_message}{}\begin{methoddesc}{construct\_change\_message}{request, form, formsets}
Construct a change message from a changed object.
\end{methoddesc}
\index{declared\_fieldsets (data.admin.StudyAdmin attribute)}

\hypertarget{data.admin.StudyAdmin.declared_fieldsets}{}\begin{memberdesc}{declared\_fieldsets}\end{memberdesc}
\index{delete\_view() (data.admin.StudyAdmin method)}

\hypertarget{data.admin.StudyAdmin.delete_view}{}\begin{methoddesc}{delete\_view}{request, object\_id, extra\_context=None}
The `delete' admin view for this model.
\end{methoddesc}
\index{form (data.admin.StudyAdmin attribute)}

\hypertarget{data.admin.StudyAdmin.form}{}\begin{memberdesc}{form}
alias of \code{ModelForm}
\end{memberdesc}
\index{formfield\_for\_choice\_field() (data.admin.StudyAdmin method)}

\hypertarget{data.admin.StudyAdmin.formfield_for_choice_field}{}\begin{methoddesc}{formfield\_for\_choice\_field}{db\_field, request=None, **kwargs}
Get a form Field for a database Field that has declared choices.
\end{methoddesc}
\index{formfield\_for\_dbfield() (data.admin.StudyAdmin method)}

\hypertarget{data.admin.StudyAdmin.formfield_for_dbfield}{}\begin{methoddesc}{formfield\_for\_dbfield}{db\_field, **kwargs}
Hook for specifying the form Field instance for a given database Field
instance.

If kwargs are given, they're passed to the form Field's constructor.
\end{methoddesc}
\index{formfield\_for\_foreignkey() (data.admin.StudyAdmin method)}

\hypertarget{data.admin.StudyAdmin.formfield_for_foreignkey}{}\begin{methoddesc}{formfield\_for\_foreignkey}{db\_field, request=None, **kwargs}
Get a form Field for a ForeignKey.
\end{methoddesc}
\index{formfield\_for\_manytomany() (data.admin.StudyAdmin method)}

\hypertarget{data.admin.StudyAdmin.formfield_for_manytomany}{}\begin{methoddesc}{formfield\_for\_manytomany}{db\_field, request=None, **kwargs}
Get a form Field for a ManyToManyField.
\end{methoddesc}
\index{get\_action() (data.admin.StudyAdmin method)}

\hypertarget{data.admin.StudyAdmin.get_action}{}\begin{methoddesc}{get\_action}{action}
Return a given action from a parameter, which can either be a callable,
or the name of a method on the ModelAdmin.  Return is a tuple of
(callable, name, description).
\end{methoddesc}
\index{get\_action\_choices() (data.admin.StudyAdmin method)}

\hypertarget{data.admin.StudyAdmin.get_action_choices}{}\begin{methoddesc}{get\_action\_choices}{request, default\_choices=, {[}('', '---------'){]}}
Return a list of choices for use in a form object.  Each choice is a
tuple (name, description).
\end{methoddesc}
\index{get\_actions() (data.admin.StudyAdmin method)}

\hypertarget{data.admin.StudyAdmin.get_actions}{}\begin{methoddesc}{get\_actions}{request}
Return a dictionary mapping the names of all actions for this
ModelAdmin to a tuple of (callable, name, description) for each action.
\end{methoddesc}
\index{get\_changelist\_form() (data.admin.StudyAdmin method)}

\hypertarget{data.admin.StudyAdmin.get_changelist_form}{}\begin{methoddesc}{get\_changelist\_form}{request, **kwargs}
Returns a Form class for use in the Formset on the changelist page.
\end{methoddesc}
\index{get\_changelist\_formset() (data.admin.StudyAdmin method)}

\hypertarget{data.admin.StudyAdmin.get_changelist_formset}{}\begin{methoddesc}{get\_changelist\_formset}{request, **kwargs}
Returns a FormSet class for use on the changelist page if list\_editable
is used.
\end{methoddesc}
\index{get\_fieldsets() (data.admin.StudyAdmin method)}

\hypertarget{data.admin.StudyAdmin.get_fieldsets}{}\begin{methoddesc}{get\_fieldsets}{request, obj=None}
Hook for specifying fieldsets for the add form.
\end{methoddesc}
\index{get\_form() (data.admin.StudyAdmin method)}

\hypertarget{data.admin.StudyAdmin.get_form}{}\begin{methoddesc}{get\_form}{request, obj=None, **kwargs}
Returns a Form class for use in the admin add view. This is used by
add\_view and change\_view.
\end{methoddesc}
\index{get\_formsets() (data.admin.StudyAdmin method)}

\hypertarget{data.admin.StudyAdmin.get_formsets}{}\begin{methoddesc}{get\_formsets}{request, obj=None}\end{methoddesc}
\index{get\_model\_perms() (data.admin.StudyAdmin method)}

\hypertarget{data.admin.StudyAdmin.get_model_perms}{}\begin{methoddesc}{get\_model\_perms}{request}
Returns a dict of all perms for this model. This dict has the keys
\code{add}, \code{change}, and \code{delete} mapping to the True/False for each
of those actions.
\end{methoddesc}
\index{get\_urls() (data.admin.StudyAdmin method)}

\hypertarget{data.admin.StudyAdmin.get_urls}{}\begin{methoddesc}{get\_urls}{}\end{methoddesc}
\index{has\_add\_permission() (data.admin.StudyAdmin method)}

\hypertarget{data.admin.StudyAdmin.has_add_permission}{}\begin{methoddesc}{has\_add\_permission}{request}
Returns True if the given request has permission to add an object.
\end{methoddesc}
\index{has\_change\_permission() (data.admin.StudyAdmin method)}

\hypertarget{data.admin.StudyAdmin.has_change_permission}{}\begin{methoddesc}{has\_change\_permission}{request, obj=None}
Returns True if the given request has permission to change the given
Django model instance.

If \emph{obj} is None, this should return True if the given request has
permission to change \emph{any} object of the given type.
\end{methoddesc}
\index{has\_delete\_permission() (data.admin.StudyAdmin method)}

\hypertarget{data.admin.StudyAdmin.has_delete_permission}{}\begin{methoddesc}{has\_delete\_permission}{request, obj=None}
Returns True if the given request has permission to change the given
Django model instance.

If \emph{obj} is None, this should return True if the given request has
permission to delete \emph{any} object of the given type.
\end{methoddesc}
\index{history\_view() (data.admin.StudyAdmin method)}

\hypertarget{data.admin.StudyAdmin.history_view}{}\begin{methoddesc}{history\_view}{request, object\_id, extra\_context=None}
The `history' admin view for this model.
\end{methoddesc}
\index{log\_addition() (data.admin.StudyAdmin method)}

\hypertarget{data.admin.StudyAdmin.log_addition}{}\begin{methoddesc}{log\_addition}{request, object}
Log that an object has been successfully added.

The default implementation creates an admin LogEntry object.
\end{methoddesc}
\index{log\_change() (data.admin.StudyAdmin method)}

\hypertarget{data.admin.StudyAdmin.log_change}{}\begin{methoddesc}{log\_change}{request, object, message}
Log that an object has been successfully changed.

The default implementation creates an admin LogEntry object.
\end{methoddesc}
\index{log\_deletion() (data.admin.StudyAdmin method)}

\hypertarget{data.admin.StudyAdmin.log_deletion}{}\begin{methoddesc}{log\_deletion}{request, object, object\_repr}
Log that an object has been successfully deleted. Note that since the
object is deleted, it might no longer be safe to call \emph{any} methods
on the object, hence this method getting object\_repr.

The default implementation creates an admin LogEntry object.
\end{methoddesc}
\index{media (data.admin.StudyAdmin attribute)}

\hypertarget{data.admin.StudyAdmin.media}{}\begin{memberdesc}{media}\end{memberdesc}
\index{message\_user() (data.admin.StudyAdmin method)}

\hypertarget{data.admin.StudyAdmin.message_user}{}\begin{methoddesc}{message\_user}{request, message}
Send a message to the user. The default implementation
posts a message using the auth Message object.
\end{methoddesc}
\index{queryset() (data.admin.StudyAdmin method)}

\hypertarget{data.admin.StudyAdmin.queryset}{}\begin{methoddesc}{queryset}{request}
Returns a QuerySet of all model instances that can be edited by the
admin site. This is used by changelist\_view.
\end{methoddesc}
\index{render\_change\_form() (data.admin.StudyAdmin method)}

\hypertarget{data.admin.StudyAdmin.render_change_form}{}\begin{methoddesc}{render\_change\_form}{request, context, add=False, change=False, form\_url='', obj=None}\end{methoddesc}
\index{response\_action() (data.admin.StudyAdmin method)}

\hypertarget{data.admin.StudyAdmin.response_action}{}\begin{methoddesc}{response\_action}{request, queryset}
Handle an admin action. This is called if a request is POSTed to the
changelist; it returns an HttpResponse if the action was handled, and
None otherwise.
\end{methoddesc}
\index{response\_add() (data.admin.StudyAdmin method)}

\hypertarget{data.admin.StudyAdmin.response_add}{}\begin{methoddesc}{response\_add}{request, obj, post\_url\_continue='../\%s/'}
Determines the HttpResponse for the add\_view stage.
\end{methoddesc}
\index{response\_change() (data.admin.StudyAdmin method)}

\hypertarget{data.admin.StudyAdmin.response_change}{}\begin{methoddesc}{response\_change}{request, obj}
Determines the HttpResponse for the change\_view stage.
\end{methoddesc}
\index{save\_form() (data.admin.StudyAdmin method)}

\hypertarget{data.admin.StudyAdmin.save_form}{}\begin{methoddesc}{save\_form}{request, form, change}
Given a ModelForm return an unsaved instance. \code{change} is True if
the object is being changed, and False if it's being added.
\end{methoddesc}
\index{save\_formset() (data.admin.StudyAdmin method)}

\hypertarget{data.admin.StudyAdmin.save_formset}{}\begin{methoddesc}{save\_formset}{request, form, formset, change}
Given an inline formset save it to the database.
\end{methoddesc}
\index{save\_model() (data.admin.StudyAdmin method)}

\hypertarget{data.admin.StudyAdmin.save_model}{}\begin{methoddesc}{save\_model}{request, obj, form, change}
Given a model instance save it to the database.
\end{methoddesc}
\index{urls (data.admin.StudyAdmin attribute)}

\hypertarget{data.admin.StudyAdmin.urls}{}\begin{memberdesc}{urls}\end{memberdesc}
\end{classdesc}
\index{TransplantationAdmin (class in data.admin)}

\hypertarget{data.admin.TransplantationAdmin}{}\begin{classdesc}{TransplantationAdmin}{model, admin\_site}~\index{action\_checkbox() (data.admin.TransplantationAdmin method)}

\hypertarget{data.admin.TransplantationAdmin.action_checkbox}{}\begin{methoddesc}{action\_checkbox}{obj}
A list\_display column containing a checkbox widget.
\end{methoddesc}
\index{action\_form (data.admin.TransplantationAdmin attribute)}

\hypertarget{data.admin.TransplantationAdmin.action_form}{}\begin{memberdesc}{action\_form}
alias of \code{ActionForm}
\end{memberdesc}
\index{add\_view() (data.admin.TransplantationAdmin method)}

\hypertarget{data.admin.TransplantationAdmin.add_view}{}\begin{methoddesc}{add\_view}{*args, **kw}
The `add' admin view for this model.
\end{methoddesc}
\index{change\_view() (data.admin.TransplantationAdmin method)}

\hypertarget{data.admin.TransplantationAdmin.change_view}{}\begin{methoddesc}{change\_view}{*args, **kw}
The `change' admin view for this model.
\end{methoddesc}
\index{changelist\_view() (data.admin.TransplantationAdmin method)}

\hypertarget{data.admin.TransplantationAdmin.changelist_view}{}\begin{methoddesc}{changelist\_view}{request, extra\_context=None}
The `change list' admin view for this model.
\end{methoddesc}
\index{construct\_change\_message() (data.admin.TransplantationAdmin method)}

\hypertarget{data.admin.TransplantationAdmin.construct_change_message}{}\begin{methoddesc}{construct\_change\_message}{request, form, formsets}
Construct a change message from a changed object.
\end{methoddesc}
\index{declared\_fieldsets (data.admin.TransplantationAdmin attribute)}

\hypertarget{data.admin.TransplantationAdmin.declared_fieldsets}{}\begin{memberdesc}{declared\_fieldsets}\end{memberdesc}
\index{delete\_view() (data.admin.TransplantationAdmin method)}

\hypertarget{data.admin.TransplantationAdmin.delete_view}{}\begin{methoddesc}{delete\_view}{request, object\_id, extra\_context=None}
The `delete' admin view for this model.
\end{methoddesc}
\index{form (data.admin.TransplantationAdmin attribute)}

\hypertarget{data.admin.TransplantationAdmin.form}{}\begin{memberdesc}{form}
alias of \code{ModelForm}
\end{memberdesc}
\index{formfield\_for\_choice\_field() (data.admin.TransplantationAdmin method)}

\hypertarget{data.admin.TransplantationAdmin.formfield_for_choice_field}{}\begin{methoddesc}{formfield\_for\_choice\_field}{db\_field, request=None, **kwargs}
Get a form Field for a database Field that has declared choices.
\end{methoddesc}
\index{formfield\_for\_dbfield() (data.admin.TransplantationAdmin method)}

\hypertarget{data.admin.TransplantationAdmin.formfield_for_dbfield}{}\begin{methoddesc}{formfield\_for\_dbfield}{db\_field, **kwargs}
Hook for specifying the form Field instance for a given database Field
instance.

If kwargs are given, they're passed to the form Field's constructor.
\end{methoddesc}
\index{formfield\_for\_foreignkey() (data.admin.TransplantationAdmin method)}

\hypertarget{data.admin.TransplantationAdmin.formfield_for_foreignkey}{}\begin{methoddesc}{formfield\_for\_foreignkey}{db\_field, request=None, **kwargs}
Get a form Field for a ForeignKey.
\end{methoddesc}
\index{formfield\_for\_manytomany() (data.admin.TransplantationAdmin method)}

\hypertarget{data.admin.TransplantationAdmin.formfield_for_manytomany}{}\begin{methoddesc}{formfield\_for\_manytomany}{db\_field, request=None, **kwargs}
Get a form Field for a ManyToManyField.
\end{methoddesc}
\index{get\_action() (data.admin.TransplantationAdmin method)}

\hypertarget{data.admin.TransplantationAdmin.get_action}{}\begin{methoddesc}{get\_action}{action}
Return a given action from a parameter, which can either be a callable,
or the name of a method on the ModelAdmin.  Return is a tuple of
(callable, name, description).
\end{methoddesc}
\index{get\_action\_choices() (data.admin.TransplantationAdmin method)}

\hypertarget{data.admin.TransplantationAdmin.get_action_choices}{}\begin{methoddesc}{get\_action\_choices}{request, default\_choices=, {[}('', '---------'){]}}
Return a list of choices for use in a form object.  Each choice is a
tuple (name, description).
\end{methoddesc}
\index{get\_actions() (data.admin.TransplantationAdmin method)}

\hypertarget{data.admin.TransplantationAdmin.get_actions}{}\begin{methoddesc}{get\_actions}{request}
Return a dictionary mapping the names of all actions for this
ModelAdmin to a tuple of (callable, name, description) for each action.
\end{methoddesc}
\index{get\_changelist\_form() (data.admin.TransplantationAdmin method)}

\hypertarget{data.admin.TransplantationAdmin.get_changelist_form}{}\begin{methoddesc}{get\_changelist\_form}{request, **kwargs}
Returns a Form class for use in the Formset on the changelist page.
\end{methoddesc}
\index{get\_changelist\_formset() (data.admin.TransplantationAdmin method)}

\hypertarget{data.admin.TransplantationAdmin.get_changelist_formset}{}\begin{methoddesc}{get\_changelist\_formset}{request, **kwargs}
Returns a FormSet class for use on the changelist page if list\_editable
is used.
\end{methoddesc}
\index{get\_fieldsets() (data.admin.TransplantationAdmin method)}

\hypertarget{data.admin.TransplantationAdmin.get_fieldsets}{}\begin{methoddesc}{get\_fieldsets}{request, obj=None}
Hook for specifying fieldsets for the add form.
\end{methoddesc}
\index{get\_form() (data.admin.TransplantationAdmin method)}

\hypertarget{data.admin.TransplantationAdmin.get_form}{}\begin{methoddesc}{get\_form}{request, obj=None, **kwargs}
Returns a Form class for use in the admin add view. This is used by
add\_view and change\_view.
\end{methoddesc}
\index{get\_formsets() (data.admin.TransplantationAdmin method)}

\hypertarget{data.admin.TransplantationAdmin.get_formsets}{}\begin{methoddesc}{get\_formsets}{request, obj=None}\end{methoddesc}
\index{get\_model\_perms() (data.admin.TransplantationAdmin method)}

\hypertarget{data.admin.TransplantationAdmin.get_model_perms}{}\begin{methoddesc}{get\_model\_perms}{request}
Returns a dict of all perms for this model. This dict has the keys
\code{add}, \code{change}, and \code{delete} mapping to the True/False for each
of those actions.
\end{methoddesc}
\index{get\_urls() (data.admin.TransplantationAdmin method)}

\hypertarget{data.admin.TransplantationAdmin.get_urls}{}\begin{methoddesc}{get\_urls}{}\end{methoddesc}
\index{has\_add\_permission() (data.admin.TransplantationAdmin method)}

\hypertarget{data.admin.TransplantationAdmin.has_add_permission}{}\begin{methoddesc}{has\_add\_permission}{request}
Returns True if the given request has permission to add an object.
\end{methoddesc}
\index{has\_change\_permission() (data.admin.TransplantationAdmin method)}

\hypertarget{data.admin.TransplantationAdmin.has_change_permission}{}\begin{methoddesc}{has\_change\_permission}{request, obj=None}
Returns True if the given request has permission to change the given
Django model instance.

If \emph{obj} is None, this should return True if the given request has
permission to change \emph{any} object of the given type.
\end{methoddesc}
\index{has\_delete\_permission() (data.admin.TransplantationAdmin method)}

\hypertarget{data.admin.TransplantationAdmin.has_delete_permission}{}\begin{methoddesc}{has\_delete\_permission}{request, obj=None}
Returns True if the given request has permission to change the given
Django model instance.

If \emph{obj} is None, this should return True if the given request has
permission to delete \emph{any} object of the given type.
\end{methoddesc}
\index{history\_view() (data.admin.TransplantationAdmin method)}

\hypertarget{data.admin.TransplantationAdmin.history_view}{}\begin{methoddesc}{history\_view}{request, object\_id, extra\_context=None}
The `history' admin view for this model.
\end{methoddesc}
\index{log\_addition() (data.admin.TransplantationAdmin method)}

\hypertarget{data.admin.TransplantationAdmin.log_addition}{}\begin{methoddesc}{log\_addition}{request, object}
Log that an object has been successfully added.

The default implementation creates an admin LogEntry object.
\end{methoddesc}
\index{log\_change() (data.admin.TransplantationAdmin method)}

\hypertarget{data.admin.TransplantationAdmin.log_change}{}\begin{methoddesc}{log\_change}{request, object, message}
Log that an object has been successfully changed.

The default implementation creates an admin LogEntry object.
\end{methoddesc}
\index{log\_deletion() (data.admin.TransplantationAdmin method)}

\hypertarget{data.admin.TransplantationAdmin.log_deletion}{}\begin{methoddesc}{log\_deletion}{request, object, object\_repr}
Log that an object has been successfully deleted. Note that since the
object is deleted, it might no longer be safe to call \emph{any} methods
on the object, hence this method getting object\_repr.

The default implementation creates an admin LogEntry object.
\end{methoddesc}
\index{media (data.admin.TransplantationAdmin attribute)}

\hypertarget{data.admin.TransplantationAdmin.media}{}\begin{memberdesc}{media}\end{memberdesc}
\index{message\_user() (data.admin.TransplantationAdmin method)}

\hypertarget{data.admin.TransplantationAdmin.message_user}{}\begin{methoddesc}{message\_user}{request, message}
Send a message to the user. The default implementation
posts a message using the auth Message object.
\end{methoddesc}
\index{queryset() (data.admin.TransplantationAdmin method)}

\hypertarget{data.admin.TransplantationAdmin.queryset}{}\begin{methoddesc}{queryset}{request}
Returns a QuerySet of all model instances that can be edited by the
admin site. This is used by changelist\_view.
\end{methoddesc}
\index{render\_change\_form() (data.admin.TransplantationAdmin method)}

\hypertarget{data.admin.TransplantationAdmin.render_change_form}{}\begin{methoddesc}{render\_change\_form}{request, context, add=False, change=False, form\_url='', obj=None}\end{methoddesc}
\index{response\_action() (data.admin.TransplantationAdmin method)}

\hypertarget{data.admin.TransplantationAdmin.response_action}{}\begin{methoddesc}{response\_action}{request, queryset}
Handle an admin action. This is called if a request is POSTed to the
changelist; it returns an HttpResponse if the action was handled, and
None otherwise.
\end{methoddesc}
\index{response\_add() (data.admin.TransplantationAdmin method)}

\hypertarget{data.admin.TransplantationAdmin.response_add}{}\begin{methoddesc}{response\_add}{request, obj, post\_url\_continue='../\%s/'}
Determines the HttpResponse for the add\_view stage.
\end{methoddesc}
\index{response\_change() (data.admin.TransplantationAdmin method)}

\hypertarget{data.admin.TransplantationAdmin.response_change}{}\begin{methoddesc}{response\_change}{request, obj}
Determines the HttpResponse for the change\_view stage.
\end{methoddesc}
\index{save\_form() (data.admin.TransplantationAdmin method)}

\hypertarget{data.admin.TransplantationAdmin.save_form}{}\begin{methoddesc}{save\_form}{request, form, change}
Given a ModelForm return an unsaved instance. \code{change} is True if
the object is being changed, and False if it's being added.
\end{methoddesc}
\index{save\_formset() (data.admin.TransplantationAdmin method)}

\hypertarget{data.admin.TransplantationAdmin.save_formset}{}\begin{methoddesc}{save\_formset}{request, form, formset, change}
Given an inline formset save it to the database.
\end{methoddesc}
\index{save\_model() (data.admin.TransplantationAdmin method)}

\hypertarget{data.admin.TransplantationAdmin.save_model}{}\begin{methoddesc}{save\_model}{request, obj, form, change}
Given a model instance save it to the database.
\end{methoddesc}
\index{urls (data.admin.TransplantationAdmin attribute)}

\hypertarget{data.admin.TransplantationAdmin.urls}{}\begin{memberdesc}{urls}\end{memberdesc}
\end{classdesc}
\index{TreatmentAdmin (class in data.admin)}

\hypertarget{data.admin.TreatmentAdmin}{}\begin{classdesc}{TreatmentAdmin}{model, admin\_site}~\index{action\_checkbox() (data.admin.TreatmentAdmin method)}

\hypertarget{data.admin.TreatmentAdmin.action_checkbox}{}\begin{methoddesc}{action\_checkbox}{obj}
A list\_display column containing a checkbox widget.
\end{methoddesc}
\index{action\_form (data.admin.TreatmentAdmin attribute)}

\hypertarget{data.admin.TreatmentAdmin.action_form}{}\begin{memberdesc}{action\_form}
alias of \code{ActionForm}
\end{memberdesc}
\index{add\_view() (data.admin.TreatmentAdmin method)}

\hypertarget{data.admin.TreatmentAdmin.add_view}{}\begin{methoddesc}{add\_view}{*args, **kw}
The `add' admin view for this model.
\end{methoddesc}
\index{change\_view() (data.admin.TreatmentAdmin method)}

\hypertarget{data.admin.TreatmentAdmin.change_view}{}\begin{methoddesc}{change\_view}{*args, **kw}
The `change' admin view for this model.
\end{methoddesc}
\index{changelist\_view() (data.admin.TreatmentAdmin method)}

\hypertarget{data.admin.TreatmentAdmin.changelist_view}{}\begin{methoddesc}{changelist\_view}{request, extra\_context=None}
The `change list' admin view for this model.
\end{methoddesc}
\index{construct\_change\_message() (data.admin.TreatmentAdmin method)}

\hypertarget{data.admin.TreatmentAdmin.construct_change_message}{}\begin{methoddesc}{construct\_change\_message}{request, form, formsets}
Construct a change message from a changed object.
\end{methoddesc}
\index{declared\_fieldsets (data.admin.TreatmentAdmin attribute)}

\hypertarget{data.admin.TreatmentAdmin.declared_fieldsets}{}\begin{memberdesc}{declared\_fieldsets}\end{memberdesc}
\index{delete\_view() (data.admin.TreatmentAdmin method)}

\hypertarget{data.admin.TreatmentAdmin.delete_view}{}\begin{methoddesc}{delete\_view}{request, object\_id, extra\_context=None}
The `delete' admin view for this model.
\end{methoddesc}
\index{form (data.admin.TreatmentAdmin attribute)}

\hypertarget{data.admin.TreatmentAdmin.form}{}\begin{memberdesc}{form}
alias of \code{ModelForm}
\end{memberdesc}
\index{formfield\_for\_choice\_field() (data.admin.TreatmentAdmin method)}

\hypertarget{data.admin.TreatmentAdmin.formfield_for_choice_field}{}\begin{methoddesc}{formfield\_for\_choice\_field}{db\_field, request=None, **kwargs}
Get a form Field for a database Field that has declared choices.
\end{methoddesc}
\index{formfield\_for\_dbfield() (data.admin.TreatmentAdmin method)}

\hypertarget{data.admin.TreatmentAdmin.formfield_for_dbfield}{}\begin{methoddesc}{formfield\_for\_dbfield}{db\_field, **kwargs}
Hook for specifying the form Field instance for a given database Field
instance.

If kwargs are given, they're passed to the form Field's constructor.
\end{methoddesc}
\index{formfield\_for\_foreignkey() (data.admin.TreatmentAdmin method)}

\hypertarget{data.admin.TreatmentAdmin.formfield_for_foreignkey}{}\begin{methoddesc}{formfield\_for\_foreignkey}{db\_field, request=None, **kwargs}
Get a form Field for a ForeignKey.
\end{methoddesc}
\index{formfield\_for\_manytomany() (data.admin.TreatmentAdmin method)}

\hypertarget{data.admin.TreatmentAdmin.formfield_for_manytomany}{}\begin{methoddesc}{formfield\_for\_manytomany}{db\_field, request=None, **kwargs}
Get a form Field for a ManyToManyField.
\end{methoddesc}
\index{get\_action() (data.admin.TreatmentAdmin method)}

\hypertarget{data.admin.TreatmentAdmin.get_action}{}\begin{methoddesc}{get\_action}{action}
Return a given action from a parameter, which can either be a callable,
or the name of a method on the ModelAdmin.  Return is a tuple of
(callable, name, description).
\end{methoddesc}
\index{get\_action\_choices() (data.admin.TreatmentAdmin method)}

\hypertarget{data.admin.TreatmentAdmin.get_action_choices}{}\begin{methoddesc}{get\_action\_choices}{request, default\_choices=, {[}('', '---------'){]}}
Return a list of choices for use in a form object.  Each choice is a
tuple (name, description).
\end{methoddesc}
\index{get\_actions() (data.admin.TreatmentAdmin method)}

\hypertarget{data.admin.TreatmentAdmin.get_actions}{}\begin{methoddesc}{get\_actions}{request}
Return a dictionary mapping the names of all actions for this
ModelAdmin to a tuple of (callable, name, description) for each action.
\end{methoddesc}
\index{get\_changelist\_form() (data.admin.TreatmentAdmin method)}

\hypertarget{data.admin.TreatmentAdmin.get_changelist_form}{}\begin{methoddesc}{get\_changelist\_form}{request, **kwargs}
Returns a Form class for use in the Formset on the changelist page.
\end{methoddesc}
\index{get\_changelist\_formset() (data.admin.TreatmentAdmin method)}

\hypertarget{data.admin.TreatmentAdmin.get_changelist_formset}{}\begin{methoddesc}{get\_changelist\_formset}{request, **kwargs}
Returns a FormSet class for use on the changelist page if list\_editable
is used.
\end{methoddesc}
\index{get\_fieldsets() (data.admin.TreatmentAdmin method)}

\hypertarget{data.admin.TreatmentAdmin.get_fieldsets}{}\begin{methoddesc}{get\_fieldsets}{request, obj=None}
Hook for specifying fieldsets for the add form.
\end{methoddesc}
\index{get\_form() (data.admin.TreatmentAdmin method)}

\hypertarget{data.admin.TreatmentAdmin.get_form}{}\begin{methoddesc}{get\_form}{request, obj=None, **kwargs}
Returns a Form class for use in the admin add view. This is used by
add\_view and change\_view.
\end{methoddesc}
\index{get\_formsets() (data.admin.TreatmentAdmin method)}

\hypertarget{data.admin.TreatmentAdmin.get_formsets}{}\begin{methoddesc}{get\_formsets}{request, obj=None}\end{methoddesc}
\index{get\_model\_perms() (data.admin.TreatmentAdmin method)}

\hypertarget{data.admin.TreatmentAdmin.get_model_perms}{}\begin{methoddesc}{get\_model\_perms}{request}
Returns a dict of all perms for this model. This dict has the keys
\code{add}, \code{change}, and \code{delete} mapping to the True/False for each
of those actions.
\end{methoddesc}
\index{get\_urls() (data.admin.TreatmentAdmin method)}

\hypertarget{data.admin.TreatmentAdmin.get_urls}{}\begin{methoddesc}{get\_urls}{}\end{methoddesc}
\index{has\_add\_permission() (data.admin.TreatmentAdmin method)}

\hypertarget{data.admin.TreatmentAdmin.has_add_permission}{}\begin{methoddesc}{has\_add\_permission}{request}
Returns True if the given request has permission to add an object.
\end{methoddesc}
\index{has\_change\_permission() (data.admin.TreatmentAdmin method)}

\hypertarget{data.admin.TreatmentAdmin.has_change_permission}{}\begin{methoddesc}{has\_change\_permission}{request, obj=None}
Returns True if the given request has permission to change the given
Django model instance.

If \emph{obj} is None, this should return True if the given request has
permission to change \emph{any} object of the given type.
\end{methoddesc}
\index{has\_delete\_permission() (data.admin.TreatmentAdmin method)}

\hypertarget{data.admin.TreatmentAdmin.has_delete_permission}{}\begin{methoddesc}{has\_delete\_permission}{request, obj=None}
Returns True if the given request has permission to change the given
Django model instance.

If \emph{obj} is None, this should return True if the given request has
permission to delete \emph{any} object of the given type.
\end{methoddesc}
\index{history\_view() (data.admin.TreatmentAdmin method)}

\hypertarget{data.admin.TreatmentAdmin.history_view}{}\begin{methoddesc}{history\_view}{request, object\_id, extra\_context=None}
The `history' admin view for this model.
\end{methoddesc}
\index{log\_addition() (data.admin.TreatmentAdmin method)}

\hypertarget{data.admin.TreatmentAdmin.log_addition}{}\begin{methoddesc}{log\_addition}{request, object}
Log that an object has been successfully added.

The default implementation creates an admin LogEntry object.
\end{methoddesc}
\index{log\_change() (data.admin.TreatmentAdmin method)}

\hypertarget{data.admin.TreatmentAdmin.log_change}{}\begin{methoddesc}{log\_change}{request, object, message}
Log that an object has been successfully changed.

The default implementation creates an admin LogEntry object.
\end{methoddesc}
\index{log\_deletion() (data.admin.TreatmentAdmin method)}

\hypertarget{data.admin.TreatmentAdmin.log_deletion}{}\begin{methoddesc}{log\_deletion}{request, object, object\_repr}
Log that an object has been successfully deleted. Note that since the
object is deleted, it might no longer be safe to call \emph{any} methods
on the object, hence this method getting object\_repr.

The default implementation creates an admin LogEntry object.
\end{methoddesc}
\index{media (data.admin.TreatmentAdmin attribute)}

\hypertarget{data.admin.TreatmentAdmin.media}{}\begin{memberdesc}{media}\end{memberdesc}
\index{message\_user() (data.admin.TreatmentAdmin method)}

\hypertarget{data.admin.TreatmentAdmin.message_user}{}\begin{methoddesc}{message\_user}{request, message}
Send a message to the user. The default implementation
posts a message using the auth Message object.
\end{methoddesc}
\index{queryset() (data.admin.TreatmentAdmin method)}

\hypertarget{data.admin.TreatmentAdmin.queryset}{}\begin{methoddesc}{queryset}{request}
Returns a QuerySet of all model instances that can be edited by the
admin site. This is used by changelist\_view.
\end{methoddesc}
\index{render\_change\_form() (data.admin.TreatmentAdmin method)}

\hypertarget{data.admin.TreatmentAdmin.render_change_form}{}\begin{methoddesc}{render\_change\_form}{request, context, add=False, change=False, form\_url='', obj=None}\end{methoddesc}
\index{response\_action() (data.admin.TreatmentAdmin method)}

\hypertarget{data.admin.TreatmentAdmin.response_action}{}\begin{methoddesc}{response\_action}{request, queryset}
Handle an admin action. This is called if a request is POSTed to the
changelist; it returns an HttpResponse if the action was handled, and
None otherwise.
\end{methoddesc}
\index{response\_add() (data.admin.TreatmentAdmin method)}

\hypertarget{data.admin.TreatmentAdmin.response_add}{}\begin{methoddesc}{response\_add}{request, obj, post\_url\_continue='../\%s/'}
Determines the HttpResponse for the add\_view stage.
\end{methoddesc}
\index{response\_change() (data.admin.TreatmentAdmin method)}

\hypertarget{data.admin.TreatmentAdmin.response_change}{}\begin{methoddesc}{response\_change}{request, obj}
Determines the HttpResponse for the change\_view stage.
\end{methoddesc}
\index{save\_form() (data.admin.TreatmentAdmin method)}

\hypertarget{data.admin.TreatmentAdmin.save_form}{}\begin{methoddesc}{save\_form}{request, form, change}
Given a ModelForm return an unsaved instance. \code{change} is True if
the object is being changed, and False if it's being added.
\end{methoddesc}
\index{save\_formset() (data.admin.TreatmentAdmin method)}

\hypertarget{data.admin.TreatmentAdmin.save_formset}{}\begin{methoddesc}{save\_formset}{request, form, formset, change}
Given an inline formset save it to the database.
\end{methoddesc}
\index{save\_model() (data.admin.TreatmentAdmin method)}

\hypertarget{data.admin.TreatmentAdmin.save_model}{}\begin{methoddesc}{save\_model}{request, obj, form, change}
Given a model instance save it to the database.
\end{methoddesc}
\index{urls (data.admin.TreatmentAdmin attribute)}

\hypertarget{data.admin.TreatmentAdmin.urls}{}\begin{memberdesc}{urls}\end{memberdesc}
\end{classdesc}
\index{TreatmentInline (class in data.admin)}

\hypertarget{data.admin.TreatmentInline}{}\begin{classdesc}{TreatmentInline}{parent\_model, admin\_site}~\index{declared\_fieldsets (data.admin.TreatmentInline attribute)}

\hypertarget{data.admin.TreatmentInline.declared_fieldsets}{}\begin{memberdesc}{declared\_fieldsets}\end{memberdesc}
\index{form (data.admin.TreatmentInline attribute)}

\hypertarget{data.admin.TreatmentInline.form}{}\begin{memberdesc}{form}
alias of \code{ModelForm}
\end{memberdesc}
\index{formfield\_for\_choice\_field() (data.admin.TreatmentInline method)}

\hypertarget{data.admin.TreatmentInline.formfield_for_choice_field}{}\begin{methoddesc}{formfield\_for\_choice\_field}{db\_field, request=None, **kwargs}
Get a form Field for a database Field that has declared choices.
\end{methoddesc}
\index{formfield\_for\_dbfield() (data.admin.TreatmentInline method)}

\hypertarget{data.admin.TreatmentInline.formfield_for_dbfield}{}\begin{methoddesc}{formfield\_for\_dbfield}{db\_field, **kwargs}
Hook for specifying the form Field instance for a given database Field
instance.

If kwargs are given, they're passed to the form Field's constructor.
\end{methoddesc}
\index{formfield\_for\_foreignkey() (data.admin.TreatmentInline method)}

\hypertarget{data.admin.TreatmentInline.formfield_for_foreignkey}{}\begin{methoddesc}{formfield\_for\_foreignkey}{db\_field, request=None, **kwargs}
Get a form Field for a ForeignKey.
\end{methoddesc}
\index{formfield\_for\_manytomany() (data.admin.TreatmentInline method)}

\hypertarget{data.admin.TreatmentInline.formfield_for_manytomany}{}\begin{methoddesc}{formfield\_for\_manytomany}{db\_field, request=None, **kwargs}
Get a form Field for a ManyToManyField.
\end{methoddesc}
\index{formset (data.admin.TreatmentInline attribute)}

\hypertarget{data.admin.TreatmentInline.formset}{}\begin{memberdesc}{formset}
alias of \code{BaseInlineFormSet}
\end{memberdesc}
\index{get\_fieldsets() (data.admin.TreatmentInline method)}

\hypertarget{data.admin.TreatmentInline.get_fieldsets}{}\begin{methoddesc}{get\_fieldsets}{request, obj=None}\end{methoddesc}
\index{get\_formset() (data.admin.TreatmentInline method)}

\hypertarget{data.admin.TreatmentInline.get_formset}{}\begin{methoddesc}{get\_formset}{request, obj=None, **kwargs}
Returns a BaseInlineFormSet class for use in admin add/change views.
\end{methoddesc}
\index{media (data.admin.TreatmentInline attribute)}

\hypertarget{data.admin.TreatmentInline.media}{}\begin{memberdesc}{media}\end{memberdesc}
\index{model (data.admin.TreatmentInline attribute)}

\hypertarget{data.admin.TreatmentInline.model}{}\begin{memberdesc}{model}
alias of \code{Treatment}
\end{memberdesc}
\end{classdesc}
\index{VendorAdmin (class in data.admin)}

\hypertarget{data.admin.VendorAdmin}{}\begin{classdesc}{VendorAdmin}{model, admin\_site}~\index{action\_checkbox() (data.admin.VendorAdmin method)}

\hypertarget{data.admin.VendorAdmin.action_checkbox}{}\begin{methoddesc}{action\_checkbox}{obj}
A list\_display column containing a checkbox widget.
\end{methoddesc}
\index{action\_form (data.admin.VendorAdmin attribute)}

\hypertarget{data.admin.VendorAdmin.action_form}{}\begin{memberdesc}{action\_form}
alias of \code{ActionForm}
\end{memberdesc}
\index{add\_view() (data.admin.VendorAdmin method)}

\hypertarget{data.admin.VendorAdmin.add_view}{}\begin{methoddesc}{add\_view}{*args, **kw}
The `add' admin view for this model.
\end{methoddesc}
\index{change\_view() (data.admin.VendorAdmin method)}

\hypertarget{data.admin.VendorAdmin.change_view}{}\begin{methoddesc}{change\_view}{*args, **kw}
The `change' admin view for this model.
\end{methoddesc}
\index{changelist\_view() (data.admin.VendorAdmin method)}

\hypertarget{data.admin.VendorAdmin.changelist_view}{}\begin{methoddesc}{changelist\_view}{request, extra\_context=None}
The `change list' admin view for this model.
\end{methoddesc}
\index{construct\_change\_message() (data.admin.VendorAdmin method)}

\hypertarget{data.admin.VendorAdmin.construct_change_message}{}\begin{methoddesc}{construct\_change\_message}{request, form, formsets}
Construct a change message from a changed object.
\end{methoddesc}
\index{declared\_fieldsets (data.admin.VendorAdmin attribute)}

\hypertarget{data.admin.VendorAdmin.declared_fieldsets}{}\begin{memberdesc}{declared\_fieldsets}\end{memberdesc}
\index{delete\_view() (data.admin.VendorAdmin method)}

\hypertarget{data.admin.VendorAdmin.delete_view}{}\begin{methoddesc}{delete\_view}{request, object\_id, extra\_context=None}
The `delete' admin view for this model.
\end{methoddesc}
\index{form (data.admin.VendorAdmin attribute)}

\hypertarget{data.admin.VendorAdmin.form}{}\begin{memberdesc}{form}
alias of \code{ModelForm}
\end{memberdesc}
\index{formfield\_for\_choice\_field() (data.admin.VendorAdmin method)}

\hypertarget{data.admin.VendorAdmin.formfield_for_choice_field}{}\begin{methoddesc}{formfield\_for\_choice\_field}{db\_field, request=None, **kwargs}
Get a form Field for a database Field that has declared choices.
\end{methoddesc}
\index{formfield\_for\_dbfield() (data.admin.VendorAdmin method)}

\hypertarget{data.admin.VendorAdmin.formfield_for_dbfield}{}\begin{methoddesc}{formfield\_for\_dbfield}{db\_field, **kwargs}
Hook for specifying the form Field instance for a given database Field
instance.

If kwargs are given, they're passed to the form Field's constructor.
\end{methoddesc}
\index{formfield\_for\_foreignkey() (data.admin.VendorAdmin method)}

\hypertarget{data.admin.VendorAdmin.formfield_for_foreignkey}{}\begin{methoddesc}{formfield\_for\_foreignkey}{db\_field, request=None, **kwargs}
Get a form Field for a ForeignKey.
\end{methoddesc}
\index{formfield\_for\_manytomany() (data.admin.VendorAdmin method)}

\hypertarget{data.admin.VendorAdmin.formfield_for_manytomany}{}\begin{methoddesc}{formfield\_for\_manytomany}{db\_field, request=None, **kwargs}
Get a form Field for a ManyToManyField.
\end{methoddesc}
\index{get\_action() (data.admin.VendorAdmin method)}

\hypertarget{data.admin.VendorAdmin.get_action}{}\begin{methoddesc}{get\_action}{action}
Return a given action from a parameter, which can either be a callable,
or the name of a method on the ModelAdmin.  Return is a tuple of
(callable, name, description).
\end{methoddesc}
\index{get\_action\_choices() (data.admin.VendorAdmin method)}

\hypertarget{data.admin.VendorAdmin.get_action_choices}{}\begin{methoddesc}{get\_action\_choices}{request, default\_choices=, {[}('', '---------'){]}}
Return a list of choices for use in a form object.  Each choice is a
tuple (name, description).
\end{methoddesc}
\index{get\_actions() (data.admin.VendorAdmin method)}

\hypertarget{data.admin.VendorAdmin.get_actions}{}\begin{methoddesc}{get\_actions}{request}
Return a dictionary mapping the names of all actions for this
ModelAdmin to a tuple of (callable, name, description) for each action.
\end{methoddesc}
\index{get\_changelist\_form() (data.admin.VendorAdmin method)}

\hypertarget{data.admin.VendorAdmin.get_changelist_form}{}\begin{methoddesc}{get\_changelist\_form}{request, **kwargs}
Returns a Form class for use in the Formset on the changelist page.
\end{methoddesc}
\index{get\_changelist\_formset() (data.admin.VendorAdmin method)}

\hypertarget{data.admin.VendorAdmin.get_changelist_formset}{}\begin{methoddesc}{get\_changelist\_formset}{request, **kwargs}
Returns a FormSet class for use on the changelist page if list\_editable
is used.
\end{methoddesc}
\index{get\_fieldsets() (data.admin.VendorAdmin method)}

\hypertarget{data.admin.VendorAdmin.get_fieldsets}{}\begin{methoddesc}{get\_fieldsets}{request, obj=None}
Hook for specifying fieldsets for the add form.
\end{methoddesc}
\index{get\_form() (data.admin.VendorAdmin method)}

\hypertarget{data.admin.VendorAdmin.get_form}{}\begin{methoddesc}{get\_form}{request, obj=None, **kwargs}
Returns a Form class for use in the admin add view. This is used by
add\_view and change\_view.
\end{methoddesc}
\index{get\_formsets() (data.admin.VendorAdmin method)}

\hypertarget{data.admin.VendorAdmin.get_formsets}{}\begin{methoddesc}{get\_formsets}{request, obj=None}\end{methoddesc}
\index{get\_model\_perms() (data.admin.VendorAdmin method)}

\hypertarget{data.admin.VendorAdmin.get_model_perms}{}\begin{methoddesc}{get\_model\_perms}{request}
Returns a dict of all perms for this model. This dict has the keys
\code{add}, \code{change}, and \code{delete} mapping to the True/False for each
of those actions.
\end{methoddesc}
\index{get\_urls() (data.admin.VendorAdmin method)}

\hypertarget{data.admin.VendorAdmin.get_urls}{}\begin{methoddesc}{get\_urls}{}\end{methoddesc}
\index{has\_add\_permission() (data.admin.VendorAdmin method)}

\hypertarget{data.admin.VendorAdmin.has_add_permission}{}\begin{methoddesc}{has\_add\_permission}{request}
Returns True if the given request has permission to add an object.
\end{methoddesc}
\index{has\_change\_permission() (data.admin.VendorAdmin method)}

\hypertarget{data.admin.VendorAdmin.has_change_permission}{}\begin{methoddesc}{has\_change\_permission}{request, obj=None}
Returns True if the given request has permission to change the given
Django model instance.

If \emph{obj} is None, this should return True if the given request has
permission to change \emph{any} object of the given type.
\end{methoddesc}
\index{has\_delete\_permission() (data.admin.VendorAdmin method)}

\hypertarget{data.admin.VendorAdmin.has_delete_permission}{}\begin{methoddesc}{has\_delete\_permission}{request, obj=None}
Returns True if the given request has permission to change the given
Django model instance.

If \emph{obj} is None, this should return True if the given request has
permission to delete \emph{any} object of the given type.
\end{methoddesc}
\index{history\_view() (data.admin.VendorAdmin method)}

\hypertarget{data.admin.VendorAdmin.history_view}{}\begin{methoddesc}{history\_view}{request, object\_id, extra\_context=None}
The `history' admin view for this model.
\end{methoddesc}
\index{log\_addition() (data.admin.VendorAdmin method)}

\hypertarget{data.admin.VendorAdmin.log_addition}{}\begin{methoddesc}{log\_addition}{request, object}
Log that an object has been successfully added.

The default implementation creates an admin LogEntry object.
\end{methoddesc}
\index{log\_change() (data.admin.VendorAdmin method)}

\hypertarget{data.admin.VendorAdmin.log_change}{}\begin{methoddesc}{log\_change}{request, object, message}
Log that an object has been successfully changed.

The default implementation creates an admin LogEntry object.
\end{methoddesc}
\index{log\_deletion() (data.admin.VendorAdmin method)}

\hypertarget{data.admin.VendorAdmin.log_deletion}{}\begin{methoddesc}{log\_deletion}{request, object, object\_repr}
Log that an object has been successfully deleted. Note that since the
object is deleted, it might no longer be safe to call \emph{any} methods
on the object, hence this method getting object\_repr.

The default implementation creates an admin LogEntry object.
\end{methoddesc}
\index{media (data.admin.VendorAdmin attribute)}

\hypertarget{data.admin.VendorAdmin.media}{}\begin{memberdesc}{media}\end{memberdesc}
\index{message\_user() (data.admin.VendorAdmin method)}

\hypertarget{data.admin.VendorAdmin.message_user}{}\begin{methoddesc}{message\_user}{request, message}
Send a message to the user. The default implementation
posts a message using the auth Message object.
\end{methoddesc}
\index{queryset() (data.admin.VendorAdmin method)}

\hypertarget{data.admin.VendorAdmin.queryset}{}\begin{methoddesc}{queryset}{request}
Returns a QuerySet of all model instances that can be edited by the
admin site. This is used by changelist\_view.
\end{methoddesc}
\index{render\_change\_form() (data.admin.VendorAdmin method)}

\hypertarget{data.admin.VendorAdmin.render_change_form}{}\begin{methoddesc}{render\_change\_form}{request, context, add=False, change=False, form\_url='', obj=None}\end{methoddesc}
\index{response\_action() (data.admin.VendorAdmin method)}

\hypertarget{data.admin.VendorAdmin.response_action}{}\begin{methoddesc}{response\_action}{request, queryset}
Handle an admin action. This is called if a request is POSTed to the
changelist; it returns an HttpResponse if the action was handled, and
None otherwise.
\end{methoddesc}
\index{response\_add() (data.admin.VendorAdmin method)}

\hypertarget{data.admin.VendorAdmin.response_add}{}\begin{methoddesc}{response\_add}{request, obj, post\_url\_continue='../\%s/'}
Determines the HttpResponse for the add\_view stage.
\end{methoddesc}
\index{response\_change() (data.admin.VendorAdmin method)}

\hypertarget{data.admin.VendorAdmin.response_change}{}\begin{methoddesc}{response\_change}{request, obj}
Determines the HttpResponse for the change\_view stage.
\end{methoddesc}
\index{save\_form() (data.admin.VendorAdmin method)}

\hypertarget{data.admin.VendorAdmin.save_form}{}\begin{methoddesc}{save\_form}{request, form, change}
Given a ModelForm return an unsaved instance. \code{change} is True if
the object is being changed, and False if it's being added.
\end{methoddesc}
\index{save\_formset() (data.admin.VendorAdmin method)}

\hypertarget{data.admin.VendorAdmin.save_formset}{}\begin{methoddesc}{save\_formset}{request, form, formset, change}
Given an inline formset save it to the database.
\end{methoddesc}
\index{save\_model() (data.admin.VendorAdmin method)}

\hypertarget{data.admin.VendorAdmin.save_model}{}\begin{methoddesc}{save\_model}{request, obj, form, change}
Given a model instance save it to the database.
\end{methoddesc}
\index{urls (data.admin.VendorAdmin attribute)}

\hypertarget{data.admin.VendorAdmin.urls}{}\begin{memberdesc}{urls}\end{memberdesc}
\end{classdesc}


\section{Animals Package}
\index{animal (module)}
\hypertarget{module-animal}{}
\declaremodule[animal]{}{animal}
\modulesynopsis{}

\subsection{Models}
\index{animal.models (module)}
\hypertarget{module-animal.models}{}
\declaremodule[animal.models]{}{animal.models}
\modulesynopsis{}

\subsection{Forms}
\index{animal.forms (module)}
\hypertarget{module-animal.forms}{}
\declaremodule[animal.forms]{}{animal.forms}
\modulesynopsis{}\index{AnimalChangeForm (class in animal.forms)}

\hypertarget{animal.forms.AnimalChangeForm}{}\begin{classdesc}{AnimalChangeForm}{data=None, files=None, auto\_id='id\_\%s', prefix=None, initial=None, error\_class=\textless{}class 'django.forms.util.ErrorList'\textgreater{}, label\_suffix=':', empty\_permitted=False, instance=None}~\index{AnimalChangeForm.Meta (class in animal.forms)}

\hypertarget{animal.forms.AnimalChangeForm.Meta}{}\begin{classdesc}{Meta}{}~\index{model (animal.forms.AnimalChangeForm.Meta attribute)}

\hypertarget{animal.forms.AnimalChangeForm.Meta.model}{}\begin{memberdesc}{model}
alias of \code{Animal}
\end{memberdesc}
\end{classdesc}
\index{add\_initial\_prefix() (animal.forms.AnimalChangeForm method)}

\hypertarget{animal.forms.AnimalChangeForm.add_initial_prefix}{}\begin{methoddesc}[AnimalChangeForm]{add\_initial\_prefix}{field\_name}
Add a `initial' prefix for checking dynamic initial values
\end{methoddesc}
\index{add\_prefix() (animal.forms.AnimalChangeForm method)}

\hypertarget{animal.forms.AnimalChangeForm.add_prefix}{}\begin{methoddesc}[AnimalChangeForm]{add\_prefix}{field\_name}
Returns the field name with a prefix appended, if this Form has a
prefix set.

Subclasses may wish to override.
\end{methoddesc}
\index{as\_p() (animal.forms.AnimalChangeForm method)}

\hypertarget{animal.forms.AnimalChangeForm.as_p}{}\begin{methoddesc}[AnimalChangeForm]{as\_p}{}
Returns this form rendered as HTML \textless{}p\textgreater{}s.
\end{methoddesc}
\index{as\_table() (animal.forms.AnimalChangeForm method)}

\hypertarget{animal.forms.AnimalChangeForm.as_table}{}\begin{methoddesc}[AnimalChangeForm]{as\_table}{}
Returns this form rendered as HTML \textless{}tr\textgreater{}s -- excluding the \textless{}table\textgreater{}\textless{}/table\textgreater{}.
\end{methoddesc}
\index{as\_ul() (animal.forms.AnimalChangeForm method)}

\hypertarget{animal.forms.AnimalChangeForm.as_ul}{}\begin{methoddesc}[AnimalChangeForm]{as\_ul}{}
Returns this form rendered as HTML \textless{}li\textgreater{}s -- excluding the \textless{}ul\textgreater{}\textless{}/ul\textgreater{}.
\end{methoddesc}
\index{changed\_data (animal.forms.AnimalChangeForm attribute)}

\hypertarget{animal.forms.AnimalChangeForm.changed_data}{}\begin{memberdesc}[AnimalChangeForm]{changed\_data}\end{memberdesc}
\index{clean() (animal.forms.AnimalChangeForm method)}

\hypertarget{animal.forms.AnimalChangeForm.clean}{}\begin{methoddesc}[AnimalChangeForm]{clean}{}\end{methoddesc}
\index{date\_error\_message() (animal.forms.AnimalChangeForm method)}

\hypertarget{animal.forms.AnimalChangeForm.date_error_message}{}\begin{methoddesc}[AnimalChangeForm]{date\_error\_message}{lookup\_type, field, unique\_for}\end{methoddesc}
\index{errors (animal.forms.AnimalChangeForm attribute)}

\hypertarget{animal.forms.AnimalChangeForm.errors}{}\begin{memberdesc}[AnimalChangeForm]{errors}
Returns an ErrorDict for the data provided for the form
\end{memberdesc}
\index{full\_clean() (animal.forms.AnimalChangeForm method)}

\hypertarget{animal.forms.AnimalChangeForm.full_clean}{}\begin{methoddesc}[AnimalChangeForm]{full\_clean}{}
Cleans all of self.data and populates self.\_errors and
self.cleaned\_data.
\end{methoddesc}
\index{has\_changed() (animal.forms.AnimalChangeForm method)}

\hypertarget{animal.forms.AnimalChangeForm.has_changed}{}\begin{methoddesc}[AnimalChangeForm]{has\_changed}{}
Returns True if data differs from initial.
\end{methoddesc}
\index{hidden\_fields() (animal.forms.AnimalChangeForm method)}

\hypertarget{animal.forms.AnimalChangeForm.hidden_fields}{}\begin{methoddesc}[AnimalChangeForm]{hidden\_fields}{}
Returns a list of all the BoundField objects that are hidden fields.
Useful for manual form layout in templates.
\end{methoddesc}
\index{is\_multipart() (animal.forms.AnimalChangeForm method)}

\hypertarget{animal.forms.AnimalChangeForm.is_multipart}{}\begin{methoddesc}[AnimalChangeForm]{is\_multipart}{}
Returns True if the form needs to be multipart-encrypted, i.e. it has
FileInput. Otherwise, False.
\end{methoddesc}
\index{is\_valid() (animal.forms.AnimalChangeForm method)}

\hypertarget{animal.forms.AnimalChangeForm.is_valid}{}\begin{methoddesc}[AnimalChangeForm]{is\_valid}{}
Returns True if the form has no errors. Otherwise, False. If errors are
being ignored, returns False.
\end{methoddesc}
\index{media (animal.forms.AnimalChangeForm attribute)}

\hypertarget{animal.forms.AnimalChangeForm.media}{}\begin{memberdesc}[AnimalChangeForm]{media}\end{memberdesc}
\index{non\_field\_errors() (animal.forms.AnimalChangeForm method)}

\hypertarget{animal.forms.AnimalChangeForm.non_field_errors}{}\begin{methoddesc}[AnimalChangeForm]{non\_field\_errors}{}
Returns an ErrorList of errors that aren't associated with a particular
field -- i.e., from Form.clean(). Returns an empty ErrorList if there
are none.
\end{methoddesc}
\index{save() (animal.forms.AnimalChangeForm method)}

\hypertarget{animal.forms.AnimalChangeForm.save}{}\begin{methoddesc}[AnimalChangeForm]{save}{commit=True}
Saves this \code{form}`s cleaned\_data into model instance
\code{self.instance}.

If commit=True, then the changes to \code{instance} will be saved to the
database. Returns \code{instance}.
\end{methoddesc}
\index{unique\_error\_message() (animal.forms.AnimalChangeForm method)}

\hypertarget{animal.forms.AnimalChangeForm.unique_error_message}{}\begin{methoddesc}[AnimalChangeForm]{unique\_error\_message}{unique\_check}\end{methoddesc}
\index{validate\_unique() (animal.forms.AnimalChangeForm method)}

\hypertarget{animal.forms.AnimalChangeForm.validate_unique}{}\begin{methoddesc}[AnimalChangeForm]{validate\_unique}{}\end{methoddesc}
\index{visible\_fields() (animal.forms.AnimalChangeForm method)}

\hypertarget{animal.forms.AnimalChangeForm.visible_fields}{}\begin{methoddesc}[AnimalChangeForm]{visible\_fields}{}
Returns a list of BoundField objects that aren't hidden fields.
The opposite of the hidden\_fields() method.
\end{methoddesc}
\end{classdesc}
\index{AnimalForm (class in animal.forms)}

\hypertarget{animal.forms.AnimalForm}{}\begin{classdesc}{AnimalForm}{data=None, files=None, auto\_id='id\_\%s', prefix=None, initial=None, error\_class=\textless{}class 'django.forms.util.ErrorList'\textgreater{}, label\_suffix=':', empty\_permitted=False, instance=None}~\index{AnimalForm.Meta (class in animal.forms)}

\hypertarget{animal.forms.AnimalForm.Meta}{}\begin{classdesc}{Meta}{}~\index{model (animal.forms.AnimalForm.Meta attribute)}

\hypertarget{animal.forms.AnimalForm.Meta.model}{}\begin{memberdesc}{model}
alias of \code{Animal}
\end{memberdesc}
\end{classdesc}
\index{add\_initial\_prefix() (animal.forms.AnimalForm method)}

\hypertarget{animal.forms.AnimalForm.add_initial_prefix}{}\begin{methoddesc}[AnimalForm]{add\_initial\_prefix}{field\_name}
Add a `initial' prefix for checking dynamic initial values
\end{methoddesc}
\index{add\_prefix() (animal.forms.AnimalForm method)}

\hypertarget{animal.forms.AnimalForm.add_prefix}{}\begin{methoddesc}[AnimalForm]{add\_prefix}{field\_name}
Returns the field name with a prefix appended, if this Form has a
prefix set.

Subclasses may wish to override.
\end{methoddesc}
\index{as\_p() (animal.forms.AnimalForm method)}

\hypertarget{animal.forms.AnimalForm.as_p}{}\begin{methoddesc}[AnimalForm]{as\_p}{}
Returns this form rendered as HTML \textless{}p\textgreater{}s.
\end{methoddesc}
\index{as\_table() (animal.forms.AnimalForm method)}

\hypertarget{animal.forms.AnimalForm.as_table}{}\begin{methoddesc}[AnimalForm]{as\_table}{}
Returns this form rendered as HTML \textless{}tr\textgreater{}s -- excluding the \textless{}table\textgreater{}\textless{}/table\textgreater{}.
\end{methoddesc}
\index{as\_ul() (animal.forms.AnimalForm method)}

\hypertarget{animal.forms.AnimalForm.as_ul}{}\begin{methoddesc}[AnimalForm]{as\_ul}{}
Returns this form rendered as HTML \textless{}li\textgreater{}s -- excluding the \textless{}ul\textgreater{}\textless{}/ul\textgreater{}.
\end{methoddesc}
\index{changed\_data (animal.forms.AnimalForm attribute)}

\hypertarget{animal.forms.AnimalForm.changed_data}{}\begin{memberdesc}[AnimalForm]{changed\_data}\end{memberdesc}
\index{clean() (animal.forms.AnimalForm method)}

\hypertarget{animal.forms.AnimalForm.clean}{}\begin{methoddesc}[AnimalForm]{clean}{}\end{methoddesc}
\index{date\_error\_message() (animal.forms.AnimalForm method)}

\hypertarget{animal.forms.AnimalForm.date_error_message}{}\begin{methoddesc}[AnimalForm]{date\_error\_message}{lookup\_type, field, unique\_for}\end{methoddesc}
\index{errors (animal.forms.AnimalForm attribute)}

\hypertarget{animal.forms.AnimalForm.errors}{}\begin{memberdesc}[AnimalForm]{errors}
Returns an ErrorDict for the data provided for the form
\end{memberdesc}
\index{full\_clean() (animal.forms.AnimalForm method)}

\hypertarget{animal.forms.AnimalForm.full_clean}{}\begin{methoddesc}[AnimalForm]{full\_clean}{}
Cleans all of self.data and populates self.\_errors and
self.cleaned\_data.
\end{methoddesc}
\index{has\_changed() (animal.forms.AnimalForm method)}

\hypertarget{animal.forms.AnimalForm.has_changed}{}\begin{methoddesc}[AnimalForm]{has\_changed}{}
Returns True if data differs from initial.
\end{methoddesc}
\index{hidden\_fields() (animal.forms.AnimalForm method)}

\hypertarget{animal.forms.AnimalForm.hidden_fields}{}\begin{methoddesc}[AnimalForm]{hidden\_fields}{}
Returns a list of all the BoundField objects that are hidden fields.
Useful for manual form layout in templates.
\end{methoddesc}
\index{is\_multipart() (animal.forms.AnimalForm method)}

\hypertarget{animal.forms.AnimalForm.is_multipart}{}\begin{methoddesc}[AnimalForm]{is\_multipart}{}
Returns True if the form needs to be multipart-encrypted, i.e. it has
FileInput. Otherwise, False.
\end{methoddesc}
\index{is\_valid() (animal.forms.AnimalForm method)}

\hypertarget{animal.forms.AnimalForm.is_valid}{}\begin{methoddesc}[AnimalForm]{is\_valid}{}
Returns True if the form has no errors. Otherwise, False. If errors are
being ignored, returns False.
\end{methoddesc}
\index{media (animal.forms.AnimalForm attribute)}

\hypertarget{animal.forms.AnimalForm.media}{}\begin{memberdesc}[AnimalForm]{media}\end{memberdesc}
\index{non\_field\_errors() (animal.forms.AnimalForm method)}

\hypertarget{animal.forms.AnimalForm.non_field_errors}{}\begin{methoddesc}[AnimalForm]{non\_field\_errors}{}
Returns an ErrorList of errors that aren't associated with a particular
field -- i.e., from Form.clean(). Returns an empty ErrorList if there
are none.
\end{methoddesc}
\index{save() (animal.forms.AnimalForm method)}

\hypertarget{animal.forms.AnimalForm.save}{}\begin{methoddesc}[AnimalForm]{save}{commit=True}
Saves this \code{form}`s cleaned\_data into model instance
\code{self.instance}.

If commit=True, then the changes to \code{instance} will be saved to the
database. Returns \code{instance}.
\end{methoddesc}
\index{unique\_error\_message() (animal.forms.AnimalForm method)}

\hypertarget{animal.forms.AnimalForm.unique_error_message}{}\begin{methoddesc}[AnimalForm]{unique\_error\_message}{unique\_check}\end{methoddesc}
\index{validate\_unique() (animal.forms.AnimalForm method)}

\hypertarget{animal.forms.AnimalForm.validate_unique}{}\begin{methoddesc}[AnimalForm]{validate\_unique}{}\end{methoddesc}
\index{visible\_fields() (animal.forms.AnimalForm method)}

\hypertarget{animal.forms.AnimalForm.visible_fields}{}\begin{methoddesc}[AnimalForm]{visible\_fields}{}
Returns a list of BoundField objects that aren't hidden fields.
The opposite of the hidden\_fields() method.
\end{methoddesc}
\end{classdesc}


\subsection{Views and URLs}
\index{animal.views (module)}
\hypertarget{module-animal.views}{}
\declaremodule[animal.views]{}{animal.views}
\modulesynopsis{}\index{animal\_change (in module animal.views)}

\hypertarget{animal.views.animal_change}{}\begin{memberdesc}[animal.views]{animal\_change}\end{memberdesc}
\index{animal\_detail (in module animal.views)}

\hypertarget{animal.views.animal_detail}{}\begin{memberdesc}[animal.views]{animal\_detail}\end{memberdesc}
\index{animal\_new (in module animal.views)}

\hypertarget{animal.views.animal_new}{}\begin{memberdesc}[animal.views]{animal\_new}\end{memberdesc}
\index{breeding (in module animal.views)}

\hypertarget{animal.views.breeding}{}\begin{memberdesc}[animal.views]{breeding}\end{memberdesc}
\index{breeding\_all (in module animal.views)}

\hypertarget{animal.views.breeding_all}{}\begin{memberdesc}[animal.views]{breeding\_all}\end{memberdesc}
\index{breeding\_change (in module animal.views)}

\hypertarget{animal.views.breeding_change}{}\begin{memberdesc}[animal.views]{breeding\_change}\end{memberdesc}
\index{breeding\_detail (in module animal.views)}

\hypertarget{animal.views.breeding_detail}{}\begin{memberdesc}[animal.views]{breeding\_detail}\end{memberdesc}
\index{breeding\_pups (in module animal.views)}

\hypertarget{animal.views.breeding_pups}{}\begin{memberdesc}[animal.views]{breeding\_pups}\end{memberdesc}
\index{strain\_detail (in module animal.views)}

\hypertarget{animal.views.strain_detail}{}\begin{memberdesc}[animal.views]{strain\_detail}\end{memberdesc}
\index{strain\_detail\_all (in module animal.views)}

\hypertarget{animal.views.strain_detail_all}{}\begin{memberdesc}[animal.views]{strain\_detail\_all}\end{memberdesc}
\index{strain\_list (in module animal.views)}

\hypertarget{animal.views.strain_list}{}\begin{memberdesc}[animal.views]{strain\_list}\end{memberdesc}
\index{animal.urls (module)}
\hypertarget{module-animal.urls}{}
\declaremodule[animal.urls]{}{animal.urls}
\modulesynopsis{}

\subsection{Administrative Site Configuration}
\index{animal.admin (module)}
\hypertarget{module-animal.admin}{}
\declaremodule[animal.admin]{}{animal.admin}
\modulesynopsis{}\index{AnimalAdmin (class in animal.admin)}

\hypertarget{animal.admin.AnimalAdmin}{}\begin{classdesc}{AnimalAdmin}{model, admin\_site}~\index{action\_checkbox() (animal.admin.AnimalAdmin method)}

\hypertarget{animal.admin.AnimalAdmin.action_checkbox}{}\begin{methoddesc}{action\_checkbox}{obj}
A list\_display column containing a checkbox widget.
\end{methoddesc}
\index{action\_form (animal.admin.AnimalAdmin attribute)}

\hypertarget{animal.admin.AnimalAdmin.action_form}{}\begin{memberdesc}{action\_form}
alias of \code{ActionForm}
\end{memberdesc}
\index{add\_view() (animal.admin.AnimalAdmin method)}

\hypertarget{animal.admin.AnimalAdmin.add_view}{}\begin{methoddesc}{add\_view}{*args, **kw}
The `add' admin view for this model.
\end{methoddesc}
\index{change\_view() (animal.admin.AnimalAdmin method)}

\hypertarget{animal.admin.AnimalAdmin.change_view}{}\begin{methoddesc}{change\_view}{*args, **kw}
The `change' admin view for this model.
\end{methoddesc}
\index{changelist\_view() (animal.admin.AnimalAdmin method)}

\hypertarget{animal.admin.AnimalAdmin.changelist_view}{}\begin{methoddesc}{changelist\_view}{request, extra\_context=None}
The `change list' admin view for this model.
\end{methoddesc}
\index{construct\_change\_message() (animal.admin.AnimalAdmin method)}

\hypertarget{animal.admin.AnimalAdmin.construct_change_message}{}\begin{methoddesc}{construct\_change\_message}{request, form, formsets}
Construct a change message from a changed object.
\end{methoddesc}
\index{declared\_fieldsets (animal.admin.AnimalAdmin attribute)}

\hypertarget{animal.admin.AnimalAdmin.declared_fieldsets}{}\begin{memberdesc}{declared\_fieldsets}\end{memberdesc}
\index{delete\_view() (animal.admin.AnimalAdmin method)}

\hypertarget{animal.admin.AnimalAdmin.delete_view}{}\begin{methoddesc}{delete\_view}{request, object\_id, extra\_context=None}
The `delete' admin view for this model.
\end{methoddesc}
\index{form (animal.admin.AnimalAdmin attribute)}

\hypertarget{animal.admin.AnimalAdmin.form}{}\begin{memberdesc}{form}
alias of \code{ModelForm}
\end{memberdesc}
\index{formfield\_for\_choice\_field() (animal.admin.AnimalAdmin method)}

\hypertarget{animal.admin.AnimalAdmin.formfield_for_choice_field}{}\begin{methoddesc}{formfield\_for\_choice\_field}{db\_field, request=None, **kwargs}
Get a form Field for a database Field that has declared choices.
\end{methoddesc}
\index{formfield\_for\_dbfield() (animal.admin.AnimalAdmin method)}

\hypertarget{animal.admin.AnimalAdmin.formfield_for_dbfield}{}\begin{methoddesc}{formfield\_for\_dbfield}{db\_field, **kwargs}
Hook for specifying the form Field instance for a given database Field
instance.

If kwargs are given, they're passed to the form Field's constructor.
\end{methoddesc}
\index{formfield\_for\_foreignkey() (animal.admin.AnimalAdmin method)}

\hypertarget{animal.admin.AnimalAdmin.formfield_for_foreignkey}{}\begin{methoddesc}{formfield\_for\_foreignkey}{db\_field, request=None, **kwargs}
Get a form Field for a ForeignKey.
\end{methoddesc}
\index{formfield\_for\_manytomany() (animal.admin.AnimalAdmin method)}

\hypertarget{animal.admin.AnimalAdmin.formfield_for_manytomany}{}\begin{methoddesc}{formfield\_for\_manytomany}{db\_field, request=None, **kwargs}
Get a form Field for a ManyToManyField.
\end{methoddesc}
\index{get\_action() (animal.admin.AnimalAdmin method)}

\hypertarget{animal.admin.AnimalAdmin.get_action}{}\begin{methoddesc}{get\_action}{action}
Return a given action from a parameter, which can either be a callable,
or the name of a method on the ModelAdmin.  Return is a tuple of
(callable, name, description).
\end{methoddesc}
\index{get\_action\_choices() (animal.admin.AnimalAdmin method)}

\hypertarget{animal.admin.AnimalAdmin.get_action_choices}{}\begin{methoddesc}{get\_action\_choices}{request, default\_choices=, {[}('', '---------'){]}}
Return a list of choices for use in a form object.  Each choice is a
tuple (name, description).
\end{methoddesc}
\index{get\_actions() (animal.admin.AnimalAdmin method)}

\hypertarget{animal.admin.AnimalAdmin.get_actions}{}\begin{methoddesc}{get\_actions}{request}
Return a dictionary mapping the names of all actions for this
ModelAdmin to a tuple of (callable, name, description) for each action.
\end{methoddesc}
\index{get\_changelist\_form() (animal.admin.AnimalAdmin method)}

\hypertarget{animal.admin.AnimalAdmin.get_changelist_form}{}\begin{methoddesc}{get\_changelist\_form}{request, **kwargs}
Returns a Form class for use in the Formset on the changelist page.
\end{methoddesc}
\index{get\_changelist\_formset() (animal.admin.AnimalAdmin method)}

\hypertarget{animal.admin.AnimalAdmin.get_changelist_formset}{}\begin{methoddesc}{get\_changelist\_formset}{request, **kwargs}
Returns a FormSet class for use on the changelist page if list\_editable
is used.
\end{methoddesc}
\index{get\_fieldsets() (animal.admin.AnimalAdmin method)}

\hypertarget{animal.admin.AnimalAdmin.get_fieldsets}{}\begin{methoddesc}{get\_fieldsets}{request, obj=None}
Hook for specifying fieldsets for the add form.
\end{methoddesc}
\index{get\_form() (animal.admin.AnimalAdmin method)}

\hypertarget{animal.admin.AnimalAdmin.get_form}{}\begin{methoddesc}{get\_form}{request, obj=None, **kwargs}
Returns a Form class for use in the admin add view. This is used by
add\_view and change\_view.
\end{methoddesc}
\index{get\_formsets() (animal.admin.AnimalAdmin method)}

\hypertarget{animal.admin.AnimalAdmin.get_formsets}{}\begin{methoddesc}{get\_formsets}{request, obj=None}\end{methoddesc}
\index{get\_model\_perms() (animal.admin.AnimalAdmin method)}

\hypertarget{animal.admin.AnimalAdmin.get_model_perms}{}\begin{methoddesc}{get\_model\_perms}{request}
Returns a dict of all perms for this model. This dict has the keys
\code{add}, \code{change}, and \code{delete} mapping to the True/False for each
of those actions.
\end{methoddesc}
\index{get\_urls() (animal.admin.AnimalAdmin method)}

\hypertarget{animal.admin.AnimalAdmin.get_urls}{}\begin{methoddesc}{get\_urls}{}\end{methoddesc}
\index{has\_add\_permission() (animal.admin.AnimalAdmin method)}

\hypertarget{animal.admin.AnimalAdmin.has_add_permission}{}\begin{methoddesc}{has\_add\_permission}{request}
Returns True if the given request has permission to add an object.
\end{methoddesc}
\index{has\_change\_permission() (animal.admin.AnimalAdmin method)}

\hypertarget{animal.admin.AnimalAdmin.has_change_permission}{}\begin{methoddesc}{has\_change\_permission}{request, obj=None}
Returns True if the given request has permission to change the given
Django model instance.

If \emph{obj} is None, this should return True if the given request has
permission to change \emph{any} object of the given type.
\end{methoddesc}
\index{has\_delete\_permission() (animal.admin.AnimalAdmin method)}

\hypertarget{animal.admin.AnimalAdmin.has_delete_permission}{}\begin{methoddesc}{has\_delete\_permission}{request, obj=None}
Returns True if the given request has permission to change the given
Django model instance.

If \emph{obj} is None, this should return True if the given request has
permission to delete \emph{any} object of the given type.
\end{methoddesc}
\index{history\_view() (animal.admin.AnimalAdmin method)}

\hypertarget{animal.admin.AnimalAdmin.history_view}{}\begin{methoddesc}{history\_view}{request, object\_id, extra\_context=None}
The `history' admin view for this model.
\end{methoddesc}
\index{log\_addition() (animal.admin.AnimalAdmin method)}

\hypertarget{animal.admin.AnimalAdmin.log_addition}{}\begin{methoddesc}{log\_addition}{request, object}
Log that an object has been successfully added.

The default implementation creates an admin LogEntry object.
\end{methoddesc}
\index{log\_change() (animal.admin.AnimalAdmin method)}

\hypertarget{animal.admin.AnimalAdmin.log_change}{}\begin{methoddesc}{log\_change}{request, object, message}
Log that an object has been successfully changed.

The default implementation creates an admin LogEntry object.
\end{methoddesc}
\index{log\_deletion() (animal.admin.AnimalAdmin method)}

\hypertarget{animal.admin.AnimalAdmin.log_deletion}{}\begin{methoddesc}{log\_deletion}{request, object, object\_repr}
Log that an object has been successfully deleted. Note that since the
object is deleted, it might no longer be safe to call \emph{any} methods
on the object, hence this method getting object\_repr.

The default implementation creates an admin LogEntry object.
\end{methoddesc}
\index{mark\_sacrificed() (animal.admin.AnimalAdmin method)}

\hypertarget{animal.admin.AnimalAdmin.mark_sacrificed}{}\begin{methoddesc}{mark\_sacrificed}{request, queryset}\end{methoddesc}
\index{media (animal.admin.AnimalAdmin attribute)}

\hypertarget{animal.admin.AnimalAdmin.media}{}\begin{memberdesc}{media}\end{memberdesc}
\index{message\_user() (animal.admin.AnimalAdmin method)}

\hypertarget{animal.admin.AnimalAdmin.message_user}{}\begin{methoddesc}{message\_user}{request, message}
Send a message to the user. The default implementation
posts a message using the auth Message object.
\end{methoddesc}
\index{queryset() (animal.admin.AnimalAdmin method)}

\hypertarget{animal.admin.AnimalAdmin.queryset}{}\begin{methoddesc}{queryset}{request}
Returns a QuerySet of all model instances that can be edited by the
admin site. This is used by changelist\_view.
\end{methoddesc}
\index{render\_change\_form() (animal.admin.AnimalAdmin method)}

\hypertarget{animal.admin.AnimalAdmin.render_change_form}{}\begin{methoddesc}{render\_change\_form}{request, context, add=False, change=False, form\_url='', obj=None}\end{methoddesc}
\index{response\_action() (animal.admin.AnimalAdmin method)}

\hypertarget{animal.admin.AnimalAdmin.response_action}{}\begin{methoddesc}{response\_action}{request, queryset}
Handle an admin action. This is called if a request is POSTed to the
changelist; it returns an HttpResponse if the action was handled, and
None otherwise.
\end{methoddesc}
\index{response\_add() (animal.admin.AnimalAdmin method)}

\hypertarget{animal.admin.AnimalAdmin.response_add}{}\begin{methoddesc}{response\_add}{request, obj, post\_url\_continue='../\%s/'}
Determines the HttpResponse for the add\_view stage.
\end{methoddesc}
\index{response\_change() (animal.admin.AnimalAdmin method)}

\hypertarget{animal.admin.AnimalAdmin.response_change}{}\begin{methoddesc}{response\_change}{request, obj}
Determines the HttpResponse for the change\_view stage.
\end{methoddesc}
\index{save\_form() (animal.admin.AnimalAdmin method)}

\hypertarget{animal.admin.AnimalAdmin.save_form}{}\begin{methoddesc}{save\_form}{request, form, change}
Given a ModelForm return an unsaved instance. \code{change} is True if
the object is being changed, and False if it's being added.
\end{methoddesc}
\index{save\_formset() (animal.admin.AnimalAdmin method)}

\hypertarget{animal.admin.AnimalAdmin.save_formset}{}\begin{methoddesc}{save\_formset}{request, form, formset, change}
Given an inline formset save it to the database.
\end{methoddesc}
\index{save\_model() (animal.admin.AnimalAdmin method)}

\hypertarget{animal.admin.AnimalAdmin.save_model}{}\begin{methoddesc}{save\_model}{request, obj, form, change}
Given a model instance save it to the database.
\end{methoddesc}
\index{urls (animal.admin.AnimalAdmin attribute)}

\hypertarget{animal.admin.AnimalAdmin.urls}{}\begin{memberdesc}{urls}\end{memberdesc}
\end{classdesc}
\index{AnimalInline (class in animal.admin)}

\hypertarget{animal.admin.AnimalInline}{}\begin{classdesc}{AnimalInline}{parent\_model, admin\_site}~\index{declared\_fieldsets (animal.admin.AnimalInline attribute)}

\hypertarget{animal.admin.AnimalInline.declared_fieldsets}{}\begin{memberdesc}{declared\_fieldsets}\end{memberdesc}
\index{form (animal.admin.AnimalInline attribute)}

\hypertarget{animal.admin.AnimalInline.form}{}\begin{memberdesc}{form}
alias of \code{ModelForm}
\end{memberdesc}
\index{formfield\_for\_choice\_field() (animal.admin.AnimalInline method)}

\hypertarget{animal.admin.AnimalInline.formfield_for_choice_field}{}\begin{methoddesc}{formfield\_for\_choice\_field}{db\_field, request=None, **kwargs}
Get a form Field for a database Field that has declared choices.
\end{methoddesc}
\index{formfield\_for\_dbfield() (animal.admin.AnimalInline method)}

\hypertarget{animal.admin.AnimalInline.formfield_for_dbfield}{}\begin{methoddesc}{formfield\_for\_dbfield}{db\_field, **kwargs}
Hook for specifying the form Field instance for a given database Field
instance.

If kwargs are given, they're passed to the form Field's constructor.
\end{methoddesc}
\index{formfield\_for\_foreignkey() (animal.admin.AnimalInline method)}

\hypertarget{animal.admin.AnimalInline.formfield_for_foreignkey}{}\begin{methoddesc}{formfield\_for\_foreignkey}{db\_field, request=None, **kwargs}
Get a form Field for a ForeignKey.
\end{methoddesc}
\index{formfield\_for\_manytomany() (animal.admin.AnimalInline method)}

\hypertarget{animal.admin.AnimalInline.formfield_for_manytomany}{}\begin{methoddesc}{formfield\_for\_manytomany}{db\_field, request=None, **kwargs}
Get a form Field for a ManyToManyField.
\end{methoddesc}
\index{formset (animal.admin.AnimalInline attribute)}

\hypertarget{animal.admin.AnimalInline.formset}{}\begin{memberdesc}{formset}
alias of \code{BaseInlineFormSet}
\end{memberdesc}
\index{get\_fieldsets() (animal.admin.AnimalInline method)}

\hypertarget{animal.admin.AnimalInline.get_fieldsets}{}\begin{methoddesc}{get\_fieldsets}{request, obj=None}\end{methoddesc}
\index{get\_formset() (animal.admin.AnimalInline method)}

\hypertarget{animal.admin.AnimalInline.get_formset}{}\begin{methoddesc}{get\_formset}{request, obj=None, **kwargs}
Returns a BaseInlineFormSet class for use in admin add/change views.
\end{methoddesc}
\index{media (animal.admin.AnimalInline attribute)}

\hypertarget{animal.admin.AnimalInline.media}{}\begin{memberdesc}{media}\end{memberdesc}
\index{model (animal.admin.AnimalInline attribute)}

\hypertarget{animal.admin.AnimalInline.model}{}\begin{memberdesc}{model}
alias of \code{Animal}
\end{memberdesc}
\end{classdesc}
\index{BreedingAdmin (class in animal.admin)}

\hypertarget{animal.admin.BreedingAdmin}{}\begin{classdesc}{BreedingAdmin}{model, admin\_site}~\index{action\_checkbox() (animal.admin.BreedingAdmin method)}

\hypertarget{animal.admin.BreedingAdmin.action_checkbox}{}\begin{methoddesc}{action\_checkbox}{obj}
A list\_display column containing a checkbox widget.
\end{methoddesc}
\index{action\_form (animal.admin.BreedingAdmin attribute)}

\hypertarget{animal.admin.BreedingAdmin.action_form}{}\begin{memberdesc}{action\_form}
alias of \code{ActionForm}
\end{memberdesc}
\index{add\_view() (animal.admin.BreedingAdmin method)}

\hypertarget{animal.admin.BreedingAdmin.add_view}{}\begin{methoddesc}{add\_view}{*args, **kw}
The `add' admin view for this model.
\end{methoddesc}
\index{change\_view() (animal.admin.BreedingAdmin method)}

\hypertarget{animal.admin.BreedingAdmin.change_view}{}\begin{methoddesc}{change\_view}{*args, **kw}
The `change' admin view for this model.
\end{methoddesc}
\index{changelist\_view() (animal.admin.BreedingAdmin method)}

\hypertarget{animal.admin.BreedingAdmin.changelist_view}{}\begin{methoddesc}{changelist\_view}{request, extra\_context=None}
The `change list' admin view for this model.
\end{methoddesc}
\index{construct\_change\_message() (animal.admin.BreedingAdmin method)}

\hypertarget{animal.admin.BreedingAdmin.construct_change_message}{}\begin{methoddesc}{construct\_change\_message}{request, form, formsets}
Construct a change message from a changed object.
\end{methoddesc}
\index{declared\_fieldsets (animal.admin.BreedingAdmin attribute)}

\hypertarget{animal.admin.BreedingAdmin.declared_fieldsets}{}\begin{memberdesc}{declared\_fieldsets}\end{memberdesc}
\index{delete\_view() (animal.admin.BreedingAdmin method)}

\hypertarget{animal.admin.BreedingAdmin.delete_view}{}\begin{methoddesc}{delete\_view}{request, object\_id, extra\_context=None}
The `delete' admin view for this model.
\end{methoddesc}
\index{form (animal.admin.BreedingAdmin attribute)}

\hypertarget{animal.admin.BreedingAdmin.form}{}\begin{memberdesc}{form}
alias of \code{ModelForm}
\end{memberdesc}
\index{formfield\_for\_choice\_field() (animal.admin.BreedingAdmin method)}

\hypertarget{animal.admin.BreedingAdmin.formfield_for_choice_field}{}\begin{methoddesc}{formfield\_for\_choice\_field}{db\_field, request=None, **kwargs}
Get a form Field for a database Field that has declared choices.
\end{methoddesc}
\index{formfield\_for\_dbfield() (animal.admin.BreedingAdmin method)}

\hypertarget{animal.admin.BreedingAdmin.formfield_for_dbfield}{}\begin{methoddesc}{formfield\_for\_dbfield}{db\_field, **kwargs}
Hook for specifying the form Field instance for a given database Field
instance.

If kwargs are given, they're passed to the form Field's constructor.
\end{methoddesc}
\index{formfield\_for\_foreignkey() (animal.admin.BreedingAdmin method)}

\hypertarget{animal.admin.BreedingAdmin.formfield_for_foreignkey}{}\begin{methoddesc}{formfield\_for\_foreignkey}{db\_field, request=None, **kwargs}
Get a form Field for a ForeignKey.
\end{methoddesc}
\index{formfield\_for\_manytomany() (animal.admin.BreedingAdmin method)}

\hypertarget{animal.admin.BreedingAdmin.formfield_for_manytomany}{}\begin{methoddesc}{formfield\_for\_manytomany}{db\_field, request=None, **kwargs}
Get a form Field for a ManyToManyField.
\end{methoddesc}
\index{get\_action() (animal.admin.BreedingAdmin method)}

\hypertarget{animal.admin.BreedingAdmin.get_action}{}\begin{methoddesc}{get\_action}{action}
Return a given action from a parameter, which can either be a callable,
or the name of a method on the ModelAdmin.  Return is a tuple of
(callable, name, description).
\end{methoddesc}
\index{get\_action\_choices() (animal.admin.BreedingAdmin method)}

\hypertarget{animal.admin.BreedingAdmin.get_action_choices}{}\begin{methoddesc}{get\_action\_choices}{request, default\_choices=, {[}('', '---------'){]}}
Return a list of choices for use in a form object.  Each choice is a
tuple (name, description).
\end{methoddesc}
\index{get\_actions() (animal.admin.BreedingAdmin method)}

\hypertarget{animal.admin.BreedingAdmin.get_actions}{}\begin{methoddesc}{get\_actions}{request}
Return a dictionary mapping the names of all actions for this
ModelAdmin to a tuple of (callable, name, description) for each action.
\end{methoddesc}
\index{get\_changelist\_form() (animal.admin.BreedingAdmin method)}

\hypertarget{animal.admin.BreedingAdmin.get_changelist_form}{}\begin{methoddesc}{get\_changelist\_form}{request, **kwargs}
Returns a Form class for use in the Formset on the changelist page.
\end{methoddesc}
\index{get\_changelist\_formset() (animal.admin.BreedingAdmin method)}

\hypertarget{animal.admin.BreedingAdmin.get_changelist_formset}{}\begin{methoddesc}{get\_changelist\_formset}{request, **kwargs}
Returns a FormSet class for use on the changelist page if list\_editable
is used.
\end{methoddesc}
\index{get\_fieldsets() (animal.admin.BreedingAdmin method)}

\hypertarget{animal.admin.BreedingAdmin.get_fieldsets}{}\begin{methoddesc}{get\_fieldsets}{request, obj=None}
Hook for specifying fieldsets for the add form.
\end{methoddesc}
\index{get\_form() (animal.admin.BreedingAdmin method)}

\hypertarget{animal.admin.BreedingAdmin.get_form}{}\begin{methoddesc}{get\_form}{request, obj=None, **kwargs}
Returns a Form class for use in the admin add view. This is used by
add\_view and change\_view.
\end{methoddesc}
\index{get\_formsets() (animal.admin.BreedingAdmin method)}

\hypertarget{animal.admin.BreedingAdmin.get_formsets}{}\begin{methoddesc}{get\_formsets}{request, obj=None}\end{methoddesc}
\index{get\_model\_perms() (animal.admin.BreedingAdmin method)}

\hypertarget{animal.admin.BreedingAdmin.get_model_perms}{}\begin{methoddesc}{get\_model\_perms}{request}
Returns a dict of all perms for this model. This dict has the keys
\code{add}, \code{change}, and \code{delete} mapping to the True/False for each
of those actions.
\end{methoddesc}
\index{get\_urls() (animal.admin.BreedingAdmin method)}

\hypertarget{animal.admin.BreedingAdmin.get_urls}{}\begin{methoddesc}{get\_urls}{}\end{methoddesc}
\index{has\_add\_permission() (animal.admin.BreedingAdmin method)}

\hypertarget{animal.admin.BreedingAdmin.has_add_permission}{}\begin{methoddesc}{has\_add\_permission}{request}
Returns True if the given request has permission to add an object.
\end{methoddesc}
\index{has\_change\_permission() (animal.admin.BreedingAdmin method)}

\hypertarget{animal.admin.BreedingAdmin.has_change_permission}{}\begin{methoddesc}{has\_change\_permission}{request, obj=None}
Returns True if the given request has permission to change the given
Django model instance.

If \emph{obj} is None, this should return True if the given request has
permission to change \emph{any} object of the given type.
\end{methoddesc}
\index{has\_delete\_permission() (animal.admin.BreedingAdmin method)}

\hypertarget{animal.admin.BreedingAdmin.has_delete_permission}{}\begin{methoddesc}{has\_delete\_permission}{request, obj=None}
Returns True if the given request has permission to change the given
Django model instance.

If \emph{obj} is None, this should return True if the given request has
permission to delete \emph{any} object of the given type.
\end{methoddesc}
\index{history\_view() (animal.admin.BreedingAdmin method)}

\hypertarget{animal.admin.BreedingAdmin.history_view}{}\begin{methoddesc}{history\_view}{request, object\_id, extra\_context=None}
The `history' admin view for this model.
\end{methoddesc}
\index{log\_addition() (animal.admin.BreedingAdmin method)}

\hypertarget{animal.admin.BreedingAdmin.log_addition}{}\begin{methoddesc}{log\_addition}{request, object}
Log that an object has been successfully added.

The default implementation creates an admin LogEntry object.
\end{methoddesc}
\index{log\_change() (animal.admin.BreedingAdmin method)}

\hypertarget{animal.admin.BreedingAdmin.log_change}{}\begin{methoddesc}{log\_change}{request, object, message}
Log that an object has been successfully changed.

The default implementation creates an admin LogEntry object.
\end{methoddesc}
\index{log\_deletion() (animal.admin.BreedingAdmin method)}

\hypertarget{animal.admin.BreedingAdmin.log_deletion}{}\begin{methoddesc}{log\_deletion}{request, object, object\_repr}
Log that an object has been successfully deleted. Note that since the
object is deleted, it might no longer be safe to call \emph{any} methods
on the object, hence this method getting object\_repr.

The default implementation creates an admin LogEntry object.
\end{methoddesc}
\index{media (animal.admin.BreedingAdmin attribute)}

\hypertarget{animal.admin.BreedingAdmin.media}{}\begin{memberdesc}{media}\end{memberdesc}
\index{message\_user() (animal.admin.BreedingAdmin method)}

\hypertarget{animal.admin.BreedingAdmin.message_user}{}\begin{methoddesc}{message\_user}{request, message}
Send a message to the user. The default implementation
posts a message using the auth Message object.
\end{methoddesc}
\index{queryset() (animal.admin.BreedingAdmin method)}

\hypertarget{animal.admin.BreedingAdmin.queryset}{}\begin{methoddesc}{queryset}{request}
Returns a QuerySet of all model instances that can be edited by the
admin site. This is used by changelist\_view.
\end{methoddesc}
\index{render\_change\_form() (animal.admin.BreedingAdmin method)}

\hypertarget{animal.admin.BreedingAdmin.render_change_form}{}\begin{methoddesc}{render\_change\_form}{request, context, add=False, change=False, form\_url='', obj=None}\end{methoddesc}
\index{response\_action() (animal.admin.BreedingAdmin method)}

\hypertarget{animal.admin.BreedingAdmin.response_action}{}\begin{methoddesc}{response\_action}{request, queryset}
Handle an admin action. This is called if a request is POSTed to the
changelist; it returns an HttpResponse if the action was handled, and
None otherwise.
\end{methoddesc}
\index{response\_add() (animal.admin.BreedingAdmin method)}

\hypertarget{animal.admin.BreedingAdmin.response_add}{}\begin{methoddesc}{response\_add}{request, obj, post\_url\_continue='../\%s/'}
Determines the HttpResponse for the add\_view stage.
\end{methoddesc}
\index{response\_change() (animal.admin.BreedingAdmin method)}

\hypertarget{animal.admin.BreedingAdmin.response_change}{}\begin{methoddesc}{response\_change}{request, obj}
Determines the HttpResponse for the change\_view stage.
\end{methoddesc}
\index{save\_form() (animal.admin.BreedingAdmin method)}

\hypertarget{animal.admin.BreedingAdmin.save_form}{}\begin{methoddesc}{save\_form}{request, form, change}
Given a ModelForm return an unsaved instance. \code{change} is True if
the object is being changed, and False if it's being added.
\end{methoddesc}
\index{save\_formset() (animal.admin.BreedingAdmin method)}

\hypertarget{animal.admin.BreedingAdmin.save_formset}{}\begin{methoddesc}{save\_formset}{request, form, formset, change}
Given an inline formset save it to the database.
\end{methoddesc}
\index{save\_model() (animal.admin.BreedingAdmin method)}

\hypertarget{animal.admin.BreedingAdmin.save_model}{}\begin{methoddesc}{save\_model}{request, obj, form, change}
Given a model instance save it to the database.
\end{methoddesc}
\index{urls (animal.admin.BreedingAdmin attribute)}

\hypertarget{animal.admin.BreedingAdmin.urls}{}\begin{memberdesc}{urls}\end{memberdesc}
\end{classdesc}
\index{CageAdmin (class in animal.admin)}

\hypertarget{animal.admin.CageAdmin}{}\begin{classdesc}{CageAdmin}{model, admin\_site}~\index{action\_checkbox() (animal.admin.CageAdmin method)}

\hypertarget{animal.admin.CageAdmin.action_checkbox}{}\begin{methoddesc}{action\_checkbox}{obj}
A list\_display column containing a checkbox widget.
\end{methoddesc}
\index{action\_form (animal.admin.CageAdmin attribute)}

\hypertarget{animal.admin.CageAdmin.action_form}{}\begin{memberdesc}{action\_form}
alias of \code{ActionForm}
\end{memberdesc}
\index{add\_view() (animal.admin.CageAdmin method)}

\hypertarget{animal.admin.CageAdmin.add_view}{}\begin{methoddesc}{add\_view}{*args, **kw}
The `add' admin view for this model.
\end{methoddesc}
\index{change\_view() (animal.admin.CageAdmin method)}

\hypertarget{animal.admin.CageAdmin.change_view}{}\begin{methoddesc}{change\_view}{*args, **kw}
The `change' admin view for this model.
\end{methoddesc}
\index{changelist\_view() (animal.admin.CageAdmin method)}

\hypertarget{animal.admin.CageAdmin.changelist_view}{}\begin{methoddesc}{changelist\_view}{request, extra\_context=None}
The `change list' admin view for this model.
\end{methoddesc}
\index{construct\_change\_message() (animal.admin.CageAdmin method)}

\hypertarget{animal.admin.CageAdmin.construct_change_message}{}\begin{methoddesc}{construct\_change\_message}{request, form, formsets}
Construct a change message from a changed object.
\end{methoddesc}
\index{declared\_fieldsets (animal.admin.CageAdmin attribute)}

\hypertarget{animal.admin.CageAdmin.declared_fieldsets}{}\begin{memberdesc}{declared\_fieldsets}\end{memberdesc}
\index{delete\_view() (animal.admin.CageAdmin method)}

\hypertarget{animal.admin.CageAdmin.delete_view}{}\begin{methoddesc}{delete\_view}{request, object\_id, extra\_context=None}
The `delete' admin view for this model.
\end{methoddesc}
\index{form (animal.admin.CageAdmin attribute)}

\hypertarget{animal.admin.CageAdmin.form}{}\begin{memberdesc}{form}
alias of \code{ModelForm}
\end{memberdesc}
\index{formfield\_for\_choice\_field() (animal.admin.CageAdmin method)}

\hypertarget{animal.admin.CageAdmin.formfield_for_choice_field}{}\begin{methoddesc}{formfield\_for\_choice\_field}{db\_field, request=None, **kwargs}
Get a form Field for a database Field that has declared choices.
\end{methoddesc}
\index{formfield\_for\_dbfield() (animal.admin.CageAdmin method)}

\hypertarget{animal.admin.CageAdmin.formfield_for_dbfield}{}\begin{methoddesc}{formfield\_for\_dbfield}{db\_field, **kwargs}
Hook for specifying the form Field instance for a given database Field
instance.

If kwargs are given, they're passed to the form Field's constructor.
\end{methoddesc}
\index{formfield\_for\_foreignkey() (animal.admin.CageAdmin method)}

\hypertarget{animal.admin.CageAdmin.formfield_for_foreignkey}{}\begin{methoddesc}{formfield\_for\_foreignkey}{db\_field, request=None, **kwargs}
Get a form Field for a ForeignKey.
\end{methoddesc}
\index{formfield\_for\_manytomany() (animal.admin.CageAdmin method)}

\hypertarget{animal.admin.CageAdmin.formfield_for_manytomany}{}\begin{methoddesc}{formfield\_for\_manytomany}{db\_field, request=None, **kwargs}
Get a form Field for a ManyToManyField.
\end{methoddesc}
\index{get\_action() (animal.admin.CageAdmin method)}

\hypertarget{animal.admin.CageAdmin.get_action}{}\begin{methoddesc}{get\_action}{action}
Return a given action from a parameter, which can either be a callable,
or the name of a method on the ModelAdmin.  Return is a tuple of
(callable, name, description).
\end{methoddesc}
\index{get\_action\_choices() (animal.admin.CageAdmin method)}

\hypertarget{animal.admin.CageAdmin.get_action_choices}{}\begin{methoddesc}{get\_action\_choices}{request, default\_choices=, {[}('', '---------'){]}}
Return a list of choices for use in a form object.  Each choice is a
tuple (name, description).
\end{methoddesc}
\index{get\_actions() (animal.admin.CageAdmin method)}

\hypertarget{animal.admin.CageAdmin.get_actions}{}\begin{methoddesc}{get\_actions}{request}
Return a dictionary mapping the names of all actions for this
ModelAdmin to a tuple of (callable, name, description) for each action.
\end{methoddesc}
\index{get\_changelist\_form() (animal.admin.CageAdmin method)}

\hypertarget{animal.admin.CageAdmin.get_changelist_form}{}\begin{methoddesc}{get\_changelist\_form}{request, **kwargs}
Returns a Form class for use in the Formset on the changelist page.
\end{methoddesc}
\index{get\_changelist\_formset() (animal.admin.CageAdmin method)}

\hypertarget{animal.admin.CageAdmin.get_changelist_formset}{}\begin{methoddesc}{get\_changelist\_formset}{request, **kwargs}
Returns a FormSet class for use on the changelist page if list\_editable
is used.
\end{methoddesc}
\index{get\_fieldsets() (animal.admin.CageAdmin method)}

\hypertarget{animal.admin.CageAdmin.get_fieldsets}{}\begin{methoddesc}{get\_fieldsets}{request, obj=None}
Hook for specifying fieldsets for the add form.
\end{methoddesc}
\index{get\_form() (animal.admin.CageAdmin method)}

\hypertarget{animal.admin.CageAdmin.get_form}{}\begin{methoddesc}{get\_form}{request, obj=None, **kwargs}
Returns a Form class for use in the admin add view. This is used by
add\_view and change\_view.
\end{methoddesc}
\index{get\_formsets() (animal.admin.CageAdmin method)}

\hypertarget{animal.admin.CageAdmin.get_formsets}{}\begin{methoddesc}{get\_formsets}{request, obj=None}\end{methoddesc}
\index{get\_model\_perms() (animal.admin.CageAdmin method)}

\hypertarget{animal.admin.CageAdmin.get_model_perms}{}\begin{methoddesc}{get\_model\_perms}{request}
Returns a dict of all perms for this model. This dict has the keys
\code{add}, \code{change}, and \code{delete} mapping to the True/False for each
of those actions.
\end{methoddesc}
\index{get\_urls() (animal.admin.CageAdmin method)}

\hypertarget{animal.admin.CageAdmin.get_urls}{}\begin{methoddesc}{get\_urls}{}\end{methoddesc}
\index{has\_add\_permission() (animal.admin.CageAdmin method)}

\hypertarget{animal.admin.CageAdmin.has_add_permission}{}\begin{methoddesc}{has\_add\_permission}{request}
Returns True if the given request has permission to add an object.
\end{methoddesc}
\index{has\_change\_permission() (animal.admin.CageAdmin method)}

\hypertarget{animal.admin.CageAdmin.has_change_permission}{}\begin{methoddesc}{has\_change\_permission}{request, obj=None}
Returns True if the given request has permission to change the given
Django model instance.

If \emph{obj} is None, this should return True if the given request has
permission to change \emph{any} object of the given type.
\end{methoddesc}
\index{has\_delete\_permission() (animal.admin.CageAdmin method)}

\hypertarget{animal.admin.CageAdmin.has_delete_permission}{}\begin{methoddesc}{has\_delete\_permission}{request, obj=None}
Returns True if the given request has permission to change the given
Django model instance.

If \emph{obj} is None, this should return True if the given request has
permission to delete \emph{any} object of the given type.
\end{methoddesc}
\index{history\_view() (animal.admin.CageAdmin method)}

\hypertarget{animal.admin.CageAdmin.history_view}{}\begin{methoddesc}{history\_view}{request, object\_id, extra\_context=None}
The `history' admin view for this model.
\end{methoddesc}
\index{log\_addition() (animal.admin.CageAdmin method)}

\hypertarget{animal.admin.CageAdmin.log_addition}{}\begin{methoddesc}{log\_addition}{request, object}
Log that an object has been successfully added.

The default implementation creates an admin LogEntry object.
\end{methoddesc}
\index{log\_change() (animal.admin.CageAdmin method)}

\hypertarget{animal.admin.CageAdmin.log_change}{}\begin{methoddesc}{log\_change}{request, object, message}
Log that an object has been successfully changed.

The default implementation creates an admin LogEntry object.
\end{methoddesc}
\index{log\_deletion() (animal.admin.CageAdmin method)}

\hypertarget{animal.admin.CageAdmin.log_deletion}{}\begin{methoddesc}{log\_deletion}{request, object, object\_repr}
Log that an object has been successfully deleted. Note that since the
object is deleted, it might no longer be safe to call \emph{any} methods
on the object, hence this method getting object\_repr.

The default implementation creates an admin LogEntry object.
\end{methoddesc}
\index{media (animal.admin.CageAdmin attribute)}

\hypertarget{animal.admin.CageAdmin.media}{}\begin{memberdesc}{media}\end{memberdesc}
\index{message\_user() (animal.admin.CageAdmin method)}

\hypertarget{animal.admin.CageAdmin.message_user}{}\begin{methoddesc}{message\_user}{request, message}
Send a message to the user. The default implementation
posts a message using the auth Message object.
\end{methoddesc}
\index{queryset() (animal.admin.CageAdmin method)}

\hypertarget{animal.admin.CageAdmin.queryset}{}\begin{methoddesc}{queryset}{request}
Returns a QuerySet of all model instances that can be edited by the
admin site. This is used by changelist\_view.
\end{methoddesc}
\index{render\_change\_form() (animal.admin.CageAdmin method)}

\hypertarget{animal.admin.CageAdmin.render_change_form}{}\begin{methoddesc}{render\_change\_form}{request, context, add=False, change=False, form\_url='', obj=None}\end{methoddesc}
\index{response\_action() (animal.admin.CageAdmin method)}

\hypertarget{animal.admin.CageAdmin.response_action}{}\begin{methoddesc}{response\_action}{request, queryset}
Handle an admin action. This is called if a request is POSTed to the
changelist; it returns an HttpResponse if the action was handled, and
None otherwise.
\end{methoddesc}
\index{response\_add() (animal.admin.CageAdmin method)}

\hypertarget{animal.admin.CageAdmin.response_add}{}\begin{methoddesc}{response\_add}{request, obj, post\_url\_continue='../\%s/'}
Determines the HttpResponse for the add\_view stage.
\end{methoddesc}
\index{response\_change() (animal.admin.CageAdmin method)}

\hypertarget{animal.admin.CageAdmin.response_change}{}\begin{methoddesc}{response\_change}{request, obj}
Determines the HttpResponse for the change\_view stage.
\end{methoddesc}
\index{save\_form() (animal.admin.CageAdmin method)}

\hypertarget{animal.admin.CageAdmin.save_form}{}\begin{methoddesc}{save\_form}{request, form, change}
Given a ModelForm return an unsaved instance. \code{change} is True if
the object is being changed, and False if it's being added.
\end{methoddesc}
\index{save\_formset() (animal.admin.CageAdmin method)}

\hypertarget{animal.admin.CageAdmin.save_formset}{}\begin{methoddesc}{save\_formset}{request, form, formset, change}
Given an inline formset save it to the database.
\end{methoddesc}
\index{save\_model() (animal.admin.CageAdmin method)}

\hypertarget{animal.admin.CageAdmin.save_model}{}\begin{methoddesc}{save\_model}{request, obj, form, change}
Given a model instance save it to the database.
\end{methoddesc}
\index{urls (animal.admin.CageAdmin attribute)}

\hypertarget{animal.admin.CageAdmin.urls}{}\begin{memberdesc}{urls}\end{memberdesc}
\end{classdesc}
\index{StrainAdmin (class in animal.admin)}

\hypertarget{animal.admin.StrainAdmin}{}\begin{classdesc}{StrainAdmin}{model, admin\_site}~\index{action\_checkbox() (animal.admin.StrainAdmin method)}

\hypertarget{animal.admin.StrainAdmin.action_checkbox}{}\begin{methoddesc}{action\_checkbox}{obj}
A list\_display column containing a checkbox widget.
\end{methoddesc}
\index{action\_form (animal.admin.StrainAdmin attribute)}

\hypertarget{animal.admin.StrainAdmin.action_form}{}\begin{memberdesc}{action\_form}
alias of \code{ActionForm}
\end{memberdesc}
\index{add\_view() (animal.admin.StrainAdmin method)}

\hypertarget{animal.admin.StrainAdmin.add_view}{}\begin{methoddesc}{add\_view}{*args, **kw}
The `add' admin view for this model.
\end{methoddesc}
\index{change\_view() (animal.admin.StrainAdmin method)}

\hypertarget{animal.admin.StrainAdmin.change_view}{}\begin{methoddesc}{change\_view}{*args, **kw}
The `change' admin view for this model.
\end{methoddesc}
\index{changelist\_view() (animal.admin.StrainAdmin method)}

\hypertarget{animal.admin.StrainAdmin.changelist_view}{}\begin{methoddesc}{changelist\_view}{request, extra\_context=None}
The `change list' admin view for this model.
\end{methoddesc}
\index{construct\_change\_message() (animal.admin.StrainAdmin method)}

\hypertarget{animal.admin.StrainAdmin.construct_change_message}{}\begin{methoddesc}{construct\_change\_message}{request, form, formsets}
Construct a change message from a changed object.
\end{methoddesc}
\index{declared\_fieldsets (animal.admin.StrainAdmin attribute)}

\hypertarget{animal.admin.StrainAdmin.declared_fieldsets}{}\begin{memberdesc}{declared\_fieldsets}\end{memberdesc}
\index{delete\_view() (animal.admin.StrainAdmin method)}

\hypertarget{animal.admin.StrainAdmin.delete_view}{}\begin{methoddesc}{delete\_view}{request, object\_id, extra\_context=None}
The `delete' admin view for this model.
\end{methoddesc}
\index{form (animal.admin.StrainAdmin attribute)}

\hypertarget{animal.admin.StrainAdmin.form}{}\begin{memberdesc}{form}
alias of \code{ModelForm}
\end{memberdesc}
\index{formfield\_for\_choice\_field() (animal.admin.StrainAdmin method)}

\hypertarget{animal.admin.StrainAdmin.formfield_for_choice_field}{}\begin{methoddesc}{formfield\_for\_choice\_field}{db\_field, request=None, **kwargs}
Get a form Field for a database Field that has declared choices.
\end{methoddesc}
\index{formfield\_for\_dbfield() (animal.admin.StrainAdmin method)}

\hypertarget{animal.admin.StrainAdmin.formfield_for_dbfield}{}\begin{methoddesc}{formfield\_for\_dbfield}{db\_field, **kwargs}
Hook for specifying the form Field instance for a given database Field
instance.

If kwargs are given, they're passed to the form Field's constructor.
\end{methoddesc}
\index{formfield\_for\_foreignkey() (animal.admin.StrainAdmin method)}

\hypertarget{animal.admin.StrainAdmin.formfield_for_foreignkey}{}\begin{methoddesc}{formfield\_for\_foreignkey}{db\_field, request=None, **kwargs}
Get a form Field for a ForeignKey.
\end{methoddesc}
\index{formfield\_for\_manytomany() (animal.admin.StrainAdmin method)}

\hypertarget{animal.admin.StrainAdmin.formfield_for_manytomany}{}\begin{methoddesc}{formfield\_for\_manytomany}{db\_field, request=None, **kwargs}
Get a form Field for a ManyToManyField.
\end{methoddesc}
\index{get\_action() (animal.admin.StrainAdmin method)}

\hypertarget{animal.admin.StrainAdmin.get_action}{}\begin{methoddesc}{get\_action}{action}
Return a given action from a parameter, which can either be a callable,
or the name of a method on the ModelAdmin.  Return is a tuple of
(callable, name, description).
\end{methoddesc}
\index{get\_action\_choices() (animal.admin.StrainAdmin method)}

\hypertarget{animal.admin.StrainAdmin.get_action_choices}{}\begin{methoddesc}{get\_action\_choices}{request, default\_choices=, {[}('', '---------'){]}}
Return a list of choices for use in a form object.  Each choice is a
tuple (name, description).
\end{methoddesc}
\index{get\_actions() (animal.admin.StrainAdmin method)}

\hypertarget{animal.admin.StrainAdmin.get_actions}{}\begin{methoddesc}{get\_actions}{request}
Return a dictionary mapping the names of all actions for this
ModelAdmin to a tuple of (callable, name, description) for each action.
\end{methoddesc}
\index{get\_changelist\_form() (animal.admin.StrainAdmin method)}

\hypertarget{animal.admin.StrainAdmin.get_changelist_form}{}\begin{methoddesc}{get\_changelist\_form}{request, **kwargs}
Returns a Form class for use in the Formset on the changelist page.
\end{methoddesc}
\index{get\_changelist\_formset() (animal.admin.StrainAdmin method)}

\hypertarget{animal.admin.StrainAdmin.get_changelist_formset}{}\begin{methoddesc}{get\_changelist\_formset}{request, **kwargs}
Returns a FormSet class for use on the changelist page if list\_editable
is used.
\end{methoddesc}
\index{get\_fieldsets() (animal.admin.StrainAdmin method)}

\hypertarget{animal.admin.StrainAdmin.get_fieldsets}{}\begin{methoddesc}{get\_fieldsets}{request, obj=None}
Hook for specifying fieldsets for the add form.
\end{methoddesc}
\index{get\_form() (animal.admin.StrainAdmin method)}

\hypertarget{animal.admin.StrainAdmin.get_form}{}\begin{methoddesc}{get\_form}{request, obj=None, **kwargs}
Returns a Form class for use in the admin add view. This is used by
add\_view and change\_view.
\end{methoddesc}
\index{get\_formsets() (animal.admin.StrainAdmin method)}

\hypertarget{animal.admin.StrainAdmin.get_formsets}{}\begin{methoddesc}{get\_formsets}{request, obj=None}\end{methoddesc}
\index{get\_model\_perms() (animal.admin.StrainAdmin method)}

\hypertarget{animal.admin.StrainAdmin.get_model_perms}{}\begin{methoddesc}{get\_model\_perms}{request}
Returns a dict of all perms for this model. This dict has the keys
\code{add}, \code{change}, and \code{delete} mapping to the True/False for each
of those actions.
\end{methoddesc}
\index{get\_urls() (animal.admin.StrainAdmin method)}

\hypertarget{animal.admin.StrainAdmin.get_urls}{}\begin{methoddesc}{get\_urls}{}\end{methoddesc}
\index{has\_add\_permission() (animal.admin.StrainAdmin method)}

\hypertarget{animal.admin.StrainAdmin.has_add_permission}{}\begin{methoddesc}{has\_add\_permission}{request}
Returns True if the given request has permission to add an object.
\end{methoddesc}
\index{has\_change\_permission() (animal.admin.StrainAdmin method)}

\hypertarget{animal.admin.StrainAdmin.has_change_permission}{}\begin{methoddesc}{has\_change\_permission}{request, obj=None}
Returns True if the given request has permission to change the given
Django model instance.

If \emph{obj} is None, this should return True if the given request has
permission to change \emph{any} object of the given type.
\end{methoddesc}
\index{has\_delete\_permission() (animal.admin.StrainAdmin method)}

\hypertarget{animal.admin.StrainAdmin.has_delete_permission}{}\begin{methoddesc}{has\_delete\_permission}{request, obj=None}
Returns True if the given request has permission to change the given
Django model instance.

If \emph{obj} is None, this should return True if the given request has
permission to delete \emph{any} object of the given type.
\end{methoddesc}
\index{history\_view() (animal.admin.StrainAdmin method)}

\hypertarget{animal.admin.StrainAdmin.history_view}{}\begin{methoddesc}{history\_view}{request, object\_id, extra\_context=None}
The `history' admin view for this model.
\end{methoddesc}
\index{log\_addition() (animal.admin.StrainAdmin method)}

\hypertarget{animal.admin.StrainAdmin.log_addition}{}\begin{methoddesc}{log\_addition}{request, object}
Log that an object has been successfully added.

The default implementation creates an admin LogEntry object.
\end{methoddesc}
\index{log\_change() (animal.admin.StrainAdmin method)}

\hypertarget{animal.admin.StrainAdmin.log_change}{}\begin{methoddesc}{log\_change}{request, object, message}
Log that an object has been successfully changed.

The default implementation creates an admin LogEntry object.
\end{methoddesc}
\index{log\_deletion() (animal.admin.StrainAdmin method)}

\hypertarget{animal.admin.StrainAdmin.log_deletion}{}\begin{methoddesc}{log\_deletion}{request, object, object\_repr}
Log that an object has been successfully deleted. Note that since the
object is deleted, it might no longer be safe to call \emph{any} methods
on the object, hence this method getting object\_repr.

The default implementation creates an admin LogEntry object.
\end{methoddesc}
\index{media (animal.admin.StrainAdmin attribute)}

\hypertarget{animal.admin.StrainAdmin.media}{}\begin{memberdesc}{media}\end{memberdesc}
\index{message\_user() (animal.admin.StrainAdmin method)}

\hypertarget{animal.admin.StrainAdmin.message_user}{}\begin{methoddesc}{message\_user}{request, message}
Send a message to the user. The default implementation
posts a message using the auth Message object.
\end{methoddesc}
\index{queryset() (animal.admin.StrainAdmin method)}

\hypertarget{animal.admin.StrainAdmin.queryset}{}\begin{methoddesc}{queryset}{request}
Returns a QuerySet of all model instances that can be edited by the
admin site. This is used by changelist\_view.
\end{methoddesc}
\index{render\_change\_form() (animal.admin.StrainAdmin method)}

\hypertarget{animal.admin.StrainAdmin.render_change_form}{}\begin{methoddesc}{render\_change\_form}{request, context, add=False, change=False, form\_url='', obj=None}\end{methoddesc}
\index{response\_action() (animal.admin.StrainAdmin method)}

\hypertarget{animal.admin.StrainAdmin.response_action}{}\begin{methoddesc}{response\_action}{request, queryset}
Handle an admin action. This is called if a request is POSTed to the
changelist; it returns an HttpResponse if the action was handled, and
None otherwise.
\end{methoddesc}
\index{response\_add() (animal.admin.StrainAdmin method)}

\hypertarget{animal.admin.StrainAdmin.response_add}{}\begin{methoddesc}{response\_add}{request, obj, post\_url\_continue='../\%s/'}
Determines the HttpResponse for the add\_view stage.
\end{methoddesc}
\index{response\_change() (animal.admin.StrainAdmin method)}

\hypertarget{animal.admin.StrainAdmin.response_change}{}\begin{methoddesc}{response\_change}{request, obj}
Determines the HttpResponse for the change\_view stage.
\end{methoddesc}
\index{save\_form() (animal.admin.StrainAdmin method)}

\hypertarget{animal.admin.StrainAdmin.save_form}{}\begin{methoddesc}{save\_form}{request, form, change}
Given a ModelForm return an unsaved instance. \code{change} is True if
the object is being changed, and False if it's being added.
\end{methoddesc}
\index{save\_formset() (animal.admin.StrainAdmin method)}

\hypertarget{animal.admin.StrainAdmin.save_formset}{}\begin{methoddesc}{save\_formset}{request, form, formset, change}
Given an inline formset save it to the database.
\end{methoddesc}
\index{save\_model() (animal.admin.StrainAdmin method)}

\hypertarget{animal.admin.StrainAdmin.save_model}{}\begin{methoddesc}{save\_model}{request, obj, form, change}
Given a model instance save it to the database.
\end{methoddesc}
\index{urls (animal.admin.StrainAdmin attribute)}

\hypertarget{animal.admin.StrainAdmin.urls}{}\begin{memberdesc}{urls}\end{memberdesc}
\end{classdesc}


\chapter{Indices and tables}
\begin{itemize}
\item {} 
\emph{Index}

\item {} 
\emph{Module Index}

\item {} 
\emph{Search Page}

\end{itemize}


\renewcommand{\indexname}{Module Index}
\printmodindex
\renewcommand{\indexname}{Index}
\printindex
\end{document}
